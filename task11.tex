\documentclass[]{article}
\usepackage{lmodern}
\usepackage{amssymb,amsmath}
\usepackage{ifxetex,ifluatex}


\usepackage[utf8]{inputenc}
\usepackage[english,russian,ukrainian]{babel}

\usepackage{fixltx2e} % provides \textsubscript
\ifnum 0\ifxetex 1\fi\ifluatex 1\fi=0 % if pdftex
  \usepackage[T1]{fontenc}
  \usepackage[utf8]{inputenc}
\else % if luatex or xelatex
  \ifxetex
    \usepackage{mathspec}
  \else
    \usepackage{fontspec}
  \fi
  \defaultfontfeatures{Ligatures=TeX,Scale=MatchLowercase}
\fi
% use upquote if available, for straight quotes in verbatim environments
\IfFileExists{upquote.sty}{\usepackage{upquote}}{}
% use microtype if available
\IfFileExists{microtype.sty}{%
\usepackage{microtype}
\UseMicrotypeSet[protrusion]{basicmath} % disable protrusion for tt fonts
}{}
\usepackage[unicode=true]{hyperref}
\hypersetup{
            pdfborder={0 0 0},
            breaklinks=true}
\urlstyle{same}  % don't use monospace font for urls
\usepackage{graphicx,grffile}
\makeatletter
\def\maxwidth{\ifdim\Gin@nat@width>\linewidth\linewidth\else\Gin@nat@width\fi}
\def\maxheight{\ifdim\Gin@nat@height>\textheight\textheight\else\Gin@nat@height\fi}
\makeatother
% Scale images if necessary, so that they will not overflow the page
% margins by default, and it is still possible to overwrite the defaults
% using explicit options in \includegraphics[width, height, ...]{}
\setkeys{Gin}{width=\maxwidth,height=\maxheight,keepaspectratio}
\IfFileExists{parskip.sty}{%
\usepackage{parskip}
}{% else
\setlength{\parindent}{0pt}
\setlength{\parskip}{6pt plus 2pt minus 1pt}
}
\setlength{\emergencystretch}{3em}  % prevent overfull lines
\providecommand{\tightlist}{%
  \setlength{\itemsep}{0pt}\setlength{\parskip}{0pt}}
\setcounter{secnumdepth}{0}
% Redefines (sub)paragraphs to behave more like sections
\ifx\paragraph\undefined\else
\let\oldparagraph\paragraph
\renewcommand{\paragraph}[1]{\oldparagraph{#1}\mbox{}}
\fi
\ifx\subparagraph\undefined\else
\let\oldsubparagraph\subparagraph
\renewcommand{\subparagraph}[1]{\oldsubparagraph{#1}\mbox{}}
\fi

\date{}


\usepackage{enumitem}
\makeatletter
\newcommand{\xslalph}[1]{\expandafter\@xslalph\csname c@#1\endcsname}
\newcommand{\@xslalph}[1]{%
    \ifcase#1\or а\or б\or в\or г\or д\or e\or є\or ж\or з\or i%
    \or й\or к\or л\or м\or н\or о\or п\or р\or с\or т%
    \or у\or ф\or х\or ц\or ч\or ш\or ю\or я\or аа\or бб\or вв%
    \else\@ctrerr\fi%
}
\AddEnumerateCounter{\xslalph}{\@xslalph}{m}
\makeatother


\begin{document}


\newpage
\subsection{4.0. Структури. Створення власного типу}
\setcounter{subsection}{1}

\begin{itemize}
\item Що таке структура та як її створити на Сі?
\item Як створити власний тип даних на Сі?
\item Як визначити структуру що має посилання на саму себе?
\item Які варіанти ініціалізації структур? Як ввести структуру? Як отримати
структуру як результат роботи функції? Через змінний аргумент?
\item Нащо використовувати typedef при створенні власної структури?
\end{itemize}

Задачі для аудиторної роботи

\begin{enumerate}
\def\labelenumi{\arabic{enumi})}
\item
  Визначити типи структури для зображення наступних понять та функції їх вводу-виводу:
 \begin{enumerate}[label=\xslalph*)]
 \item дата (число, місяць, рік);
 \item поле шахової дошки (напр., а5, b8);
 \item прямокутник зі сторонами, паралельними осям координат --- заданий через дві вершини.
Вершина в свою чергу --- теж структура яка містить дві дійсні координати.
 \end{enumerate}

\item
 Використовуючи тип Поле шахової дошки описати булеву функцію, яка
перевіряє, чи може ферзь за один хід перейти з одного заданого поля
шахової дошки на інше задане поле.

\item
 Визначимо тип Rational (Раціональне число) як:

typedef struct \{

int numerator; // чисельник

unsigned int denominator; // знаменник

\} Rational;

Визначити функції для:
\begin{itemize}
\item обчислення суми двох раціональних чисел;
\item обчислення добутку двох раціональних чисел;
\item порівняння двох раціональних чисел;
\item зведення раціонального числа до нескоротного виду.
\end{itemize}

\item
 Використовуючи опис типу Дата, визначити функції обчислення:
дати вчорашнього дня та дня тижня за його датою в поточному році.

\item
 Задано масив розмірності N, компонентами якого є структури, що містять відомості про вершини гір. У
відомостях про кожну вершину вказуються назва гори та її висота.
Визначити функції введення/виведення гір та функції пошуку назви
найвищої вершини та виведення висоти вершини з заданою назвою (якщо
вершини з такою назвою немає в масиви --- вивести відповідне
повідомлення).
\end{enumerate}

Задачі для самостійної роботи

\begin{enumerate}
\def\labelenumi{\arabic{enumi})}
\setcounter{enumi}{5}
\item
 Визначити типи запису для зображення наступних понять та реалізуйте
їх функції введення виведення:
\begin{enumerate}[label=\xslalph*)]
\item ціна (гривні, копійки);
\item час (година, хвилина, секунда);
\item повна дата (число, місяць, рік, година, хвилина);
\item адреса (місто, вулиця, будинок, квартира);
\item семінар (предмет, викладач, № групи, день тижня, години занять, аудиторія);
\item бланк вимоги на книгу (відомості про книгу: шифр, автор, назва;
відомості про читача: № читацького квитка, прізвище; дата замовлення);
\item коло (радіус, координати центра);
\item сфера в просторі;
\item прямокутний паралеліпипед (сторони якого паралельні вісям координат);
\item поліном довільного ступеня (дійсні коефіцієнти --- безрозмірний
масив).

\end{enumerate}

\item
  В масиві структур записано вартість та вік кожної з N моделей легкових
  автомобілів. Визначити середню вартість автомобілів, вік яких більший
  за 5 років.
\item
  В масиві структур записано інформацію про ціну та наклад кожного з N
  журналів. Знайти середню вартість журналів, наклад яких менший за
  10000 примірників.
\item
  В масиві структур записано дані про масу й об'єм N предметів,
  виготовлених із різних матеріалів. Знайти предмет, густина матеріалу
  якого найбільша.
\item
  В масиві структур записано дані про чисельність населення (у мільйонах
  жителів) та площі N держав. Знайти країну з мінімальною щільністю
  населення.
\item
  Задано масив С розмірності N, компонентами якого є відомості про
  мешканців деяких міст. Інформація про кожного мешканця містить його
  прізвище, назву міста, місцеву адресу у вигляді вулиці, будинку,
  квартири. Визначити функцію пошуку двох будь-яких жителів, що мешкають
  у різних містах за однаковою адресою.
\item
  Відомо дані про вартість кожного з N найменувань товарів: кількість
  гривень, кількість копійок. Скласти підпрограми пошуку:
\begin{enumerate}[label=\xslalph*)]
\item найдешевшого товару в магазині;
\item найдорожчого товару в магазині;
\item товару, вартість якого відрізняється від середньої вартості товару в магазині не більш ніж на 5 гривень:

\end{enumerate}

\item
  Задано масив Р розмірності N, компонентами якого є стурктури, що
  містять анкети службовців деякого закладу. У кожній анкеті вказуться
  прізвище та ім'я службовця, його стать, дата народження у вигляі
  числа, місяця, року. Визначити підпрограми пошуку:
\begin{enumerate}[label=\xslalph*)]
\item посади, яку обіймає найбільша кількість співробітників;
\item співробітників з однаковими іменами;
\item співробітників, прізвища яких починаються із заданої літери;
\item найстаршого з чоловіків цього закладу;
\item співробітників, вік яких менший за середній по організації;
\item пенсійного віку (урахувати, що пенсійний вік чоловіків і жінок --
різний).
\end{enumerate}

\item
  Задано масив $P$, компонентами $P_i$ якого є записи, що містять дані про
  людину на ім'я $i$ з указаного списку. Кожне дане складається зі статі
  людини та її зросту. Визначити підпрограми для:
\begin{enumerate}[label=\xslalph*)]
\item обчислення середнього зросту жінок;
\item пошуку найвищого чоловіка;
\item перевірки, чи є дві людини, однакові на зріст.
\end{enumerate}

\item
  Задано масив розмірності N, компоненти якого містять інформацію про
  студентів деякого вишу. Відомості про кожного студента містять дані
  про його прізвище, ім'я, по батькові, стать, вік, курс. Визначити
  процедуру пошуку:
\begin{enumerate}[label=\xslalph*)]
\item найпоширеніших чоловічих і жіночих імен;
\item прізвищ та ініціалів усіх студентів, вік яких є найпоширенішим.
\end{enumerate}

\item
  Задано масив розмірності N, компонентами якого є відомості про
  складання іспитів студентами деякого вишу. Інформація про кожного
  студента задана в такому вигляді: прізвище, номер групи, оцінка\_1,
  оцінка\_2, оцінка\_3. Визначити процедуру пошуку:
\begin{enumerate}[label=\xslalph*)]
\item студентів, що мають заборгованості принаймні з одного з предметів;
\item предмета, складеного найуспішніше;
\item студентів, що склали всі іспити на 5 і 4.
\end{enumerate}

\end{enumerate}

Додаткові задачі:

\begin{enumerate}
\def\labelenumi{\arabic{enumi})}
\setcounter{enumi}{17}
\item
  Визначити універсальний тип, який допускає зображення точки на площині
  у прямокутній або полярній системі координат (3-тє поле -- тип
  координат). Побудувати функцію обчислення площі трикутника з вершинами
  A, B, C.
\end{enumerate}

\end{document}
