\documentclass[]{article}
\usepackage{lmodern}
\usepackage{amssymb,amsmath}
\usepackage{ifxetex,ifluatex}


\usepackage[utf8]{inputenc}
\usepackage[english,russian,ukrainian]{babel}

\usepackage{fixltx2e} % provides \textsubscript
\ifnum 0\ifxetex 1\fi\ifluatex 1\fi=0 % if pdftex
  \usepackage[T1]{fontenc}
  \usepackage[utf8]{inputenc}
\else % if luatex or xelatex
  \ifxetex
    \usepackage{mathspec}
  \else
    \usepackage{fontspec}
  \fi
  \defaultfontfeatures{Ligatures=TeX,Scale=MatchLowercase}
\fi
% use upquote if available, for straight quotes in verbatim environments
\IfFileExists{upquote.sty}{\usepackage{upquote}}{}
% use microtype if available
\IfFileExists{microtype.sty}{%
\usepackage{microtype}
\UseMicrotypeSet[protrusion]{basicmath} % disable protrusion for tt fonts
}{}
\usepackage[unicode=true]{hyperref}
\hypersetup{
            pdfborder={0 0 0},
            breaklinks=true}
\urlstyle{same}  % don't use monospace font for urls
\usepackage{graphicx,grffile}
\makeatletter
\def\maxwidth{\ifdim\Gin@nat@width>\linewidth\linewidth\else\Gin@nat@width\fi}
\def\maxheight{\ifdim\Gin@nat@height>\textheight\textheight\else\Gin@nat@height\fi}
\makeatother
% Scale images if necessary, so that they will not overflow the page
% margins by default, and it is still possible to overwrite the defaults
% using explicit options in \includegraphics[width, height, ...]{}
\setkeys{Gin}{width=\maxwidth,height=\maxheight,keepaspectratio}
\IfFileExists{parskip.sty}{%
\usepackage{parskip}
}{% else
\setlength{\parindent}{0pt}
\setlength{\parskip}{6pt plus 2pt minus 1pt}
}
\setlength{\emergencystretch}{3em}  % prevent overfull lines
\providecommand{\tightlist}{%
  \setlength{\itemsep}{0pt}\setlength{\parskip}{0pt}}
\setcounter{secnumdepth}{0}
% Redefines (sub)paragraphs to behave more like sections
\ifx\paragraph\undefined\else
\let\oldparagraph\paragraph
\renewcommand{\paragraph}[1]{\oldparagraph{#1}\mbox{}}
\fi
\ifx\subparagraph\undefined\else
\let\oldsubparagraph\subparagraph
\renewcommand{\subparagraph}[1]{\oldsubparagraph{#1}\mbox{}}
\fi

\date{}


\usepackage{enumitem}
\makeatletter
\newcommand{\xslalph}[1]{\expandafter\@xslalph\csname c@#1\endcsname}
\newcommand{\@xslalph}[1]{%
    \ifcase#1\or а\or б\or в\or г\or д\or e\or є\or ж\or з\or i%
    \or й\or к\or л\or м\or н\or о\or п\or р\or с\or т%
    \or у\or ф\or х\or ц\or ч\or ш\or ю\or я\or аа\or бб\or вв%
    \else\@ctrerr\fi%
}
\AddEnumerateCounter{\xslalph}{\@xslalph}{m}
\makeatother


\begin{document}


\newpage
\subsection{18. Стандартна бібліотека С++. Контейнери.}
\setcounter{subsection}{1}


\begin{itemize}

\item
  Створіть власний клас-шаблон vector\textless{}T\textgreater{} з
  методом Норма(). Порівняйте його дію з стандартним шаблоном vector в
  головній програмі.
\item
  З яких частин складається бібліотека шаблонів Сі++?
\item
  Для чого потрібні контейнери-адаптори? Які контейнери-адаптори
  визначені в Сі++?
\item
  Які контейнери прямого доступу визначені в Сі++?
\item
  Яка різниця між контейнерами list, forward\_list, vector, array?
\item
  Основні методи контейнеру вектор (доступ до елементів, заміна
  елементів, розміри)?
\item
  Які переваги array або vector перед стандартним масивом чи
  вказівником?
\item
  Як додавати елемент в вектор, стек, список?
\item
  Як видаляти елементи в list, forward\_list, vector, array?
\item
  Які варіанти проітеруватись по елементах послідовних контейнерів?

\item
Як визначити кількість елементів будь-якого контейнеру?
\item
Які коректні шляхи ітерації по вектору? Мультивідображенню? Будь-якому
контейнеру?
\item
Як коректно пройти по всім елементам відображення?

\end{itemize}

Задачі для аудиторної роботи

\begin{enumerate}
\def\labelenumi{\arabic{enumi})}

\item
Реалізувати функції для введення d-вимірних векторів 
(d вводиться з клавіатури). Ввести n d-вимірних векторів x
 та обчислити значення суми норм векторів.

\item
  Створіть клас-шаблон Поліном, який приймає вектор чисел (будь-якого
  типу) --- вектор (на базі стандартного класу vector) коефіцієнтів
  поліному. Методи: введення-виведення, додавання, множення та
  обчислення значення. Перевірте, що клас працює коректно для дійсних,
  цілих чисел та для типу Раціональний дріб з попередніх завдань.


\end{enumerate}

Задачі для самостійної роботи

\begin{enumerate}
\def\labelenumi{\arabic{enumi})}
\setcounter{enumi}{7}
\item
 
  Біля прилавка в магазині вишикувалася черга з п покупців. Час
  обслуговування продавцем i-го покупця число
  \(t_{i},\ i = 1,\cdots,n\). Нехай дано натуральне n і дійсні числа
  \(t_{1},t_{2},\cdots,t_{n}\). Отримати \(c_{1},c_{2},\cdots,c_{n}\) де
  з \(c_{i}\ \)-- час перебування i-го покупця в черзі
  \(i = 1,\cdots,n\). Вказати номер покупця, для обслуговування якого
  продавцеві потрібно найменше часу.
\item
  Дана матриця з цілих чисел. Знайти в ній прямокутну підматрицю, що
  складається з максимальної кількості однакових елементів.
  Використовувати клас Stack.
\item
  Реалізувати структуру «чорний ящик» на базі Queue, що зберігає множину
  чисел і має внутрішній лічильник K, спочатку рівний нулю. Cтруктура
  повинна підтримувати операції додавання числа в множину і повернення
  K-го по мінімальності числа з множини.
\item
  На клітковому аркуші намальований круг. Вивести в файл опису всіх
  клітин, цілком лежать всередині кола в порядку зростання відстані від
  клітини до центру кола. Використовувати клас PriorityQueue.
\item
  На базі шаблону List реалізувати структуру зберігання чисел з
  підтримкою наступних операцій:

  \begin{itemize}
    \item
    додавання / видалення числа;
  \item
    пошук числа, найбільш близького до заданого (тобто модуль різниці
    мінімальний).
  \end{itemize}
\item
  У вхідному файлі розташовані два набору додатніх чилих чисел; між наборами
  -- роздільник від'ємне число. Побудувати два списки C1 і С2, елементи яких
  містять відповідно числа 1-го і 2-го набору таким чином, щоб усередині
  одного списку числа були впорядковані по зростанню. Потім об'єднати
  списки C1 і С2 в один відсортований список.

\item

Реалізуйте клас Auto, що містить члени: назва, модель, номер, ідентифікатор власника.
Визначте для цього класу методи введення/виведення.
Реалізуйте за допомогою стандартних шаблонів наступні задачі:
\begin{enumerate}[label=\xslalph*)]
\item
    в шаблоні vector даний масив даних про авто, потрібно вивести всіх власників даної марки;
\item
    в шаблоні list є дані про авто, відсортуйте їх по назві та виведіть всі їх номери в цьому порядку; 
\item
    в шаблоні deque зберігаються дані по черги з авто на заправці --- промоделюйте запвоення черги
на заправці виводячи стан черги при кожному вибуванні чи прибуванні авто на заправку;
\item
    в шаблоні stack зберігаються авто на складі ринку, промоделюйте роботу складу;
\item
    використайте шаблон queue для моделювання черги з авто на мойці;
\item
    використайте шаблон priority\_queue для моделювання черги замовлень по ремонту в 
залежності від вартості ремонту (додатковий член класу, що вводиться окремим методом). 
\end{enumerate}    

\item
  Складіть клас Employee із двома членами даних: hours та hourlyPay.
  Працівник також повинен мати функцію calcSalary(), яка повертає
  заробітну плату за цього працівника. Генеруйте довільну погодинну
  оплату праці та години для довільної кількості працівників. Зберігайте
  вектор Співробітник. Дізнайтеся, скільки
  грошей компанія витратить за даний період оплати праці.


\item
  Створіть шаблон класу Matrix, який створений з вектору
  \textless{}vector \textless{}T\textgreater{}\textgreater{}. Надайте
  його дружньому методу ostream \& operator \textless{}\textless{}
  (ostream \&, const Matrix \&) для відображення матриці. Створіть
  наступні бінарні операції, використовуючи об'єкти функції STL, де це
  можливо: оператор + (const Matrix \&, const Matrix \&) для додавання
  матриці, оператор * (const Matrix \&, const vector
  \textless{}int\textgreater{} \&) для множення матриці на вектор та
  оператор * ( const Matrix \&, const Matrix \&) для множення матриць.
  Перевірте шаблон класу Matrix, використовуючи int і float.

\item
Реалізувати функцію, що виконує додавання чисел, заданих вектором unsigned char
в різних системах числення: 

vector <int> addition (const vector <UCHAR> \& A, int baseA,
                      const vector <UCHAR> \& B, int baseB, int baseResult);

Функція повинна перевіряти вхідні дані про коректність і повертати пустий вектор у разі виявлення помилки.
В текстових файлах записані перше та друге число та останнім числом -- основа числення.
Основа числення результату вводиться з консолі та результат записується в третій файл.
Тестування

Для основних функцій плюс повинен бути створений набір тестів, що перевіряють функції на наборі прикладів, 
та коректність введення в разі некоректних даних. 

Додатково реалізуйте також підтримку записів вхідних даних у рядках. Наприклад

16: "FF"
10: "256"
2
Результат:
"111111111" 

\item
Даний текстовий файл, що містить рядкові
представлення цілих чисел. Заповнити вектор V числами з цього файлу
та вивести їх у вихідному порядку. У випадку некоректних даних видайте
змістовне повідомлення.

\item 
В консолі вводиться масив цілих чисел. Заповнить список L ціми чис-
лами і вивести елементи списки L в початковому порядку у вихідному, а 
потім в оберненому порядку. Відсортуйте дані за зростанням у списку, але виведіть
навпаки за спаданням.

\item 
Даний вектор цілих чисел з парною кількістю елементів.
Заповнить дек D даними числами так, щоб перша полвина чисел співпадала
з порядком заповнення вектору, а друга була в зворотньому порядку.
 
\item 
Ввести список цілих чисел з консолі. Вставити перед кожним ненульовим 
елементом вихідного списку число $-1$, а після кожного рівного 2 -- нуль. 
\item 
Ввести з консолі список L натуральних чисел. Вставити після 
кожного непарного елементу з першої половини вихідного списку число -1, а 
перед кожним парним елементом другої половини -2.
\item 
Ввести з текстового файлу дек довільного типу D. Видалить середній елемент дека, якщо 
кількість елементів непарна або 2 середні елементи -- якщо парна. 

\item 
Ввести з текстового файлу дійсний вектор V з непарною кількістю елементів $N$ ($N \ge 5$).
Якщо там парна кількість елементів -- додати до вектору 3 дійсні числа з консолі.
Видалити три середніх елемента вектора за один виклик erase. 

\item 
Ввести список L з консолі та вектор V з бінарного файлу (тип -- рядок). Пе 
Переместити елемент списку L з даним номером в кінець списку V. 

\item 
Ввести список L з елементами $A_1,A_2, \ldots, A_{N-1}, A_N$ ($N$ -- парне,
якщо ні, то додайте нуль до списку.
Змінити порядок елементів у списку на наступний: $A_1, A_N,
A_{2}, A_{N-1}, A_3, A_{N-2}, \ldots, A_{N/2}, A_{N/2-1}$. 

\item 
Ввести два списки L1 і L2 з одинаковим числом елементів -- $N$.
Якщо це не так, то видалить з кінця більшого списку потрібну кількість
елементів. Отримати в списку L2 комбінований набір елементів елементів ---
списоку вигляду $B_1, A_1, B_2, A_2,\ldots, B_N, A_ N$, де $A_I$ --
елементи вихідного списку L1, а $B_I$ -- елементи списку L2.
\emph{Вказівка.} Використайте splice для L2 з інкрементами у другому та третьому аргументах. 

\end{enumerate}


\end{document}
