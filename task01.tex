\documentclass[]{article}
\usepackage{lmodern}
\usepackage{amssymb,amsmath}
\usepackage{ifxetex,ifluatex}


\usepackage[utf8]{inputenc}
\usepackage[english,russian,ukrainian]{babel}

\usepackage{fixltx2e} % provides \textsubscript
\ifnum 0\ifxetex 1\fi\ifluatex 1\fi=0 % if pdftex
  \usepackage[T1]{fontenc}
  \usepackage[utf8]{inputenc}
\else % if luatex or xelatex
  \ifxetex
    \usepackage{mathspec}
  \else
    \usepackage{fontspec}
  \fi
  \defaultfontfeatures{Ligatures=TeX,Scale=MatchLowercase}
\fi
% use upquote if available, for straight quotes in verbatim environments
\IfFileExists{upquote.sty}{\usepackage{upquote}}{}
% use microtype if available
\IfFileExists{microtype.sty}{%
\usepackage{microtype}
\UseMicrotypeSet[protrusion]{basicmath} % disable protrusion for tt fonts
}{}
\usepackage[unicode=true]{hyperref}
\hypersetup{
            pdfborder={0 0 0},
            breaklinks=true}
\urlstyle{same}  % don't use monospace font for urls
\usepackage{graphicx,grffile}
\makeatletter
\def\maxwidth{\ifdim\Gin@nat@width>\linewidth\linewidth\else\Gin@nat@width\fi}
\def\maxheight{\ifdim\Gin@nat@height>\textheight\textheight\else\Gin@nat@height\fi}
\makeatother
% Scale images if necessary, so that they will not overflow the page
% margins by default, and it is still possible to overwrite the defaults
% using explicit options in \includegraphics[width, height, ...]{}
\setkeys{Gin}{width=\maxwidth,height=\maxheight,keepaspectratio}
\IfFileExists{parskip.sty}{%
\usepackage{parskip}
}{% else
\setlength{\parindent}{0pt}
\setlength{\parskip}{6pt plus 2pt minus 1pt}
}
\setlength{\emergencystretch}{3em}  % prevent overfull lines
\providecommand{\tightlist}{%
  \setlength{\itemsep}{0pt}\setlength{\parskip}{0pt}}
\setcounter{secnumdepth}{0}
% Redefines (sub)paragraphs to behave more like sections
\ifx\paragraph\undefined\else
\let\oldparagraph\paragraph
\renewcommand{\paragraph}[1]{\oldparagraph{#1}\mbox{}}
\fi
\ifx\subparagraph\undefined\else
\let\oldsubparagraph\subparagraph
\renewcommand{\subparagraph}[1]{\oldsubparagraph{#1}\mbox{}}
\fi

\date{}


\usepackage{enumitem}
\makeatletter
\newcommand{\xslalph}[1]{\expandafter\@xslalph\csname c@#1\endcsname}
\newcommand{\@xslalph}[1]{%
    \ifcase#1\or а\or б\or в\or г\or д\or e\or є\or ж\or з\or i%
    \or й\or к\or л\or м\or н\or о\or п\or р\or с\or т%
    \or у\or ф\or х\or ц\or ч\or ш\or ю\or я\or аа\or бб\or вв %
    \else\@ctrerr\fi%
}
\AddEnumerateCounter{\xslalph}{\@xslalph}{m}
\makeatother


\begin{document}

\section*{ Методичні рекомендації з курсу «Мова програмування С++» }

Вступ

1. Лінійні програми на Сі. Введення/виведення. Дійсний тип даних.

2. Використання математичної бібліотеки С. Створення власних функцій

3. Цілі типи Сі. Умовні конструкції.

4. Цикли.

5. Цикли. Рекурентні співвідношення. Рекурсія

6. Бітові операції

7. Статичні масиви. Лінійні масиви та багатовимірні масиви

8. Динамічні масиви. Робота з вказівниками

9. Робота з рядком, що закінчується нулем на С.

10. Структури. Створення власного типу

11. Робота з бінарним файлами на Сі

12. Введення/виведення на С++. Робота з текстовими файлами

13. Робота з класом рядок на С++.

14. Створення власних класів. Інкапсуляція.

15. Робота з класами. Наслідування та поліморфізм.

16. Перетворення типів та робота з виключеннями.

17. Створення шаблонів функцій та шаблонів класів

18. Стандартна бібліотека С++. Послідовні контейнери.

19. Стандартна бібліотека С++. Асоціативні контейнери.

20. Стандартна бібліотека С++. Алгоритми та функтори.

\subsection{ ВСТУП }

Мета цього посібника, надати студенту завдання для того, щоб практично
оволодіти потрібними навичками програмування на мовах С та С++ в рамках
дисципліни «Мова програмування С++». Теми обиралися автором таким чином,
щоб найбільш швидким темпом здобути навичкі для практичного
програмування за 20 занять, тому деякі теми та розділи програмування на
С та С++, які автор вважає занадто складним або не обовязковими з точки
зору практики програмування, не входять до цього задачника, а винесені
на самостійну роботу або в якості завдань на курсові проекти.

Завдання посібника розділені на 20 лабораторних робіт, кожна з яких
присвячена окремій темі, що вивчається в дисципліні. Завдання та теми
підбиралися таким чином, щоб вивчення синтаксису мови виходило
поступовим тому послідовне виконання лабораторних робіт є найкращим для
засвоєння та набуття відповідних навичок. Тому наполегливо рекомендуємо
дотримуватися послідовного виконання лабораторних робіт.

Матеріал кожної лабораторної роботи посібника складається з п'яти
блоків: контрольних запитань, завдань для аудиторної роботи та трьох
блоків завдань для самостійної роботи. Під час підготовки до практичного
заняття, студент повинен опрацювати блок контрольних запитань та знати
вичерпні відповіді на них. Блок завдань для аудиторної містять перелік
типових задач відповідної теми. Ці завдання студент має виконати
протягом практичного заняття самостійно або під керівництвом викладача.
Завдання для самостійної роботи студент виконує самостійно та звітує про
їхнє виконання викладачу. Як було зазначено вище, завдання для
самостійної роботи складається з трьох блоків, перший з яких є
обов'язковим для виконання.

Другий блок завдань є ідентичним по складності основному блоку завдань
для самостійної роботи та призначений для кращого засвоєння матеріалу.

Третій блок завдань складається з задач підвищеної складності та вимагає
від студента не лише досконалого опанування методів поточної теми, а й
матеріалу, що виходить за межі нормативного курсу.


\newpage
\subsection{ Лінійні програми на Сі. Введення/виведення. Дійсний тип даних. }
\setcounter{subsection}{1}

Питання по темі 1:

\begin{itemize}
\item
Як запустити програму на Сі через консоль? На Сі++? Як створити
проект у вашому улюбленому середовищі?
\item
Як ініціалізувати дійсне та подвійне дійсні числа в Сі без попереджень 
компілятору? 

\item
Як вивести дійсне число на Сі? Як вивести його в десятковому вигляді?
З заданою точністю?

\item
Як ввести дійсне число на Сі? Як ввести його в
експоненційному вигляді? Які розміри дійсних чисел в байтах на Сі/Сі++
бувають?

\item
  Як ввести два дійсних числа через пробіли в одному рядку? А якщо
  роздільник --- 2 пробіли? А якщо кома?
\item
  Як ввести два дійсних числа в різних рядках?

\end{itemize}

Аудиторні завдання:

\begin{enumerate}
\def\labelenumi{\arabic{enumi}.}
\item
  Обчисліть наступні математичні вирази та виведіть результати:

2+31; 45*54-11; 15/4; 15.0/4; 67\%5; (2*45.1 +3.2)/2;

\item
  Ініціалізуйте наступні числа як дійсні, подвійні дійсні та довгі
  дійсні:$10^{-4}$, $24.33E5$, $\pi$, $e$, $\sqrt{5}$,
  $\ln(100)$ та виведить їх з 2 знаками після коми.

\item
  Вивести на екран текст:

а) 

-\/ a -\/ a -\/ a

a \textbar{} a \textbar{} a

-\/ a -\/ a -\/ a,

де a -- введена з клавіатури цифра.

\item
  Обчислити силу притягання $F$ в науковому (екоспоненційному) форматі між двома тілами,
  що мають маси $m_{1},m_{2}$ на відстані $r$. 
  \emph{\emph{Вказівка}}. Шукана силa визначається за формулою 
  $ F=\gamma \frac{m_{1}*m_{2}}{r^{2}}$,
  де $\gamma = 6.673*10^{-11}$ Н*м\textsuperscript{2}/кг\textsuperscript{2}. Всі потрібні змінні
  присвоюються всередині програми. Результат вивести в окремому рядку
  вигляду «F=*** », де замість зірок представлення в науковому
  (експоненційному) вигляді.

\item
  Дано дійсне число \(x\). Користуючись лише операцією множення,
  отримати:
  \begin{enumerate}[label=\xslalph*)]
  \item  \(x\textsuperscript{4}\) за дві операції; 
  \item  \(x\textsuperscript{6}\) за три операції;
  \item \(x\textsuperscript{9}\) за чотири операції; 
  \item \(x\textsuperscript{15}\) за п'ять операцій;
  \item \(x\textsuperscript{28}\) за шість операцій; 
  \item \(x\textsuperscript{64}\) за шість операцій.
  \end{enumerate}

\item
  Ввести дійсне число градусів Цельсія $C$ (на екрані повинна бути
  підказка, що ввести) та обчислити й вивести число $F$ в дійсному форматі
  -- та сама температура в градусах Фаренгейта за формулою $F = \frac{9C}{5} + 32 $.
 Результат вивести в окремому рядку вигляду «F=***», де замість зірок представлення числа в найкоротшому вигляді
  з можливих.

\item
  Ввести дійсне число x та підрахуйте без та за допомогою математичних
  функцій Сі її цілу та дробову частину, найменше ціле число, що більше
  x та найбільше ціле, що менше x, а також його округлене значення.
  Перевірте результат роботи для від'ємного числа.
\item
  Ввести в двох різних рядках послідовно два дійсних числа та обчислити
  значення їх різниці та добутку. Результат вивести в десятковому
  представленні (з фіксованою крапкою).
\item
  Ввести два дійсних числа записаних через пробіли в одному рядку та
  обчислити значення їх середнього арифметичного та середнього
  гармонічного. Результат вивести в науковому та десятковому
  представленні.
\end{enumerate}

Завдання для самостійної роботи

\begin{enumerate}
\def\labelenumi{\arabic{enumi}.}
\setcounter{enumi}{9}
\item
  Задайте в програмі довільні 5 цілих та 5 дійсних чисел. Вивести на екран таблицю
з цих значень у вигляді:\\

x | \  1 \  | \ 2 \  | \ 3 \ | \ 4 \  | \ 5 \ \\
- - |- - | - - | - -| - -| - - \\
y |  3.0  | 1.0 |5.0 | 4.0| 2.1\\


\item
  Зобразити на екрані сила з задачі 2) один під одним, так щоби десяткова крапка була
на одній вертикальній лінії.

\item
  Вивести на екран напис:

* * * * * * * * * * * * *\\
* * * * * * * * * Hello *\\
* * * * * * * * * * * * *\\ 
* * * * * * * * World!  *\\
* * * * * * * * * * * * *\\

\item
  Наближено визначити період обертання Землі навколо Сонця,
  використовуючи ланцюговий дріб

\[T = 365 + \frac{1}{4 + \frac{1}{7 + \frac{1}{1 + \frac{1}{3}}}}\]

Результат вивести в форматі плаваючої крапки.

\item
  Обчислити значення функції десяткового логарифму для даного числа та
  вивести результат з точністю до 3 знаків.
\item
  Тіло починає рухатися без початкової швидкості з прискоренням
  \(a\). Обчислити: відстань, яку воно пройде за час \(t\) від початку руху та
час, за який тіло досягне швидкості \(v\).

\item
  Обчислити кінетичну енергію тіла масою \(m\), що рухається зі
  швидкістю \(v\) відносно поверхні Землі.
\item
  Вивести на екран таблицю

\ x \ \textbar{} \  1 \ \textbar{} \  2 \ \textbar{} \ 3 \ \textbar{} \ 4 \textbar{} 5 

- - \textbar{} - - \textbar{} - - \textbar{} - - \textbar{} - - \textbar{} - - 

F(x)\textbar{} y \textbar{} y \textbar{} y \textbar{} y \textbar{} y

де замість символу y - значення у форматі з плаваючої крапкою з точністю
до двох знаків після крапки або ціле, вирівняне по центру, функцій:

а) F(x) = exp(-x*x); б) F(x) -- квадратний корінь з x.

\item
  Ввести дійсне число від 0 до 10000 та вивести його 8 ступінь з
  точністю до 20 знаків до десяткової коми та 4 значками після
  десяткової коми.

\item
Позиція у  грі «Хрестики-нулики» представлена в програмі за допомогою 9 символів виду ' ','O','X'. 
Відобразити на екрані позицію у грі «Хрестики-нулики». 
Наприклад, для позицій 'O','X',' ', ' ', 'X', 'O',' ' вона буде:\\
O | X | \hspace*{7pt}  \\   
\hspace*{7pt} | X | O \\
X | O | \hspace*{7pt}  \\ 



\end{enumerate}

Додаткові задачі:

\begin{enumerate}
\def\labelenumi{\arabic{enumi}.}
\setcounter{enumi}{19}
\item
  Три дійсні числа вводяться як рядок вигляду:

А=ххх.ххх, B=xxExxx C=xxx.xxxx (тут ``A='',''B='', ``C='' символи, що
повинні бути присутніми та ігноруються при введенні. Обчисліть їх
середнє арифметичне та середнє гармонічне та виведіть у науковому та
форматі з фіксованою крапкою.

\item
  Вивести на екран текст:

-- \textbar{} -- \textbar{} a \textbar{} -- \textbar{} --

-- \textbar{} a \textbar{} a \textbar{} а \textbar{} --

a \textbar{} a \textbar{} a \textbar{} a \textbar{} a

де a -- введене з клавіатури дійсне число менше 100 (прослідкуйте, щоб
воно а) мало не більше 5 значущих цифр, б) мало рівно 5 значущих цифр).

\item
Ввести користуючись лише однією функцією вводу ціле число записане в шістнадцятковому вигляді та вивести його зменшене на одиницю в  шістнадцятковому та десятковому вигляді.

\item
Дійсне число записано в рядку, при цьому перед ним може стояти будь-яка послідовність з пробілів та символів «*». Ввести його  користуючись лише одним викликом функції вводу та виведіть значення його кубу.

\item
Введіть два натуральних числа $n, m$ та виведіть числа $m$, $m^{2}$ в різних рядках
 на відстанях від лівого краю консолі рівних $n$ та $2n$ відповідно не користуючись циклами.

\end{enumerate}



\end{document}

