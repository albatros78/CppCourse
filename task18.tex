\documentclass[]{article}
\usepackage{lmodern}
\usepackage{amssymb,amsmath}
\usepackage{ifxetex,ifluatex}


\usepackage[utf8]{inputenc}
\usepackage[english,russian,ukrainian]{babel}

\usepackage{fixltx2e} % provides \textsubscript
\ifnum 0\ifxetex 1\fi\ifluatex 1\fi=0 % if pdftex
  \usepackage[T1]{fontenc}
  \usepackage[utf8]{inputenc}
\else % if luatex or xelatex
  \ifxetex
    \usepackage{mathspec}
  \else
    \usepackage{fontspec}
  \fi
  \defaultfontfeatures{Ligatures=TeX,Scale=MatchLowercase}
\fi
% use upquote if available, for straight quotes in verbatim environments
\IfFileExists{upquote.sty}{\usepackage{upquote}}{}
% use microtype if available
\IfFileExists{microtype.sty}{%
\usepackage{microtype}
\UseMicrotypeSet[protrusion]{basicmath} % disable protrusion for tt fonts
}{}
\usepackage[unicode=true]{hyperref}
\hypersetup{
            pdfborder={0 0 0},
            breaklinks=true}
\urlstyle{same}  % don't use monospace font for urls
\usepackage{graphicx,grffile}
\makeatletter
\def\maxwidth{\ifdim\Gin@nat@width>\linewidth\linewidth\else\Gin@nat@width\fi}
\def\maxheight{\ifdim\Gin@nat@height>\textheight\textheight\else\Gin@nat@height\fi}
\makeatother
% Scale images if necessary, so that they will not overflow the page
% margins by default, and it is still possible to overwrite the defaults
% using explicit options in \includegraphics[width, height, ...]{}
\setkeys{Gin}{width=\maxwidth,height=\maxheight,keepaspectratio}
\IfFileExists{parskip.sty}{%
\usepackage{parskip}
}{% else
\setlength{\parindent}{0pt}
\setlength{\parskip}{6pt plus 2pt minus 1pt}
}
\setlength{\emergencystretch}{3em}  % prevent overfull lines
\providecommand{\tightlist}{%
  \setlength{\itemsep}{0pt}\setlength{\parskip}{0pt}}
\setcounter{secnumdepth}{0}
% Redefines (sub)paragraphs to behave more like sections
\ifx\paragraph\undefined\else
\let\oldparagraph\paragraph
\renewcommand{\paragraph}[1]{\oldparagraph{#1}\mbox{}}
\fi
\ifx\subparagraph\undefined\else
\let\oldsubparagraph\subparagraph
\renewcommand{\subparagraph}[1]{\oldsubparagraph{#1}\mbox{}}
\fi

\date{}


\usepackage{enumitem}
\makeatletter
\newcommand{\xslalph}[1]{\expandafter\@xslalph\csname c@#1\endcsname}
\newcommand{\@xslalph}[1]{%
    \ifcase#1\or а\or б\or в\or г\or д\or e\or є\or ж\or з\or i%
    \or й\or к\or л\or м\or н\or о\or п\or р\or с\or т%
    \or у\or ф\or х\or ц\or ч\or ш\or ю\or я\or аа\or бб\or вв%
    \else\@ctrerr\fi%
}
\AddEnumerateCounter{\xslalph}{\@xslalph}{m}
\makeatother


\begin{document}


\newpage
\subsection{17. Створення шаблонів функцій та шаблонів класів}
\setcounter{subsection}{1}


\begin{itemize}
\item
Як створити функцію-шаблон? В яких ситуаціях вона корисна?
\item
Як створити клас-шаблон? Що потрібно зробити якщо шаблоном є лише
єдиний метод класу?
\item
  Навіщо потрібні простори імен та що таке стандартний простір імен? Як
  його підключити та що робити коли не можна його підключати на весь
  файл програми?
\item
  Як створити власний простір імен що містить власні математичні функції
  sin, cos, pow. Як їх коректно використати разом зі стандартними
  функціями?
\item
  Створіть вкладені простори імен та функції з однаковими
  ідентифікаторами в них та функцію з таким самим ідентифікатором
  глобально. Як правильно використати ці функції використовуючи ключове
  слово using?
\end{itemize}

Задачі для аудиторної роботи

\begin{enumerate}
\def\labelenumi{\arabic{enumi})}

\item
  Перепишіть функцію шаблон для пошуку максимуму двох чисел,
  так щоб вона працювала для всіх стандартних числових типів. 
Чи запрацює вона для рядків? 
Що потрібно зробити, щоб вона запрацювала і для типу 
Раціонального дробу з попередніх лекцій? 
(Вказівка: щось потрібно визначити для класу Раціональний дріб)

\item 
 Написати функцію, що вводить масив цілих чисел доки не введеться нуль 
та повертає результат через змінний аргумент та 
кількість елементів масиву повертається як
результат роботи функції. Для невідомої заздалегідь кількості елементів потрібно 
  робити реалізацію стеку. 
Створіть власну реалізацію класу шаблону Стек для будь-якого типу. Перевірте її роботу за
  допомогою стандартного класу Stack з STL для даної задачі та іншого типу.



\end{enumerate}

Задачі для самостійної роботи

\begin{enumerate}
\def\labelenumi{\arabic{enumi})}
\setcounter{enumi}{3}
\item
  Створити клас-шаблон BlackBox, який містить конструктор
  (порожній та від масиву (вказівника) будь-якого типу), метод push(),
  що дозволяє додати елемент певного типу, та метод pop(), що видає та
  видаляє випадковий елемент, що вже міститься в класі та виключення,
  якщо БлекБокс порожній, метод xpop(), що просто повертає випадковий
  елемент цього класу. Кількість елементів обмежена 100.
\item
  Створити клас-шаблон Mediana, який містить конструктор (порожній та
  від масиву (вказівника) будь-якого типу), що містить операції
  порівняння, метод push() який дозволяє додати елемент будь-якого типу,
  що містить операції порівняння, метод pop(int n), що видає та
  видаляє елемент за номером $n$ по порядку, або виключення якщо $n$ більше
  розміру всіх елементів та метод mediana(), що повертає медіану елементів
  цього класу. Кількість елементів обмежена 100.
\item
Визначити клас Масив, який містить розмір масиву та 
відповідний масив даних довільного типу. 

Реалізувати в ньому методи сортування як для самого масиву та як статичні методи (inplace):
\begin{enumerate}[label=\xslalph*)]
\item
обмінне сортування (метод бульбашки); 
\item
обмінне сортування «Шейкер-сортування»;
\item
сортування за допомогою вибору (метод простого вибору);
\item
сортування вставками;
\item
сортування методом хешування (сортування з обчисленням адреси);
\item
сортування вставками (метод простих вставок);
\item
сортування бінарним злиттям;
\item
сортування Шелла (сортування зі спадаючим кроком);
\item
швидке сортування;
\item
сортування купою.
\end{enumerate}

\item
Створить клас раціональне число на базі шаблону пари для довільних типів знаменника та чисельника.
Перевантажте методи для всіх арифметичних операцій та порівнянь 
(зокрема, остача від ділення -- це ділення після якого видаляється ціла частина). 
Зробіть наступну спеціалізацію, якщо знаменник або чисельник -- рядок:
створюється рядок вигляду ''{чисельник} /{знаменник}'' з виключеннями на всі арифметичні операції,
крім додавання (для нього -- це конкатинація), але коректною роботою з 
порівнянням/введенням/виведенням/доступом.

\item
Створіть клас рядок, що приймає у якості символу будь-який тип (зокрема інший рядок) 
та роздільник(того самого типу) - що відокремлює в запису ці символи. 
Методи класу:
\begin{itemize}
\item
перезавантажте методи введення/виведення в/з консолі та в/з текстового файлу;
\item
введення та заміна роздільника;
\item
метод конкатинації (з додаванням між рядками роздільника);
\item
довжина рядку;
\item
злиття символів -- тобто перетворення масиву символів на єдиний символ типу рядок;
\item
доступ до даного символу за квадратним дужками;
\item
видалення данного символу.
\end{itemize}
Забезпечити ініціювання помилки при неправильному введенні та роботі з рядками 
та роботі з файлами та спеціалізацію як звичайний рядок при символі типу char.

\item
Визначити клас Інтервал с урахуванням включення/невключення країв та нескінченості на інтервалах,
на базі шаблону пара. Якщо тип на одному з країв --- рядок, то вважається 
що це відповідна нескінченість.  
Створити методи по знаходженню перетину і об'єднанню інтервалів, 
причому інтервали, що немають спільних точок, перетинатися /об'єднуватися неможуть. 
Створить масив з $n$ інтервалів та знайдіть їх спільний перетин.
 
\item
Реалізуйте функцію sumAll(T*, size\_t n), яка рахує суму будь якого масиву, що передається їй як аргумент.
При цьому тип char сумується як символ, а тип вказівник вважається масивом розміру 1 та сумується утворюючи 
масив розміру n (нульові вказівники просто ігноруються в додаванні):
\begin{verbatim}
int v1[]  = { 1, 2, 3 }; // sumAll(v1,3) =6 
double v2[] = { 1, 2, 3 };//sumAll(v2,3) =6.0 
string v3[] = { "a", "bc", "def" };// sumAll(v3,3) ="abcdef"
char v4[]   = { 'a', 'b', 'c' }; // sumAll(v4,3) ="abcdef"
int* v5[] = { {1,4}, {2}, {3} }; // sumAll(v5,3) ={1,2,3}
\end{verbatim}

\item
Визначить клас Визначений інтеграл аналітичної підінтегральної функції. 
Клас дозволяє задавати інтервал де шукається корінь та створювати функцію 
від ступнів дійсних чисел та від функцій косінус, корінь та логарифм. 
Створити методи для обчислення значення за формулою лівих прямокутників, 
за формулою правих прямокутників, формулою середніх прямокутників, 
по формулі трапецій, по формулі Сімпсона (параболічних трапецій).

Створіть метод для семплювання функції (
обчислення дискретних значень в даних точках і побудова таблиці, 
що містить пари точкі-значення).

\end{enumerate}

\end{document}
