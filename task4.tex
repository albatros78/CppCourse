\documentclass[]{article}
\usepackage{lmodern}
\usepackage{amssymb,amsmath}
\usepackage{ifxetex,ifluatex}


\usepackage[utf8]{inputenc}
\usepackage[english,russian,ukrainian]{babel}

\usepackage{fixltx2e} % provides \textsubscript
\ifnum 0\ifxetex 1\fi\ifluatex 1\fi=0 % if pdftex
  \usepackage[T1]{fontenc}
  \usepackage[utf8]{inputenc}
\else % if luatex or xelatex
  \ifxetex
    \usepackage{mathspec}
  \else
    \usepackage{fontspec}
  \fi
  \defaultfontfeatures{Ligatures=TeX,Scale=MatchLowercase}
\fi
% use upquote if available, for straight quotes in verbatim environments
\IfFileExists{upquote.sty}{\usepackage{upquote}}{}
% use microtype if available
\IfFileExists{microtype.sty}{%
\usepackage{microtype}
\UseMicrotypeSet[protrusion]{basicmath} % disable protrusion for tt fonts
}{}
\usepackage[unicode=true]{hyperref}
\hypersetup{
            pdfborder={0 0 0},
            breaklinks=true}
\urlstyle{same}  % don't use monospace font for urls
\usepackage{graphicx,grffile}
\makeatletter
\def\maxwidth{\ifdim\Gin@nat@width>\linewidth\linewidth\else\Gin@nat@width\fi}
\def\maxheight{\ifdim\Gin@nat@height>\textheight\textheight\else\Gin@nat@height\fi}
\makeatother
% Scale images if necessary, so that they will not overflow the page
% margins by default, and it is still possible to overwrite the defaults
% using explicit options in \includegraphics[width, height, ...]{}
\setkeys{Gin}{width=\maxwidth,height=\maxheight,keepaspectratio}
\IfFileExists{parskip.sty}{%
\usepackage{parskip}
}{% else
\setlength{\parindent}{0pt}
\setlength{\parskip}{6pt plus 2pt minus 1pt}
}
\setlength{\emergencystretch}{3em}  % prevent overfull lines
\providecommand{\tightlist}{%
  \setlength{\itemsep}{0pt}\setlength{\parskip}{0pt}}
\setcounter{secnumdepth}{0}
% Redefines (sub)paragraphs to behave more like sections
\ifx\paragraph\undefined\else
\let\oldparagraph\paragraph
\renewcommand{\paragraph}[1]{\oldparagraph{#1}\mbox{}}
\fi
\ifx\subparagraph\undefined\else
\let\oldsubparagraph\subparagraph
\renewcommand{\subparagraph}[1]{\oldsubparagraph{#1}\mbox{}}
\fi

\date{}


\usepackage{enumitem}
\makeatletter
\newcommand{\xslalph}[1]{\expandafter\@xslalph\csname c@#1\endcsname}
\newcommand{\@xslalph}[1]{%
    \ifcase#1\or а\or б\or в\or г\or д\or e\or є\or ж\or з\or i%
    \or й\or к\or л\or м\or н\or о\or п\or р\or с\or т%
    \or у\or ф\or х\or ц\or ч\or ш\or ю\or я\or аа\or бб\or вв %
    \else\@ctrerr\fi%
}
\AddEnumerateCounter{\xslalph}{\@xslalph}{m}
\makeatother


\begin{document}

\section*{ Методичні рекомендації з курсу «Мова програмування С++» }

Вступ

1. Лінійні програми на Сі. Введення/виведення. Дійсний тип даних.

2. Використання математичної бібліотеки С. Створення власних функцій

3. Цілі типи Сі. Умовні конструкції.

4. Цикли.

5. Цикли. Рекурентні співвідношення. Рекурсія

6. Бітові операції

7. Статичні масиви. Лінійні масиви та багатовимірні масиви

8. Динамічні масиви. Робота з вказівниками

9. Робота з рядком, що закінчується нулем на С.

10. Структури. Створення власного типу

11. Робота з бінарним файлами на Сі

12. Введення/виведення на С++. Робота з текстовими файлами

13. Робота з класом рядок на С++.

14. Створення власних класів. Інкапсуляція.

15. Робота з класами. Наслідування та поліморфізм.

16. Перетворення типів та робота з виключеннями.

17. Створення шаблонів функцій та шаблонів класів

18. Стандартна бібліотека С++. Послідовні контейнери.

19. Стандартна бібліотека С++. Асоціативні контейнери.

20. Стандартна бібліотека С++. Алгоритми та функтори.

\subsection{ ВСТУП }

Мета цього посібника, надати студенту завдання для того, щоб практично
оволодіти потрібними навичками програмування на мовах С та С++ в рамках
дисципліни «Мова програмування С++». Теми обиралися автором таким чином,
щоб найбільш швидким темпом здобути навичкі для практичного
програмування за 20 занять, тому деякі теми та розділи програмування на
С та С++, які автор вважає занадто складним або не обовязковими з точки
зору практики програмування, не входять до цього задачника, а винесені
на самостійну роботу або в якості завдань на курсові проекти.

Завдання посібника розділені на 20 лабораторних робіт, кожна з яких
присвячена окремій темі, що вивчається в дисципліні. Завдання та теми
підбиралися таким чином, щоб вивчення синтаксису мови виходило
поступовим тому послідовне виконання лабораторних робіт є найкращим для
засвоєння та набуття відповідних навичок. Тому наполегливо рекомендуємо
дотримуватися послідовного виконання лабораторних робіт.

Матеріал кожної лабораторної роботи посібника складається з п'яти
блоків: контрольних запитань, завдань для аудиторної роботи та трьох
блоків завдань для самостійної роботи. Під час підготовки до практичного
заняття, студент повинен опрацювати блок контрольних запитань та знати
вичерпні відповіді на них. Блок завдань для аудиторної містять перелік
типових задач відповідної теми. Ці завдання студент має виконати
протягом практичного заняття самостійно або під керівництвом викладача.
Завдання для самостійної роботи студент виконує самостійно та звітує про
їхнє виконання викладачу. Як було зазначено вище, завдання для
самостійної роботи складається з трьох блоків, перший з яких є
обов'язковим для виконання.

Другий блок завдань є ідентичним по складності основному блоку завдань
для самостійної роботи та призначений для кращого засвоєння матеріалу.

Третій блок завдань складається з задач підвищеної складності та вимагає
від студента не лише досконалого опанування методів поточної теми, а й
матеріалу, що виходить за межі нормативного курсу.


\newpage
\subsection{ Цикли }
\setcounter{subsection}{1}

\begin{itemize}
\item
  Які типи циклів на Сі/Сі++? Напишіть цикл для введення n цілих чисел
  за допомогою трьох різних типів циклів.
\item
  Напишіть цикл для введення дійсних чисел доки не введемо 0. Обчисліть
  максимум з цих чисел.
\item
  Які інструкції та команди дозволяють закінчити (перервати цикл)?
\item
  Як можна уникнути виконання однієї (чи декількох) ітерацій циклу?
\item
  Як обчислити факторіал за допомогою арифметичного циклу на Сі?
\end{itemize}

Аудиторні задачі

\begin{enumerate}
\def\labelenumi{\arabic{enumi})}
\item
  Скласти функцію обчислення за даним дійсним x та натуральним n число
  \(y = \sin(\sin(\ldots\sin(x)\ldots))\) ($n$ разів).
\item
  Вивести на екран такий рядок:

n! = 1*2*3*4*5*...*n,

де n -- введене з клавіатури натуральне число, використовуючи
\begin{itemize}
\item цикл по діапазону із зростанням;
\item цикл по діапазону зі спаданням.
\end{itemize}

\item
  Скласти функції для обчислення значень многочленів і виконати їх при
  заданих значеннях аргументів:
\begin{enumerate}[label=\xslalph*)]
\item
\(y = x^{n} + x^{n - 1} + \ldots + x^{2} + x + 1, \ \  n = 3,x = 2\);
\item
\(y = x^{2^{n}}y^{n} + x^{2^{n - 1}}y^{n - 1} + \ldots + x^{2}y + 1, \ \ n = 4,x = 1,y = 2;\).
\end{enumerate}

\item
  Дано натуральне число \(n\). Написати програми обчислення
  значення виразу при заданому значенні \(x\):

$x + 2x^{2} + \ldots + (n - 1)x^{n - 1} + nx^{n}$.

\item
  Скласти функцію обчислення подвійного факторіала натурального числа
  \(n\): \(y = n!!\).

\emph{\emph{Вказівка}}. За означенням

\[n!! = \left\{ \begin{matrix}
1 \cdot 3 \cdot 5 \cdot \ldots \cdot n,\textup{при n непарному}, \\
2 \cdot 4 \cdot 6 \cdot \ldots \cdot n,\textup{при n парному} \\
\end{matrix} \right.\ \]

\item
  Скласти програму обчислення
\begin{enumerate}[label=\xslalph*)]
\item
\(\sqrt{2 + \sqrt{2 + \ldots + \sqrt{2}}}\) (n коренів),

\item
 \(\sqrt{3 + \sqrt{6 + \ldots + \sqrt{3(n - 1) + \sqrt{3n}}}}.\)

\end{enumerate}

\item
  Скласти програми обчислення значень многочлену для 
  \(x \in \bf{R}\), що по модулю менше за одиницю та
  \( n \geq 0\):

\(y = 1 + \frac{x}{1!} + \frac{x^{2}}{2!} + \frac{x^{3}}{3!} + \ldots + \frac{x^{n}}{n!} \).

\item
  Для довільного цілого числа \(m \geq 1\)знайти найбільше ціле \(k\),
  при якому \(4^{k} \leq m\).
\item
  Для заданого натурального числа \(n\)одержати найменше число вигляду
  \(2^{r}\), яке перевищує \(n\) .
\item
  Знайдіть машинний нуль для вашого компілятора, тобто таке дійсне число
  \(a > 0,\) що \(1 + a = 1\ \) буде істиною.

\emph{Вказівка:} в циклі ділить значення \(a\)на 2 доки не виконується
вказана вище рівність.

\item
  Ввести послідовність наступним чином: користувачу виводиться напис
  ``a{[}**{]}= '', де замість ** стоїть номер числа, що вводиться. Тобто
  там виводяться написи ``a{[}0{]}= '', і після знаку рівності
  користувач вводить число, ``a{[}1{]}= '', і після знаку рівності
  користувач вводить число і так далі доки користувач не введе число 0.
  Після цього потрібно вивести суму введених чисел (масив чисел заводити
  необов'язково). Введіть послідовність цілих ненульових чисел (тобто введення
  закінчується коли ми вводимо 0) та виведіть середнє арифметичне
  введених чисел та середнє геометричне.

\end{enumerate}

Для самостійної роботи

\begin{enumerate}
\def\labelenumi{\arabic{enumi})}
\setcounter{enumi}{11}
\item
  Скласти функції для обчислення значень многочленів і виконати їх при
  заданих значеннях аргументів:
\begin{enumerate}[label=\xslalph*)]
\item \(y = x^{2^{n}} + x^{2^{n - 1}} + \ldots + x^{4} + x^{2} + 1;\)

\item \(y = x^{3^{n}} + x^{3^{n - 1}} + \ldots + x^{9} + x^{3} + 1;\)

\item \(y = x^{1^{2}} + x^{2^{2}} + \ldots + x^{n^{2}}.\)

\end{enumerate}

\item
  Дано натуральне число \(n\). Написати програми обчислення
  значень виразів при заданому значенні \(x\):

\begin{enumerate}[label=\xslalph*)]
\item

\(1 + (x - 1) + (x - 1)^{2} + \ldots + (x - 1)^{n}\);
\item
\(1 + \frac{1}{x^{2} + 1} + \frac{1}{(x^{2} + 1)^{2}} + \ldots + \frac{1}{(x^{2} + 1)^{n}}\);
\item
\(1 + \sin{x} + \sin^2{x} + \ldots + \sin^{n}{x}\);
\item
\(y = nx^{n - 1} + (n - 1)x^{n - 2} + \ldots + 2x + 1\);
\item
\(y = \sum_{k = 0}^{n}{kx^{k}(1 - x)^{n - k}}\), за умови \(0 < x < 1,n \geq 0)\).

\end{enumerate}

\item
  Введіть послідовність цілих ненульових чисел (тобто введення
  закінчується коли ми вводимо 0). Визначити кількість змін знаку в цій
  послідовності. Наприклад, у послідовності 1,-34, 8, 14, -5, 0 знак
  змінюється три рази.

\item
  Введіть послідовність натуральних ненульових чисел (тобто введення
  закінчується коли ми вводимо 0). Визначити порядковий номер найменшого
  з них.
\item
  Введіть послідовність дійсних ненульових чисел (тобто введення
  закінчується коли ми вводимо 0). Визначити величину найбільшого серед
  від`ємних членів цієї послідовності. Якщо від'ємних чисел немає
  вивести найменший серед додатних членів.
\item
  Банк пропонує річну ставку по депозиту A та 15\% по вкладу додаються
  до основної суми депозиту кожен рік. Ви кладете в цей банк D гривень.
  Скільки років потрібно чекати, щоб сума вкладу зросла до очікуваної
  суми P?
\item
  Скласти програми для обчислення елементів послідовностей. Операцію
  піднесення до степені та функцію обчислення факторіалу не
  використовувати.
\begin{enumerate}[label=\xslalph*)]
\item
\(x_{k} = \frac{x^{k}}{k}\ (k \geq 1)\) 
\item
\(x_{k} = \frac{x^{2k}}{(2k)!}\ (k \geq 0)\);
\item \(x_{k} = \frac{( - 1)^{k}x^{k}}{k}\ (k \geq 1)\); 
\item
\(x_{k} = \frac{x^{2k + 1}}{(2k + 1)!}\ (k \geq 0)\);
\item \(x_{k} = \frac{x^{k}}{k!}\ (k \geq 0)\) ;
\item \(x_{k} = \frac{(-1)^{k}x^{2k}}{(2k)!}\ (k \geq 0)\);
\item \(x_{k} = \frac{(-1)^{k}x^{k}}{k!}\ (k \geq 0)\); 
\item \(x_{k} = \frac{(-1)^{k}x^{2k + 1}}{(2k + 1)!}\ (k \geq  0)\).
 \end{enumerate}

\item
  Задане натуральне число \(n\). Скласти програми обчислення добутків:

а)
\(p = \left( 1 + \frac{1}{1^{2}} \right)\left( 1 + \frac{1}{2^{2}} \right)\ldots\left( 1 + \frac{1}{n^{2}} \right),\mathrm{\ \ \ \ n > 2};\)

б)
\(p = \left( 1 - \frac{1}{2^{2}} \right)\left( 1 - \frac{1}{3^{2}} \right)\ldots\left( 1 + \frac{1}{n^{2}} \right),\mathrm{\ \ \ \ n > 2.}\)

\item
  Скласти програму друку таблиці значень функції \(y = \sin x\) на
  відрізку {[}0,1{]} з кроком \(h = 0.1\).
\item
  Скласти програму визначення кількості тризначних натуральних чисел,
  сума цифр яких дорівнює \(n\) \((n > 1)\). Операцію ділення не
  використовувати.
\item
  Дано \(n\) цілих чисел. Скласти програму, що визначає, скільки з
  них більші за своїх "сусідів", тобто попереднього та наступного чисел.
\item
  Задані натуральне число \emph{n}, дійсні числа
  \(y_{1},\ldots y_{n}.\)Скласти програму визначення

\begin{enumerate}[label=\xslalph*)]

\item \(\max(\left| z_{1} \right|,\ldots,\left| z_{n} \right|),\) де
\(z_{i} = \left\{ \begin{matrix}
y_{i},\textup{ при }\left| y_{i} \right| \leq 2, \\
0.5,\textup{у інших випадках} \\
\end{matrix} \right.\ \);
\item \(\min(\left| z_{1} \right|,\ldots,\left| z_{n} \right|),\) де
\(z_{i} = \left\{ \begin{matrix}
 y_{i},\textup{при}\left| y_{i} \right| \geq 1, \\
 2,\textup{у інших випадках} \\
\end{matrix} \right.\ \);
\item \(z_{1} + z_{2} + \ldots + z_{n},\) де
\(z_{i} = \left\{ \begin{matrix}
 y_{i},\textup{при} {y}_{i} < 10, \\
 1,\textup{у інших випадках} \\
\end{matrix} \right.\ \)
 \end{enumerate}

\item
  Дано натуральне число n. Викинути із запису числа n цифри 0 і 5,
  залишивши порядок інших цифр. Наприклад, з числа 59015509 повинно
  вийти 919.
\item
  Знайти період десяткового дробу для відношення n/m для заданих
  натуральних чисел n та m.
\item
  Скоротити дріб n/m для заданих цілого числа n та натурального числа m.
\end{enumerate}

Підвищеної складності

\begin{enumerate}
\def\labelenumi{\arabic{enumi})}
\setcounter{enumi}{26}
\item
  Ввести натуральні числа $a$ і $b$ та натуральне число $n$. Чи можна
  представити число $n$ у вигляді $n= k*a + m*b$, де $k$ та $m$ -- натуральні
  числа? Якщо можна -- то знайдіть такі числа $k$ та $m$, що мають найменшу
  суму модулів.
\item
  Представити дане натуральне число як суму двох квадратів натуральних
  чисел. Якщо це неможливо представити як суму трьох квадратів. Якщо і
  це неможливо, представити у вигляді суми чотирьох квадратів
  натуральних чисел.
\item
  Знайти всі цілі корені кубічного рівняння $ax^3 + bx^2 + cx + d$ ($a,b,c,d$ 
-- задані цілі числа). \emph{Вказівка}: цілі корені повинні бути від'ємними
 або додатними дільниками вільного члену $d$.
\item
  Напишіть функцію, яка розраховує для даного натурального числа n
  значення функції Ойлера --- тобто кількість чисел від 1 до $n$, взаємно простих з
  n.
\item
  Ввести натуральне число \(d > 1\) та натуральне число $m$. Знайдіть
  мінімальну кількість натуральних чисел вигляду \(x^{d}\ \)(\(d\)-ступенів
  натуральних чисел) сума яких дорівнює $m$.
\end{enumerate}

\end{document}

