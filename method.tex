\documentclass[]{article}
\usepackage{lmodern}
\usepackage{amssymb,amsmath}
\usepackage{ifxetex,ifluatex}
\usepackage{fixltx2e} % provides \textsubscript
\ifnum 0\ifxetex 1\fi\ifluatex 1\fi=0 % if pdftex
  \usepackage[T1]{fontenc}
  \usepackage[utf8]{inputenc}
\else % if luatex or xelatex
  \ifxetex
    \usepackage{mathspec}
  \else
    \usepackage{fontspec}
  \fi
  \defaultfontfeatures{Ligatures=TeX,Scale=MatchLowercase}
\fi
% use upquote if available, for straight quotes in verbatim environments
\IfFileExists{upquote.sty}{\usepackage{upquote}}{}
% use microtype if available
\IfFileExists{microtype.sty}{%
\usepackage{microtype}
\UseMicrotypeSet[protrusion]{basicmath} % disable protrusion for tt fonts
}{}
\usepackage[unicode=true]{hyperref}
\hypersetup{
            pdfborder={0 0 0},
            breaklinks=true}
\urlstyle{same}  % don't use monospace font for urls
\usepackage{graphicx,grffile}
\makeatletter
\def\maxwidth{\ifdim\Gin@nat@width>\linewidth\linewidth\else\Gin@nat@width\fi}
\def\maxheight{\ifdim\Gin@nat@height>\textheight\textheight\else\Gin@nat@height\fi}
\makeatother
% Scale images if necessary, so that they will not overflow the page
% margins by default, and it is still possible to overwrite the defaults
% using explicit options in \includegraphics[width, height, ...]{}
\setkeys{Gin}{width=\maxwidth,height=\maxheight,keepaspectratio}
\IfFileExists{parskip.sty}{%
\usepackage{parskip}
}{% else
\setlength{\parindent}{0pt}
\setlength{\parskip}{6pt plus 2pt minus 1pt}
}
\setlength{\emergencystretch}{3em}  % prevent overfull lines
\providecommand{\tightlist}{%
  \setlength{\itemsep}{0pt}\setlength{\parskip}{0pt}}
\setcounter{secnumdepth}{0}
% Redefines (sub)paragraphs to behave more like sections
\ifx\paragraph\undefined\else
\let\oldparagraph\paragraph
\renewcommand{\paragraph}[1]{\oldparagraph{#1}\mbox{}}
\fi
\ifx\subparagraph\undefined\else
\let\oldsubparagraph\subparagraph
\renewcommand{\subparagraph}[1]{\oldsubparagraph{#1}\mbox{}}
\fi

\date{}

\begin{document}

Методичні рекомендації з курсу «Мова програмування С++»

Вступ

1. Лінійні програми на Сі. Введення/виведення. Дійсний тип даних.

2. Використання математичної бібліотеки С. Створення власних функцій

3. Цілі типи Сі. Умовні конструкції.

4. Цикли.

5. Цикли. Рекурентні співвідношення. Рекурсія

6. Бітові операції

7. Статичні масиви. Лінійні масиви та багатовимірні масиви

8. Динамічні масиви. Робота з вказівниками

9. Робота з рядком, що закінчується нулем на С.

10. Структури. Створення власного типу

11. Робота з бінарним файлами на Сі

12. Введення/виведення на С++. Робота з текстовими файлами

13. Робота з класом рядок на С++.

14. Створення власних класів. Інкапсуляція.

15. Робота з класами. Наслідування та поліморфізм.

16. Перетворення типів та робота з виключеннями.

17. Створення шаблонів функцій та шаблонів класів

18. Стандартна бібліотека С++. Послідовні контейнери.

19. Стандартна бібліотека С++. Асоціативні контейнери.

20. Стандартна бібліотека С++. Алгоритми та функтори.

ВСТУП

Мета цього посібника, надати студенту завдання для того, щоб практично
оволодіти потрібними навичками програмування на мовах С та С++ в рамках
дисципліни «Мова програмування С++». Теми обиралися автором таким чином,
щоб найбільш швидким темпом здобути навичкі для практичного
програмування за 20 занять, тому деякі теми та розділи програмування на
С та С++, які автор вважає занадто складним або не обовязковими з точки
зору практики програмування, не входять до цього задачника, а винесені
на самостійну роботу або в якості завдань на курсові проекти.

Завдання посібника розділені на 20 лабораторних робіт, кожна з яких
присвячена окремій темі, що вивчається в дисципліні. Завдання та теми
підбиралися таким чином, щоб вивчення синтаксису мови виходило
поступовим тому послідовне виконання лабораторних робіт є найкращим для
засвоєння та набуття відповідних навичок. Тому наполегливо рекомендуємо
дотримуватися послідовного виконання лабораторних робіт.

Матеріал кожної лабораторної роботи посібника складається з п'яти
блоків: контрольних запитань, завдань для аудиторної роботи та трьох
блоків завдань для самостійної роботи. Під час підготовки до практичного
заняття, студент повинен опрацювати блок контрольних запитань та знати
вичерпні відповіді на них. Блок завдань для аудиторної містять перелік
типових задач відповідної теми. Ці завдання студент має виконати
протягом практичного заняття самостійно або під керівництвом викладача.
Завдання для самостійної роботи студент виконує самостійно та звітує про
їхнє виконання викладачу. Як було зазначено вище, завдання для
самостійної роботи складається з трьох блоків, перший з яких є
обов'язковим для виконання.

Другий блок завдань є ідентичним по складності основному блоку завдань
для самостійної роботи та призначений для кращого засвоєння матеріалу.

Третій блок завдань складається з задач підвищеної складності та вимагає
від студента не лише досконалого опанування методів поточної теми, а й
матеріалу, що виходить за межі нормативного курсу.

1. Лінійні програми на Сі. Введення/виведення. Дійсний тип даних.

Питання по темі 1:

1) Як запустити програму на Сі через консоль? На Сі++? Як створити
проект у вашому улюбленому середовищі?

2) Як ініціалізувати дійсне та подвійне дійсні числа в Сі без
попереджень компілятора? Як ввести дійсне число на Сі? Як ввести його в
експоненційному вигляді? Які розміри дійсних чисел в байтах на Сі/Сі++
бувають?

3) Як вивести дійсне число на Сі? Як вивести його в десятковому вигляді?
З заданою точністю?

Аудиторні завдання:

\begin{enumerate}
\def\labelenumi{\arabic{enumi}.}
\item
  Обчисліть наступні математичні вирази та виведіть результати:
\end{enumerate}

2+31; 45*54-11; 15/4; 15.0/4; 67\%5; (2*45.1 +3.2)/2;

\begin{enumerate}
\def\labelenumi{\arabic{enumi}.}
\item
  Ініціалізуйте наступні числа як дійсні, подвійні дійсні та довгі
  дійсні:\(10^{- 4}\), 2.33E5, \(\pi\), \(e\), \(\sqrt{5}\),
  \(\ln(100)\)
\item
  Вивести на екран текст:
\end{enumerate}

а) б)

a a a a-\/-\/-\/-\/-\/-\/-\/-\/-\/-a

a a \textbar{} a \textbar{}

a a a a-\/-\/-\/-\/-\/-\/-\/-\/-\/-a

де a -- введена з клавіатури цифра.

\begin{enumerate}
\def\labelenumi{\arabic{enumi}.}
\item
  Обчислити силу притягання \(F\)(в науковому форматі) між двома тілами,
  що мають маси \emph{m\textsubscript{1},m\textsubscript{2}} \emph{,} на
  відстані \emph{r}. \emph{\emph{Вказівка}}. Шукана силa визначається за
  формулою
  \emph{F=γ*m\textsubscript{1}*m\textsubscript{2}/r\textsuperscript{2},}
  де \emph{γ = 6.673*10\textsuperscript{-11}
  Н*м\textsuperscript{2}/кг\textsuperscript{2}. Всі потрібні змінні
  присвоюються всередині програми.} Результат вивести в окремому рядку
  вигляду «F=*** », де замість зірок представлення в науковому
  (експоненційному) вигляді.
\item
  Дано дійсне число \emph{x}. Користуючись лише операцією множення,
  отримати:
\end{enumerate}

а) \emph{x\textsuperscript{4}} за дві операції; б)
\emph{x\textsuperscript{6}} за три операції;

в) \emph{x\textsuperscript{9}} за чотири операції; г)
\emph{x\textsuperscript{15}} за п'ять операцій;

ґ) \emph{x\textsuperscript{28}} за шість операцій; д)
\emph{x\textsuperscript{64}} за шість операцій.

\begin{enumerate}
\def\labelenumi{\arabic{enumi}.}
\item
  Ввести дійсне число градусів Цельсія C (на екрані повинна бути
  підказка, що ввести) та обчислити й вивести число F в дійсному форматі
  -- та сама температура в градусах Фаренгейта за формулою
  \(F = \frac{9C}{5} + 32\). Результат вивести в окремому рядку вигляду
  «F=*** », де замість зірок представлення числа в найкоротшому вигляді
  з можливих.
\item
  Ввести дійсне число x та підрахуйте без та за допомогою математичних
  функцій Сі її цілу та дробову частину, найменше ціле число, що більше
  x та найбільше ціле, що менше x, а також його округлене значення.
  Перевірте результат роботи для від'ємного числа.
\item
  Ввести в двох різних рядках послідовно два дійсних числа та обчислити
  значення їх різниці та добутку. Результат вивести в десятковому
  представленні (з фіксованою крапкою).
\item
  Ввести два дійсних числа записаних через пробіли в одному рядку та
  обчислити значення їх середнього арифметичного та середнього
  гармонічного. Результат вивести в науковому та десятковому
  представленні.
\end{enumerate}

Завдання для самостійної роботи

\begin{enumerate}
\def\labelenumi{\arabic{enumi}.}
\item
  Вивести на екран таблицю
\end{enumerate}

x \textbar{} 1 \textbar{} 2 \textbar{} 3 \textbar{} 4 \textbar{} 5

-\/-\/-\/-\/-\/-\/-\/-\/-\/-\/-\/-\/-\/-\/-\/-\/-\/-

у \textbar{} 3 \textbar{} 1 \textbar{} 5 \textbar{} 4 \textbar{} 2

\begin{enumerate}
\def\labelenumi{\arabic{enumi}.}
\item
  Зобразити на екрані декартову систему координат у вигляді

  \begin{enumerate}
  \def\labelenumii{\roman{enumii}.}
  \item
    \^{} y
  \item
    \textbar{} x
  \end{enumerate}

  \begin{enumerate}
  \def\labelenumii{\alph{enumii}.}
  \item
    -\/-\/-\/-\/-\/-\/-\/-\/-\/-\/-\/-\/-\/-\/-\/-\/-\/-\/-\textgreater{}

    \begin{enumerate}
    \def\labelenumiii{\roman{enumiii}.}
    \item
      \textbar{} 1
    \item
      \textbar{}
    \end{enumerate}
  \end{enumerate}
\item
  Вивести на екран рисунки:
\end{enumerate}

а) б) в)

* * * * * * * * * * * * *

* * * * * * * * *

* * * * * * * * * Hello *

* * * * * * * * *

* * * * * * * * * * * * *

\begin{enumerate}
\def\labelenumi{\arabic{enumi}.}
\item
  Наближено визначити період обертання Землі навколо Сонця,
  використовуючи ланцюговий дріб
\end{enumerate}

\[T = \mathrm{365} + \frac{1}{4 + \frac{1}{7 + \frac{1}{1 + \frac{1}{3}}}}\]

Результат вивести в форматі плаваючої крапки.

\begin{enumerate}
\def\labelenumi{\arabic{enumi}.}
\item
  Обчислити значення функції десяткового логарифму для даного числа --
  вивести результат з точністю до 3 знаків.
\item
  Тіло починає рухатися без початкової швидкості з прискоренням
  \emph{a}. Обчислити:
\end{enumerate}

а) відстань, яку воно пройде за час \emph{t} від початку руху;

б) час, за який тіло досягне швидкості \emph{v}.

\begin{enumerate}
\def\labelenumi{\arabic{enumi}.}
\item
  Обчислити кінетичну енергію тіла масою \emph{m}, що рухається зі
  швидкістю \emph{v} відносно поверхні Землі.
\item
  Вивести на екран таблицю
\end{enumerate}

x \textbar{} 1 \textbar{} 2 \textbar{} 3 \textbar{} 4 \textbar{} 5

-\/-\/- +-\/-\/-+-\/-\/-+-\/-\/-+-\/-\/-+-\/-\/-

F(x)\textbar{} y \textbar{} y \textbar{} y \textbar{} y \textbar{} y

де замість символу y - значення у форматі з плаваючої крапкою з точністю
до двох знаків після крапки або ціле, вирівняне по центру функцій:

а) F(x) = exp(-x*x); б) F(x) -- квадратний корінь з x

\begin{enumerate}
\def\labelenumi{\arabic{enumi}.}
\item
  Ввести дійсне число від 0 до 10000 та вивести його 8 ступінь з
  точністю до 20 знаків до десяткової коми та 4 значками після
  десяткової коми.
\end{enumerate}

Додаткові задачі:

\begin{enumerate}
\def\labelenumi{\arabic{enumi}.}
\item
  Три дійсні числа вводяться як рядок вигляду:
\end{enumerate}

А=ххх.ххх, B=xxExxx C=xxx.xxxx (тут ``A='',''B='', ``C='' символи, що
повинні бути присутніми та ігноруються при введенні). Обчисліть їх
середнє арифметичне та середнє гармонічне та виведіть у науковому та
форматі з фіксованою крапкою.

\begin{enumerate}
\def\labelenumi{\arabic{enumi}.}
\item
  Вивести на екран текст:
\end{enumerate}

а) б)

a a a a-\/-\/-\/-\/-\/-\/-\/-\/-\/-a

a a а \textbar{} a \textbar{}

a a a a-\/-\/-\/-\/-\/-\/-\/-\/-\/-a

де a -- введене з клавіатури дійсне число менше 100 (прослідкуйте, щоб
воно а) мало не більше 5 значущих цифр, б) мало рівно 5 значущих цифр).

2. Використання математичної бібліотеки С. Створення власних функцій

Питання по темі 2:

\begin{itemize}
\item
  Як підключити математичні функції та скомпілювати програму, що
  використовує sin та arctan?
\item
  Як ввести два дійсних числа через пробіли в одному рядку? А якщо
  роздільник --- 2 пробіли? А якщо кома?
\end{itemize}

\begin{itemize}
\item
  Як ввести два дійсних числа в різних рядках?
\item
  Як узнати скільки максимальна кількість значущих цифр в даному
  дійсному типі? Максимальну експоненту та мантису?
\item
  Як записати власну функцію на Сі? Як запустити її зі сталими
  аргументами та як з аргументами, що є змінними в програмі?
\end{itemize}

Аудиторні завдання:

\begin{enumerate}
\def\labelenumi{\arabic{enumi})}
\item
  Ввести дійсне число х та обчислити значення функції тригонометричного
  косинуса для нього.
\item
  Обчислити гіпотенузу \emph{c} прямокутного трикутника за катетами
  \emph{a} та \emph{b}.
\item
  Обчислити площу трикутника \emph{S} за трьома сторонами \emph{a},
  \emph{b}, \emph{c}.
\item
  Обчислити відстань від точки \((x_{0},y_{0})\)до:
\end{enumerate}

а) заданої точки \((x,y);\)

б) заданої прямої \(\mathrm{\text{ax}} + \mathrm{\text{by}} + c = 0\);

в) точки перетину прямих \(x + \mathrm{\text{by}} + c = 0\) і
\(\mathrm{\text{ax}} + y + c = 0,\ \)де
\(\mathrm{\text{ab}} \neq 1\mathrm{.}\)

\begin{enumerate}
\def\labelenumi{\arabic{enumi})}
\item
  Напишіть функцію, яка за найменшу кількість арифметичних операцій,
  обчислює значення многочлена для введеного з клавіатури значення
  \emph{x}:
\end{enumerate}

а) \(y = x^{4} + 2x^{2} + 1;\) б)\(y = x^{4} + x^{3} + x^{2} + x + 1;\)

в)
\(y = x^{5} + 5x^{4} + \mathrm{10}x^{3} + \mathrm{10}x^{2} + 5x + 1;\)
г) \(y = x^{9} + x^{3} + 1;\)

ґ) \(y = \mathrm{16}x^{4} + 8x^{3} + 4x^{2} + 2x + 1;\) д)
\(y = x^{5} + x^{3} + x\mathrm{.}\)

\begin{enumerate}
\def\labelenumi{\arabic{enumi})}
\item
  Напишіть функцію Rosenbrock2d(x,y) =
  \(100(x^{2} - y)^{2} + (x - 1)^{2}\)та перевірте її результат на
  довільних трьох парах дійсних чисел.
\item
  Трикутник вводиться координатами своїх вершин, які вводяться так: в
  першому рядку через пробіл два дійсних числа --- координати точки А,
  пропускається рядок, в третьому рядку через пробіл два дійсних числа
  --- координати Б, пропускається рядок, через пробіл --- координати
  точки С. Підрахувати площу трикутника. (Вказівка: напишіть функції
  підрахунку довжини відрізка та функції обчислення площі трикутника за
  довжинами сторін)
\end{enumerate}

Завдання для самостійної роботи

\begin{enumerate}
\def\labelenumi{\arabic{enumi})}
\item
  Обчислити площу трикутника \emph{S} за трьома сторонами \emph{a},
  \emph{b}, \emph{c}.
\item
  Обчислити площу еліпса за координатами його радіусів.
\item
  В трикутнику відомо довжини всіх сторін. Обчислити довжини його:
\item
  а) медіан,
\item
  б) бісектрис
\item
  в) висот.
\item
  Трикутник заданий величинами своїх кутів та радіусом вписаного кола.
  Обчисліть його площу.
\item
  Трикутник заданий довжиною своїх сторін. Знайти та вивести величину
  кутів трикутника у радіанах та градусах.
\item
  Знайти об'єм циліндра, якщо відомо його радіус основи та висоту.
\item
  Знайти об'єм конуса, якщо відомо його радіус основи та висоту.
\item
  Знайти об'єм тора з внутрішнім радіусом \emph{r} і зовнішнім радіусом
  \emph{R.}
\item
  Знайти корені квадратного рівняння з коефіцієнтами a,b,c, якщо відомо,
  що обидва корені в ньому існують. Перевірте ваш розв'язок на
  коефіцієнтах рівняння a=3,b=100,c=2.
\item
  Скласти функцію для обчислення значення многочлена від двох змінних
  для введеної з клавіатури пари чисел \((x,y)\):
\item
  а) \(f(x,y) = x^{3} + 3x^{2}y + 3\mathrm{\text{xy}}^{2} + y^{3};\)
\item
  б) \(f(x,y) = x^{2}y^{2} + x^{3}y^{3} + x^{4}y^{4};\)
\item
  в)
  \(f(x,y) = x + y + x^{2} + y^{2} + x^{3} + y^{3} + x^{4} + y^{4}\mathrm{.}\)
\item
  Напишіть власні функції, що обчислюють наступні вирази та відповідні
  власні функції, що будуть рахувати похідні даних функцій(Приклад,
  функція \(f(x) = identity(x) = x\)\emph{,} її похідна
  \(g(x) = \text{identity}_{\text{derivative}}(x) = 1\)) :
\end{enumerate}

\emph{а)} \(f(x) = th(x) = \frac{(e^{x} - e^{- x})}{(e^{x} + e^{- x})}\)

б) \(f(x) = Bent(x) = \frac{\sqrt{x^{2} + 1} - 1}{2} + x\)

в) \(f(x) = Softsign(x) = \frac{x}{1 + |x|}\)

г) \(f(x) = arctg(x) = tg^{- 1}(x)\)

д)\(f(x) = gauss(x) = e^{- x^{2}}\)

е) \(f(x) = SoftPlus(x) = \ln(1 + e^{x})\)

ж) \(f(x) = sigmoid(x) = {(1 + e^{- x})}^{- 1}\)

з) \(f(x) = invsqrt(x,\alpha) = \frac{x}{\sqrt{1 + \alpha x^{2}}}\)

і)\(f(x) = sigmweight(x) = x*{(1 + e^{- x})}^{- 1}\)

3. Цілі типи Сі. Умовні конструкції.

Питання по темі 3:

1) Які типи цілих чисел використовуються в Сі/С++?

2) Які варіанти використання булевого типу є в Сі?

3) Як перевести число із знакового до беззнакового типу? Як навпаки?

4) Як ввести найдовше можливе ціле число? Як узнати його розмір в
байтах?

5) Як виконати цілочисельне ділення в Сі? Як поділити не цілочисельно
два цілих числа?

6) Як коректно та без поперджень компілятора ініціалізувати довге
натуральне число? Натуральне коротке? Ціле довге?

7) Як коректно та без попереджень ввести та вивести натуральне число?
Натуральне коротке? Ціле довге?

8) Як записати умовне розгалудження в Сі/Сі++?

9) Які типи умовних виразів на Сі/Сі++? Напишіть два варіанти з ними для
пошуку мінимума двох чисел. Напишіть за допомогою виразу альтернативи
функцію, що повертає парність цілого числа.

Завдання для аудиторної роботи

1)Дано натуральне тризначне число. Знайти:

а) кількість одиниць, десятків і сотень цього числа;

б) суму цифр цього числа;

в) число, утворене при читанні заданого числа справа наліво.

2)Ввести натуральне тризначне число. Якщо в ньому всі 3 цифри різні, то
вивести всі числа, які утворюються при перестановці цифр заданого числа.

3) Введіть три цілих числа, записаних через кому в одному рядку та
підрахуйте їх добуток якщо всі ці числа гарантовано по модулю менші а)
\(2^{10}\)б) \(2^{21}\)

4) Напишіть програму, що з'ясовує скільки байтів на цілий та довгий
цілий тип виділяє компілятор, а також чи підтримує він довгий тип та
скільки на нього виділяється байтів.

5) Напишіть функцію, що гарантовано приймає у якості аргументів 8-бітні
натуральні числа та обчислює їх добуток як гарантовано 16-бітне
натуральне число.

6) Визначити більше та менше з двох чисел, введених з клавіатури.

Дано три дійсних числа. Скласти програму для знаходження числа:

a) найбільшого за модулем;

б) найменшого за модулем.

7) Визначити, скільки розв'язків має рівняння та розв'язати його:

а) \(\mathrm{\text{ax}}^{2} + \mathrm{\text{bx}} + c = 0;\) б)
\(\mathrm{\text{ax}}^{4} + \mathrm{\text{bx}}^{2} + c = 0\mathrm{.}\)

Завдання для самостійної роботи

\begin{enumerate}
\def\labelenumi{\arabic{enumi})}
\setcounter{enumi}{7}
\item
  Введіть два натуральних 32-бітних числа та виведіть їх суму як
  32-бітне число, якщо немає переповнення типу. В противному випадку
  виведіть про це повідомлення. Аналогічно підрахуйте добуток двох цілих
  32-бітних чисел.
\item
  Дано три дійсних числа \emph{x, y} і \emph{z}. Скласти програму для
  обчислення:
\item
  а)
  \(\mathrm{\max}(x + y + z,\mathrm{\text{xy}} - \mathrm{\text{xz}} + \mathrm{\text{yz}},\mathrm{\text{xyz}});\)
  б)
  \(\mathrm{\max}(\mathrm{\text{xy}},\mathrm{\text{xz}},\mathrm{\text{yz}})\mathrm{.}\)
\item
  Дано три дійсних числа \emph{x, y} і \emph{z}. Визначити кількість:
\end{enumerate}

а) різних серед них; б) однакових серед них;

в) чисел, що є більшими за їхнє середнє арифметичне значення;

г) чисел, що є більшими за введене з клавіатури число \(a\).

\begin{enumerate}
\def\labelenumi{\arabic{enumi})}
\setcounter{enumi}{7}
\item
  Обчислити значення функцій:
\end{enumerate}

а) \(f(x) = |x|;\) б) \(f(x) = ||x| - 1| - 1;\)

в) \(f(x) = sign(x)\) г) \(f(x) = \mathrm{\sin}|x|;\)

\begin{enumerate}
\def\labelenumi{\arabic{enumi})}
\setcounter{enumi}{7}
\item
  Перевірити, чи існує трикутник із заданими сторонами \emph{a,b,c}.
  Якщо так, то визначити, який він: (гострокутний, прямокутний,
  тупокутний).
\item
  Визначити, скільки розв'язків має рівняння та розв'язати його:
\end{enumerate}

а) \(\mathrm{\text{ax}}^{2} + \mathrm{\text{bx}} + c = 0;\) б)
\(\mathrm{\text{ax}}^{4} + \mathrm{\text{bx}}^{2} + c = 0\mathrm{.}\)

\begin{enumerate}
\def\labelenumi{\arabic{enumi})}
\setcounter{enumi}{7}
\item
  Визначити, скільки розв'язків має система рівнянь і розв'язати її:
\end{enumerate}

а) \(\left\{ \begin{matrix}
a_{1}x + b_{1}y + c_{1} = 0 \\
a_{2}x + b_{2}y + c_{2} = 0; \\
\end{matrix} \right.\ \) б) \(\left\{ \begin{matrix}
\left| x + y \right| = 1 \\
a_{2}x + b_{2}y + c_{2} = 0 \\
\end{matrix} \right.\ \)

\begin{enumerate}
\def\labelenumi{\arabic{enumi})}
\setcounter{enumi}{7}
\item
  Знайти число точок пеpетину кола \(x^{2} + y^{2} = r^{2}\) з відpізком
  \(x = a,\ b \leq y \leq b + c^{2}\) .
\item
  Скласти програму, яка по колу
  \({(x - v)}^{2} + ({y - u)}^{2} = r^{2}\) та пpямій
  \(ax + by + c = 0\) встановлює, який випадок має місце:
\end{enumerate}

а) дві точки пеpетину;

б) одна точка дотику;

в) жодної спільної точки.

\begin{enumerate}
\def\labelenumi{\arabic{enumi})}
\setcounter{enumi}{7}
\item
  З'ясувати, чи пеpетинаються два кола на площині.
\item
  Задано два квадрати, сторони яких паралельні координатним осям.
  З'ясувати, чи перетинаються вони. Якщо так, то знайти координати
  лівого нижнього та правого верхнього кутів прямокутника, що є їхнім
  перетином.
\item
  Дано два прямокутники, сторони яких паралельні координатним осям.
  Відомо координати лівого нижнього та правого верхнього кутів кожного з
  прямокутників. Знайти координати лівого нижнього та правого верхнього
  кутів мінімального прямокутника, що містить задані прямокутники.
\item
  Записати функції, що істинні тоді й тільки тоді, коли:
\end{enumerate}

а) натуральне число n -- парне;

б) остання цифра числа n -- 0;

в) ціле число n кратне натуральному числу m;

г) натуральні числа n і k одночасно кратні натуральному числу m

ґ) сума першої і другої цифри двозначного натурального числа - двозначне
число;

д) число x більше за число y не менше, ніж на 6;

е) принаймні одне з чисел x, y або z більше за 100;

є) тільки одне з чисел x, y або z менше за 1000.

\begin{enumerate}
\def\labelenumi{\arabic{enumi})}
\setcounter{enumi}{7}
\item
  Створити функцію, яка перевіряє, чи належить початок координат
  трикутнику, що заданий координатами своїх вершин.
\item
  Точка простору задана декартовими координатами (x, y, z). Перевірити,
  чи належить вона кулі з радіусом R i центром у початку координат.
\item
  Точка простору задана декартовими координатами (x, y, z). Перевірити,
  чи належить вона циліндру, вісь якого збігається з віссю O. Висота
  дорівнює h, а нижня основа лежить у площині Oxy та має радіус r
\item
  Реалізуйте функції та напишіть відповідну до кожної з них функцію, що
  буде рахувати їх похідні (за нескінченість прийміть число MAXDOUBLE):
\end{enumerate}

а) onestep(x) = \(\left\{ \begin{matrix}
1,x \geq 0 \\
0,x < 0 \\
\end{matrix} \right.\ \)

б)ReLu(x) =\(max(0,x)\)

в)leakyReLu(x,a)= \(\left\{ \begin{matrix}
ax,\ x < 0 \\
0,\ x \geq 0 \\
\end{matrix} \right.\ \)

г) \includegraphics{media/image1.png}eLu(a,x) =\(\left\{ \begin{matrix}
a(e^{x} - 1),x < 0 \\
0,\ x \geq 0 \\
\end{matrix} \right.\ \)

д) sReLu(tl,tr,al,ar,x)=\(\left\{ \begin{matrix}
tl + al\left( x - tl \right),x \leq tl \\
0,tl < x < tr \\
tr + ar\left( x - tr \right),x \geq tr \\
\end{matrix} \right.\ \)

е) isReLu(a,x)= \(\left\{ \begin{matrix}
\frac{x}{\sqrt{1 + ax^{2}}},x < 0 \\
x,\ x \geq 0 \\
\end{matrix} \right.\ \)

ж) SoftExponential(a,x) = \(\left\{ \begin{matrix}
 - \frac{ln(1 - a(x + a)}{a},a < 0 \\
x,a = 0 \\
\frac{e^{\text{ax}} - 1}{a} + a,a > 0 \\
\end{matrix} \right.\ \)

з) sinc(x)= \(\left\{ \begin{matrix}
1,\ x = 0 \\
\frac{\sin x}{x},x \neq 0 \\
\end{matrix} \right.\ \)

4. Цикли.

\begin{itemize}
\item
  Які типи циклів на Сі/Сі++? Напишіть цикл для введення n цілих чисел
  за допомогою трьох різних типів циклів.
\item
  Напишіть цикл для введення дійсних чисел доки не введемо 0. Обчисліть
  максимум з цих чисел.
\item
  Які інструкції та команди дозволяють закінчити (перервати цикл)?
\item
  Як можна уникнути виконання однієї (чи декількох) ітерацій циклу?
\item
  Як обчислити факторіал за допомогою арифметичного циклу на Сі?
\end{itemize}

Аудиторні задачі

\begin{enumerate}
\def\labelenumi{\arabic{enumi})}
\item
  Скласти функцію обчислення за даним дійсним x та натуральним n число
  \(y = \sin(\sin(\ldots\sin(x)\ldots))\) \((\)
  \(\mathrm{\ }n\mathrm{\ \ raziv}).\)
\item
  Вивести на екран такий рядок:
\end{enumerate}

n! = 1*2*3*4*5*...*n,

де n -- введене з клавіатури натуральне число, використовуючи

а) цикл по діапазону із зростанням;

б) цикл по діапазону зі спаданням.

\begin{enumerate}
\def\labelenumi{\arabic{enumi})}
\item
  Скласти функції для обчислення значень многочленів і виконати їх при
  заданих значеннях аргументів:
\end{enumerate}

а)
\(y = x^{n} + x^{n - 1} + \ldots + x^{2} + x + 1\mathrm{\text{\ \ \ \ \ \ \ \ \ \ \ \ \ \ \ \ \ \ \ \ }}n = 3,x = 2;\)

б)
\(y = x^{2^{n}}y^{n} + x^{2^{n - 1}}y^{n - 1} + \ldots + x^{2}y + 1\mathrm{\text{\ \ \ \ \ \ \ \ \ \ }}n = 4,x = 1,y = 2;\)

\begin{enumerate}
\def\labelenumi{\arabic{enumi})}
\item
  Дано натуральне число \(\text{n.}\)Написати програми обчислення
  значень виразів при заданому значенні \(x\):
\end{enumerate}

\[x + (2x)^{2} + \ldots + ((n - 1)x)^{n - 1} + (nx)^{n};\]

\begin{enumerate}
\def\labelenumi{\arabic{enumi})}
\item
  Скласти функцію обчислення подвійного факторіала натурального числа
  \(n\mathrm{\text{\ \ }}y = n!!.\)
\end{enumerate}

\emph{\emph{Вказівка}}. За означенням

\[n!! = \left\{ \begin{matrix}
1 \cdot 3 \cdot 5 \cdot \ldots \cdot n,\mathrm{\ \ iakshcho\ }n - \mathrm{neparne,} \\
2 \cdot 4 \cdot 6 \cdot \ldots \cdot n,\mathrm{\ \ iakshcho\ }n - \mathrm{\ \ parne.} \\
\end{matrix} \right.\ \]

\begin{enumerate}
\def\labelenumi{\arabic{enumi})}
\item
  Скласти програму обчислення
\end{enumerate}

а) \(\sqrt{2 + \sqrt{2 + \ldots + \sqrt{2}}}\) (\emph{п} коренів),

б) \(\sqrt{3 + \sqrt{6 + \ldots + \sqrt{3(n - 1) + \sqrt{3n}}}}.\)

\begin{enumerate}
\def\labelenumi{\arabic{enumi})}
\item
  Скласти програми обчислення значень многочлену для
  \(\mathrm{diisnogo}\) x, що по модулю менше за одиницю та
  \(\mathrm{\ }n \geq 0:\)
\end{enumerate}

\(y = 1 + \frac{x}{1!} + \frac{x^{2}}{2!} + \frac{x^{3}}{3!} + \ldots + \frac{x^{n}}{n!}.\mathrm{\ }\)

\begin{enumerate}
\def\labelenumi{\arabic{enumi})}
\item
  Для довільного цілого числа \(m \geq 1\)знайти найбільше ціле \(k\),
  при якому \(4^{k} \leq m.\)
\item
  Для заданого натурального числа \(n\)одержати найменше число вигляду
  \(2^{r}\), яке перевищує \(n\)\emph{.}
\item
  Знайдіть машинний нуль для вашого компілятора, тобто таке дійсне число
  \(a > 0,\) що \(1 + a = 1\ \)буде істиною.
\end{enumerate}

\emph{Вказівка:} в циклі ділить значення \(a\)на 2 доки не виконується
вказана вище рівність.

\begin{enumerate}
\def\labelenumi{\arabic{enumi})}
\item
  Ввести послідовність наступним чином: користувачу виводиться напис
  ``a{[}**{]}= '', де замість ** стоїть номер числа, що вводиться. Тобто
  там виводяться написи ``a{[}0{]}= '', і після знаку рівності
  користувач вводить число, ``a{[}1{]}= '', і після знаку рівності
  користувач вводить число і так далі доки користувач не введе число 0.
  Після цього потрібно вивести суму введених чисел (масив чисел заводити
  необов'язково).
\item
  Введіть послідовність цілих ненульових чисел (тобто введення
  закінчується коли ми вводимо 0) та виведіть середнє арифметичне
  введених чисел та середнє геометричне.
\end{enumerate}

Для самостійної роботи

\begin{enumerate}
\def\labelenumi{\arabic{enumi})}
\item
  Скласти функції для обчислення значень многочленів і виконати їх при
  заданих значеннях аргументів:
\end{enumerate}

а) \(y = x^{2^{n}} + x^{2^{n - 1}} + \ldots + x^{4} + x^{2} + 1;\)

б) \(y = x^{3^{n}} + x^{3^{n - 1}} + \ldots + x^{9} + x^{3} + 1;\)

в) \(y = x^{1^{2}} + x^{2^{2}} + \ldots + x^{n^{2}}.\)

\begin{enumerate}
\def\labelenumi{\arabic{enumi})}
\item
  Дано натуральне число \(\text{n.\ }\)Написати програми обчислення
  значень виразів при заданому значенні \(x\):
\end{enumerate}

\[1 + (x - 1) + (x - 1)^{2} + \ldots + (x - 1)^{n};\]

\[1 + \frac{1}{x^{2} + 1} + \frac{1}{(x^{2} + 1)^{2}} + \ldots + \frac{1}{(x^{2} + 1)^{n}};\]

\[1 + \sin x + \operatorname{}x + \ldots + \operatorname{}x,\]

\[y = nx^{n - 1} + (n - 1)x^{n - 2} + \ldots + 2x + 1,\]

\(y = \sum_{k = 0}^{n}{kx^{k}(1 - x)^{n - k}},\mathrm{\ }\mathrm{\ }\)
\(za\ umovi\mathrm{(}0 < x < 1,n \geq 0)\)

\begin{enumerate}
\def\labelenumi{\arabic{enumi})}
\item
  Введіть послідовність цілих ненульових чисел (тобто введення
  закінчується коли ми вводимо 0). Визначити кількість змін знаку в цій
  послідовності. Наприклад, у послідовності 1, −34, 8,14, −5, 0 знак
  змінюється три рази.
\item
  Введіть послідовність натуральних ненульових чисел (тобто введення
  закінчується коли ми вводимо 0). Визначити порядковий номер найменшого
  з них.
\item
  Введіть послідовність дійсних ненульових чисел (тобто введення
  закінчується коли ми вводимо 0). Визначити величину найбільшого серед
  від`ємних членів цієї послідовності. Якщо від'ємних чисел немає
  вивести найменший серед додатних членів.
\item
  Банк пропонує річну ставку по депозиту A та 15\% по вкладу додаються
  до основної суми депозиту кожен рік. Ви кладете в цей банк D гривень.
  Скільки років потрібно чекати, щоб сума вкладу зросла до очікуваної
  суми P?
\item
  Скласти програми для обчислення елементів послідовностей. Операцію
  піднесення до степені та функцію обчислення факторіалу не
  використовувати.
\end{enumerate}

а) \(x_{k} = \frac{x^{k}}{k}\ (k \geq 1);\) д)
\(x_{k} = \frac{x^{2k}}{(2k)!}\ (k \geq 0);\);

б) \(x_{k} = \frac{( - 1)^{k}x^{k}}{k}\ (k \geq 1);\) е)
\(x_{k} = \frac{x^{2k + 1}}{(2k + 1)!}\ (k \geq 0);\);

в) \(x_{k} = \frac{x^{k}}{k!}\ (k \geq 0);\) ж)
\(x_{k} = \frac{( - 1)^{k}x^{2k}}{(2k)!}\ (k \geq 0);\);

г) \(x_{k} = \frac{( - 1)^{k}x^{k}}{k!}\ (k \geq 0);\) з)
\(x_{k} = \frac{( - 1)^{k}x^{2k + 1}}{(2k + 1)!}\ (k \geq 0);\)

\begin{enumerate}
\def\labelenumi{\arabic{enumi})}
\item
  Задане натуральне число \emph{n}. Скласти програми обчислення добутків
\end{enumerate}

а)
\(p = \left( 1 + \frac{1}{1^{2}} \right)\left( 1 + \frac{1}{2^{2}} \right)\ldots\left( 1 + \frac{1}{n^{2}} \right),\mathrm{\ \ \ \ n > 2};\)

б)
\(p = \left( 1 - \frac{1}{2^{2}} \right)\left( 1 - \frac{1}{3^{2}} \right)\ldots\left( 1 + \frac{1}{n^{2}} \right),\mathrm{\ \ \ \ n > 2.}\)

\begin{enumerate}
\def\labelenumi{\arabic{enumi})}
\item
  Скласти програму друку таблиці значень функції \(y = \sin x\)на
  відрізку {[}0,1{]} з кроком \(h = 0.1.\)
\item
  Скласти програму визначення кількості тризначних натуральних чисел,
  сума цифр яких дорівнює \(n\ (n > 1).\)Операцію ділення не
  використовувати.
\item
  Дано \emph{n} цілих чисел. Скласти програму, що визначає, скільки з
  них більші за своїх "сусідів", тобто попереднього та наступного чисел.
\item
  Задані натуральне число \emph{n}, дійсні числа
  \(y_{1},\ldots y_{n}.\)Скласти програму визначення
\end{enumerate}

а) \(\max(\left| z_{1} \right|,\ldots,\left| z_{n} \right|),\) де
\(z_{i} = \left\{ \begin{matrix}
\& y_{i},\mathrm{\ \ \ pri\ }\left| y_{i} \right| \leq 2, \\
\& 0.5,\mathrm{\ \ u\ inshikh\ vipadkakh\ \ \ } \\
\end{matrix} \right.\ \);

б) \(\min(\left| z_{1} \right|,\ldots,\left| z_{n} \right|),\) де
\(z_{i} = \left\{ \begin{matrix}
\& y_{i},\mathrm{\ \ \ pri\ }\left| y_{i} \right| \geq 1, \\
\& 2,\mathrm{\ \ u\ inshikh\ vipadkakh\ \ \ } \\
\end{matrix} \right.\ \);

в) \(z_{1} + z_{2} + \ldots + z_{n},\) де
\(z_{i} = \left\{ \begin{matrix}
\& y_{i},\mathrm{\ \ \ pri\ 0 <}\mathrm{y}_{i} < 10, \\
\& 1,\mathrm{\ \ u\ inshikh\ vipadkakh\ \ \ } \\
\end{matrix} \right.\ \)

\begin{enumerate}
\def\labelenumi{\arabic{enumi})}
\item
  Дано натуральне число n. Викинути із запису числа n цифри 0 і 5,
  залишивши порядок інших цифр. Наприклад, з числа 59015509 повинно
  вийти 919.
\item
  Знайти період десяткового дробу для відношення n/m для заданих
  натуральних чисел n та m.
\item
  Скоротити дріб n/m для заданих цілого числа n та натурального числа m.
\end{enumerate}

Підвищеної складності

\begin{enumerate}
\def\labelenumi{\arabic{enumi})}
\item
  Ввести натуральні числа a і b та натуральне число n. Чи можна
  представити число n у вигляді n= k*a + m*b, де k та m --- натуральні
  числа? Якщо можна --- то знайдіть такі числа k та m, що мають найменшу
  суму модулів.
\item
  Представити дане натуральне число як суму двох квадратів натуральних
  чисел. Якщо це неможливо представити як суму трьох квадратів. Якщо і
  це неможливо, представити у вигляді суми чотирьох квадратів
  натуральних чисел.
\item
  Знайти всі цілі корені кубічного рівняння . Вказівка: цілі корені
  повинні бути дільниками (від'ємними або додатними дільниками вільного
  члену d).
\item
  Напишіть функцію, яка розраховує для даного натурального числа n
  значення функції Ойлера (кількість чисел від 1 до n, взаємно простих з
  n).
\item
  Ввести натуральне число \(d > 1\)та натуральне число m. Знайдіть
  мінімальну кількість натуральних чисел вигляду \(x^{d}\ \)(d-ступенів
  натуральних чисел) сума яких дорівнює m.
\end{enumerate}

5. Цикли. Рекурентні співвідношення. Рекурсія

\begin{itemize}
\item
  Яким чином обчислити числа Фібоначчі на Сі за допомогою циклів?
\item
  Який загальний метод обчислення рекурентних послідовномтей для Сі?
\item
  Що таке рекурсія та як її застосувати для обчислення, наприклад,
  факторіалу? Чисел Фібоначчі?
\item
  Що таке бінарний пошук та як його застосувати?
\end{itemize}

\begin{enumerate}
\def\labelenumi{\arabic{enumi})}
\item
  Маємо дійсне число \emph{a}. Скласти програми обчислення:
\end{enumerate}

а) серед чисел
\(1,1 + \frac{1}{2},1 + \frac{1}{2} + \frac{1}{3},\ldots\)першого,
більшого за \emph{;}

б) такого найменшого , що
\(1 + \frac{1}{2} + \ldots + \frac{1}{n} > a.\)

\begin{enumerate}
\def\labelenumi{\arabic{enumi})}
\item
  Числами Фібоначчі називається числова послідовність
  \(\left\{ F_{n} \right\}\), задана рекурентним співвідношенням другого
  порядку
  \(F_{0} = 0,F_{1} = 1,F_{k} = F_{k - 1} + F_{k},\ k = 2,3,\ldots\).
\end{enumerate}

Скласти функції:

а) для обчислення \(F_{n}\ \)за номером члену;

б) номера найбільшого числа Фібоначчі, яке не перевищує задане число
\emph{a;}

в) номера найменшого числа Фібоначчі, яке більше заданого числа
\emph{a;}

г) суми всіх чисел Фібоначчі, які не перевищують 1000.

\begin{enumerate}
\def\labelenumi{\arabic{enumi})}
\item
  Введіть натуральне число n. Далі утворить рекурентну послідовність
  \(a_{i}\)за наступним правилом: \(a_{0} = n\). Якщо \(a_{k}\) парне,
  то \(a_{k}\), якщо --- непарне, то\(a_{k + 1} = 4a_{k} + 1\). Доведіть
  що для n\textless{}1000 ця послідовність буду містити член рівний
  одиниці. Знайдіть серед цих n число, якому потрібно максимальна
  кількість кроків для досягнення одиниці.
\item
  Скласти програми для обчислення добутків:
\end{enumerate}

а) \(P_{n} = \prod_{i = 1}^{n}\left( 2 + \frac{1}{i!} \right);\) б)
\(P_{n} = \prod_{i = 1}^{n}\left( \frac{i + 1}{i + 2} \right);\)

в) \(P_{n} = \prod_{i = 1}^{n}\frac{1}{(i + 1)!};\); г)
\(P_{n} = \prod_{i = 1}^{n}\frac{1}{i^{i} + 1}.\)

\emph{\emph{Вказівка}}. Добуток \emph{P\textsubscript{n}} обчислити за
допомогою рекурентного співвідношення
\(P_{0} = 1,P_{k} = P_{k - 1}*a_{k},k = 1,2,\ldots,n,\)\emph{k=}1,2\emph{,...,n,}
де \(a_{k}\)- \emph{k}- тий множник.

\begin{enumerate}
\def\labelenumi{\arabic{enumi})}
\item
  Скласти програми для обчислення найменшого додатного члена числових
  послідовностей, які задаються рекурентними співвідношеннями, та його
  номера:
\end{enumerate}

а)
\(x_{n} = x_{n - 1} + x_{n - 3} + 100,\mathrm{\text{\ \ \ \ \ }}x_{1} = x_{2} = x_{3} = - 99,\mathrm{\text{\ \ \ }}n = 3,4,\ldots;\)

\begin{enumerate}
\def\labelenumi{\arabic{enumi})}
\item
  Скласти програми для обчислення ланцюгових дробів
\end{enumerate}

а) \(b_{n} = b + \frac{1}{b + \frac{1}{b + \ddots + \frac{1}{b}};}\); б)
\(\lambda_{n} = 2 + \frac{1}{6 + \frac{1}{10 + \ddots + \frac{1}{4n + 2}};}\)

в)
\(x_{2n} = 1 + \frac{1}{2 + \frac{1}{1 + \frac{1}{2 + \frac{1}{1 + \ddots + \frac{1}{2}}}.};}\)

\emph{\emph{Вказівка}}. Використати рекурентні співвідношення

а)
\(b_{0} = b,b_{k} = b + \frac{1}{b_{k - 1}},\mathrm{\text{\ \ }}k = 1,2,\ldots,n;\)

б)
\(b_{0} = 4n + 2,b_{k} = 4(n - k) + 2 + \frac{1}{b_{k - 1}},\mathrm{\text{\ \ }}k = 1,2,\ldots,n.\)

\begin{enumerate}
\def\labelenumi{\arabic{enumi})}
\item
  Скласти програми для обчислення суми:
\end{enumerate}

\(S_{n} = \sum_{k = 1}^{n}\frac{2^{k}}{a_{k} + b_{k}},\) ,

де \(\left\{ \begin{matrix}
\& a_{1} = 0,a_{2} = 1, \\
\& a_{k} = \frac{a_{k - 1}}{k} + a_{k - 2}b_{k}, \\
\end{matrix} \right.\ \) \(\left\{ \begin{matrix}
\& b_{1} = 1,b_{2} = 0, \\
\& b_{k} = b_{k - 1} + a_{k - 1}, \\
\end{matrix} \right.\ \) \(k = 3,4,\ldots;\)

\emph{Самостійна}

\begin{enumerate}
\def\labelenumi{\arabic{enumi})}
\item
  Скласти програми обчислення довільного елемента послідовностей,
  заданих рекурентними співвідношеннями
\end{enumerate}

а)
\(v_{0} = 1,v_{1} = 0.3,\mathrm{\text{\ \ \ \ \ \ \ \ }}v_{i} = (i + 2)v_{i - 2},\mathrm{\text{\ \ \ }}i = 2,3,\ldots\)

б)
\(v_{0} = v_{1} = v_{2} = 1,\mathrm{\text{\ \ \ \ \ \ \ \ }}v_{i} = (i + 4)(v_{i - 1} - 1) + (i + 5)v_{i - 3},\mathrm{\text{\ \ }}i = 3,4,\ldots\)

в)
\(v_{0} = v_{1} = 0,\ v_{2} = \frac{3}{2}\mathrm{,\ \ \ }v_{i} = \frac{i - 2}{(i - 3)^{2} + 1}v_{i - 1} - v_{i - 2}v_{i - 3} + 1,\ i = 2,3,\ldots\)

\begin{enumerate}
\def\labelenumi{\arabic{enumi})}
\item
  Скласти програму обчислення довільного елемента послідовності
  \(v_{n}\), визначеної системою співвідношень
\end{enumerate}

\[v_{0} = v_{1} = 1,\mathrm{\text{\ \ \ \ }}v_{i} = \frac{u_{i - 1} - v_{i - 1}}{\left| u_{i - 2} + v_{i - 1} \right| + 2},\mathrm{\text{\ \ \ }}i = 2,3,\ldots;\]

де
\(u_{0} = u_{1} = 0,\mathrm{\text{\ \ \ \ }}\mathrm{u}_{i} = \frac{u_{i - 1} - u_{i - 2}v_{i - 1} - v_{i - 2}}{1 + u_{i - 1}^{2} + v_{i - 1}^{2}},\mathrm{\text{\ \ \ }}i = 2,3,\ldots;\)

\begin{enumerate}
\def\labelenumi{\arabic{enumi})}
\item
  Скласти програми для обчислення сум:
\end{enumerate}

а)
\(S_{n} = \sum_{k = 1}^{n}{2^{k}a_{k}},\mathrm{\ de\ \ }a_{1} = 0,a_{2} = 1,a_{k} = a_{k - 1} + k*a_{k - 2},\ k = 3,4,\ldots;\)
б)
\(S_{n} = \sum_{k = 1}^{n}\frac{3^{k}}{a_{k}},\mathrm{\ de\ \ }a_{1} = a_{2} = 1,\ a_{k} = \frac{a_{k - 1}}{k} + a_{k - 2},\mathrm{\text{\ \ }}k = 3,4,\ldots;\)

в)
\(S_{n} = \sum_{k = 1}^{n}\frac{k!}{a_{k}},\mathrm{\ de\ \ }a_{1} = a_{2} = 1,\ a_{k} = a_{k - 1} + \frac{a_{k - 1}}{2^{k}},\mathrm{\text{\ \ }}k = 3,4,\ldots;\)

г\emph{)}
\(S_{n} = \sum_{k = 1}^{n}{k!a_{k}},\mathrm{\ de\ \ }a_{1} = 0,a_{2} = 1,\ a_{k} = a_{k - 1} + \frac{a_{k - 2}}{(k - 1)!},\ k = 3,4,\ldots;\)

ґ)
\(S_{n} = \sum_{k = 1}^{n}\frac{a_{k}}{2^{k}},\mathrm{\ de\ \ }a_{1} = a_{2} = a_{3} = 1,\ a_{k} = a_{k - 1} + a_{k - 3},\mathrm{\text{\ \ }}k = 4,5,\ldots;\)

д)
\(S_{n} = \sum_{k = 1}^{n}{\frac{2^{k}}{k!}a_{k}},\mathrm{\ de\ \ }a_{0} = 1,a_{k} = ka_{k - 1} + \frac{1}{k},\mathrm{\text{\ \ }}k = 1,2,\ldots.\)

\begin{enumerate}
\def\labelenumi{\arabic{enumi})}
\item
  \protect\hypertarget{_Hlk65241776}{}{}Скласти програми для обчислення
  сум:
\end{enumerate}

а) \(S_{n} = \sum_{k = 1}^{n}\frac{3^{2k + 1}}{a_{k}*b_{k} + 1},\) ,

де \(\left\{ \begin{matrix}
\& a_{1} = 2,a_{2} = 1, \\
\& a_{k} = \frac{a_{k}}{k + 1} + a_{k - 2} + b_{k}, \\
\end{matrix} \right.\ \) \(\left\{ \begin{matrix}
\& b_{1} = 1,b_{2} = 0, \\
\& b_{k} = 2b_{k - 1} + a_{k - 1}, \\
\end{matrix} \right.\ \) \(k = 3,4,\ldots;\)

б) \(S_{n} = \sum_{k = 1}^{n}\frac{a_{k}b_{k}}{(k + 1)!},\)

де \(\left\{ \begin{matrix}
\& a_{1} = u, \\
\& a_{k} = 2b_{k - 1} + a_{k - 1}, \\
\end{matrix} \right.\ \) \(\left\{ \begin{matrix}
\& b_{1} = v, \\
\& b_{k} = 2a_{k = 1}^{2} + b_{k - 1}, \\
\end{matrix} \right.\ \) \(k = 2,3,\ldots;\)

\emph{u,v} -- задані дійсні числа;

в)
\(\ S_{n} = \sum_{k = 1}^{n}\frac{2^{k}}{{(1 + a}_{k} + b_{k}){k!}^{}}\)

де \(\left\{ \begin{matrix}
\& a_{1} = 1, \\
\& a_{k} = 3b_{k - 1} + 2a_{k - 1}, \\
\end{matrix} \right.\ \) \(\left\{ \begin{matrix}
\& b_{1} = 1, \\
\& b_{k} = 2a_{k - 1} + b_{k - 1}, \\
\end{matrix} \right.\ \) \(k = 2,3,\ldots;\)

г) \(S_{n} = \sum_{k = 1}^{n}\left( \frac{a_{k}}{b_{k}} \right)^{k},\)

де \(\left\{ \begin{matrix}
\& a_{0} = 1,a_{1} = 2, \\
\& a_{k} = b_{k - 2} + \frac{b_{k}}{2}, \\
\end{matrix} \right.\ \) \(\left\{ \begin{matrix}
\& a_{0} = 5,b_{1} = 5, \\
\& b_{k} = b_{k - 2}^{2} - a_{k - 1}, \\
\end{matrix} \right.\ \) \(k = 2,3,\ldots;\)

д) \(S_{n} = \sum_{k = 1}^{n}\frac{a_{k}}{1 + b_{k}},\)

де \(\left\{ \begin{matrix}
\& a_{0} = 1, \\
\& a_{k} = b_{k - 1}a_{k - 1}, \\
\end{matrix} \right.\ \) \(\left\{ \begin{matrix}
\& b_{0} = 1, \\
\& b_{k} = b_{k - 1} + a_{k - 1}, \\
\end{matrix} \right.\ \) \(k = 1,2,\ldots.\)\emph{.}

\begin{enumerate}
\def\labelenumi{\arabic{enumi})}
\item
  Скласти програми для обчислення добутків
\end{enumerate}

а) \(P_{n} = \prod_{k = 0}^{n}{\frac{a_{k}}{3^{k}},}\) де
\(\left\{ \begin{matrix}
\& a_{0} = a_{1} = 1,\ a_{2} = 3, \\
\& a_{k} = a_{k - 3} + \frac{a_{k - 2}}{2^{k - 1}}, \\
\end{matrix} \right.\ \), \(k = 3,4,\ldots;\)

б) \(P_{n} = \prod_{k = 1}^{n}{a_{k}b_{k},}\)

де \(\left\{ \begin{matrix}
\& a_{1} = 1, \\
\& a_{k} = \left( \sqrt{b_{k - 1}} + a_{k - 1} \right)/5, \\
\end{matrix} \right.\ \) \(\left\{ \begin{matrix}
\& b_{1} = 1, \\
\& b_{k} = 2b_{k - 1} + 5a_{k - 1}^{2}, \\
\end{matrix} \right.\ \) \(k = 2,3,\ldots\)\emph{.}

\begin{enumerate}
\def\labelenumi{\arabic{enumi})}
\item
  Реалізувати функцію яка з`ясовує, чи входить задана цифра до запису
  заданого натурального числа.
\item
  Реалізувати функцію "обернення" (запису в оберненому порядку цифр)
  заданого натурального числа.
\end{enumerate}

\emph{\emph{Вказівка. Для побудови числа використати рекурентне
співвідношення}} \(y_{0} = 0,y_{i} = y_{i - 1}*10 + a_{i},\)\emph{де}
\(a_{i}\) \emph{- наступна цифра числа} \(n\)\emph{при розгляді цифр
справа наліво.}

\begin{enumerate}
\def\labelenumi{\arabic{enumi})}
\item
  Скласти програми наближеного обчислення суми всіх доданків, абсолютна
  величина яких не менше ε\emph{\textgreater{}}0:
\end{enumerate}

а) \(y = \sin x = x - \frac{x^{3}}{3!} + \frac{x^{5}}{5!} - \ldots;\)

б) \(y = \cos x = 1 - \frac{x^{2}}{2!} + \frac{x^{4}}{4!} - \ldots;\)

в)
\(y = \operatorname{s}hx = x + \frac{x^{3}}{3!} + \frac{x^{5}}{5!} + \ldots;\)

г) \(y = chx = 1 + \frac{x^{2}}{2!} + \frac{x^{4}}{4!} + \ldots;\)

д) \(y = e^{x} = 1 + \frac{x}{1!} + \frac{x^{2}}{2!} + \ldots;\)

е)
\(y = \ln(1 + x) = x - \frac{x^{2}}{2!} + \frac{x^{3}}{3!} - \ldots,\mathrm{\text{\ \ \ \ \ }}(\left| x \right| < 1);\)

ж)
\(y = \frac{1}{1 + x} = 1 - x + x^{2} - x^{3} + \ldots,\mathrm{\text{\ \ \ \ \ }}(\left| x \right| < 1);\)

з)
\(y = \ln\frac{1 + x}{1 - x} = 2*\left\lbrack \frac{x}{1} + \frac{x^{3}}{3} + \frac{x^{5}}{5} + \ldots \right\rbrack\mathrm{,\ \ \ \ \ }(\left| x \right| < 1);\)

і)
\(y = \frac{1}{(1 + x)^{2}} = 1 - 2*x + 3*x^{2} - \ldots,\mathrm{\text{\ \ \ \ \ }}(\left| x \right| < 1);\)

к)
\(y = \frac{1}{(1 + x)^{3}} = 1 - \frac{2*3}{2}x + \frac{3*4}{2}x^{2} - \frac{4*5}{2}x^{3} + \ldots,\mathrm{\text{\ \ \ \ \ }}(\left| x \right| < 1);\)

л)
\(y = \frac{1}{1 + x^{2}} = 1 - x^{2} + x^{4} - x^{6} + \ldots,\mathrm{\text{\ \ \ \ \ }}(\left| x \right| < 1);\)

м)
\(y = \sqrt{1 + x} = 1 + \frac{1}{2}x - \frac{1}{2*4}x^{2} + \frac{1*3}{2*4*6}x^{3} - \ldots,\mathrm{\text{\ \ \ \ \ }}(\left| x \right| < 1);\)

н)
\(y = \frac{1}{\sqrt{1 + x}} = 1 - \frac{1}{2}x + \frac{1*3}{2*4}x^{2} - \frac{1*3*5}{2*4*6}x^{3} - \ldots,\mathrm{\text{\ \ \ \ \ }}(\left| x \right| < 1);\)

о)
\(y = \mathrm{\text{arc}}\sin x = x + \frac{1}{2}\frac{x^{3}}{3!} + \frac{1*3}{2*4}\frac{x^{5}}{5!} + \ldots,\mathrm{\text{\ \ \ \ \ \ \ }}(\left| x \right|\mathrm{< 1}).\)

\emph{\emph{Вказівка}}. Суму \emph{y} обчислювати за допомогою
рекурентного співвідношення
\(S_{0} = 0,\ S_{k} = S_{k - 1} + a_{k},\ k = 1,2,\ldots,\) де
\(a_{k} - k\)\emph{-}тий доданок, для обчислення якого також складається
рекурентне співвідношення. В якості умови повторення циклу розглядається
умова \(\left| a_{k} \right| \geq \varepsilon.\)

\begin{enumerate}
\def\labelenumi{\arabic{enumi})}
\item
  Ввести дійсні числа
  \(x,\varepsilon\ (x \neq 0,\varepsilon > 0)\)\emph{.} Обчислити з
  точністю \(\varepsilon\) нескінченну суму і вказати кількість
  врахованих доданків.
\end{enumerate}

а) \(\sum_{k = 0}^{\infty}\frac{x^{2k}}{2k!};\) б)
\(\sum_{k = 0}^{\infty}\frac{( - 1)^{k}x^{k}}{(k + 1)^{2}};\)

в) \(\sum_{k = 0}^{\infty}\frac{x^{2k}}{2^{k}k!};\) г)
\(\sum_{k = 0}^{\infty}\frac{( - 1)^{k}x^{2k + 1}}{k!(2k + 1)!}.\)

Додаткові задачі:

\begin{enumerate}
\def\labelenumi{\arabic{enumi})}
\item
  Дано натуральне число \emph{k} . Скласти програму одержання
  \emph{к}-тої цифри послідовності
\end{enumerate}

а) 110100100010000 ... , в якій виписані підряд степені 10;

б) 123456789101112 ... , в якій виписані підряд всі натуральні числа;

в) 149162536 ... , в якій виписані підряд квадрати всіх натуральних
чисел;

г) 01123581321 ... , в якій виписані підряд всі числа Фібоначчі.

\begin{enumerate}
\def\labelenumi{\arabic{enumi})}
\item
  Скласти програму знаходження кореня рівняння \(tgx = x\)на відрізку
  {[}0,001;1,5{]} із заданою точністю \(\varepsilon\), використовуючи
  метод ділення відрізку навпіл.
\item
  Знайти корінь рівняння \(x^{3} + 4x^{2} + x - 6 = 0,\) який міститься
  на відрізку {[}0,2{]}, з заданою точністю
\end{enumerate}

\emph{\emph{Вказівка.}} Одним з методів розв`язування рівняння є метод
хорд, який полягає в обчисленні елементів послідовності

\[u_{0} = a,\backslash n\]

до виконання умови \(\left| u_{n} - u_{n - 1} \right| < \varepsilon\). В
умовах нашої задачі \(a = 0,b = 2,\ y(x) = x^{3} + 4x^{2} + x - 6.\)

\begin{enumerate}
\def\labelenumi{\arabic{enumi})}
\item
  а)Скласти програму, яка визначає потрібний спосіб розміну будь-якої
  суми грошей до 99 коп. за допомогою монет вартістю 1, 2, 5, 10, 25, 50
  коп.
\end{enumerate}

б) Розв'яжить цю задачу для будь-якого натурального числа m
(1\textless{}m\textless{}100000) копійок так щоб кількість монет при
цьому була найменша.

6. Бітові операції

\begin{itemize}
\item
  Що таке та які бітові операції існують? Який пріоритет цих операцій?
\item
  \protect\hypertarget{_Hlk65235798}{}{}Чому дорівнюють наступні вирази:
  3\textless{}\textless{}2, 5\textgreater{}\textgreater{}2, 5 \& 3, n
  \&1, n \textbar{} 1, n\^{}n, \textasciitilde{}0.
\item
  Як знайти значення самого лівого біту? Самого правого? Третього зліва?
  Як встановити 5-й байт зліва в 1? В нуль?
\item
  Для яких типів Сі краще не застосовувати бітові операції?
\item
  Який нюанс для першого біту є при використанні бітових операцій для
  цілого типу С/С++?
\end{itemize}

Аудиторні задачі

\begin{enumerate}
\def\labelenumi{\arabic{enumi})}
\item
  \protect\hypertarget{_Hlk65235695}{}{}Ввести натуральне 8-бітове число
  n і вивести 2\textsuperscript{n}, використовуючи бітові операції.
\item
  Ввести ціле число n та натуральне k і вивести ціле число, яке у якого
  k-й біт встановлений в 1, а всі інші біти збігаються з бітами числа n
  на тих же позиціях. Наприклад, якщо введені 9 і 1, відповіддю буде 11.
\item
  Вести натуральне довге число
  \protect\hypertarget{__DdeLink__4_8035966481}{}{}M. Встановіть її біт
  з номером j рівним нулеві та виведіть отримане число в десятковому та
  шістнадцятковому вигляді.
\item
  Поміняйте місцями перші 8 біт та останні 8 біт натурального числа,
  виведіть отримане число в десятковому та шістнадцятковому вигляді.
\item
  Підрахуйте найбільшу кількість одиничок серед бітів даного числа, що
  йдуть підряд.
\item
  Описати словами результат наступного виразу: x \& (x-1).
\item
  Описати словами результат наступного виразу: x \& (-x).
\item
  Напишіть функцію що визначає до якої архітектури (big, high, little
  endian) належить даний комп'ютер.
\end{enumerate}

Самостійна

\begin{enumerate}
\def\labelenumi{\arabic{enumi})}
\item
  Ввести натуральне(32-бітне) число M. Встановіть її j-тий рівним нулеві
  та виведіть отримане число виведіть отримане число в десятковому та
  шістнадцятковому вигляді.
\item
  Визначить номер першого значущого зліва та зправа біта натурального
  числа M.
\item
  Поміняйте місцями перші 8 біт та останні 8 біт натурального числа
  (розмір в бітах якого вважаємо невідомим до вводу) та виведіть
  отримане число в десятковому та шістнадцятковому вигляді.
\item
  Ввести натуральне 64-бітне число M. Встановіть її ліві n біт рівним
  нулеві та виведіть отримане число. Встановіть її праві n біт рівним
  нулеві та виведіть отримане число в десятковому та вісімковому
  вигляді. Розвяжить задачу для типу M unsigned та long long unsigned.
\item
  Ввести натуральне число M. Поміняйте місцями біти її двійкового запису
  з номерами i та j (що теж вводяться) та виведіть отримане число в
  десятковому та шістнадцятковому вигляді.
\item
  Знайдіть кількість значущих (не рівних 0) бітів натурального
  32-бітного числа.
\item
  За допомогою лище бітових операцій та операції декременту зясуйте чи є
  дане натуральне число ступінню двійки. Спробуйте з циклом та без
  циклу. (Підказка: подумайте, як виглядає бітове представлення
  декременту ступеню двійки, та використайте далі конюнкцію).
\item
  Ввести натуральні 32-бітні числа M та N та визначить скільки в них
  спільних одиничок бітового представлення. Визначить скільки в цих
  числах взагалі співпадає бітів.
\item
  Виведіть бітове (двійкове) представлення натурального числа.
\item
  Інвертуйте (тобто прочитайте зліва направо) бітове представлення
  даного числа та виведіть двійкове представлення та десяткове для цієї
  інверсії.
\item
  Ввести ціле число n (однобайтове) і вивести число, отримане в
  результаті циклічного зсуву числа n на один розряд вліво, тобто
  старший біт зсунитий в позицію молодшого, а всі інші біти зсуваються
  на один розряд вліво. Наприклад, якщо введено 130, відповіддю буде 5.
\item
  Визначити, скільки разів зустрічається 11 в двійковому поданні цілого
  додатнього числа (в двійковому поданні 11110111 воно зустрічається 5
  разів).
\item
  Викреслити i-й біт з двійкового представлення натурального числа
  (молодші i-го біти залишаються на місці, старші зсуваються на один
  розряд вправо). Наприклад, якщо введені 11 і 2, відповіддю буде 7.
\item
  \protect\hypertarget{_Hlk65236142}{}{} Написати функцію, результатом
  якого є дане значення x, у якого молодший нульовий біт та найстарший
  біт встановлені в 1.
\item
  Написати функцію, результатом якого є дане значення x, у якого все
  біти встановлені в 1, крім молодших n бітів.
\item
  Підрахуйте кількість нулів серед бітів даного числа.
\item
  Знайдіть номер найстаршого значущого біта в даному 32-бітному числі.
\item
  Напишіть функцію, що визначає чи два натуральних числа не мають
  одиничних бітів на однакових позиціях.
\item
  Напишіть функцію, що визначає чи два натуральних числа не мають
  нульових бітів на однакових позиціях.
\item
  Напишіть функцію, що визначає чи два натуральних числа не мають
  однакових бітів на однакових позиціях.
\end{enumerate}

7. Статичні масиви. Лінійні масиви та багатовимірні масиви

\begin{itemize}
\item
  Які варіанти декларації масивів на Сі. На Сі++?
\item
  Які варіанти ініціалізації масивів на Сі. На Сі++?
\item
  Створення багатовимірного масиву. Введіть розміри та вміст двовимірної
  дійсної матриці. Виведіть її красиво рідок за рядком.
\item
  Як найкраще передавати масив у аргументи функції?
\item
  Чи можна повернути масив фіксованого розміру як результат функції?
\item
  Чому масив як аргумент краще передавати через вказівник чи посилання?
\item
  Як повернути коректно дані з масиву з функції?
\end{itemize}

Задачі для аудиторної роботи

\begin{enumerate}
\def\labelenumi{\arabic{enumi})}
\item
  Ініціалізуйте масив 5 цілих чисел в програмі довільним чином. Введіть
  дійсне число та знайдіть кількість чисел у вашому масиві, що менше зі
  це число.
\end{enumerate}

\begin{enumerate}
\def\labelenumi{\arabic{enumi})}
\item
  Масив заповнений таким чином: 5, 112, 4, 3. Вивести його елементи
  навпаки (3,4,112,5). При цьому використання циклу є обов'язковим.
\item
  Заповнити масив типу double з 10 елементів з клавіатури (по черзі в
  циклі вводяться всі елементи) і знайти суму всіх елементів більших за
  число Ейлера \(e\).
\item
  Масив типу int з 5 елементів заповнюється з клавіатури. Знайти і
  вивести на екран максимальне значення у вашому масиві.
\item
  Знайти суму всіх парних і непарних елементів масиву натуральних чисел.
  Масив заповнюється з клавіатури, 5 елементів.
\item
  \protect\hypertarget{_Hlk65236602}{}{}Написати функції, що
\end{enumerate}

а) вводить n-вимірний вектор дійсних чисел;

б) виводить n-вимірний вектор дійсних чисел;

в) рахує суму двох векторів (результат : аргумент функції --- масив);

г) рахує скалярний добуток двох векторів.

Протестувати роботи цих функцій: ввести в головній програмі розмірність
векторів, 2 вектори цієї розмірності та підрахувати їх суму та скалярний
добуток і вивести результати.

Задачі для самостійної роботи

\begin{enumerate}
\def\labelenumi{\arabic{enumi})}
\item
  Написати функцію, що вводить послідовність ненульових цілих чисел,
  введення завершується при вводі нуля. Кількість елементів масиву
  обмежена числом 20. Визначити кількість добуток та середнє гармонічне
  цієї послідовності.
\item
  Вводиться масив натуральних чисел заданого розміру N:
\end{enumerate}

а) визначити скільки серед цих чисел повних квадратів простих чисел

б) визначити скільки серед цих чисел парних повних кубів

в) визначити скільки серед цих чисел n-тих ступенів цілих чисел (для
всіх n\textgreater{}1)

г) визначити скільки серед них цілих ступенів двійки

д) визначити скільки серед них ступенів чисел, що кратні 3

е) визначити скільки серед них простих чисел

ж) визначити скільки серед них чисел Фібоначчі

з) визначити скільки серед них чисел, у яких 5-й, 6-й та 8-й біт
двійкового запису дорівнюють 1

і) визначити скільки серед них чисел, які містять рівно 5 біт в
двійковому записі, що дорівнюють 1

к) визначити скільки серед них чисел, у яких сума цифр в десятковому
запису ділиться на 7

10) Задані натуральне число \(n\)\emph{,} дійсні числа
\(a_{1},a_{2},\ldots,a_{n}.\)Скласти програму для знаходження:

а) \(\max\left( a_{1},a_{2},\ldots,a_{n} \right);\) б)
\(\min\left( a_{1},a_{2},\ldots,a_{n} \right);\)

в) \(\max\left( a_{2},a_{4},\ldots \right);\) г)
\(\min\left( a_{1},a_{3},\ldots \right);\)

д)
\(\min\left( a_{2},a_{4},\ldots \right) + \max\left( a_{1},a_{3},\ldots \right);\)

е)
\(\max\left( \left| a_{1} \right|,\ldots,\left| a_{n} \right| \right);\)
ж) \(\max\left( - a_{1},a_{2}, - a_{3}\ldots,( - 1)^{n}a_{n} \right);\)

з)
\(\left( \min\left( a_{1},\ldots,a_{n} \right) \right)^{2} - \min\left( a_{1n}^{2},\ldots,a_{n}^{2} \right).\)

11) Дано натуральне число n, цілі числа \(a_{1},a_{2},\ldots,a_{n}.\)
Скласти програму знаходження

а) \(min(a_{1},2a_{2},\ldots,na_{n});\)

б) \(min(a_{1} + a_{2},\ldots,a_{n - 1} + a_{n});\)

в) \(max(a_{1},a_{1}a_{2},\ldots,a_{1}a_{2}\ldots a_{n});\)

г) кількості парних серед \(a_{1},a_{2},\ldots,a_{n}\);

д) кількості повних квадратів серед\(a_{1},a_{2},\ldots,a_{n}\);

е) кількості квадратів непарних чисел серед
\(a_{1},a_{2},\ldots,a_{n}.\)

12) Скласти функції для обчислення

а) Значення многочлена Чебишова заданого степеню \(n\)в точці \(x\)

\(T_{0}(x) = 1,T_{1}(x) = x,\)

\(T_{n}(x) = 2xT_{n - 1}(x) - T_{n - 2}(x),n = 2,3,\ldots;\)

та функцію, що виводить коефіцієнти поліному ступеня n\textless{}256.

б) многочлена Ерміта заданого степеню \(n\)в точці \(x\)

\(H_{0}(x) = 1,H_{1}(x) = 2x,\)

\(H_{n}(x) = 2xH_{n - 1}(x) - 2(n - 1)H_{n - 2}(x),n = 2,3,\ldots\)

та функцію, що виводить коефіцієнти поліному ступеня n\textless{}256.

\begin{enumerate}
\def\labelenumi{\arabic{enumi})}
\setcounter{enumi}{12}
\item
  В цілочисельному масиві A{[}N{]} знайдіть моду, тобто елемент, що
  зустрчається найбільшу кількість разів. Якщо таких елементів декілька
  виведіть всі такі елементи.
\end{enumerate}

\begin{enumerate}
\def\labelenumi{\arabic{enumi})}
\setcounter{enumi}{12}
\item
  В цілочисельному масиві A{[}N{]} знайдіть елемент, що є найближчим до
  середнього арифметичного найбільшого та найменшого елементу масиву.
\item
  Напишіть функцію, яка в дійсному масиві A{[}N{]} знаходить середнє
  відхилення (варіацію) масиву
\item
  Знайдіть в даному цілому числі цифру десяткового запису, яка
  зустрічається найбільшу кількість разів. Якщо їх декілька, виведіть
  найбільшу цифру.
\item
  Напишіть функцію, яка за заданим масивом значень
  \({\{ x_{i}\}}_{i = 1}^{d}\)обчислює:
\end{enumerate}

\includegraphics[width=3.23976in,height=0.66654in]{media/image4.png}

\begin{enumerate}
\def\labelenumi{\arabic{enumi})}
\setcounter{enumi}{12}
\item
  Біля прилавка в магазині вишикувалася черга з n покупців. Час
  обслуговування продавцем t-го покупця \(t_{i}\) (i = 1, ...,n). Нехай
  дано натуральне n і дійсні \({\{ t_{i}\}}_{i = 1..n}\) . Отримати
  \({\{ c_{i}\}}_{i = 1..n},\)де \(c_{i}\)- час перебування i-го покупця
  в черзі (i = 1..n). Вказати номер покупця, для обслуговування якого
  продавцеві потрібно найменше часу.
\item
  В деяких видах спортивних змагань виступ кожного спортсмена незалежно
  оцінюється деякими суддями, потім з усієї сукупності оцінок
  видаляються найбільш висока і найнижча, а для решти оцінок
  обчислюється середнє арифметичне, яке і йде в залік спортсмену. Якщо
  найбільш високу оцінку виставило декілька суддів, то з сукупності
  оцінок видаляється лише одна така оцінка; аналогічно надходять з
  найбільш низькими оцінками. Дано натуральне число n, дійсні числа
  \(a_{1},a_{2},\cdots,a_{n}\). Вважаючи, що
  \(a_{1},a_{2},\cdots,a_{n}\)оцінки, виставлені суддями одному з
  учасників змагань, визначити оцінку, яка піде в залік цього
  спортсмену.
\end{enumerate}

Додаткові задачі:

По заданим значенням коефіцієнтів поліномів P(x) та Q(x) знайдіть
значення коефіцієнтів поліному P(Q(x)).

\begin{enumerate}
\def\labelenumi{\arabic{enumi})}
\setcounter{enumi}{12}
\item
  Обчислити коефіцієнти багаточлена з заданими дійсними коренями
  x{[}0{]},x{[}1{]}, \ldots{}, x{[}n{]}. Кількість коефіцієнтів обмежена
  числом 100.
\item
  Побудувати N-розрядний код Грея. Кодом Грея зветься така послідовність
  дворозрядних двійкових чисел, в яких кожні два сусідних а також перше
  й останнє числа відрізняються лише одним розрядом. Так, для N=2 код
  Грея наступний: 00,01,11,10. Для N=3: 000,001,011,010,110,111,101.
  Переведіть всі числа з цього двійкового коду до десяткової системи
  числення.
\item
  В цілочисельному масиві A{[}N{]} (не обов'язково впорядкованому)
  знайдіть медіану, тобто величину, що ділить ряд навпіл: по обидві
  сторони від неї знаходиться однакова кількість одиниць сукупності.
  Тобто, якщо кількість чисел непарна, обирається елемент, що є середнім
  за зростанням. Наприклад, для впорядкованого набору чисел 1, 3, 3, 6,
  7, 8, 9 медіаною є четверте із них, число 6. Якщо кількість елементів
  парна, тоді медіану зазвичай визначають як середнє значення між двома
  числами по середині впорядкованого масиву Наприклад, для наступного
  набору 1, 2, 3, 4, 5, 6, 8, 9 - медіана є середнім значенням для двох
  чисел по середині: вона дорівнюватиме (4 + 5)/2=4.5.
\end{enumerate}

*********

7. Статичні масиви. Лінійні масиви та багатовимірні масиви

Задачі для аудиторної роботи

\begin{enumerate}
\def\labelenumi{\arabic{enumi})}
\item
  Двовимірна матриця 3х3 ініціалізована числами
  \{\{1,2,3,\},\{4,5,6\},\{7,8,9\}\}. Транспонуйте цю матрицю, введіть
  натуральні числа N і M та замініть елемент, що рівний числу M (якщо
  він є в матриці) на число N. Виведіть отриману матрицю рядок за
  рядком.
\item
  Двовимірна матриця 3х3 ініціалізована дійсними числами \{\{1.0,
  2,3,\},\{4,5,6\},\{7,8,9\}\}. Транспонуйте цю матрицю, введіть
  натуральні числа I і J та дійсне число A замініть елемент з індексами
  IJ на число A (відслідкуйте при цьому коректність індексів). Виведіть
  отриману матрицю рядок за рядком.
\item
  Напишіть функцію для вводу двовимірної дійсної матриці довільного
  розміру m x n, яка вводить з підказкою для користувача (вказуючи
  індекси елементів) кожен елемент в одному рядку. Величини m, n
  вводяться з консолі та їх значення менші за 20.
\item
  Напишіть функцію для вводу двовимірної цілої (дійсної) матриці
  довільного розміру m x n, яка вводить з підказкою для користувача
  (номер рядку) матрицю рядок за рядком (числа в рядку розділяються
  одним пробілом). Величини m, n вводяться з консолі та їх значення
  менші за 25.
\item
  В двовимірному масиві A{[}N,M{]} знайдіть суму елементів A{[}i,j{]},
  що i-j=k . Ціле число k може бути від'ємним, якщо таких елементів
  немає, то вивести нуль. Величини m, n вводяться з консолі та їх
  значення менші за 100.
\end{enumerate}

Задачі для самостійної роботи

\begin{enumerate}
\def\labelenumi{\arabic{enumi})}
\item
  Дана матриця розміру n * m. Поміняти місцями її стовпці так, щоб їх
  максимальні елементи утворювали спадаючу послідовність.
\item
  Знайдіть квадратну матрицю, зворотну даної з розміром n x n.
\item
  Дана квадратна матриця порядку 2n. Повернути її на 180 градусів в
  позитивному напрямку.
\item
  Заповнити двовимірний квадратний масив цілими числами від 1 до 100 по
  спіралі, як показано на наступному малюнку.
\item
  Дана матриця розміру n x m. Поміняти місцями стовпці, що містять
  мінімальний і максимальний елементи матриці.
\item
  Дано дві матриці n x m і m x k. Отримайте їх добуток.
\item
  Дана матриця розміру n х m. Поміняти місцями її рядки так, щоб їх
  максимальні елементи утворювали зростаючу послідовність.
\item
  У даній дійсної квадратної матриці порядку n знайти найбільший по
  модулю елемент.
\item
  Отримати квадратну матрицю порядку n - 1 шляхом викидання з вихідної
  матриці будь-якого рядка і стовпця, на перетині яких розташований
  елемент зі знайденим значенням. Виконуйте до тих пір, поки не
  залишиться останній елемент.
\item
  Дана квадратна матриця порядку 2n + 1. Дзеркально відобразити її
  елементи відносно побічної діагоналі матриці.
\item
  Дана дійсна квадратна матриця порядку 2n + 1. Отримати нову матрицю,
  повернувши її блоки, обмежені діагоналями, на 180 градусів.
\item
  Дана матриця розміру n x m. Поміняти місцями її перший і останній
  рядки, що містять тільки негативні елементи.
\item
  Дана цілочисельна матриця розміру n x m. Знайти елемент, який є
  максимальним у своєму рядку і мінімальним в своєму стовпці. Якщо такий
  елемент відсутній, то вивести 0.
\item
  Складіть програму циклічної перестановки стовпців двовимірного масиву
  m x k, при якій зсуві зсувається вправо на n стовпців.
\item
  Дана матриця розміру n x m. Поміняти місцями її стовпці так, щоб їх
  мінімальні елементи утворювали зростаючу послідовність.
\item
  Дана квадратна матриця порядку 2n + 1. Дзеркально відобразити її
  елементи відносно вертикальної осі симетрії матриці.
\item
  Дана квадратна матриця порядку 2n. Повернути її на 270 градусів в
  позитивному напрямку щодо її центру.
\item
  Дана матриця розміру n x m. Поміняти місцями рядки, що містять
  мінімальний і максимальний елементи матриці.
\item
  У квадратній таблиці обміняйте місцями елементи рядка і стовпця, на
  перетині яких знаходиться мінімальний з позитивних елементів.
\item
  Дана квадратна матриця порядку 2n. Повернути її на 90 градусів в
  позитивному напрямку щодо її центру.
\item
  Дана квадратна матриця порядку 2n + 1. Дзеркально відобразити її
  елементи відносно головної діагоналі матриці.
\item
  Складіть програму циклічної перестановки рядків двовимірного масиву m
  x k, при якій зсув відбувається вниз на n рядків.
\item
  Дана матриця розміру n x m. Поміняти місцями її перший і останній
  стовпці, що містять тільки позитивні елементи.
\item
  Заповнити двовимірний квадратний масив цілими числами від 1 до 100 по
  спіралі, починаючи від центру і закручуючи за годинниковою стрілкою.
\item
  Заповніть квадратну матрицю n x n за принципом латинського квадрата: в
  кожному рядку і кожному стовпці використовуються лише числа від 1 до n
  що не повторюються між собою.
\item
  Дана матриця дійсних коефіцієнтів. Впорядкувати її рядки по неспаданню
  перших елементів, суми значень рядків, величині найменших елементів
  рядків.
\end{enumerate}

Додаткові задачі:

\begin{enumerate}
\def\labelenumi{\arabic{enumi})}
\item
  Дана матриця n*m з нулів та одиниць. Знайти найбільший за площиною
  прямокутник з одних одиниць. Зробіть цю задачу без вкладених 4-х
  циклів для 0\textless{}n,m\textless{}255.
\end{enumerate}

8. Динамічні масиви. Робота з вказівниками

\begin{itemize}
\item
  Як можна створити лінійний динамічний масив та коректно завершити при
  цьому програму?
\item
  Що таке вказівники? Які операції визначені на вказівниках? Як
  проітеруватись по даному масиву за допомогою вказівника?
\item
  Як визначити динамічну матрицю за допомогою масиву вказівників та
  коректно її обробити?
\item
  Які функції та з якої бібліотеки використовуються на Сі для виділення
  памяті? В чому їх різниця? Що відбудеться якщо потрібної памяті не
  було ними виділено?
\item
  Які функції існують для очищення памяті? Що відбудеться, якщо їх не
  використовувати? Які ще проблеми виникають при некоректному очищенні
  чи його відсутності?
\end{itemize}

Задачі для аудіторної роботи

\begin{enumerate}
\def\labelenumi{\arabic{enumi})}
\item
  Ввести натуральне число n. Створити масив з n дійсних чисел та
  підрахувати суму квадратів елементів цього масиву. Підрахувати
  кількість ступенів двійки та трійки в цьому масиві.
\end{enumerate}

\begin{enumerate}
\def\labelenumi{\arabic{enumi})}
\item
  Написати функцію, що вводить масив цілих чисел доки не введеться нуль
  через змінний аргумент та кількість елементів масиву повертається як
  результат роботи функції. Кількість елементів обмежена числом 100.
  Підрахувати кількість повних квадратів та кубів в цьому масиві.
\item
  Створити функцію, що вводить n-вимірний вектор, виділяючи відповідну
  пам'ять та функцію, що відповідно очищує пам'ять. Напишіть програму,
  що вводить два вектори, підраховує та створює як окремий масив їх
  векторний добуток, якщо це можливо, та в будь-якому варіанті коректно
  завершує програму без витоків пам'яті.
\item
  Створити функцію, що вводить дійсну квадратну n-вимірну матрицю (n
  задається як аргумент функції), виділяючи відповідну пам'ять та
  функцію, що відповідно очищує пам'ять. Напишіть програму, що вводить
  дві матриці, підраховує та обчислює як окремий масив їх добуток, якщо
  це можливо, та в будь-якому варіанті коректно завершує програму без
  витоків пам'яті.
\end{enumerate}

Задачі для самостійної роботи

\begin{enumerate}
\def\labelenumi{\arabic{enumi})}
\item
  Створити функцію, що вводить матрицю цілих чисел довільних
  розмірностей, виділяючи відповідну пам'ять (розміри масивів) та
  функцію, що відповідно очищує пам'ять. Напишіть функцію, що підраховує
  ранг матриці. Коректно протестуйте роботу цих функцій.
\item
  Створити функцію, що вводить матриці довільних розмірностей, виділяючи
  відповідну пам'ять та функцію, що відповідно очищує пам'ять. Напишіть
  програму, що вводить масив таких матриць, підраховує та створює як
  окремий масив добуток всього масиву матриць, якщо це можливо, та в
  будь-якому варіанті коректно завершує програму без витоків пам'яті.
\item
  Вирішіть завдання виконуючи наступні вимоги:
\end{enumerate}

Сформувати динамічний двовимірний масив, заповнити його випадковими
числами і вивести на екран.

\begin{enumerate}
\def\labelenumi{\alph{enumi})}
\item
  Додати рядок із заданим номером.
\item
  Додати стовпець із заданим номером.
\item
  Додати рядок в кінець матриці.
\item
  Додати стовпець в кінець матриці.
\item
  Додати рядок в початок матриці.
\item
  Додати стовпець в початок матриці.
\item
  Додати К рядків в кінець матриці.
\item
  Додати К стовпців в кінець матриці.
\item
  Додати К рядків в початок матриці.
\item
  Додати К стовпців в початок матриці.
\item
  Видалити рядок з номером К.
\item
  Видалити стовпець з номером К.
\item
  Видалити рядки, починаючи з рядка К1 і до рядка К2.
\item
  Видалити стовпці, починаючи з стовпця К1 і до стовпчика К2.
\item
  Видалити всі парні рядки.
\item
  Видалити всі парні стовпці.
\item
  Видалити всі рядки, в яких є хоча б один нульовий елемент.
\item
  Видалити всі стовпці, в яких є хоча б один нульовий елемент.
\item
  Видалити рядок, в якій знаходиться найбільший елемент матриці.
\item
  Додати рядки після кожної парної рядки матриці.
\item
  Додати стовпці після кожного парного стовпця матриці.
\item
  Додати К рядків, починаючи з рядка з номером N.
\item
  Додати К стовпців, починаючи зі стовпчика з номером N.
\item
  Додати рядок після рядка, що містить найбільший елемент.
\item
  Додати стовпець після стовпця, що містить найбільший елемент.
\item
  Додати рядок після рядка, що містить найменший елемент.
\item
  Додати стовпець після стовпця, що містить найменший елемент.
\item
  Видалити рядок і стовпець, на перетині яких знаходиться найбільший
  елемент масиву.
\end{enumerate}

Додаткові задачі

\begin{enumerate}
\def\labelenumi{\arabic{enumi})}
\item
  Користувачу надається можливість декілька разів вводити розмірність
  вектору дійсних чисел та самі ці значення. Після кожного вводу
  потрібно підрахувати середнє арифметичне та дисперсію всіх введених
  значень.
\item
  Петя та Вася кожен день на протязі
  \protect\hypertarget{__DdeLink__55546_11145444801}{}{}N днів вимірюють
  декілька (від 0 до 1000) разів температуру повітря (хоча інколи хтось
  може забути це зробити). Створіть програму, що дозволить їм ввести ці
  результати за кожен день спостережень та підрахує середню температуру
  кожного з цих днів, де сумарна кількість вимірювань була більше 1.
  Програма повинна передбачити, що після вводу цих N днів вони можуть
  захотіти ввести наступні M днів таки спостережень. Передбачте
  можливість коректного завершення при нестачі ресурсів ПК для
  зберігання та обробки даних.
\item
  * В масиві натуральних чисел A{[}N{]} всі числа є меншими 16. Напишить
  функцію, що зберігає дані цього масиву у масиві N/2 чисел типу
  uint8\_t (тобто в кожному числі uint8\_t зберігається два числа масиву
  A{[}i{]}).
\item
  *В масиві натуральних чисел A{[}N{]} всі числа є меншими 64. Напишить
  функцію, що зберігає дані цього масиву у масиві {[}N*4/3{]} чисел типу
  uint8\_t (тобто в кожних трьох числах uint8\_t зберігається чотири
  числа масиву A{[}i{]}).
\item
  **В масиві натуральних чисел A{[}N{]} всі числа є меншими \(2^{k}\).
  Знайдіть це число k та напишить функцію, що зберігає цей масив в N*k
  біт найбільш економічним чином( int A{[}3{]}, k=5 → uint8 B{[}2{]}
  ,тобто використовує 16 біт, або int A{[}8{]}, k=14 → uint16 B{[}7{]} ,
  тобто використовує 112 біт) та функцію що обратно повертає числа з
  масиву B у масив A.
\end{enumerate}

9. Робота з рядком, що закінчується нулем на С.

Які є символьні типи в Сі/Сі++? Як їх коректно ввести/вивести на Сі? Які
є функції для роботи з символьним типом?

Як ініціалізувати рядок на Сі? Як ввести/вивести рядок?

Як порівняти два рядки? Як конкатенувати два рядки?

Як з'ясувати, що даний рядок є словом? Натуральним числом?

Як ввести речення та підрахувати кількість слів у ньому?

Які варіанти є для переведення рядку в ціле число? Дійсне число? Як
обробити помилку цих приведень?

Як перевести ціле число в рядок на Сі? Дійсне число?

Задачі для аудиторної роботи

1) Надрукувати заданий рядок:

а) виключивши з нього всі цифри і подвоївши знаки '+' та '-';

б) виключивши з нього всі знаки '+', безпосередньо за якими знаходиться
цифра;

в) виключивши з нього всі літери '\emph{в}', безпосередньо перед якими
знаходиться літера '\emph{с}';

г) замінивши в ньому всі пари '\emph{ph}' на літеру '\emph{f}';

д) виключивши з нього всі зайві пропуски, тобто з кількох, що йдуть
підряд, залишити один.

2) Дано рядок, серед символів якого є принаймні одна кома, а може й
немає її. Знайти номер

а) першої по порядку коми;

б) останньої по порядку коми;

в) кількості ком.

3) Виключити з заданого рядка групи символів, які знаходяться між '(' та
')'. Самі дужки теж мають бути виключені. Перевірте перед цим, що дужки
розставлено правильно (парами) та всередині кожної пари дужок немає
інших дужок.

4) Задана послідовність символів, яка має вигляд:

\emph{d\textsubscript{1}} ± \emph{d\textsubscript{2}} ± \emph{...} ±
\emph{d\textsubscript{n }}(\emph{d\textsubscript{i }}-- натуральні
числа, \emph{n}\textgreater{}1), за якою знаходиться знак рівності.
Перевірити, що рядок задовольняє вказаний вигляд та обчислити значення
цієї алгебраїчної суми.

5) Задане натуральне число \emph{n}. Надрукувати в заданій системі
числення b цілі числа від 0 до \emph{n}.

Задачі для самостійної роботи

\begin{enumerate}
\def\labelenumi{\arabic{enumi})}
\setcounter{enumi}{5}
\item
  Заданий рядок, серед символів якого міститься двокрапка ':'. Отримати
  всі символи, розміщені
\end{enumerate}

а) до першої двокрапки включно;

б) після першої двокрапки;

в) між першою і другою двокрапкою. Якщо другої двокрапки немає, то
отримати всі символи, розміщені після єдиної двокрапки.

\begin{enumerate}
\def\labelenumi{\arabic{enumi})}
\setcounter{enumi}{5}
\item
  Заданий текст надрукувати по рядках, розуміючи під рядком або наступні
  6 символів, якщо серед них немає коми (знак оклику, питання), або
  частину тексту до коми включно.
\item
  В заданий рядок входять тільки цифри та літери. Визначити, чи
  задовольняє він наступній властивості:
\end{enumerate}

а) рядок є десятковим записом числа, кратного 9 (6, 4);

б) рядок починається з деякої ненульової цифри, за якою знаходяться
тільки літери і їх кількість дорівнює числовому значенню цієї цифри;

в) рядок містить (крім літер) тільки одну цифру, причому її числове
значення дорівнює довжині рядка;

г) сума числових значень цифр, які входять в рядок, дорівнює довжині
рядка;

д) рядок співпадає з початковим (кінцевим, будь-яким) відрізком ряду
0123456789;

е) рядок складається тільки з цифр, причому їх числові значення
складають арифметичну прогресію (наприклад, 3 5 7 9, 8 5 2, 2).

\begin{enumerate}
\def\labelenumi{\arabic{enumi})}
\item
  Знайти у даному рядку символ та довжину найдовшої послідовності
  однакових символів, що йдуть підряд.
\item
  Скласти програму підрахунку загального числа входжень символів '+',
  '-', '*' у рядок \emph{А}.
\item
  Скласти програму перетворення рядка \emph{А}, замінивши у ньому всі
  знаки оклику '!' крапками '.', кожну крапку -- трьома крапками '...',
  кожну зірочку '*'- знаком '+'.
\item
  Рядок називається симетричним, якщо його символи, рівновіддалені від
  початку та кінця рядка, співпадають. Порожній рядок вважається
  симетричним. Перевірити рядок \emph{A} на симетричність.
\item
  Скласти програму видалення із рядка \emph{А} всіх входжень заданої
  групи символів.
\item
  Скласти програму перетворення слова \emph{А}, видаливши у ньому кожний
  символ '*' та подвоївши кожний символ, відмінний від '*'.
\item
  Скласти функцію підрахунку найбільшої кількості цифр, що йдуть підряд
  у рядку \emph{А}.
\item
  Скласти функція підрахунку числа входжень у рядок \emph{А} заданої
  послідовності літер.
\item
  Скласти функцію, яка за рядком \emph{А} та символом \emph{S} будує
  новий рядок, отриманий заміною кожного символу, наступного за
  \emph{S}, заданим символом \emph{С}.
\item
  Cкласти функцію перетворення рядка \emph{А} видаленням із нього всіх
  ком, які передують першій крапці, та заміною у ньому знаком '+' усіх
  цифр '3', які зустрічаються після першої крапки.
\item
  Cкласти функцію виведення на друк усіх цифр, які входять в заданий
  рядок, та окремо решти символів, зберігаючи при цьому взаємне
  розташування символів у кожній з цих двох груп.
\item
  Рядок називається монотонним, якщо він складається з зростаючої або
  спадної послідовності символів. Cкласти програму перевірки
  монотонності рядка.
\item
  Перевірити, чи складається рядок з
\end{enumerate}

а) 2 симетричних підрядків;

б) n симетричних підрядків.

\begin{enumerate}
\def\labelenumi{\arabic{enumi})}
\item
  Знайти символ, кількість входжень якого у рядок \emph{A}
\end{enumerate}

а) максимальна;

б) мінімальна.

\begin{enumerate}
\def\labelenumi{\arabic{enumi})}
\item
  Дано рядок \emph{A}, що містить послідовність слів. Скласти функції,
  що визначають:
\end{enumerate}

а) кількість усіх слів;

б) кількість слів, що починаються із заданого символу \emph{c};

в) кількість слів, що закінчуються заданим символом \emph{c};

г) кількість слів, що починаються й закінчуються заданим символом
\emph{c};

ґ) кількість слів, що починаються й закінчуються однаковим символом.

\begin{enumerate}
\def\labelenumi{\arabic{enumi})}
\item
  Виділити з рядка \emph{A} найбільший підрядок, перший і останній
  символи якого співпадають.
\item
  Виділити з рядка найбільший монотонний підрядок, коди послідовних
  символів якого відрізняються на 1.
\item
  Замінити всі пари однакових символів рядка, які йдуть підряд, одним
  символом. Наприклад, рядок \emph{`aabcbb'} перетворюється у
  \emph{`abcb'}.
\item
  Побудувати рядок \emph{S} з рядків \emph{S1}, \emph{S2} так, щоб у
  \emph{S} входили
\end{enumerate}

а) ті символи \emph{S1}, які не входять у S2;

б) всі символи \emph{S1}, які не входять у \emph{S2}, та всі символи
\emph{S2}, які не входять у \emph{S1}.

28) Видалити з рядка симетричні початок та кінець. Наприклад, рядок
\emph{`abcdefba'} перетворюється у \emph{`cdef'}. Написати програму, яка
виконує зсув по ключу (ключ задається) тільки для малих латинських
літер. Наприклад: вхідні дані anz -- рядок, 2 -- ключ. Результат: cpb.

Додаткові задачі:

\begin{enumerate}
\def\labelenumi{\arabic{enumi})}
\setcounter{enumi}{28}
\item
  Встановити, чи задовольняє заданий рядок заданому шаблону. Шаблон ---
  це рядок, що складається з символів а також наступних спецсимволів:
  символ «?» позначає будь-який символ, «*» означає будь-яку
  послідовність символів, у тому числі порожню, а «+» будь-яку непорожню
  послідовність символів (приклад, «ab*ra??da+ra»).
\end{enumerate}

30) Хеш даного рядку (довжина рядку більше одиниці) обчислюється так:

а) Кожні послідовні 4 байти конкатинуються щоб утворити натуральне
число. Якщо кількість символів не кратна 4, то до рядка дописуються
потрібна кількість символів, що взята з кінця рядку справа наліво
(зеркальний падінг). Всі ці числа додаються за допомогою ``виключного
або'' (xor).

б) Кожні послідовні 4 байти конкатинуються щоб утворити натуральне
число. Якщо кількість символів не кратна 4, то до рядка дописуються
потрібна кількість нулевих символів (нульовий падінг). До всіх цих чисел
додається за допомогою ``виключного або'' номер по порядку цього числа.
Потім всі ці числа додаються за допомогою ``виключного або''.

в) Береться просте число p. Кожен послідовні байт множиться на
p\textsuperscript{i}, де I -- номер по порядку цього числа та береться
остача від ділення на 2\textsuperscript{32}. Потім всі ці числа
додаються по модулю 2\textsuperscript{32}.

10. Структури. Створення власного типу

1) Що таке структура та як її створити на Сі?

2) Як створити власний тип даних на Сі?

3) Як визначити структуру що має посилання на саму себе?

4) Які варіанти ініціалізації структур? Як ввести структуру? Як отримати
структуру як результат роботи функції? Через змінний аргумент?

5) Нащо використовувати typedef при створенні власної структури?

Задачі для аудиторної роботи

\begin{enumerate}
\def\labelenumi{\arabic{enumi})}
\item
  \protect\hypertarget{_Hlk65237419}{}{}Визначити типи структури для
  зображення наступних понять та функції їх вводу-виводу:
\end{enumerate}

а) дата (число, місяць, рік);

б) поле шахової дошки (напр., а5, b8);

в) \protect\hypertarget{_Hlk65237388}{}{}прямокутник зі сторонами,
паралельними осям координат (Точка А, Точка Б). Точка --- дві дійсні
координати

2) Використовуючи тип Поле шахової дошки описати булеву функцію, яка
перевіряє, чи може ферзь за один хід перейти з одного заданого поля
шахової дошки на інше задане поле.

3) Визначимо тип Rational (Раціональне число) як:

typedef struct \{

int numerator; // чисельник

unsigned int denominator; // знаменник

\} Rational;

Визначити функції для:

а) обчислення суми двох раціональних чисел;

б) обчислення добутку двох раціональних чисел;

в) порівняння двох раціональних чисел;

г) зведення раціонального числа до нескоротного виду.

4) Використовуючи опис типу Дата, визначити функції обчислення:

а) дати вчорашнього дня;

б) дня тижня за його датою в поточному році.

\protect\hypertarget{_Hlk65237504}{}{}5) Задано масив розмірності N,
компонентами якого є структури, що містять відомості про вершини гір. У
відомостях про кожну вершину вказуються назва гори та її висота.
Визначити функції введення/виведення гір та функції пошуку назви
найвищої вершини та виведення висоти вершини з заданою назвою (якщо
вершини з такою назвою немає в масиви --- вивести відповідне
повідомлення).

Задачі для самостійної роботи

6) Визначити типи запису для зображення наступних понять та реалізуйте
їх функції введення виведення:

а) ціна (гривні, копійки);

б) час (година, хвилина, секунда);

в) повна дата (число, місяць, рік, година, хвилина);

г) адреса (місто, вулиця, будинок, квартира);

д) семінар (предмет, викладач, № групи, день тижня, години занять,

аудиторія);

е) бланк вимоги на книгу (відомості про книгу: шифр, автор, назва;

відомості про читача: № читацького квитка, прізвище; дата замовлення);

є) коло (радіус, координати центра).

ж) \protect\hypertarget{_Hlk65237605}{}{}сфера в просторі

з) прямокутний паралеліпипед (сторони якого паралельні осям координат)

і) поліном довільного ступеня (дійсні коефіцієнти --- безрозмірний
масив).

\begin{enumerate}
\def\labelenumi{\arabic{enumi})}
\setcounter{enumi}{6}
\item
  В масиві структур записано вартість та вік кожної з N моделей легкових
  автомобілів. Визначити середню вартість автомобілів, вік яких більший
  за 5 років.
\item
  В масиві структур записано інформацію про ціну та наклад кожного з N
  журналів. Знайти середню вартість журналів, наклад яких менший за
  10000 примірників.
\item
  В масиві структур записано дані про масу й об'єм N предметів,
  виготовлених із різних матеріалів. Знайти предмет, густина матеріалу
  якого найбільша.
\item
  В масиві структур записано дані про чисельність населення (у мільйонах
  жителів) та площі N держав. Знайти країну з мінімальною щільністю
  населення.
\item
  Задано масив С розмірності N, компонентами якого є відомості про
  мешканців деяких міст. Інформація про кожного мешканця містить його
  прізвище, назву міста, місцеву адресу у вигляді вулиці, будинку,
  квартири. Визначити функцію пошуку двох будь-яких жителів, що мешкають
  у різних містах за однаковою адресою.
\item
  Відомо дані про вартість кожного з N найменувань товарів: кількість
  гривень, кількість копійок. Скласти підпрограми пошуку:
\end{enumerate}

а) найдешевшого товару в магазині;

б) найдорожчого товару в магазині;

в) товару, вартість якого відрізняється від середньої вартості товару

в магазині не більш ніж на 5 гривень:

\begin{enumerate}
\def\labelenumi{\arabic{enumi})}
\setcounter{enumi}{6}
\item
  Задано масив Р розмірності N, компонентами якого є стурктури, що
  містять анкети службовців деякого закладу. У кожній анкеті вказуться
  прізвище та ім'я службовця, його стать, дата народження у вигляі
  числа, місяця, року. Визначити підпрограми пошуку:
\end{enumerate}

а) посади, яку обіймає найбільша кількість співробітників;

б) співробітників з однаковими іменами;

в) співробітників, прізвища яких починаються із заданої літери;

г) найстаршого з чоловіків цього закладу;

ґ) співробітників, вік яких менший за середній по організації;

д) пенсійного віку (урахувати, що пенсійний вік чоловіків і жінок --
різний).

\begin{enumerate}
\def\labelenumi{\arabic{enumi})}
\setcounter{enumi}{6}
\item
  Задано масив Р, компонентами якого Рi є записи, що містять дані про
  людину на ім'я i з указаного списку. Кожне дане складається зі статі
  людини та її зросту. Визначити підпрограми для:
\end{enumerate}

а) обчислення середнього зросту жінок;

б) пошуку найвищого чоловіка;

в) перевірки, чи є дві людини, однакові на зріст.

\begin{enumerate}
\def\labelenumi{\arabic{enumi})}
\setcounter{enumi}{6}
\item
  Задано масив розмірності N, компоненти якого містять інформацію про
  студентів деякого вишу. Відомості про кожного студента містять дані
  про його прізвище, ім'я, по батькові, стать, вік, курс. Визначити
  процедуру пошуку:
\end{enumerate}

а) найпоширеніших чоловічих і жіночих імен;

б) прізвищ та ініціалів усіх студентів, вік яких є найпоширенішим.

\begin{enumerate}
\def\labelenumi{\arabic{enumi})}
\setcounter{enumi}{6}
\item
  Задано масив розмірності N, компонентами якого є відомості про
  складання іспитів студентами деякого вишу. Інформація про кожного
  студента задана в такому вигляді: прізвище, номер групи, оцінка\_1,
  оцінка\_2, оцінка\_3. Визначити процедуру пошуку:
\end{enumerate}

а) студентів, що мають заборгованості принаймні з одного з предметів;

б) предмета, складеного найуспішніше;

в) студентів, що склали всі іспити на 5 і 4.

Додаткові задачі:

\begin{enumerate}
\def\labelenumi{\arabic{enumi})}
\setcounter{enumi}{6}
\item
  Визначити універсальний тип, який допускає зображення точки на площині
  у прямокутній або полярній системі координат (3-тє поле -- тип
  координат). Побудувати функцію обчислення площі трикутника з вершинами
  A, B, C.
\end{enumerate}

11. Робота з бінарними файлами на Сі

\begin{itemize}
\item
  Цикл роботи з файлами на Сі/Сі++.
\item
  Як створити та працювати з текстовим файлом на Сі? Як можна вводити та
  виводити файл посимвольно? Порядково?
\item
  Як створити та працювати з бінарним файлом?
\item
  Як записати та прочитати масив дійсних чисел в/з бінарного файлу?
\item
  Як прочитати всі цілі числа з файлу, якщо на початку роботи невідомо,
  скільки їх там насправді?
\item
  Які додаткові речі можна робити з бінарним файлом, що неможна робити з
  текстовим?
\item
  Як записати масив структур у файл та прочитати k-тий запис у файлі?
\end{itemize}

Задачі для аудиторної роботи

\begin{enumerate}
\def\labelenumi{\arabic{enumi})}
\item
  Дано файл F, компоненти якого є цілими числами. Побудувати файл G,
  який містив би всі компоненти файлу F:
\end{enumerate}

а) що є парними числами; б) що діляться на 3 і на 5;

в) що є точними квадратами; г) записані у зворотному порядку;

ґ) за винятком повторних входжень одного й того самого числа.

\begin{enumerate}
\def\labelenumi{\arabic{enumi})}
\item
  Використовуючи файл F, компоненти якого є цілими числами, побудувати
  файл G, що містить усі парні числа файлу F, і файл H -- усі непарні.
  Послідовність чисел зберігається.
\item
  Дано файл, який містить відомості про прямокутники: вказано номер
  прямокутника у файлі, координати (дійсні числа) верхнього лівого кута,
  нижнього правого кута прямокутника. Скласти функцію пошуку
  прямокутника з найбільшою площею й визначення цієї площі.
\item
  Дано файл, компоненти якого є записи (koef, st) -- дійсний коефіцієнт
  і степінь членів полінома (koef ≠ 0). Визначити підпрограми для
  виконання таких дій над поліномом:
\end{enumerate}

а) введення полінома; б) друк полінома;

в) обчислення похідної від полінома;

г) додання поліному у файл заданий іменем;

д) отримання поліному з даного файлу;

е) заміна коефіцієнту заданого ступенем;

ж) заміна коефіцієнту заданого номером запису у файлі.

Задачі для самостійної роботи

\begin{enumerate}
\def\labelenumi{\arabic{enumi})}
\item
  Дано файл, компоненти якого є натуральними числами. Скласти
  підпрограми для обчислення:
\end{enumerate}

а) кількості парних чисел серед компонент;

б) кількості квадратів непарних чисел серед компонент;

в) різниці між найбільшим парним і найменшим непарним числами компонент;

г) кількості компонент у найдовшій зростаючій послідовності компонент
файлу.

\begin{enumerate}
\def\labelenumi{\arabic{enumi})}
\item
  Дано файл, компоненти якого є дійсними числами. Скласти підпрограми
  для обчислення:
\end{enumerate}

а) суми компонент файлу;

б) кількості від'ємних компонент файлу;

в) останньої компоненти файлу;

г) найбільшого зі значень компонент файлу;

ґ) найменшого зі значень компонент файлу з парними номерами;

д) суми найбільшого та найменшого зі компонент;

е) різниці першої й останньої компоненти файлу;

є) кількості компонент файлу, які менші за середнє арифметичне всіх його
компонент.

\begin{enumerate}
\def\labelenumi{\arabic{enumi})}
\item
  Дано файл F, компоненти якого є цілими числами. Написати фінкції, які
  створюють файл G, який містив би всі компоненти файлу F:
\end{enumerate}

а) що є непарними числами;

б) що є простими числами;

в) що є точними квадратами;

г) записані у зворотному порядку;

д) за винятком повторних входжень одного й того самого числа;

е) числами Фібоначчі.

\begin{enumerate}
\def\labelenumi{\arabic{enumi})}
\item
  Задано натуральне число n та файл F, компоненти якого є цілими
  числами. Побудувати файл G, записавши до нього найбільше значення
  перших n компонент файлу F, потім -- наступних n компонент тощо.
  Розглянути два випадки:
\end{enumerate}

а) кількість компонент файлу ділиться на n;

б) кількість компонент файлу не ділиться на n. Остання компонента файлу
g має дорівнювати найбільшій із компонент файлу F, які утворюють останню
(неповну) групу.

\begin{enumerate}
\def\labelenumi{\arabic{enumi})}
\item
  Дано файл F, компоненти якого є цілими числами. Файл містить рівне
  число додатних і від'ємних чисел -- перевірте це і в противному
  випадку видайтие відповідне повідомлення та не робить нічого.
  Використовуючи допоміжний файл H, переписати компоненти файлу F до
  файлу G так, щоб у файлі G:
\end{enumerate}

а) не було двох сусідніх чисел одного знаку;

б) спочатку йшли додатні, потім -- від'ємні числа;

в) числа йшли таким чином: два додатних, два від'ємних тощо. Якщо це
неможливо -- то переписати допоки можливо в такому вигляді, а останні 2
числа вивести на консоль.

\begin{enumerate}
\def\labelenumi{\arabic{enumi})}
\item
  Дано файл F, компонентами якого є записи (структури) вигляду
\end{enumerate}

struct T \{

unsigned Key; // ключ

char Data{[}10{]}; // дані

\};

Такий файл називатимемо впорядкованим за ключами, якщо записи в ньому
розташовуються в порядку зростання (спадання) ключів. Скласти процедуру
пошуку запису за ключем у впорядкованому файлі. Скласти процедуру
вилучення запису із заданим ключем:

а) з впорядкованого файлу;

б) з невпорядкованого файлу.

\begin{enumerate}
\def\labelenumi{\arabic{enumi})}
\item
  Багаж пасажира характеризується номером пасажира, кількістю речей і
  їхньою загальною вагою. Дано файл пасажирів, який містить прізвища
  пасажирів, і файл, що містить інформацію про багаж декілька пасажирів
  (номер пасажира -- це номер запису у файлі пасажирів)
\item
  Скласти процедури для:
\end{enumerate}

а) знаходження пасажира, у багажі якого середня вага однієї речі
відрізняється не більш ніж на 1 кг від загальної середньої ваги речей;

б) визначення пасажирів, які мають більше двох речей, і пасажирів
кількість речей у яких більша за середню кількість речей;

в) видачі відомостей про пасажира, кількість речей у багажі якого не
менша, ніж у будь-якому іншому багажі, а вага речей -- не більша, ніж у
будь-якому іншому багажі із цією самою кількістю речей;

г) визначення, чи мають принаймні два пасажири багажі, які не
відрізняються за кількістю речей і відрізняються вагою не більш ніж на 1
кг (якщо такі пасажири є, то показати їхні прізвища);

ґ) визначення пасажира, багаж якого складається з однієї речі вагою не
менше 30 кг.

\begin{enumerate}
\def\labelenumi{\arabic{enumi})}
\item
  Дано файл, який містить відомості про іграшки: указано назву іграшки
  (напр., м'яч, лялька, конструктор тощо), її вартість у гривнях і
  вікові межі для дітей, яким іграшка призначається (напр., для дітей
  від двох до п'яти років). Скласти процедури:
\end{enumerate}

а) пошуку назв іграшок, вартість яких не перевищує 40 грн, призначених
дітям п'яти років;

б) пошуку назв іграшок, призначені дітям і чотирьох, і десяти років;

в) пошуку назв найдорожчих іграшок (ціна яких відрізняється від ціни
найдорожчої іграшки не більш ніж на 50 грн);

г) визначення ціни найдорожчого конструктора;

ґ) визначення ціни всіх кубиків;

д) пошуку двох іграшок, що призначені дітям трьох років, сумарна
вартість яких не перевищує 20 грн;

е) пошуку конструктора ціною 22 грн, призначеного дітям від п'яти до
десяти років. Якщо такої іграшки немає, то занести відомості про її
відсутність до файлу.

\begin{enumerate}
\def\labelenumi{\arabic{enumi})}
\item
  Дано файл, який містить відомості про прямокутники: указано номер
  прямокутника у файлі, координати верхнього лівого кута, нижнього
  правого кута прямокутника. Скласти процедуру пошуку прямокутника з
  найбільшою площею й визначення цієї площі.
\end{enumerate}

Додаткові задачі:

\begin{enumerate}
\def\labelenumi{\arabic{enumi})}
\item
  У двох файлах міститься таблиця футбольного турніру, у першому --
  записано назви команд; у другому -- результати матчів, що зберігаються
  у записах типу T\_Match
\end{enumerate}

typedef struct \{

unsigned int n1, n2;

unsigned int b1, b2;

\} T\_Match;

Тут у структурі типу T\_Match поля n1, n2 -- номери першої і другої

команд (тобто номери назв команд у файлі команд); b1, b2 -- кількість

м'ячів, забитих першою та другою командами, відповідно.

Кожній команді за перемогу нараховується 3 очки, за нічию -- 1, за

поразку -- 0.

Із двох команд, які мають однакову кількість очок, першою вважається та,
що має кращу різницю забитих і пропущених м'ячів;

за однакової різниці має більше забитих м'ячів;

за всіма однаковими попередніми показниками визначається жеребкуванням
(для жеребкування використати генератор випадкових чисел).

Знайти команду, яка є лідером.

Вказівка. Описати підпрограми створення файлів команд і матчів,

додавання результату матчу, визначення лідера.

11) Файл бази даних з малюнками містить на початку ціле 32-бітне число
2051, потім ціле 32-бітне число -- кількість малюнків, а наступні два
32-бітних числа -- кількість пікселів висоту та ширину кожного малюнку у
пікселах. При цьому ці числа задані в форматі high-indian (MSB first).
Наступний вміст файлу -- беззнакові натуральні байти (K*n*m байтів),
кожен з яких -- значення яскравостей пікселів (число від 0 до 255)
кожного з цих малюнків, що проходяться у порядку зліва-направо та
зверху-вниз.

Напишіть функцію, що перевіряє даний файл (заданий ім'ям) на
відповідність даному формату, та виводить масив яскравостей малюнка з
заданим номером, якщо такий номер та сам файл коректно задані. В
противному випадку вивести змістовне повідомлення про помилку.

12. Введення/виведення на С++. Робота з текстовими файлами на Сі++

\begin{enumerate}
\def\labelenumi{\arabic{enumi})}
\item
  Як використовувати бібліотеки Сі на Сі++? Що потрібно для того щоб код
  на Сі працював так само на Сі++?
\item
  Яка різниця булевого типу та його використання на Сі та Сі++?
\item
  Як вивести в Сі++ використовуючи потоки виведення дійсне число з
  заданою точністю? В науковому представленні? З заданою шириною?
\item
  Як записати у текстовий файл масив цілих чисел через кому у якості
  роздільника та прочитати потім цей масив?
\item
  Що таке перевантаження функцій та навіщо воно може бути потрібно?
\item
  Що таке new та new{[}{]}? Коли потрібно перше та коли друге?
\item
  В чому різниця між new та malloc?
\item
  Як очищувати пам'ять після new та new{[}{]}?
\end{enumerate}

Задачі для аудиторної роботи

\begin{enumerate}
\def\labelenumi{\arabic{enumi})}
\item
  Ввести в двох різних рядках послідовно два дійсних числа x та y та
  обчислити значення x в ступені y. Результат вивести в десятковому та
  науковому представленні.
\item
  На терміналі вводяться 10*n цифр. Перші 10 цифр -- це перше натуральне
  число, наступні 10 -- друге і так далі. Введіть всі ці числа в масив
  розміру n та обчисліть і виведіть їх суму (вважайте що сума влазить в
  точність unsigned long long ).
\item
  Вивести на екран таблицю, слідкуючи, щоб виведення було рівним та
  кількість цифр після коми була або 0 або 2:
\end{enumerate}

+++++++++++++++ +++++++++++

+число + 1 + 2 + 3 + 4 + 5

++++++++++++++++++++++++++++

+експонента+ 1 +1.44 + 1.69 + 2

++++++++++++++++ ++++++++++

\begin{enumerate}
\def\labelenumi{\arabic{enumi})}
\item
  Ввести з текстового файлу та з консолі натуральне число n та масиви з
  n цілих чисел \(\left\{ m_{i} \right\}_{i = 1}^{n}\) та дійсних чисел
  \(\left\{ x_{i} \right\}_{i = 1}^{n}\). Обчислить та виведіть у файл
  числа \(\left\{ x_{i}^{m_{i}} \right\}_{i = 1}^{n}\).
\item
  Вхідний потік заданий текстовим файлом містить набір цілих чисел Ai (0
  ≤ Ai ≤ 1018), відділений один від іншого довільною кількістю пробілів
  і переводів рядків. Розмір вхідного потоку не перевищує 256 КБ. Для
  кожного числа Ai, починаючи з останнього та завершуючи першим, в
  окремому рядку вивести його квадратний корінь не менш ніж з чотирма
  знаками після десяткової крапки.
\end{enumerate}

Приклад:

\textbf{Вхід:}

1427 0

876652098643267843

5276538

\textbf{Вихід: }

2297.0716

936297014.1164

0.0000

37.7757

Задачі для самостійної роботи

\begin{enumerate}
\def\labelenumi{\arabic{enumi})}
\item
  Ввести декілька (невідомо зазделегідь скільки) дійсних числа записаних
  через коми та обчислити значення функції log() для кожного з них. Якщо
  значення виходить за межі області вивести слово ``None'', для інших
  значень результат вивести в науковому та десятковому представленні
  шириною 5 символів.
\item
  Три додатніх дійсні числа вводяться як рядок вигляду
\item
  А=ххх.ххх, B=xxExxx C=xxx.xxxx
\item
  Обчисліть їх середнє гармонійне та виведіть у науковому та звичайному
  форматі.
\item
  Ввести дійсне число від -10000 до 10000 та вивести його k-ту ступінь
  (k\textless{}10) з точністю до 20 знаків до десяткової коми та 4
  значками після десяткової коми.
\item
  Організуйте роботу з текстовим файлом. Вихідні файли не передбачають
  зміни. Змінені дані збережіть в іншому файлі.
\item
  Дано два текстові файли з іменами Name1 і Name2. Додати в кінець
  кожного рядка файлу Name1 відповідний рядок файлу Name2. Якщо файл
  Name2 коротший файлу Name1, то виконайте перехід до початку файлу
  Name2.
\item
  Організувати текстовий файл, що складається з N рядків. Визначити
  максимальний і мінімальний розмір рядків в файлі і вивести їх в інший
  файл.
\item
  Дан текстовий файл з ім'ям NameT. Підрахувати число повторень в ньому
  малих латинських літер ('a' - 'z') і створити файл з ім'ям NameS,
  рядки якого мають вигляд: "\textless{}літера\textgreater{} -
  \textless{}число повторень даної літери\textgreater{}". Літери,
  відсутні в тексті, в файл не включати. Рядки впорядкувати за спаданням
  кількості повторень літер, а при однаковій кількості повторень - по
  зростанню кодів літер.
\item
  Дан символ с (прописна латинська літера) і текстовий файл. Створити
  текстовий файл, який містить всі слова з вихідного файлу, що
  починаються цією літерою (як великої, так і малої). Розділові знаки,
  розташовані на початках і в кінцях слів, не враховувати. Якщо вихідний
  файл не містить відповідних слів, залишити результуючий файл порожнім.
\end{enumerate}

\textbf{Організуйте роботу з текстовим файлом. Вхідний файл потрібно
змінити згідно вказаних умов, тобто вхідний та вихідні файли
співпадають.}

\begin{enumerate}
\def\labelenumi{\arabic{enumi})}
\item
  Дано число N і текстовий файл. Видалити з файлу рядки з номерами,
  кратними N. Порожні рядки не враховувати і не видаляти. Якщо рядки з
  необхідними номерами відсутня, то залишити файл без змін. Зміна
  вивести в другий файл.
\item
  Дан текстовий файл, що містить текст, вирівняний по лівому краю
  (довжина кожного рядка не перевищує 50 символів). Вирівняти його по
  центру, додавши в початок кожної непорожній рядки необхідну кількість
  прогалин. Рядки непарної довжини перед центруванням доповнювати зліва
  прогалиною. Вирівняний текст записати в інший файл.
\item
  Організувати текстовий файл, що складається з N рядків. Перетворити
  файл, видаливши в кожній його рядку зайві пробіли. Зміни вивести в
  другий файл.
\item
  Дан файл з текстом із символів латинського алфавіту. Зашифрувати файл,
  виконавши циклічний зсув кожної букви вперед на n позицій в алфавіті.
  Розділові знаки і пропуски не змінювати.
\item
  Дано числа N1, N2 і текстовий файл. Видалити з файлу рядки з номерами
  між N1, N2, не включаючи меж. Зміни вивести в другий файл. Якщо
  виконати видалення неможливо, видайте про це повідомлення на екран і в
  вихідний файл.
\item
  Дан файл з текстом із символів латинського алфавіту, цифр та знаків.
  Замініть всі цифри їх назвами на англійській мові.
\item
  Організувати текстовий файл f складається з N рядків. Після цього
  організувати файли h і g. У файл h записати рядки файлу f непарної
  довжини, в файл g парної довжини.
\end{enumerate}

29) Визначити функцію, яка:

а) підраховує кількість порожніх рядків;

б) обчислює максимальну довжину рядків текстового файлу.

30) Визначити процедуру виведення:

а) усіх рядків текстового файлу;

б) рядків, які містять більше 60 символів.

31) Визначити функцію, що визначає кількість рядків текстового файлу,
які:

а) починаються із заданого символу;

б) закінчуються заданим символом;

в) починаються й закінчуються одним і тим самим символом;

г) що складаються з однакових символів.

33) В даному текстовому файлі знаходиться англомовний текст. Вирівняйте
його по лівий та правий границі так щоб розподіл слів у рядках був
найбільш рівномірним.

35) Визначити процедуру, яка переписує до текстового файлу G усі

рядки текстового файлу F:

а) із заміною в них символу '0' на '1', і навпаки;

б) в інвертованому вигляді.

36) Визначити процедуру пошуку найдовшого рядка в текстовому

файлі. Якщо таких рядків кілька, знайти перший із них.

37) Визначити процедуру, яка переписує компоненти текстового

файлу F до файлу G, вставляючи до початку кожного рядка один сим-

вол пропуску. Порядок компонент не має змінюватися.

38) У текстовому файлі записано непорожню послідовність дійсних чисел,
які розділяються пропусками. Визначити функцію обчислення найбільшого з
цих чисел.

39) У текстовому файлі F записано послідовність цілих чисел, як

розділяються пропусками. Визначити процедуру запису до текстового

файлу g усіх додатних чисел із F.

40) У текстовому файлі кожний рядок містить кілька натуральних

чисел, які розділяються пропусками. Числа визначають вигляд геометричної
фігури (номер) та її розміри. Прийнято такі домовленості:

відрізок прямої задається координатами своїх кінців і має номер 1;

прямокутник задається координатами верхнього лівого й нижнього

правого кутів і має номер 2;

коло задається координатами центра й радіусом і має номер 3.

Визначити процедури обчислення:

а) відрізка з найбільшою довжиною;

б) прямокутника з найбільшим периметром;

в) кола з найменшою площею.

42) У файлі записані координати точок на площині задані парою цілих
чисел. Точки записуються в форматі : ( х1 , х2 ) (х1 , х2) , \ldots{} -
саме так через коми та дужки. Створити файл, в якому будуть записані
координати всіх відрізків з точок цього файлу, при цьому ці відрізки
відсортовані за зростанням довжини.

43) У файлі записані координати Точок в просторі задані трійкою цілих
чисел. Точки записуються в форматі : х1 , х2 , х3 ; х1 , х2, х3;
\ldots{} Знайти відрізок з точок цього файлу, що має найбільшу довжину.

44) У файлі записані координати матеріальних точок на площині задані
парою цілих чисел та масою(дійсне число). Точки записуються в форматі :
{[}х1 , y1, m1 {]}, {[}х2 , y2, m2{]} , \ldots{} \textbf{- саме так
через коми та дужки. Знайдіть дві точки} з найбільшим важілем сили (m*(х
+y)).

45) У файлі записані дати , що задані трійкою цілих чисел у форматі
(чч1./мм1/рр1),(чч2./мм2/рр2), \ldots{} - саме в такому форматі.
Створити файл, в якому будуть записано найстарша та найсвіжіша дати
(врахуйте, що роки дат з 1951 по 2049).

Додаткові задачі:

46) Розглянемо послідовність чисел \(a_{i}\) , i = 0, 1, 2, \ldots{}, що
задовольняють умовам:

\(a_{0} = 0\), \(a_{1} = 1\), \(a_{2i} = a_{i}\) a,
\(a_{2i + 1} = {2a}_{i} + 1\) для кожного i = 1, 2, 3, \ldots{} .

Напишіть програму, яка для заданого значення n знаходить максимальне
серед чисел \(a_{0},a_{1},\cdots,a_{n}\). Вхідні дані складаються з
декількох тестів (не більше 10). Кожен тест - рядок, в якому записано
ціле число n (1 ≤ n ≤ 99 999). В останньому рядку вхідних даних записано
число 0. Для кожного n у виводі запишіть максимальне значення.

\emph{47) Створити текстовий (.txt) файл з 100,000,000 рядків з числами
в діапазоні від 0 до 99,999,999:}

\emph{формат чисел - 8 нулів (1 = 00000001, 65535 = 00065535) , діапазон
від 0 до 99999999, всі числа розташовані в випадковому порядку без
повторів (кожен рядок -- унікальне число)}

\emph{Приклад.}

\emph{00306453 }

\emph{99645283 }

\emph{70000021 }

\emph{06847127 }

13. Робота з класом рядок на С++.

\begin{enumerate}
\def\labelenumi{\arabic{enumi})}
\item
  Які конструктори для класу рядок? Які копі-конструкторів? Скільки та
  які оператори є перевантаженими для класу рядок?
\item
  Як видалити підрядок використовуючи методи класу String?
\item
  Як можна проітеруватись по рядку Сі++?
\item
  Як узнати довжину рядка?
\item
  Як знайти перше входження даного підрядку в рядку? Останнє?
\item
  Як вивести всі слова в реченні, що розділено пробілами? Комами?
\end{enumerate}

Задачі для аудиторної роботи

В даній групі задач потрібно реалізувати функції та в тих функціях де
потрібно виводити рядок зробіть 2 варіанти: 1) Результат записати в
новий рядок. 2) Результат замінює рядок, що є аргументом функції.

\begin{enumerate}
\def\labelenumi{\arabic{enumi})}
\item
  Даний рядок, що складається з символів латинського алфавіту,
  розділених пробілами (одним або декількома). Визначити кількість слів,
  які починаються і закінчуються однією і тією ж літерою.
\item
  Даний рядок, що складається з символів латинського алфавіту,
  розділених пробілами (одним або декількома). Перетворити кожне слово в
  рядку, видаливши з нього всі входження останньої літери цього самого
  слова (кількість пропусків між словами не змінювати).
\item
  Даний рядок із символів латинського алфавіту. Перевірте правильність
  розстановки тега \textless{}td\textgreater{}: кожному відкритого тегу
  повинен відповідати закритий \textless{}/ td\textgreater{}.
\item
  Даний рядок -- речення з символів латинського алфавіту. Вивести
  найкоротше слово в реченні \protect\hypertarget{_Hlk65949285}{}{}(якщо
  таких слів декілька, то: а) вивести перше з них; б) останнє з них в)
  всі такі слова.
\item
  У реченні, що складається зі слів, відокремлених одним пропуском,
  замінити першу букву у слів, що настають за словами die, der, das, на
  прописну.
\item
  Напишіть функцію часткового спліттінгу рядку. Тобто функція, що
  приймає рядок та повертає перше слово з рядку (роздільник -- задається
  як аргумент функції)
\item
  Напишіть функцію, що приймає рядок та повертає масив (як
  аргумент-змінний) всі дійсні числа, що містяться в рядку (роздільник
  -- задається як аргумент функції)
\item
  У записці слова зашифровані - кожне з них записано навпаки.
  Розшифрувати повідомлення
\end{enumerate}

Задачі для самостійної роботи

\begin{enumerate}
\def\labelenumi{\arabic{enumi})}
\item
  Даний рядок, що складається з символів латинського алфавіту,
  розділених пробілами (одним або декількома). Вивести рядок, що містить
  ці ж слова, але розділені одним символом '.' (точка, крапка). В кінці
  крапку не ставити.
\item
  Даний рядок, що складається з символів латинського алфавіту,
  розділених пробілами (одним або декількома). Перетворити кожне слово в
  рядку, видаливши з нього всі входження останньої літери цього слова
  (кількість пропусків між словами не змінювати).
\item
  Речення складається з слів, розділених одним або декількома
  пропусками. Написати програму, що друкує все слова, що закінчуються на
  заданий символ.
\item
  Даний рядок, що складається з символів латинського алфавіту,
  розділених пробілами (одним або декількома). Перетворити кожне слово в
  рядку видаливши з нього всі входження заданого символу (кількість
  пропусків між словами не змінювати).
\item
  Даний рядок-речення з символів латинського алфавіту. Перетворити рядок
  так, щоб кожне слово починалося з великої літери.
\item
  Даний рядок-речення з символів латинського алфавіту. Вивести найдовше
  слово в реченні (якщо таких слів кілька, то вивести останнє з них).
\item
  Визначити, скільки разів в рядку зустрічається задане слово.
\item
  \protect\hypertarget{_Hlk65949361}{}{}У записці слова зашифровані -
  кожне з них записано навпаки. Розшифрувати повідомлення.
\item
  Даний рядок, що складається з символів латинського алфавіту,
  розділених пробілами (одним або декількома). Визначити кількість слів,
  які містять введений символ.
\item
  З'ясуйте, чи є серед введених символів всі букви, що входять в задане
  слово.
\item
  Речення складається з слів, розділених одним або декількома
  пропусками. Написати програму, що друкує все слова, що починаються на
  введений символ.
\item
  У англійському реченні слова розділені одним пропуском. У всіх словах,
  наступних за артиклями a, an та the, першу букву замінити на прописну.
  Написати програму, що виконує цю роботу.
\item
  Написати програму, що визначає, який відсоток слів в англійському
  тексті містить подвоєну приголосну.
\item
  У мові використовується латинський алфавіт, причастя завжди
  закінчується суфіксом "ings". Задана рядок слів, в якій слова
  відокремлюються одним або декількома пропусками. Надрукувати причастя
  з цього рядку.
\item
  Даний рядок з малих символів латинського алфавіту. Замініть кожен
  символ на наступний за ним за алфавітом, символ 'z' замініть на 'a'.
\item
  Даний рядок із символів латинського алфавіту. Замініть всі входження
  рядків ``one'', "two","three",\ldots{},''nine'' на символи `1',
  '2','3',\ldots{},'9'.
\item
  Відредагувати задане речення, видаляючи з нього ті слова, які
  зустрічаються в реченні задану кількість разів.
\item
  Визначте, який відсоток символи кожного слова складають з символів
  даного речення.
\item
  Дан текст, що складається з символів латинського алфавіту, пробілів і
  знаків пунктуації. Знайдіть найпоширенішу голосну букву (без
  урахування регістру).
\item
  Даний рядок. Групи символів, що відокремлені пропусками (одним або
  кількома) і не містять пропусків усередині, називатимемо словами.
  Скласти підпрограми для:
\end{enumerate}

а) знаходження найдовшого слова;

б) визначення кількості слів

в) вилучення з рядку зайвих пропусків і всіх слів, що складаються з
однієї літери;

г) вилучення всіх пропусків на початку рядків, у кінці рядків і між
словами (крім одного);

д) вставки пропусків до рядків рівномірно між словами так, щоб довжина
всіх рядків (якщо в них більше 1 слова) була 80 символів і кількість

пропусків між словами в одному рядку відрізнялась не більше ніж на 1

(вважати, що рядки файлу мають не більш ніж 80 символів).

21) В заданий рядок входять тільки цифри та літери. Перевірте це.
Визначити, чи задовольняє він наступній властивості:

а) рядок є десятковим записом числа, кратного 9 (6, 4);

б) рядок починається з деякої ненульової цифри, за якою знаходяться
тільки літери і їх кількість дорівнює числовому значенню цієї цифри;

в) рядок містить (крім літер) тільки одну цифру, причому її числове
значення дорівнює довжині рядка;

г) сума числових значень цифр, які входять в рядок, дорівнює довжині
рядка;

д) рядок співпадає з початковим (кінцевим, будь-яким) відрізком ряду
0123456789;

е) рядок складається тільки з цифр, причому їх числові значення
складають арифметичну прогресію (наприклад, 3 5 7 9, 8 5 2, 2).

Додаткові задачі:

\begin{enumerate}
\def\labelenumi{\arabic{enumi})}
\item
  Даний рядок в якому зустрічаються слова, які складаються з восьми
  цифрових символів. Переведіть всі їх в формат дати "dd-mm-yyyy" і
  перевірте коректність такої дати.
\item
  В текстовому файлі записані в кожному рядку значення поліномів за
  допомогою знаків +, -, *, **(ступінь) та цифр і літери x. Введіть
  значення x з консолі та для всіх коректних записів поліномів обчисліть
  їх значення для даного x та виведіть в новий текстовий файл.
\end{enumerate}

14. Створення власних класів. Інкапсуляція.

1) Що таке класи і які шляхи визначення класів в Сі++?

2) Яким чином можна визначити методи класу?

3) Приватний та публічний доступ до членів та методів. Яка різниця?

4) Які методи в класі визначені за замовченням? Як і коли потрібно ці
методи визначати самостійно?

5) Шляхи визначення конструктору класу. Як викликати конструктор в
головній функції?

6) Статичні члени та методи класу. Як визначити і коли вони потрібні?

7) Дружні класи та методи. Як вони використовуються?

Задачі для аудиторної роботи

1) а) Визначити клас раціональне число з членами: nominator --- ціле
число, denominator --- натуральне число. Визначити методи введення та
виведення з терміналу, методи додавання та множення раціонального числа

б) Зробіть члени класу приватними та визначить методи ініціалізації
окремо чисельника і знаменника (при цьому не дайте користувачу
можливість ініціалізувати знаменник нулем)

в) Створіть приватний метод класу для скорочення раціонального числа
через НСД

г) Визначить конструктори класу який ініціалізує за замовченням
раціональне число одиницями та конструктор, що ініціалізує його двома
довільними числами

д) Також у класі перевантажте основні арифметичні оператори, оператори
порівняння та інші оператори, що необхідні для роботи з раціональними
числами.

Використовуючи цей клас, розв'яжіть такі задачі:

а) знайдіть найбільше за модулем серед послідовності раціональних чисел

б) підрахуйте суму 20-ти членів ряду за формулою Грегорі

\[\frac{\pi}{4} = 1 - \frac{1}{3} + \frac{1}{5} - \frac{1}{7} + \ldots\]

\begin{enumerate}
\def\labelenumi{\arabic{enumi})}
\item
  На базі класу Точка напишіть програму, що дозволяє вводити
  багатокутник з будь якої кількості вершин вводячи точки доки
  користувач не відповість на запитання «Ввести точку?» - «Ні». Після
  цього виведіть інформацію про кількість точок у багатокутнику та
  виведе його периметр.
\item
  Визначить клас Поліном, що ініціалізується кількістю елементів масиву
  N та виділяє при цьому пам'ять під N дійсних чисел. Створіть методи
  для заповнення членів цього масиву (через конструктор та окремим
  методом) та конкретного коефіцієнту за номером. Визначить деструктор
  та копіконструктор.
\end{enumerate}

Визначить свою дружню функцію для цього класу для введення/виведення
його в/з текстового файл/бінарний файл.

Реалізувати методи: введення поліному, виведення поліному, обчислення
значення поліному у точці x, взяття похідної поліному, суми, різниці та
добутку поліномів. Використати цей клас для розв'язання задачі: ввести 2
поліноми P1, P2 та рядок, який містить вираз, що залежить від 2
поліномів. Наприклад, P1 + P2*P1 -- P2

Обчислити поліном, який буде значенням цього виразу.

Задачі для самостійної роботи

Описати класи розділивши інтерфейс та реалізацію та заборонивши введення
некоректних даних, з методами введення/виведення та де можливо
додавання:

\begin{enumerate}
\def\labelenumi{\arabic{enumi})}
\item
  Описати клас \textbf{Точка} (на площині). Реалізуйте методи введення,
  виведення. Описати клас \textbf{Відрізок} (на площині), що складається
  з 2-х точок та містить крім введення/виведення методи підрахунку
  середини відрізку, довжини відрізку. \emph{За допомогою визначення
  порожньої Точки реалізуйте метод перетину двох відрізків, що повертає
  Точку (у випадку, якщо цих точок декілька виведіть будь-яку з них, а
  якщо жодної -- порожній відрізок).}
\item
  Описати клас \textbf{Коло} (на площині), що задається координатами
  центру та радіусом. Описати методи отримання довжини діаметру, площі
  та периметру кола, перетину двох кіл (повертає відповідно 0,1 або 2
  точки як масив через змінний аргумент).
\item
  Описати клас \textbf{Прямокутник}. Сторони прямокутника паралельні
  осям координат. Для прямокутника задані координати лівого верхнього
  кута та довжини сторін. Описати методи отримання довжини кожної зі
  сторін, площі та периметру, перетину двох прямокутників (якщо перетин
  порожній -- поверніть Прямокутник вигляду(-1,-1,-1,-1)).
\item
  Описати клас \textbf{Трикутник}. Основа трикутника паралельна осі
  \emph{x} координат. Для трикутника задані лівий нижній кут та довжини
  2 сторін. Описати методи отримання довжини кожної зі сторін, кутів,
  площі та периметру.
\item
  Описати класи:
\end{enumerate}

А) \textbf{Час} (години, хвилини, секунди)

Б) \textbf{Дата}(рік, місяць, день)

\begin{enumerate}
\def\labelenumi{\arabic{enumi})}
\item
  Описати клас ігрова \textbf{Дошка}(визначається розміром та назвою
  гри: шашки (міжнародні, російські та турецькі), шахи, нарди) та
  \textbf{Фігура} (назва, гра, масив можливих ходів -- ходи описуються в
  термінах зрозумілих класу Дошка)
\item
  Описати класи:
\end{enumerate}

А) \textbf{Валюта}( назва валюти, значення, центи(копійки))

Б) \textbf{Товар} (назва товару, вартість, валюта в який вимірюється
вартість, одиниця в який вимірюються товар).

\begin{enumerate}
\def\labelenumi{\arabic{enumi})}
\item
  \begin{quote}
  Написати клас Book (Книжка -- назва, автор, кількість сторінок, рік
  видання) та реалізувати програму пошуку книжки за авторами та назвою в
  каталозі (каталог -- масив книжок, що зберігається у файлі).
  \end{quote}
\item
  \begin{quote}
  Визначить клас Вектор, що ініціалізується кількістю елементів масиву N
  та виділяє при цьому пам'ять під N дійсних чисел. Створіть методи для
  заповнення членів цього масиву (через конструктор та окремим методом)
  та конкретного елементу вектору за номером. Визначить деструктор та
  копіконструктор. Із використанням динамічних масивів розв'язати
  задачу: у двох масивах містяться коефіцієнти векторів степеню m і n
  відповідно. Написати методи для введення/ виведення з файлу,
  скалярного та векторного добутку (за можливості) цих векторів.
  \end{quote}
\item
  \begin{quote}
  Опишіть класи Matrix3 та Vector3, що є відповідно матрицею розмірності
  3 на 3 та тривімірним вектором. Перевантажте математичні оператори для
  цих класів та спеціальні методи (множення матриці на вектор у тому
  числі). Оператор abs() перевантажте для матриці методом, що визначає
  її норму. Для матриці опишіть метод det(), що повертає визначник цієї
  матриці.
  \end{quote}
\end{enumerate}

Додаткові задачі:

15. Робота з класами. Наслідування та поліморфізм.

\begin{enumerate}
\def\labelenumi{\arabic{enumi})}
\item
  Що таке перевантаження методів? Чому воно зручно в мовах зі строгою
  типізацією?
\item
  Чим перевантаження операторів відрізняється від перевантаження інших
  методів?
\item
  Які оператори не можна перевантажувати? Коли перевантаження операторів
  може бути небезпечним?
\item
  Чому при перевантаженні операторів вводу-виводу нам потрібно ключове
  слово friend?
\item
  Які типи наслідування є на Сі++ та яка між ними різниця?
\item
  Поясніть на прикладі, що таке раннє та пізнє зв'язування
\item
  Що таке чисто віртуальний клас та чисто віртуальний метод? Коли вони
  потрібні?
\item
  Що таке віртуальний деструктор, та чому він потрібний?
\item
  Як реалізувати множинне наслідування на Сі++?
\item
  Що робити та які шляхи правильного множинного наслідування якщо й
  класи батьки й клас-син мають метод з однаковою назвою? Що зміниться,
  якщо це не метод, а перевантажений оператор?
\end{enumerate}

Задачі для аудиторної роботи

Клас Person описано таким чином:

\textbf{class} \textbf{Person\{} //Клас Особа

string name; //прізвище

unsigned byear\textbf{;//}рік народження

public:

\textbf{int} input()\textbf{\{} //ввести особу

\textbf{cin\textgreater{}\textgreater{}}name;

\textbf{cin\textgreater{}\textgreater{}byer;}

\textbf{\}}

\textbf{void} \textbf{print()\{ //}вивести особу

\textbf{cout\textless{}\textless{}}name\textless{}\textless{}'',''\textless{}\textless{}byear\textbf{\textless{}\textless{}endl;}

\}

\begin{enumerate}
\def\labelenumi{\arabic{enumi})}
\item
  Описати клас Знайомий на базі класу Person.
\end{enumerate}

У цьому класі повинно бути як мінімум одне додаткове поле «номер
телефону» а також методи введення та виведення інформації про знайомого.

Використати цей клас для побудови класу телефонного довідника (кількість
знайомих обмежена числом 100).

Передбачити дії: створення довідника, додавання запису про знайомого,
пошуку номера телефону за прізвищем та заміни номера телефону.

Телефонний довідник зберігає дані про знайомих у файлі.

\emph{\emph{Вказівка}}: телефонний довідник представити у вигляді класу
що зчитує дані з (текстового) файлу.

\begin{enumerate}
\def\labelenumi{\arabic{enumi})}
\item
  На базі класу \textbf{Точка} (на площині) створіть клас Точка3Д (точка
  в просторі). Реалізуйте методи введення, виведення. Аналогічно на базі
  Відрізка2Д реалізуйте клас Відрізок3Д. Методи
  введення\textbackslash{}виведення, визначення довжини відрізка та
  визначення чи перетинаються 2 відрізка.
\end{enumerate}

Задачі для самостійної роботи

\begin{enumerate}
\def\labelenumi{\arabic{enumi})}
\item
  Описати клас Пасажир на базі класу Person. Клас містить дані про місце
  відправлення та місце слідування, а також місце пасажира. Створіть
  клас Каса, який дозволяє додавати та виводити інформацію про
  Пасижирів, містить методи пошуку по прізвищу, місцям відправлення,
  прибуття та місцю. Також серед заданого масиву місць у потягу знайдіть
  місце яке не зайняте (у випадку якщо таких місць декілька -- виведіть
  найменше за значенням, якщо їх немає відповідне повідомлення).
\end{enumerate}

\emph{\emph{Вказівка}}: інформацію про пасажирів представити у вигляді
бінарного файлу.

\begin{enumerate}
\def\labelenumi{\arabic{enumi})}
\item
  Описати клас Студент на базі класу Person.
\end{enumerate}

У класі Студент повинна бути інформація про оцінки отримані ним протягом
сесії (за 5-ти бальною та 100 бальною шкалами).

Скласти програму для обчислення нарахованої студентам стипендії в
залежності від результатів сесії:

\begin{itemize}
\item
  За старим підходом нарахування стипендії (середній бал за всі іспити
  має бути не меншим ніж 4 за 5-ти бальною шкалою).
\item
  З новим підходом нарахування стипендії (стипендію отримують 40\% від
  загального числа студентів, які є найкращими по рейтингу)
\end{itemize}

\emph{\emph{Вказівка}}: інформацію про студентів представити у вигляді
масиву. Дані зчитувати з клавіатури.

\begin{enumerate}
\def\labelenumi{\arabic{enumi})}
\item
  Реалізувати клас СЛОВО, який має члени типу Рядок: ПРИСТАВКА,
  ПРИСТАВКА2, КОРІНЬ, СУФІКС, ЗАКІНЧЕННЯ (клас повинен мати геттери та
  сеттери).
\end{enumerate}

Створіть наслідники цього класу: ГЛАГОЛ, ІМЕННИК, ПРИКМЕТНИК.

Реалізуйте для них методи: Род, Число, Лице, Відмінок -- які будуть
відповідним чино змінювати (якщо це можливо) дане слово.

Створіть декілька слів, що є екземплярами ГЛАГОЛу, ІМЕННИКу, ПРИКМЕТНИКу
та виконайте відповідні методи для них щоб можна було побачити
результат.

\begin{enumerate}
\def\labelenumi{\arabic{enumi})}
\item
  Реалізувати наступні класи:
\end{enumerate}

Описати клас \textbf{Прямокутник}. Сторони прямокутника паралельні осям
координат. Для прямокутника задані лівий верхній кут та довжини сторін.
Описати методи отримання довжини кожної з сторін, площі прямокутника,
периметру, метод знаходження перетину двох прямокутників. Методи
переміщення прямокутника. Скласти програму створення заданої кількості
прямокутників та знаходження їх спільного перетину.

Описати клас \textbf{Трикутник}. Основа трикутника паралельна осі
\emph{x} координат. Для трикутника задані лівий нижній кут (координати)
та довжини сторін. Описати методи отримання довжини кожної зі сторін.
Описати методи отримання довжини кожної з сторін, площі прямокутника,
периметру, метод знаходження перетину двох прямокутників. Методи
переміщення прямокутника. Скласти програму створення заданої кількості
прямокутників та знаходження їх спільного перетину.

Описати клас \textbf{Трикутник}. Основа трикутника паралельна осі
\emph{x} координат. Для трикутника задані лівий нижній кут (координати)
та довжини сторін. Описати методи отримання довжини кожної зі сторін.
Описати методи отримання довжини кожної з сторін, площі, периметру,
метод знаходження перетину двох трикутників. Методи переміщення. Скласти
програму створення заданої кількості трикутників та знаходження їх
спільного перетину.

Описати клас \textbf{Еліпс}. Для нього є заданими фокуси та радіуси.
Описати методи отримання геометричних характеристик. Описати методи
отримання довжини радіусів, площі, периметру, метод знаходження площі
перетину двох еліпсів. Методи переміщення та повороту. Скласти програму
створення заданої кількості еліпсів та знаходження їх спільного
перетину.

Створити клас Фігура, який є базою.

Опишіть класи для таких геометричних фігур та реалізуйте зазначені
методи:

\begin{enumerate}
\def\labelenumi{\alph{enumi})}
\item
  Клас Трапеція. У цьому класі реалізуйте операції знаходження периметра
  і площі;
\item
  Клас Паралелограм. У цьому класі реалізуйте операції знаходження
  периметра і площі.
\item
  Клас Круг. Реалізуйте методи відшукання площі круга, довжини кола,
  цього круга.
\item
  Клас Піраміда. Реалізуйте методи пошуку площі бічної поверхні і
  об'єму;
\item
  Клас П'ятикутник, що містить масив вершин. Реалізуйте метод перевірки
  чи є цей п'ятикутник опуклим.
\item
  Клас Багатокутник. Реалізуйте метод перевірки чи є цей багатокутник
  опуклим.
\end{enumerate}

Дано список фігур вищенаведених класів. Серед фігур, що належать до
перших трьох класів знайдіть фігуру, що має найбільшу площу та периметр
(довжину кола). Також знайдіть всі опуклі багатокутники

7) Опишіть класи

1. \textbf{Гість}, що містить всю необхідну інформацію про жильця
деякого готелю: ім'я, період проживання тощо.

2. \textbf{Кімната}, що містить інформацію про кімнату готелю у тому
числі вартість проживання за добу.

3. \textbf{Готель}, що містить список кімнат цього готелю, інформацію
про те ким і коли вони зайняті, а також методи на кшталт тощо.

Використовуючи вищенаведені класи розв'язати задачі:

а) Вивести відомість про кількість вільних кімнат у готелі;

б) Пошуку вільної кімнати у зазначений період;

в) Поселити жильця на вказаний термін;

г) Вартості проживання жильця у зазначений період;

д) Прибутку, який отримає готель за вказаний період;

е) Пошуку гостя у готелі (у заданий період);

\begin{enumerate}
\def\labelenumi{\arabic{enumi})}
\item
  Опишіть клас Фігура, що інкапсулює основні геометричні характеристики
  та методи. Для фігури визначено методи:
\end{enumerate}

\begin{enumerate}
\def\labelenumi{\arabic{enumi}.}
\item
  calculateVolume() -- віртуальний метод, що обчислює міру фігури (для
  плоскої фігури -- площу, для об'ємної -- відповідно об'єм).
\item
  getVolume() -- що повертає міру фігури.
\end{enumerate}

Від класу Фігура наслідуються такі класи

\begin{itemize}
\item
  Трикутник
\item
  Прямокутник
\item
  Трапеція
\item
  Паралелограм
\item
  Круг
\item
  Куля
\item
  Трикутна Піраміда (який успадковується від класу Трикутник)
\item
  Чотирикутна піраміда (який успадковується від класу Прямокутник)
\item
  Паралелепіпед (який успадковується від класу Прямокутник)
\end{itemize}

Нехай дано список фігур. Серед заданих фігур, знайдіть фігуру, що має
найбільшу міра якої є найбільшою

\begin{enumerate}
\def\labelenumi{\arabic{enumi})}
\item
  Опишіть клас Pet -- домашня тварина, що має метод to\_feed(feed,
  count) -- годувати (feed -- тип корму, count -- кількість).
\end{enumerate}

Клас Pet має віртуальні методи

\begin{itemize}
\item
  to\_sniff () («нюхати» -- визначає, чи може їсти тварина заданий тип
  корму),
\item
  to\_ask() («просити» -- метод повертає True, якщо тип корму не
  підходить або тварина ще хоче їсти і виводить на екран прохання
  «тваринною мовою», наприклад, «Мяв\ldots{}» для кота),
\item
  to\_eat() (їсти, якщо тип корму підходить).
\end{itemize}

Клас Pet має нащадки -- Cat, Dog, Parrot (папуга), у яких перевизначено
вищезгадані віртуальні методи.

Задано список тварин та список кормів (тип та загальна вага). Пропонуючи
по черзі кожній тварині порцію їжі, потрібно нагодувати всіх тварин.
Якщо корму не вистачить -- вивести відповідне повідомлення.

\begin{enumerate}
\def\labelenumi{\arabic{enumi})}
\item
  Опишіть клас Car, що має метод go(distance), який змінює пройдений
  кілометраж автомобілем та залишок пального. Метод go(\ldots{})
  залежить від віртуального методу fuelPerKm(), який визначає скільки
  потрібно пального автомобілю для проїзду одного кілометру. Нехай
  Personal (легковий автомобіль) і Truck (вантажівка) -- класи, що
  наслідують клас Car і перевизначають метод fuelPerKm(). При цьому
  потрібно врахувати, що цей метод залежить від кількості пасажирів
  (+10\% на кожного пасажира) для авто класу Personal або ваги вантажу
  для Truck (+25\% на кожну тонну вантажу). Визначити чи зможе задане
  авто проїхати задану відстань.
\item
  Задано клас Flower, що має нащадками конкретні класи квітів (напр.,
  тюльпан, троянд, тощо). Ви зайшли у квітковий магазин у якому
  продаються різні типи квітів. Необхідно зібрати букет з квітів (букет
  може містити квітки одного класу) та визначити:
\end{enumerate}

\begin{enumerate}
\def\labelenumi{\arabic{enumi}.}
\item
  Його вартість.
\item
  Скільки часу зможе тішити букет очі (до моменту поки не зів'яне перша
  квітка).
\item
  Колір, що домінує у цьому букеті.
\item
  Чи припустимий цей букет за інтенсивністю запаху.
\end{enumerate}

Додаткові задачі:

16. Перетворення типів та робота з виключеннями.

\begin{enumerate}
\def\labelenumi{\arabic{enumi})}
\item
  Які варіанти перетворень стандартних типів один між іншим можливі в
  Сі++?
\item
  Яким перетворенням краще скористатись для перетворень між цілими
  типами? Яким при перетворення цілих до дійсного та навпаки?
\item
  Чим відрізняються перетворення вгору та вниз? Яке перетворення типу
  краще для перетворення вгору, а яке вниз?
\item
  Чому не можна відловити виключення при діленні на нуль в Сі++ зі
  стандартними типами?
\item
  Як створити власне виключення в Сі++? Як його коректно обробити?
\item
  Яке стандартне виключення дозволяє коректно обробити static\_cast?
\item
  Як складнощі виникають якщо виключення виникає в деструкторі класу?
\item
  Як коректно працювати з виключенням, що виникає в конструкторі класу?
\end{enumerate}

Задачі для аудиторної роботи

\begin{enumerate}
\def\labelenumi{\arabic{enumi})}
\item
\end{enumerate}

\begin{enumerate}
\def\labelenumi{\arabic{enumi})}
\item
  В класі Раціональній дріб з попередньої лекції напишіть методи
  введення, виведення (cin\textgreater{}\textgreater{},
  cout\textless{}\textless{}) та оператори віднімання, ділення як
  перевантажені оператори. Тобто з типом Раціональній дріб можна тепер
  працювати як зі стандартним типом. Чому краще перевантажити два
  оператори віднімання? Перепишіть методи введення
  (cin\textgreater{}\textgreater{}) та конструктор і сеттери, щоб вони
  кидали виключення при ініціалізації знаменнику нулем. Коректно
  обробить в коді це виключення. Напишіть дружню функцію запису
  Раціонального дробу в файл, яка буде викидати виключення при
  некоректному відкритті файлу та обробить його в тілі програми.
\end{enumerate}

\begin{enumerate}
\def\labelenumi{\arabic{enumi})}
\item
\end{enumerate}

\begin{enumerate}
\def\labelenumi{\arabic{enumi})}
\item
  Створіть клас Людина (члени: ПІБ, стать, вік) та його наслідники
  Студент (додано: курс, група, ВУЗ), Викладач (додано: ВУЗ, посада,
  з.п.). Методи введення, виведення, конструктори для різної кількості
  вхідних даних.
\end{enumerate}

Створіть клас Аспірант, що є наслідником і студента і викладача.
Коректно визначте член ВУЗ для нього.

Створить програму що буде вводити масив Людей, серед яких є Студенти,
Викладачі, Аспіранти. Без створення нових членів класу виведіть коректно
ВУЗ для кожного екземпляру масиву.

Задачі для самостійної роботи

1)Скласти підпрограму та програму для обчислення значення натурального
числа за заданим рядком символів, який є записом цього числа у системі
числення за основою b (\(2 \leq b \leq 16\)). Використати функцію, яка
за заданим символом повертає відповідну цифру у системі числення за
основою b. Використати у цій функції твердження про стан програми assert
для перевірки того, що відповідний символ є цифрою у системі числення за
основою b. Обробити у підпрограмі помилку неправильного символу рядка та
показати змістовне повідомлення про помилку.

2)Скласти функцію та програму для обчислення суми всіх доданків, модуль
яких не менше ε \textgreater{} 0, у комплексній точці \emph{z}

\(\text{arctg}\left( z \right) = z - \frac{z^{3}}{3} + \frac{z^{5}}{5} - \cdots + {( - 1)}^{n}\frac{z^{2n + 1}}{2n + 1} + \cdots,\ \ \ \ (\left| z \right| < 1)\).

Використати у цій функції твердження про стан програми для перевірки
того, що параметр \emph{z} відповідає заданій умові та зробить обробку
всіх можливих виключень -- включаючи некоректне введення та виділення
пам'яті під масиви. Обробити у програмі помилку неправильного значення
\emph{z} та показати змістовне повідомлення про помилку.

3)Задані натуральне число \emph{і} файл \emph{f}, компоненти якого є
цілими числами. Побудувати файл \emph{g}, записавши в нього найбільше
значення перших \emph{n} компонент файлу \emph{f}, потім-наступних
\emph{n} компонент і т.д. Розглянути два випадки:

а) число компонент файлу ділиться на \emph{n};

б) число компонент файлу не ділиться на \emph{n}.

В цьому випадку остання компонента файлу \emph{g} повинна дорівнювати
найбільшій із компонент файлу \emph{f}, які утворюють останню (неповну)
групу.

Забезпечити обробку помилок при роботі з файлами.

4)У текстовому файлі записана непорожня послідовність дійсних чисел, які
розділяються пропусками в одному рядку та можуть бути розташовані у
різних рядках. Визначити функцію обчислення найбільшого з цих чисел.

Забезпечити обробку помилок, якщо у файлі зустрічаються не дійсні числа.

5)Описати клас Трьохбайтне ціле число для роботи з цілими числами,
представленими трьома байтами. Інтервал представлення при цьому -- від
-2\textsuperscript{23} до 2\textsuperscript{23}-1. Операції не можуть
вивести за межі інтервалу представлення. Наприклад,
2\textsuperscript{23}-1 + 1 == -2\textsuperscript{23} й т.д. Якщо
результат операції виводить за межі інтервалу представлення, повинна
ініціюватися помилка переповнення.

Перевизначити у цьому класі операції +, -, *, //, \%.

Описати також 3 класи обробки помилок для трьохбайтних цілих чисел:
загальний клас обробки помилок та два його підкласи для обробки помилки
переповнення та помилки ділення на 0.

Використати цей клас для розв'язання задач:

а) обчислення \emph{n}!

б) обчислення \emph{x\textsuperscript{n}}, де \emph{x} -- ціле, \emph{n}
-- невід'ємне ціле.

Забезпечити обробку помилок при виконанні обчислень.

6)Описати клас Поліном та реалізувати методи: введення поліному,
виведення поліному, обчислення значення поліному у точці x, взяття
похідної поліному, суми, різниці та добутку поліномів.

Описати також клас обробки помилок при неправильному введенні поліному
(степінь -- не невід'ємне ціле число, коефіцієнт -- не дійсне число) та
забезпечити ініціювання помилки при неправильному введенні.

Використати цей клас для розв'язання задачі: ввести 2 поліноми P1, P2 та
рядок, який містить вираз, що залежить від 2 поліномів. Наприклад,

P1 + P2*P1 -- P2

Обчислити поліном, який буде значенням цього виразу.

Забезпечити обробку помилок неправильного введення поліному.

\emph{\emph{Вказівка:}} поліном представити у вигляді словника.

7)Описати клас для реалізації мультимножини на базі масиву чисел розміру
N=100. Мультимножина - це множина в якій для кожного елемента
запам'ятовується не лише його входження, але й кількість входжень.

Кількість входжень елемента \emph{k} (\(0 \leq k \leq n\)) у
мультимножину - це значення елемента словника з ключем \emph{k}.

Реалізувати дії над мультимножинами:

1) зробити мультимножину порожньою;

2) чи є мультимножина порожньою;

3) додати елемент до мультимножини;

4) забрати елемент з мультимножини (кількість входжень елемента
зменшується на 1, якщо елемент не входить - відмова);

5) кількість входжень елемента у мультимножину;

6) об'єднання двох мультимножин (в результаті об'єднання кількість
входжень елемента визначається як максимальна з двох мультимножин);

7) перетин двох мультимножин (в результаті кількість входжень елемента
визначається як мінімальна з двох мультимножин);

Описати клас обробки помилки взяття елементу, який не входить до
мультимножини.

З використанням класу розв'язати задачі:

а) перевірити, чи складаються рядки \emph{S1}, \emph{S2} з одних і тих
же символів, які входять у ці рядки однакову кількість разів;

б) перевірити, чи вірно, що всі символи рядка \emph{S1}, входять також у
рядок \emph{S2}, причому не меншу кількість разів, ніж у \emph{S1}.

Забезпечити обробку помилок.

Додаткові задачі:

17. Створення шаблонів функцій та шаблонів класів

1) Як створити функцію-шаблон? В яких ситуаціях вона корисна?

2) Як створити клас-шаблон? Що потрібно зробити якщо шаблоном є лише
єдиний метод класу?

\begin{enumerate}
\def\labelenumi{\arabic{enumi})}
\item
  Навіщо потрібні простори імен та що таке стандартний простір імен? Як
  його підключити та що робити коли не можна його підключати на весь
  файл програми?
\item
  Як створити власний простір імен що містить власні математичні функції
  sin, cos, pow. Як їх коректно використати разом зі стандартними
  функціями?
\item
  Створіть вкладені простори імен та функції з однаковими
  ідентифікаторами в них та функцію з таким самим ідентифікатором
  глобально. Як правильно використати ці функції використовуючи ключове
  слово using?
\end{enumerate}

Задачі для аудиторної роботи

\begin{enumerate}
\def\labelenumi{\arabic{enumi})}
\item
  Перепишіть функцію шаблон для пошуку максимуму, так щоб вона працювала
  для всіх стандартних числових типів. Що потрібно зробити, щоб вона
  запрацювала і для типу Раціонального дробу з попередніх лекцій?
  (Вказівка: щось потрібно визначити для класу Раціональний дріб)
\item
  Створіть клас-шаблон Поліном, який приймає вектор чисел (будь-якого
  типу) --- вектор (на базі стандартного класу Вектор) коефіцієнтів
  поліному. Методи: введення-виведення, додавання, множення та
  обчислення значення. Перевірте, що клас працює коректно для дійсних,
  цілих чисел та для типу Раціональний дріб з попередніх завдань.
\end{enumerate}

Задачі для самостійної роботи

\begin{enumerate}
\def\labelenumi{\arabic{enumi})}
\item
  Створити клас-шаблон BlackBox БлекБокс, який містить конструктор
  (порожній та від масиву (вказівника) будь-якого типу), метод push(),
  що дозволяє додати елемент певного типу, та метод pop(), що видає та
  видаляє випадковий елемент, що вже міститься в класі та виключення,
  якщо БлекБокс порожній, метод xpop(), що просто повертає випадковий
  елемент цього класу.
\item
  Створити клас-шаблон Mediana, який містить конструктор (порожній та
  від масиву (вказівника) будь-якого типу, що містить операції
  порівняння), метод push(), що дозволяє додати елемент будь-якого типу,
  що містить операції порівняння, та метод pop(int n), що видає та
  видаляє елемент, з номером n по порядку, або виключення, якщо n більше
  розміру всіх елементів, метод mediana(), що повертає медіану елементів
  цього класу.
\end{enumerate}

Додаткові задачі:

18. Стандартна бібліотека С++. Послідовні контейнери.

\begin{enumerate}
\def\labelenumi{\arabic{enumi})}
\item
  Створіть власний клас-шаблон vector\textless{}T\textgreater{} з
  методом Норма(). Порівняйте його дію з стандартним шаблоном vector в
  головній програмі.
\item
  З яких частин складається бібліотека шаблонів Сі++?
\item
  Для чого потрібні контейнери-адаптори? Які конетейнери-адаптори
  визначені в Сі++?
\item
  Які контейнери прямого доступу визначені в Сі++?
\item
  Яка різниця між контейнерами list, forward\_list, vector, array?
\item
  Основні методи конетйнеру вектор (доступ до елементів, заміна
  елементів, розміри)?
\item
  Які переваги array або vector перед стандартним масивом чи
  вказівником?
\item
  Як додавати елемент в вектор, стек, список?
\item
  Як видаляти елементи в list, forward\_list, vector, array?
\item
  Які варіанти проітеруватись по елементах послідовних контейнерів?
\end{enumerate}

Задачі для аудиторної роботи

\begin{enumerate}
\def\labelenumi{\arabic{enumi})}
\item
  Створіть власну реалізацію класу шаблону Стек. Перевірте її роботу за
  допомогою стандартного класу Стек з STL.
\item
  Ввести n d-вимірних векторів x (n, d вводяться з клавіатури) та
  обчислити значення функції f(x) (реалізувати її) для кожного з цих x.
\end{enumerate}

\includegraphics[width=3.23889in,height=0.66667in]{media/image4.png}

\begin{enumerate}
\def\labelenumi{\arabic{enumi})}
\item
  Створить список цілих чисел List і число X. Не використовуючи
  допоміжних об'єктів і не змінюючи розміру списку, переставити елементи
  списку так, щоб спочатку йшли числа, що не перевищують X, а потім
  числа що є більшими за X.
\end{enumerate}

Задачі для самостійної роботи

\begin{enumerate}
\def\labelenumi{\arabic{enumi})}
\item
  Біля прилавка в магазині вишикувалася черга з п покупців. Час
  обслуговування продавцем i-го покупця число
  \(t_{i},\ i = 1,\cdots,n\). Нехай дано натуральне n і дійсні числа
  \(t_{1},t_{2},\cdots,t_{n}\). Отримати \(c_{1},c_{2},\cdots,c_{n}\) де
  з \(c_{i}\ \)-- час перебування i-го покупця в черзі
  \(i = 1,\cdots,n\). Вказати номер покупця, для обслуговування якого
  продавцеві потрібно найменше часу.
\item
  Дана матриця з цілих чисел. Знайти в ній прямокутну підматрицю, що
  складається з максимальної кількості однакових елементів.
  Використовувати клас Stack.
\item
  Реалізувати структуру «чорний ящик» на базі Queue, що зберігає множину
  чисел і має внутрішній лічильник K, спочатку рівний нулю. структура
  повинна підтримувати операції додавання числа в множину і повернення
  K-го по мінімальності числа з множини.
\item
  На клітковому аркуші намальований круг. Вивести в файл опису всіх
  клітин, цілком лежать всередині кола в порядку зростання відстані від
  клітини до центру кола. Використовувати клас PriorityQueue.
\item
  На базі шаблону List реалізувати структуру зберігання чисел з
  підтримкою наступних операцій:

  \begin{enumerate}
  \def\labelenumii{\alph{enumii}.}
  \item
    додавання / видалення числа;
  \item
    пошук числа, найбільш близького до заданого (тобто модуль різниці
    мінімальний).
  \end{enumerate}
\item
  У вхідному файлі розташовані два набору позитивних чисел; між наборами
  -- від'ємне число. Побудувати два списки C1 і С2, елементи яких
  містять відповідно числа 1-го і 2-го набору таким чином, щоб усередині
  одного списку числа були впорядковані по зростанню. Потім об'єднати
  списки C1 і С2 в один відсортований список.
\item
  Армія хоче набрати людей зі свого виборчого списку служб. Вони
  вирішили набрати тих, хто записався на службу в 1997 році, починаючи
  від найстаршого і закінчуючи молодшим. Згенеруйте довільну кількість
  людей (надайте їм такі дані, як вік та рік, зареєстровані) у вектор.
  Розділіть вектор так, щоб ті, хто вступив у 1997 році, були
  упорядковані на початку списку, починаючи від наймолодшого до
  найстаршого, а решту частину списку залишали сортувати за віком.
\item
  Створіть клас «Місто» з даними про населення, висоту та погоду.
  Зробіть погоду переліченою за допомогою \{ДОЩИТЬ, СНІЖНО, ХМАРНО,
  ЯСНО\}. Створіть клас, який генерує об'єкти Town. Створіть назви міст
  (незалежно від того, мають вони сенс чи ні, це не має значення) або
  витягніть їх з Інтернету. Переконайтеся, що назва всього міста має
  маленькі регістри, а дублікатів назв немає. Для простоти радимо
  зводити назви міст одним словом. Для населення, висот та погодних
  полів створіть генератор, який випадковим чином генеруватиме погодні
  умови, популяції в межах {[}100-1000000) та висоти між {[}0, 8000)
  футами. Заповніть вектор об'єктами міста. Перепишіть вектор у новий
  файл під назвою Towns.txt.
\item
  Складіть клас Employee із двома членами даних: hours та hourlyPay.
  Працівник також повинен мати функцію calcSalary(), яка повертає
  заробітну плату за цього працівника. Генеруйте довільну погодинну
  оплату праці та години для довільної кількості працівників. Зберігайте
  вектор \textless{}Співробітник *\textgreater{}. Дізнайтеся, скільки
  грошей компанія витратить за цей період оплати праці.
\item
  Створіть шаблон класу Matrix, який створений з вектору
  \textless{}vector \textless{}T\textgreater{}\textgreater{}. Надайте
  його дружньому методу ostream \& operator \textless{}\textless{}
  (ostream \&, const Matrix \&) для відображення матриці. Створіть
  наступні двійкові операції, використовуючи об'єкти функції STL, де це
  можливо: оператор + (const Matrix \&, const Matrix \&) для додавання
  матриці, оператор * (const Matrix \&, const vector
  \textless{}int\textgreater{} \&) для множення матриці на вектор та
  оператор * ( const Matrix \&, const Matrix \&) для множення матриць.
  Перевірте шаблон класу Matrix, використовуючи int і float.
\end{enumerate}

Додаткові задачі:

19. Стандартна бібліотека С++. Асоціативні контейнери.

Які типи ітераторів існують в STL? Які з них визначені для вектору? Які
до списку? Які для множини?

Які асоціативні контейнери існують в Сі++? Що додає приставка multi до
назви контейнера?

Для яких стандартних класів-шаблонів не визначений метод push\_back()?
Чому? Як в ці класи додаються елементи?

Як визначити кількість елементів будь-якого контейнеру?

Які коректні шляхи ітерації по вектору? Мультивідображенню? Будь-якому
контейнеру?

Як коректно пройти по всім елементам відображення?

Задачі для аудиторної роботи

\begin{enumerate}
\def\labelenumi{\arabic{enumi})}
\item
  Заданий файл з текстом англійською мовою. Виділити все різні слова.
  Для кожного слова підрахувати частоту його входження. Слова, що
  відрізняються регістром літер, вважати різними. Використовувати Map.
\item
  Введіть відображення в якому ключ --- це слово, а значення декілька
  слів, які визначають це слово. При цьому порочного кола немає.
  Пронумеруйте слова таким чином, щоб слова з більшим номером
  визначались лише словами з меншими номерами.
\item
  З використанням Set виконати попарне підсумовування довільного
  кінцевого ряду чисел за такими правилами: на першому етапі
  підсумовуються попарно сусідні числа, на другому етапі підсумовуються
  результати першого етапу і т. д. до тих пір, поки не залишиться одне
  число.
\end{enumerate}

Задачі для самостійної роботи

\begin{enumerate}
\def\labelenumi{\arabic{enumi})}
\item
  На площині задано N точок. Вивести в файл описи всіх прямих, які
  проходять більш ніж через одну точку із заданих. Для кожної прямий
  вказати, через скільки точок вона проходить. Використовувати клас
  MultiMap.
\item
  На площині задано N відрізків. Знайти точку перетину двох відрізків,
  що має мінімальну абсцису. Використовувати клас Map.
\item
  На клітковому аркуші паперу зафарбована частина клітин. Виділити все
  різні фігури, які утворилися при цьому. Фігурою вважається набір
  зафарбованих клітин, які сусідні один з одного при руху в чотирьох
  напрямах. Дві фігури є різними, якщо їх не можна сумістити поворотом
  на кут, кратний 90 градусам, і паралельним переносом. Використовуйте
  клас MultiSet.
\item
  Нехай значення функції f(n)- кількість літер у письмовому
  представленні числа n (f(1)=4 („один``), f(3)=3(«три»), f(42)=8
  («сорок два», а(2001) =13 («дві тисячи один»))). Знайдіть всі числа до
  10000, для яких f(n) = n. (Вказівка: Використовуйте
  відображення(словник) для зберігання кількості літер у представленні
  цифри)
\item
  Напишіть програму, яка знаходить усі спільні слова між двома вхідними
  файлами, використовуючи set\_intersection (). Змініть його, щоб
  показати слова, які не є спільними, за допомогою
  set\_symmetric\_difference ().
\end{enumerate}

Додаткові задачі:

20. Стандартна бібліотека С++. Алгоритми та функтори.

\begin{enumerate}
\def\labelenumi{\arabic{enumi})}
\item
  Які існують бібліотеки алгоритмів в STL?
\item
  Як скористатись алгоритмами сортування? Акумульованої суми? Пошуку?
\item
  Як знайти потрібний елемент в будь-якому контейнері за допомогою
  стандартних алгоритмів?
\item
  Що таке придикат та функтор? Як ними скористатись?
\item
  Як відсортувати вектор по незростанню? Неспаданню?
\item
  Коли і як можна скористатись бінарним пошуком в колекції?
\item
  Як за допомогою алгоритмів знайти суму масиву одним викликом функції?
  А як норму вектору двома викликами за допомогою функтору?
\end{enumerate}

Задачі для аудиторної роботи

\begin{enumerate}
\def\labelenumi{\arabic{enumi})}
\item
  У файлі записані координати точок на площині задані парою цілих чисел.
  Точки записуються в форматі : ( х1 , х2 ) (х1 , х2) , \ldots{} - саме
  так через коми та дужки. Створити файл, в якому будуть записані
  координати всіх відрізків з точок цього файлу, при цьому ці відрізки
  відсортовані за зростанням довжини.
\item
  Напишіть функцію, як повертає а) суму найбільших k чисел даного
  вектору, б) масив з k найменших чисел даного вектору, якщо k не
  перевищує розмір масиву та а) нуль б) порожній масив в протилежному
  випадку.
\item
  Створіть генератор, який повертає поточне значення clock () (у
  \textless{}ctime\textgreater{}). Створіть список
  \textless{}clock\_t\textgreater{} і заповніть його своїм генератором
  за допомогою create\_n (). Видаліть усі дублікати зі списку та
  роздрукуйте його на cout за допомогою copy ().
\item
  За допомогою transform () і toupper () (у
  \textless{}cctype\textgreater{}) напишіть один виклик функції, який
  перетворить рядок на всі великі літери.
\item
  Створіть шаблон об'єкта функції Sum, який буде накопичувати всі
  значення в діапазоні при використанні з for\_each ().
\item
  Напишіть генератор анаграм, який приймає слово як аргумент командного
  рядка і створює всі можливі перестановки літер.
\item
  Напишіть генератор анаграм речень, який приймає речення як аргумент
  командного рядка і створює всі можливі перестановки слів у реченні.
  (Це залишає слова в спокої і просто рухає їх навколо).
\end{enumerate}

Задачі для самостійної роботи

\begin{enumerate}
\def\labelenumi{\arabic{enumi})}
\item
  У файлі записані координати Точок в просторі задані трійкою цілих
  чисел. Точки записуються в форматі : х1 , х2 , х3 ; х1 , х2, х3 ;
  \ldots{}
\end{enumerate}

Створити файл, в якому будуть записані відрізки з точок цього файлу, при
цьому ці відрізки відсортовані за зростанням довжини.

\begin{enumerate}
\def\labelenumi{\arabic{enumi})}
\item
  У файлі записані координати Точок на площині задані парою цілих чисел
  та масою(дійсне число). Точки записуються в форматі : (х1 , х2): m1 ,
  (х1 , х2): m2 , \ldots{} Створити файл, в якому будуть записані
  відрізки з точок цього файлу, при цьому ці точки відсортовані за
  важилем сили (m1*(х1 +х2)).
\item
  У файлі записані дати , що трійкою цілих чисел у форматі: чч/мм/рр,
  \ldots{} Створити файл, в якому будуть записані дати з цього файлу без
  повторень, при цьому ці дати відсортовані за спадання дати (врахуйте,
  що роки дат з 1951 по 2049).
\item
  У файлі записані дати , що двома цілими числами та рядком (англійські
  або числові назви місяця) у форматі: чч1 місяць1 рік1, чч2 місяць2
  рік2\ldots{}Вивести дати без повторень з цього файлу у форматі:
  рік1/місяць1/число1, рік2/місяць1/число2,... (місяць заданий назвою)
  при цьому ці дати відсортовані за зростанням дати
\item
  Створіть ієрархію класів з базовим класом B та похідним класом D.
  Помістіть функцію віртуального члена void f () у B таким чином, щоб
  вона надрукувала повідомлення, що вказує, що було викликано B sf (), і
  перевизначити цю функцію для D щоб надрукувати інше повідомлення.
  Створіть вектор \textless{}B *\textgreater{} і заповніть його
  об'єктами B і D. Використовуйте for\_each () для виклику f () для
  кожного з об'єктів у вашому векторі.
\item
  Створіть програму, яка, отримуючи ціле число в командному рядку,
  створює таблицю факторіалів з усіх факторіалів, включаючи число в
  командному рядку. Для цього напишіть генератор для заповнення вектора
  \textless{}int\textgreater{}, а потім використовуйте парциальну\_суму
  () зі стандартним об'єктом функції.
\item
  Використовуючи символи "\textasciitilde{}`! @ \# \$\% \^{} \& * () \_-
  + =\} \{{[}{]} \textbar{} \textbackslash{} :;
  "'\textless{}.\textgreater{},? /", згенеруйте кодову книгу,
  використовуючи вхідний файл, вказаний у командному рядку як словник
  слів. Не турбуйтеся про вилучення не алфавітних символів і не
  турбуйтеся про регістр слів у файлі словника. Співставте кожну
  перестановку рядка символів із таким словом, наприклад:
\end{enumerate}

"= ') /\% {[}\}{]} \textbar{} \{* @ ?!" `,;\textgreater{} \& \^{} -
\textasciitilde{} \_: \$ +. \# (\textless{}\textbackslash{}" apple ",

\textbar{}{]} \textbackslash{} \textasciitilde{}\textgreater{} \#. +\%
(/ -\_ {[}` ':; =\} \{* "\$ \^{}! \&?), @ \textless{}"carrot ",

@ = \textasciitilde{} {[}'{]}. \textbackslash{} /
\textless{}-`\textgreater{} \# *) \^{}\% +, "; \&?! \_ \{: \textbar{}
\$\} " Carrot'' тощо .

Переконайтеся, що у вашій книзі кодів немає повторюваних кодів або слів.
Використовуйте lexicographic\_compare (), щоб виконати сортування кодів.
Використовуйте книгу кодів для кодування файлу словника. Розшифруйте
своє кодування файлу словника та переконайтеся, що ви повернули той
самий вміст.

\begin{enumerate}
\def\labelenumi{\arabic{enumi})}
\item
  Створіть алгоритм стилю STL transform\_if (), слідуючи першій формі
  transform (), яка виконує перетворення лише на об'єктах, які
  задовольняють одинарний предикат. Об'єкти, які не задовольняють
  предикату, опускаються з результату. Потрібно повернути новий кінцевий
  ітератор.
\item
  Створіть алгоритм стилю STL, який є перевантаженою версією for\_each
  (), яка слідує за другою формою перетворення () і займає два діапазони
  введення, щоб він міг передавати об'єкти другого діапазону введення a
  двійковій функції, яку він застосовує до кожного об'єкта першого
  діапазону.
\end{enumerate}

\protect\hypertarget{_Hlk65952131}{}{}

\begin{enumerate}
\def\labelenumi{\arabic{enumi})}
\item
  Створіть клас «Місто» з даними про населення, висоту та погоду.
  Зробіть погоду переліченою за допомогою \{ДОЩИТЬ, СНІЖНО, ХМАРНО,
  ЯСНО\}. Створіть клас, який генерує об'єкти Town. Створіть назви міст
  (незалежно від того, мають вони сенс чи ні, це не має значення) або
  витягніть їх з Інтернету. Переконайтеся, що назва всього міста має
  маленькі регістри, а дублікатів назв немає. Для простоти радимо
  зводити назви міст одним словом. Для населення, висот та погодних
  полів створіть генератор, який випадковим чином генеруватиме погодні
  умови, популяції в межах {[}100-1000000) та висоти між {[}0, 8000)
  футами. Заповніть вектор об'єктами міста. Перепишіть вектор у новий
  файл під назвою Towns.txt.
\end{enumerate}

Відбувся бебі-бум, що призвело до збільшення населення на 10\% у кожному
місті. Оновіть дані про місто за допомогою transform (), перепишіть дані
назад у файл.

\begin{enumerate}
\def\labelenumi{\arabic{enumi})}
\item
  Знайдіть міста з найбільшим і найменшим населенням. Для цієї вправи
  застосуйте оператор \textless{}для вашого класу Town. Також спробуйте
  реалізувати функцію, яка повертає true, якщо її перший параметр менше,
  ніж другий. Використовуйте його як предикат для виклику
  використовуваного вами алгоритму.
\end{enumerate}

Знайдіть усі міста на висоті 2500--3500 футів включно. За необхідності
реалізуйте оператори рівності для класу Town.

\begin{enumerate}
\def\labelenumi{\arabic{enumi})}
\item
  Нам потрібно розмістити аеропорт на певній висоті, але розташування не
  є проблемою. Впорядкуйте свій список міст так, щоб не було дублікатів
  (дублікат означає, що жодні дві висоти не знаходяться в одному
  діапазоні 100 футів. До таких класів належать {[}100, 199), {[}200,
  199) і т.д. Відсортуйте цей список за зростанням принаймні двома
  різними способами, використовуючи об'єкти функції в
  \textless{}functional\textgreater{}. Зробіть те ж саме для порядку
  зменшення. За необхідності впроваджуйте реляційні оператори для міста.
\item
  Створіть довільну кількість випадкових чисел у масиві на основі стеку.
  Використовуйте max\_element (), щоб знайти найбільше число в масиві.
  Поміняйте його номером у кінці масиву. Знайдіть наступне найбільше
  число та росташуйте його в масиві в позиції перед попереднім числом.
  Продовжуйте це робити, доки всі елементи не будуть переміщені. Коли
  алгоритм буде завершено, ви отримаєте відсортований масив. (Це
  сортування виділенням)
\item
  Напишіть програму, яка знімає телефонні номери з файлу (що також
  містить імена та іншу відповідну інформацію) та змінює номери, що
  починаються з 222 на 863. Обов'язково збережіть старі номери. Формат
  файлу такий:
\end{enumerate}

222 8945

756 3920

222 8432

тощо

Напишіть програму, яка за прізвищем знайде кожного з цим прізвищем із
відповідним номером телефону. Використовуйте алгоритми, які мають справу
з діапазонами (upper\_bound, lower\_bound, equal\_range тощо). Сортуйте
за прізвищем, що діє як первинний ключ, а за іменем, що діє як вторинний
ключ. Припустимо, що ви прочитаєте імена та номери з файлу, формат якого
буде таким. (Обов'язково впорядкуйте їх так, щоб прізвища були
впорядковані, а імена впорядковані в межах прізвищ.):

Джон Доу 345 9483

Нік Бонем 349 2930

Джейн Доу 283 2819

Отримавши файл із даними, подібними до наведених нижче, витягніть із
нього всі державні абревіатури та помістіть їх в окремий файл. (Зверніть
увагу, що ви не можете залежати від номера рядка для типу даних. Дані
містяться на випадкових рядках.)

\begin{enumerate}
\def\labelenumi{\arabic{enumi})}
\item
  Порівняйте роботу функцій sort(), partial\_sort() та nth\_element()
  один проти одного і з'ясуйте, чи дійсно варто використовувати одне із
  слабких сортувань, коли вони можуть спрацювати коректно.
\item
  Міські об'єкти. Створіть назви міст (незалежно від того, мають вони
  сенс чи ні, це не має значення) або витягніть їх з Інтернету.
  Переконайтеся, що назва всього міста має маленькі регістри, а
  дублікатів назв немає. Для простоти радимо зводити назви міст одним
  словом. Для населення, висот та погодних полів створіть генератор,
  який випадковим чином генеруватиме погодні умови, популяції в межах
  {[}100-1000000) та висоти між {[}0, 8000) футами. Заповніть вектор
  об'єктами міста. Перепишіть вектор у новий файл під назвою Towns.txt.
\end{enumerate}

Додаткові задачі:

\end{document}
