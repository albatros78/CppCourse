\documentclass[]{article}
\usepackage{lmodern}
\usepackage{amssymb,amsmath}
\usepackage{ifxetex,ifluatex}


\usepackage[utf8]{inputenc}
\usepackage[english,russian,ukrainian]{babel}

\usepackage{fixltx2e} % provides \textsubscript
\ifnum 0\ifxetex 1\fi\ifluatex 1\fi=0 % if pdftex
  \usepackage[T1]{fontenc}
  \usepackage[utf8]{inputenc}
\else % if luatex or xelatex
  \ifxetex
    \usepackage{mathspec}
  \else
    \usepackage{fontspec}
  \fi
  \defaultfontfeatures{Ligatures=TeX,Scale=MatchLowercase}
\fi
% use upquote if available, for straight quotes in verbatim environments
\IfFileExists{upquote.sty}{\usepackage{upquote}}{}
% use microtype if available
\IfFileExists{microtype.sty}{%
\usepackage{microtype}
\UseMicrotypeSet[protrusion]{basicmath} % disable protrusion for tt fonts
}{}
\usepackage[unicode=true]{hyperref}
\hypersetup{
            pdfborder={0 0 0},
            breaklinks=true}
\urlstyle{same}  % don't use monospace font for urls
\usepackage{graphicx,grffile}
\makeatletter
\def\maxwidth{\ifdim\Gin@nat@width>\linewidth\linewidth\else\Gin@nat@width\fi}
\def\maxheight{\ifdim\Gin@nat@height>\textheight\textheight\else\Gin@nat@height\fi}
\makeatother
% Scale images if necessary, so that they will not overflow the page
% margins by default, and it is still possible to overwrite the defaults
% using explicit options in \includegraphics[width, height, ...]{}
\setkeys{Gin}{width=\maxwidth,height=\maxheight,keepaspectratio}
\IfFileExists{parskip.sty}{%
\usepackage{parskip}
}{% else
\setlength{\parindent}{0pt}
\setlength{\parskip}{6pt plus 2pt minus 1pt}
}
\setlength{\emergencystretch}{3em}  % prevent overfull lines
\providecommand{\tightlist}{%
  \setlength{\itemsep}{0pt}\setlength{\parskip}{0pt}}
\setcounter{secnumdepth}{0}
% Redefines (sub)paragraphs to behave more like sections
\ifx\paragraph\undefined\else
\let\oldparagraph\paragraph
\renewcommand{\paragraph}[1]{\oldparagraph{#1}\mbox{}}
\fi
\ifx\subparagraph\undefined\else
\let\oldsubparagraph\subparagraph
\renewcommand{\subparagraph}[1]{\oldsubparagraph{#1}\mbox{}}
\fi

\date{}


\usepackage{enumitem}
\makeatletter
\newcommand{\xslalph}[1]{\expandafter\@xslalph\csname c@#1\endcsname}
\newcommand{\@xslalph}[1]{%
    \ifcase#1\or а\or б\or в\or г\or д\or e\or є\or ж\or з\or i%
    \or й\or к\or л\or м\or н\or о\or п\or р\or с\or т%
    \or у\or ф\or х\or ц\or ч\or ш\or ю\or я\or аа\or бб\or вв%
    \else\@ctrerr\fi%
}
\AddEnumerateCounter{\xslalph}{\@xslalph}{m}
\makeatother


\begin{document}


\newpage
\subsection{ 7. Статичні масиви. Лінійні масиви }
\setcounter{subsection}{1}


\begin{itemize}
\item
  Які варіанти декларації масивів на Сі. На Сі++?
\item
  Які варіанти ініціалізації масивів на Сі. На Сі++?
\item
  Створення багатовимірного масиву. Введіть розміри та вміст двовимірної
  дійсної матриці. Виведіть її красиво рідок за рядком.
\item
  Як найкраще передавати масив у аргументи функції?
\item
  Чи можна повернути масив фіксованого розміру як результат функції?
\item
  Чому масив як аргумент краще передавати через вказівник чи посилання?
\item
  Як повернути коректно дані з масиву з функції?
\end{itemize}

Задачі для аудиторної роботи

\begin{enumerate}
\def\labelenumi{\arabic{enumi})}
\item
  Ініціалізуйте масив 5 цілих чисел в програмі довільним чином. Введіть
  дійсне число та знайдіть кількість чисел у вашому масиві, що менше зі
  це число.

\item
  Масив заповнений таким чином: 5, 112, 4, 3. Вивести його елементи
  навпаки (3,4,112,5). При цьому використання циклу є обов'язковим.
\item
  Заповнити масив типу double з 10 елементів з клавіатури (по черзі в
  циклі вводяться всі елементи) і знайти суму всіх елементів більших за
  число Ейлера \(e\).
\item
  Масив типу int з 5 елементів заповнюється з клавіатури. Знайти і
  вивести на екран максимальне значення у вашому масиві.
\item
  Знайти суму всіх парних і непарних елементів масиву натуральних чисел.
  Масив заповнюється з клавіатури, 5 елементів.
\item
Написати функції, в яких якщо потрібно повернути результат -- масив,
то це робиться за допомогою змінного аргументу функції:
\begin{itemize}
\item вводить n-вимірний вектор дійсних чисел;
\item виводить n-вимірний вектор дійсних чисел;
\item рахує суму двох векторів;
\item рахує скалярний добуток двох векторів.
 \end{itemize}
Протестувати роботи цих функцій: ввести в головній програмі розмірність
векторів, два вектори цієї розмірності та підрахувати їх суму та скалярний
добуток і вивести результати.
\end{enumerate}

Задачі для самостійної роботи

\begin{enumerate}
\def\labelenumi{\arabic{enumi})}
\setcounter{enumi}{6}
\item
  Написати функцію, що вводить послідовність ненульових цілих чисел,
  введення завершується при вводі нуля. Кількість елементів масиву
  обмежена числом 20. Визначити кількість добуток та середнє гармонічне
  цієї послідовності.
\item
  Вводиться масив натуральних чисел заданого розміру N:
\begin{enumerate}[label=\xslalph*)]
\item визначити скільки серед цих чисел повних квадратів простих чисел;
\item визначити скільки серед цих чисел парних повних кубів;
\item визначити скільки серед цих чисел $n$-тих ступенів цілих чисел (для
всіх $n>1$);
\item визначити скільки серед них цілих ступенів двійки;
\item визначити скільки серед них ступенів чисел, що кратні 3;
\item визначити скільки серед них простих чисел;
\item визначити скільки серед них чисел Фібоначчі;
\item визначити скільки серед них чисел, у яких 5-й, 6-й та 8-й біт
двійкового запису дорівнюють 1;
\item визначити скільки серед них чисел, які містять рівно 5 біт в
двійковому записі, що дорівнюють 1;
\item визначити скільки серед них чисел, у яких сума цифр в десятковому
запису ділиться на 7.
 \end{enumerate}

\item Задані натуральне число \(n\), дійсні числа
\(a_{1},a_{2},\ldots,a_{n}\). Скласти програму для знаходження:
\begin{enumerate}[label=\xslalph*)]
\item
 \(\max\left( a_{1},a_{2},\ldots,a_{n} \right)\); 
\item
\(\min\left( a_{1},a_{2},\ldots,a_{n} \right)\);
\item \(\max\left( a_{2},a_{4},\ldots \right)\); 
\item
\(\min\left( a_{1},a_{3},\ldots \right)\);
\item
\(\min\left( a_{2},a_{4},\ldots \right) + \max\left( a_{1},a_{3},\ldots \right)\);
\item
\(\max\left( \left| a_{1} \right|,\ldots,\left| a_{n} \right| \right)\);
\item \(\max\left( -a_{1},a_{2}, -a_{3}\ldots,(-1)^{n}a_{n} \right)\);
\item
\(\left( \min\left( a_{1},\ldots,a_{n} \right) \right)^{2} - \min\left( a_{1}^{2},\ldots,a_{n}^{2} \right)\).

\end{enumerate}

\item Дано натуральне число n, цілі числа \(a_{1},a_{2},\ldots,a_{n}\).
Скласти програму знаходження
\begin{enumerate}[label=\xslalph*)]
\item
 \(min(a_{1},2a_{2},\ldots,na_{n})\);

\item \(min(a_{1} + a_{2},\ldots,a_{n - 1} + a_{n})\);

\item \(max(a_{1},a_{1}a_{2},\ldots,a_{1}a_{2}\ldots a_{n})\);
\item кількості парних серед \(a_{1^2},\ a_{2^2},\ldots,a_{k^2},\; k=[\sqrt{n}] \);
\item кількості повних квадратів серед \(a_{1}a_{n},\ a_{1}a_{n-1},\ldots,\ a_{k}a_{n-k},\; k=[n/2]\);
\item кількості квадратів непарних чисел серед
\(a_{1},a_{2},\ldots,a_{n}\).
\end{enumerate}

\item
Скласти функції для обчислення
\begin{enumerate}[label=\xslalph*)]
\item
Значення поліному Чебишова заданого степеню \(n\) в точці \(x\)

\(T_{0}(x) = 1,T_{1}(x) = x,\)

\(T_{n}(x) = 2xT_{n - 1}(x) - T_{n - 2}(x),n = 2,3,\ldots;\)

та функцію, що виводить коефіцієнти поліному Чебишова ступеня $n<256$.

\item
Значення поліному Ерміта заданого степеню \(n\)в точці \(x\)

\(H_{0}(x) = 1,H_{1}(x) = 2x,\)

\(H_{n}(x) = 2xH_{n - 1}(x) - 2(n - 1)H_{n - 2}(x),n = 2,3,\ldots\)

та функцію, що виводить коефіцієнти поліному Ерміта ступеня $n<256$.
\end{enumerate}
\item
  В цілочисельному масиві A{[}N{]} знайдіть моду, тобто елемент, що
  зустрчається найбільшу кількість разів. Якщо таких елементів декілька
  виведіть всі такі елементи.

\item
  В цілочисельному масиві A{[}N{]} знайдіть елемент, що є найближчим до
  середнього арифметичного найбільшого та найменшого елементу масиву.
\item
  Напишіть функцію, яка в дійсному масиві A{[}N{]} знаходить середнє
  відхилення (варіацію) масиву
\item
  Знайдіть в даному цілому числі цифру десяткового запису, яка
  зустрічається найбільшу кількість разів. Якщо їх декілька, виведіть
  найбільшу цифру.
\item
  Напишіть функцію, яка за заданим масивом значень
  \({\{ x_{i}\}}_{i = 1}^{d}\) обчислює:

  $$ f(x) = \sum\limits_{i=1}^{d} (100x_{i+1} -x_{i})^{2} + (x_{i}-1)^2. $$  

\item
  В деяких видах спортивних змагань виступ кожного спортсмена незалежно
  оцінюється деякими суддями, потім з усієї сукупності оцінок
  видаляються найбільш висока і найнижча, а для решти оцінок
  обчислюється середнє арифметичне, яке і йде в залік спортсмену. Якщо
  найбільш високу оцінку виставило декілька суддів, то з сукупності
  оцінок видаляється лише одна така оцінка; аналогічно надходять з
  найбільш низькими оцінками. Дано натуральне число n, дійсні числа
  \(a_{1},a_{2},\cdots,a_{n}\). Вважаючи, що
  \(a_{1},a_{2},\cdots,a_{n}\)оцінки, виставлені суддями одному з
  учасників змагань, визначити оцінку, яка піде в залік цього
  спортсмену.
\end{enumerate}

Додаткові задачі:

\begin{enumerate}
\def\labelenumi{\arabic{enumi})}
\setcounter{enumi}{17}
\item
По заданим значенням коефіцієнтів поліномів $P(x)$ та $Q(x)$ знайдіть
значення коефіцієнтів поліному $P(Q(x))$.
\item
  Обчислити коефіцієнти багаточлена з заданими дійсними коренями 
$ x{[}0{]},x{[}1{]}, \ldots{}, x{[}n{]}$. Кількість коефіцієнтів обмежена
  числом 100.
\item
  Побудувати N-розрядний код Грея. Кодом Грея зветься така послідовність
  дворозрядних двійкових чисел, в яких кожні два сусідних а також перше
  й останнє числа відрізняються лише одним розрядом. Так, для N=2 код
  Грея наступний: 00,01,11,10. Для N=3: 000,001,011,010,110,111,101.
  Переведіть всі числа з цього двійкового коду до десяткової системи
  числення.
\item
  В цілочисельному масиві A{[}N{]} (не обов'язково впорядкованому)
  знайдіть медіану, тобто величину, що ділить ряд навпіл: по обидві
  сторони від неї знаходиться однакова кількість одиниць сукупності.
  Тобто, якщо кількість чисел непарна, обирається елемент, що є середнім
  за зростанням. Наприклад, для впорядкованого набору чисел 1, 3, 3, 6,
  7, 8, 9 медіаною є четверте із них, число 6. Якщо кількість елементів
  парна, тоді медіану зазвичай визначають як середнє значення між двома
  числами по середині впорядкованого масиву Наприклад, для наступного
  набору 1, 2, 3, 4, 5, 6, 8, 9 - медіана є середнім значенням для двох
  чисел по середині: вона дорівнюватиме (4 + 5)/2=4.5.
\end{enumerate}


\end{document}
