\documentclass[]{article}
\usepackage{lmodern}
\usepackage{amssymb,amsmath}
\usepackage{ifxetex,ifluatex}


\usepackage[utf8]{inputenc}
\usepackage[english,russian,ukrainian]{babel}

\usepackage{fixltx2e} % provides \textsubscript
\ifnum 0\ifxetex 1\fi\ifluatex 1\fi=0 % if pdftex
  \usepackage[T1]{fontenc}
  \usepackage[utf8]{inputenc}
\else % if luatex or xelatex
  \ifxetex
    \usepackage{mathspec}
  \else
    \usepackage{fontspec}
  \fi
  \defaultfontfeatures{Ligatures=TeX,Scale=MatchLowercase}
\fi
% use upquote if available, for straight quotes in verbatim environments
\IfFileExists{upquote.sty}{\usepackage{upquote}}{}
% use microtype if available
\IfFileExists{microtype.sty}{%
\usepackage{microtype}
\UseMicrotypeSet[protrusion]{basicmath} % disable protrusion for tt fonts
}{}
\usepackage[unicode=true]{hyperref}
\hypersetup{
            pdfborder={0 0 0},
            breaklinks=true}
\urlstyle{same}  % don't use monospace font for urls
\usepackage{graphicx,grffile}
\makeatletter
\def\maxwidth{\ifdim\Gin@nat@width>\linewidth\linewidth\else\Gin@nat@width\fi}
\def\maxheight{\ifdim\Gin@nat@height>\textheight\textheight\else\Gin@nat@height\fi}
\makeatother
% Scale images if necessary, so that they will not overflow the page
% margins by default, and it is still possible to overwrite the defaults
% using explicit options in \includegraphics[width, height, ...]{}
\setkeys{Gin}{width=\maxwidth,height=\maxheight,keepaspectratio}
\IfFileExists{parskip.sty}{%
\usepackage{parskip}
}{% else
\setlength{\parindent}{0pt}
\setlength{\parskip}{6pt plus 2pt minus 1pt}
}
\setlength{\emergencystretch}{3em}  % prevent overfull lines
\providecommand{\tightlist}{%
  \setlength{\itemsep}{0pt}\setlength{\parskip}{0pt}}
\setcounter{secnumdepth}{0}
% Redefines (sub)paragraphs to behave more like sections
\ifx\paragraph\undefined\else
\let\oldparagraph\paragraph
\renewcommand{\paragraph}[1]{\oldparagraph{#1}\mbox{}}
\fi
\ifx\subparagraph\undefined\else
\let\oldsubparagraph\subparagraph
\renewcommand{\subparagraph}[1]{\oldsubparagraph{#1}\mbox{}}
\fi

\date{}


\usepackage{enumitem}
\makeatletter
\newcommand{\xslalph}[1]{\expandafter\@xslalph\csname c@#1\endcsname}
\newcommand{\@xslalph}[1]{%
    \ifcase#1\or а\or б\or в\or г\or д\or e\or є\or ж\or з\or i%
    \or й\or к\or л\or м\or н\or о\or п\or р\or с\or т%
    \or у\or ф\or х\or ц\or ч\or ш\or ю\or я\or аа\or бб\or вв %
    \else\@ctrerr\fi%
}
\AddEnumerateCounter{\xslalph}{\@xslalph}{m}
\makeatother


\begin{document}

\section*{ Методичні рекомендації з курсу «Мова програмування С++» }

Вступ

1. Лінійні програми на Сі. Введення/виведення. Дійсний тип даних.

2. Використання математичної бібліотеки С. Створення власних функцій

3. Цілі типи Сі. Умовні конструкції.

4. Цикли.

5. Цикли. Рекурентні співвідношення. Рекурсія

6. Бітові операції

7. Статичні масиви. Лінійні масиви та багатовимірні масиви

8. Динамічні масиви. Робота з вказівниками

9. Робота з рядком, що закінчується нулем на С.

10. Структури. Створення власного типу

11. Робота з бінарним файлами на Сі

12. Введення/виведення на С++. Робота з текстовими файлами

13. Робота з класом рядок на С++.

14. Створення власних класів. Інкапсуляція.

15. Робота з класами. Наслідування та поліморфізм.

16. Перетворення типів та робота з виключеннями.

17. Створення шаблонів функцій та шаблонів класів

18. Стандартна бібліотека С++. Послідовні контейнери.

19. Стандартна бібліотека С++. Асоціативні контейнери.

20. Стандартна бібліотека С++. Алгоритми та функтори.

\subsection{ ВСТУП }

Мета цього посібника, надати студенту завдання для того, щоб практично
оволодіти потрібними навичками програмування на мовах С та С++ в рамках
дисципліни «Мова програмування С++». Теми обиралися автором таким чином,
щоб найбільш швидким темпом здобути навичкі для практичного
програмування за 20 занять, тому деякі теми та розділи програмування на
С та С++, які автор вважає занадто складним або не обовязковими з точки
зору практики програмування, не входять до цього задачника, а винесені
на самостійну роботу або в якості завдань на курсові проекти.

Завдання посібника розділені на 20 лабораторних робіт, кожна з яких
присвячена окремій темі, що вивчається в дисципліні. Завдання та теми
підбиралися таким чином, щоб вивчення синтаксису мови виходило
поступовим тому послідовне виконання лабораторних робіт є найкращим для
засвоєння та набуття відповідних навичок. Тому наполегливо рекомендуємо
дотримуватися послідовного виконання лабораторних робіт.

Матеріал кожної лабораторної роботи посібника складається з п'яти
блоків: контрольних запитань, завдань для аудиторної роботи та трьох
блоків завдань для самостійної роботи. Під час підготовки до практичного
заняття, студент повинен опрацювати блок контрольних запитань та знати
вичерпні відповіді на них. Блок завдань для аудиторної містять перелік
типових задач відповідної теми. Ці завдання студент має виконати
протягом практичного заняття самостійно або під керівництвом викладача.
Завдання для самостійної роботи студент виконує самостійно та звітує про
їхнє виконання викладачу. Як було зазначено вище, завдання для
самостійної роботи складається з трьох блоків, перший з яких є
обов'язковим для виконання.

Другий блок завдань є ідентичним по складності основному блоку завдань
для самостійної роботи та призначений для кращого засвоєння матеріалу.

Третій блок завдань складається з задач підвищеної складності та вимагає
від студента не лише досконалого опанування методів поточної теми, а й
матеріалу, що виходить за межі нормативного курсу.


\newpage
\subsection{ 3. Цілі типи Сі. Умовні конструкції.}
\setcounter{subsection}{1}

Питання по темі 3:

\begin{itemize}
\item Які типи цілих чисел використовуються в Сі/С++?
\item Які варіанти використання булевого типу є в Сі?
\item Як перевести число із знакового до беззнакового типу? Як навпаки?
\item Як ввести найдовше можливе ціле число? Як узнати його розмір в
байтах?
\item Як з'ясовує скільки байтів на цілий та довгий
цілий тип виділяє компілятор, а також чи підтримує він довгий тип та
скільки на нього виділяється байтів?
\item Як виконати цілочисельне ділення в Сі? Як поділити не цілочисельно
два цілих числа?
\item Як коректно та без поперджень компілятора ініціалізувати довге
натуральне число? Натуральне коротке? Ціле довге?
\item Як коректно та без попереджень ввести та вивести натуральне число?
Натуральне коротке? Ціле довге?
\item Як записати умовне розгалудження в Сі/Сі++?
\item Які типи умовних виразів на Сі/Сі++? Напишіть два варіанти з ними для
пошуку мінимума двох чисел. Напишіть за допомогою виразу альтернативи
функцію, що повертає парність цілого числа.
\end{itemize}

Завдання для аудиторної роботи
\begin{enumerate}
\def\labelenumi{\arabic{enumi})}
\item
Дано натуральне тризначне число. Знайти:
  \begin{enumerate}[label=\xslalph*)]
\item кількість одиниць, десятків і сотень цього числа;
\item суму цифр цього числа;
\item число, утворене при читанні заданого числа справа наліво.
\end{enumerate}

\item
Ввести натуральне тризначне число. Якщо в ньому всі 3 цифри різні, то
вивести всі числа, які утворюються при перестановці цифр заданого числа.
\item
Введіть три цілих числа, записаних через кому в одному рядку та
підрахуйте їх добуток якщо всі ці числа гарантовано по модулю менші а)
\(2^{10}\); б) \(2^{21}\).

\item
Напишіть функцію, що гарантовано приймає у якості аргументів 8-бітні
натуральні числа та обчислює їх добуток як гарантовано 16-бітне
натуральне число.
\item
Визначити більше та менше з двох чисел, введених з клавіатури.
\item
Дано три дійсних числа. Скласти програму для знаходження числа:
найбільшого за модулем та найменшого за модулем.
\item
Визначити, скільки розв'язків має рівняння та розв'язати його:
  \begin{enumerate}[label=\xslalph*)]
\item \(ax^{2} + bx + c = 0\);
\item \(ax^{4} + bx^{2} + c = 0\).
  \end{enumerate}
\end{enumerate}

Завдання для самостійної роботи

\begin{enumerate}
\def\labelenumi{\arabic{enumi})}
\setcounter{enumi}{7}
\item
  Введіть два натуральних 32-бітних числа та виведіть їх суму як
  32-бітне число, якщо немає переповнення типу. В противному випадку
  виведіть про це повідомлення. Аналогічно підрахуйте добуток двох цілих
  32-бітних чисел.
\item
  Дано три дійсних числа $x$, $y$ і $z$. Скласти програму для
  обчислення:
\begin{enumerate}[label=\xslalph*)]
\item
  \(max(x + y + z,xy- xz + yz,xyz)\);
\item
  \(max(xy,xz,yz)\).
\end{enumerate}

\item
  Дано три дійсних числа $x$, $y$ і $z$. Визначити кількість:
\begin{enumerate}[label=\xslalph*)]
\item різних серед них; 
\item однакових серед них;
\item чисел, що є більшими за їхнє середнє арифметичне значення;
\item чисел, що є більшими за введене з клавіатури число \(a\).
\end{enumerate}

\item
  Обчислити значення функцій:
\begin{enumerate}[label=\xslalph*)]
\item \(f(x) = |x|;\) \item \(f(x) = ||x| - 1| - 1;\)
\item \(f(x) = sign(x)\) \item \(f(x) = sin(|x|);\)

\end{enumerate}


\item
  Перевірити, чи існує трикутник із заданими сторонами $a,b,c$.
  Якщо так, то визначити, який він: гострокутний, прямокутний чи
  тупокутний.

\item
  Визначити, скільки розв'язків має система рівнянь і розв'язати її:
\begin{enumerate}[label=\xslalph*)]
\item \(\left\{ \begin{matrix}
a_{1}x + b_{1}y + c_{1} = 0 \\
a_{2}x + b_{2}y + c_{2} = 0; \\
\end{matrix} \right.\ \) 

\item \(\left\{ \begin{matrix}
\left| x + y \right| = 1 \\
ax + by + c = 0 \\
\end{matrix} \right.\ \)
\end{enumerate}

\item
  Знайти число точок пеpетину кола \(x^{2} + y^{2} = r^{2}\) з відpізком
  \(x = a,\ b \leq y \leq b + c^{2}\) .
\item
  Скласти програму, яка по колу
  \({(x - v)}^{2} + ({y - u)}^{2} = r^{2}\) та пpямій
  \(ax + by + c = 0\) встановлює, який випадок має місце:
\begin{itemize}
\item дві точки пеpетину;
\item одна точка дотику;
\item жодної спільної точки.
\end{itemize}

\item
  З'ясувати, чи пеpетинаються два кола на площині.
\item
  Задано два квадрати, сторони яких паралельні координатним осям.
  З'ясувати, чи перетинаються вони. Якщо так, то знайти координати
  лівого нижнього та правого верхнього кутів прямокутника, що є їхнім
  перетином.
\item
  Дано два прямокутники, сторони яких паралельні координатним осям.
  Відомо координати лівого нижнього та правого верхнього кутів кожного з
  прямокутників. Знайти координати лівого нижнього та правого верхнього
  кутів мінімального прямокутника, що містить задані прямокутники.
\item
  Записати функції, що повертають значення 1 тоді й тільки тоді, коли:
\begin{enumerate}[label=\xslalph*)]
\item натуральне число n -- непарне;
\item остання цифра числа n -- 5;
\item ціле число n кратне натуральному числу m;
\item натуральні числа n і k одночасно кратні натуральному числу m
\item сума першої і другої цифри двозначного натурального числа - двозначне
число;
\item число x більше за число y не менше, ніж на 7;
\item принаймні одне з чисел x, y або z більше за 99;
\item тільки одне з чисел x, y або z менше за 1001.
\end{enumerate}

\item
  Реалізувати функцію, яка перевіряє, чи належить початок координат
  трикутнику, що заданий координатами своїх вершин.
\item
  Точка простору задана декартовими координатами $(x, y, z)$. Перевірити,
  чи належить вона кулі з радіусом $R$ i центром у початку координат.
\item
  Точка простору задана декартовими координатами $(x, y, z)$. Перевірити,
  чи належить вона циліндру, вісь якого збігається з віссю Oz. Висота
  дорівнює $h$, а нижня основа лежить у площині Oxy та має радіус $r$.
\item
  Реалізуйте функції та напишіть відповідну до кожної з них функцію, що
  буде рахувати їх похідні (за нескінченість прийміть найбільше можливе 
число типу double):
\begin{enumerate}[label=\xslalph*)]
\item onestep(x) = \(\left\{ \begin{matrix}
1,x \geq 0 \\
0,x < 0 \\
\end{matrix} \right.\ \)

\item 
ReLu(x) =\(max(0,x)\)

\item
leakyReLu(x,a)= \(\left\{ \begin{matrix}
ax,\ x < 0 \\
0,\ x \geq 0 \\
\end{matrix} \right.\ \)

\item 
eReLu(a,x) =\(\left\{ \begin{matrix}
a(e^{x} - 1),x < 0 \\
0,\ x \geq 0 \\
\end{matrix} \right.\ \)

\item 
sReLu(tl,tr,al,ar,x)=\(\left\{ \begin{matrix}
tl + al\left( x - tl \right),x \leq tl \\
0,tl < x < tr \\
tr + ar\left( x - tr \right),x \geq tr \\
\end{matrix} \right.\ \)

\item
 isReLu(a,x)= \(\left\{ \begin{matrix}
\frac{x}{\sqrt{1 + ax^{2}}},x < 0 \\
x,\ x \geq 0 \\
\end{matrix} \right.\ \)

\item 
softExponential(a,x) = \(\left\{ \begin{matrix}
 - \frac{ln(1 - a(x + a)}{a},a < 0 \\
x,a = 0 \\
\frac{e^{\text{ax}} - 1}{a} + a,a > 0 \\
\end{matrix} \right.\ \)

\item 
sinc(x)= \(\left\{ \begin{matrix}
1,\ x = 0 \\
\frac{\sin x}{x},x \neq 0 \\
\end{matrix} \right.\ \)

 \end{enumerate}
\end{enumerate}

\end{document}

