\documentclass[]{article}
\usepackage{lmodern}
\usepackage{amssymb,amsmath}
\usepackage{ifxetex,ifluatex}
\usepackage{fixltx2e} % provides \textsubscript
\ifnum 0\ifxetex 1\fi\ifluatex 1\fi=0 % if pdftex
  \usepackage[T1]{fontenc}
  \usepackage[utf8]{inputenc}
\else % if luatex or xelatex
  \ifxetex
    \usepackage{mathspec}
  \else
    \usepackage{fontspec}
  \fi
  \defaultfontfeatures{Ligatures=TeX,Scale=MatchLowercase}
\fi
% use upquote if available, for straight quotes in verbatim environments
\IfFileExists{upquote.sty}{\usepackage{upquote}}{}
% use microtype if available
\IfFileExists{microtype.sty}{%
\usepackage{microtype}
\UseMicrotypeSet[protrusion]{basicmath} % disable protrusion for tt fonts
}{}
\usepackage[unicode=true]{hyperref}
\hypersetup{
            pdfborder={0 0 0},
            breaklinks=true}
\urlstyle{same}  % don't use monospace font for urls
\usepackage{graphicx,grffile}
\makeatletter
\def\maxwidth{\ifdim\Gin@nat@width>\linewidth\linewidth\else\Gin@nat@width\fi}
\def\maxheight{\ifdim\Gin@nat@height>\textheight\textheight\else\Gin@nat@height\fi}
\makeatother
% Scale images if necessary, so that they will not overflow the page
% margins by default, and it is still possible to overwrite the defaults
% using explicit options in \includegraphics[width, height, ...]{}
\setkeys{Gin}{width=\maxwidth,height=\maxheight,keepaspectratio}
\IfFileExists{parskip.sty}{%
\usepackage{parskip}
}{% else
\setlength{\parindent}{0pt}
\setlength{\parskip}{6pt plus 2pt minus 1pt}
}
\setlength{\emergencystretch}{3em}  % prevent overfull lines
\providecommand{\tightlist}{%
  \setlength{\itemsep}{0pt}\setlength{\parskip}{0pt}}
\setcounter{secnumdepth}{0}
% Redefines (sub)paragraphs to behave more like sections
\ifx\paragraph\undefined\else
\let\oldparagraph\paragraph
\renewcommand{\paragraph}[1]{\oldparagraph{#1}\mbox{}}
\fi
\ifx\subparagraph\undefined\else
\let\oldsubparagraph\subparagraph
\renewcommand{\subparagraph}[1]{\oldsubparagraph{#1}\mbox{}}
\fi

\date{}

\begin{document}

\begin{quote}
Задачник

План
\end{quote}

1. Компіляція програм та прості програми на Сі++/Сі:

Прості розрахунки та виведення (дійсні числа)

Введення (дійсні числа)

(Локалізація)

2. Введення/виведення на Сі та прості функції дійсних аргументів.
Математичні функції.

Використання математичних функцій

Створення власних простих функцій

3. Цілі числа та умовні конструкції

Цілі числа та їх типи

Прості умовні конструкції

4. Цикли (дійсні та цілі числа)

Цикли з лічильником

Цикли з перед- та післяумовами

(Комплексні числа)

5. Бітові операції

6. Масиви на Сі

Одновимірні масиви

Двовимірні та багатовимірні масиви

7. Вказівники та виділення пам'яті

(Статичні та глобальні змінні)

(вказівники на функції)

8. Робота з рядками Сі

Символьний тип

Тип рядку --- рядок з символом кінця рядку

(Широкі символи та юнікод)

9. Символьні файли на Сі

10. Текстові файли на Сі

11. Структури на Сі

(Бітові поля)

(Об'єднання)

12. Бінарні файли на Сі

(Командний рядок)

(Змінні оточення)

(Функції зі змінною кількістю аргументів)

(Макроси?)

(Дати та час?)

(Багатопоточність?)

13. Потоки вводу/виводу Сі++

14. Текстові файли Сі++

15. Робота з класом string

16. Створення власних класів на Сі++

17. Наслідування та віртуальні методи

18. Шаблони функцій та шаблон пари

19. Класи-шаблони

20. Стандартна бібліотека STL

\subsection{}\label{section}

\subsection{0. Компіляція програм та прості програми на Сі++/Сі.
Форматоване виведення. Прості розрахунки та виведення (дійсні
числа)}\label{ux43aux43eux43cux43fux456ux43bux44fux446ux456ux44f-ux43fux440ux43eux433ux440ux430ux43c-ux442ux430-ux43fux440ux43eux441ux442ux456-ux43fux440ux43eux433ux440ux430ux43cux438-ux43dux430-ux441ux456ux441ux456.-ux444ux43eux440ux43cux430ux442ux43eux432ux430ux43dux435-ux432ux438ux432ux435ux434ux435ux43dux43dux44f.-ux43fux440ux43eux441ux442ux456-ux440ux43eux437ux440ux430ux445ux443ux43dux43aux438-ux442ux430-ux432ux438ux432ux435ux434ux435ux43dux43dux44f-ux434ux456ux439ux441ux43dux456-ux447ux438ux441ux43bux430}

\subsubsection{0.0 Компіляція програм та прості програми на Сі++/Сі.
Форматоване
виведення}\label{ux43aux43eux43cux43fux456ux43bux44fux446ux456ux44f-ux43fux440ux43eux433ux440ux430ux43c-ux442ux430-ux43fux440ux43eux441ux442ux456-ux43fux440ux43eux433ux440ux430ux43cux438-ux43dux430-ux441ux456ux441ux456.-ux444ux43eux440ux43cux430ux442ux43eux432ux430ux43dux435-ux432ux438ux432ux435ux434ux435ux43dux43dux44f}

\begin{quote}
1) Обчисліть наступні математичні вирази та виведіть результати:

2+31; 45*54-11; 15/4; 15.0/4; 67\%5; (2*45.1 +3.2)/2;

2) Ініціалізуйте наступні числа як дійсні, подвійні дійсні та довгі
дійсні:\(10^{- 4}\), 2.33E5, \(\pi\) , \(e\), \(\sqrt{5}\), \(ln(100)\)

3) Вивести на екран таблицю
\end{quote}

x \textbar{} 1 \textbar{} 2 \textbar{} 3 \textbar{} 4 \textbar{} 5

-\/-\/-\/-\/-\/-\/-\/-\/-\/-\/-\/-\/-\/-\/-\/-\/-\/-

у \textbar{} 3 \textbar{} 1 \textbar{} 5 \textbar{} 4 \textbar{} 2

\begin{quote}
\textbf{4)} Зобразити на екрані декартову систему координат у вигляді
\end{quote}

\^{} y

\textbar{} x

-\/-\/-\/-\/-\/-\/-\/-\/-\/-\/-\/-\/-\/-\/-\/-\/-\/-\/-\textgreater{}

\textbar{} 1

\textbar{}

\begin{quote}
5)Вивести на екран рисунки:

а) б) в)

* * * * * * * * * * * * *

* * * * * * * * *

* * * * * * * * * Hello *

* * * * * * * * *

* * * * * * * * * * * * *

\textbf{6)} Вивести на екран текст:

а) б)

a a a a-\/-\/-\/-\/-\/-\/-\/-\/-\/-a

a a \textbar{} a \textbar{}

a a a a-\/-\/-\/-\/-\/-\/-\/-\/-\/-a

де a -- введена з клавіатури цифра.

7) Обчислити силу притягання \(F\) (в науковому форматі) між двома
тілами, що мають маси \emph{m\textsubscript{1},m\textsubscript{2}}
\emph{,} на відстані \emph{r}. \emph{\emph{Вказівка}}. Шукана силa
визначається за формулою
\emph{F=γ*m\textsubscript{1}*m\textsubscript{2}/r\textsuperscript{2},}
де \emph{γ = 6.673*10\textsuperscript{-11}
Н*м\textsuperscript{2}/кг\textsuperscript{2}. Всі потрібні змінні
присвоюються всередині програми.} Результат вивести в окремому рядку
вигляду «F=*** », де замість зірок представлення в науковому
(експоненційному) вигляді.

8)Наближено визначити період обертання Землі навколо Сонця,
використовуючи ланцюговий дріб
\end{quote}

\[T = \mathrm{365} + \frac{1}{4 + \frac{1}{7 + \frac{1}{1 + \frac{1}{3}}}}\]

\begin{quote}
Результат вивести в форматі плаваючої крапки.

9) Обчислити значення функції десяткового логарифму для даного числа --
вивести результат з точністю до 3 знаків.

10) Дано дійсне число \emph{x}. Користуючись лише операцією множення,
отримати:

а) \emph{x\textsuperscript{4}} за дві операції; б)
\emph{x\textsuperscript{6}} за три операції;

в) \emph{x\textsuperscript{9}} за чотири операції; г)
\emph{x\textsuperscript{15}} за п'ять операцій;

ґ) \emph{x\textsuperscript{28}} за шість операцій; д)
\emph{x\textsuperscript{64}} за шість операцій.

11) Тіло починає рухатися без початкової швидкості з прискоренням
\emph{a}. Обчислити:

а) відстань, яку воно пройде за час \emph{t} від початку руху;

б) час, за який тіло досягне швидкості \emph{v}.

12) Обчислити кінетичну енергію тіла масою \emph{m}, що рухається зі
швидкістю \emph{v} відносно поверхні Землі.

13) Вивести на екран таблицю

x \textbar{} 1 \textbar{} 2 \textbar{} 3 \textbar{} 4 \textbar{} 5

-\/-\/- +-\/-\/-+-\/-\/-+-\/-\/-+-\/-\/-+-\/-\/-

F(x)\textbar{} y \textbar{} y \textbar{} y \textbar{} y \textbar{} y

де замість символу y - значення у форматі з плаваючої крапкою з точністю
до двох знаків після крапки або ціле, вирівняне по центру функцій:

а) F(x) = exp(-x*x); б) F(x) -- квадратний корінь з x
\end{quote}

\subsubsection{0.1. Введення (дійсні
числа)}\label{ux432ux432ux435ux434ux435ux43dux43dux44f-ux434ux456ux439ux441ux43dux456-ux447ux438ux441ux43bux430}

\begin{enumerate}
\def\labelenumi{\arabic{enumi})}
\item
  Ввести дійсне число градусів Цельсія C (на екрані повинна бути
  підказка, що ввести) та обчислити й вивести число F в дійсному форматі
  -- та сама температура в градусах Фаренгейта за формулою
  \(F = \frac{9C}{5} + 32\). Результат вивести в окремому рядку вигляду
  «F=*** », де замість зірок представлення числа в найкоротшому вигляді
  з можливих.
\item
  Ввести дійсне число x та підрахуйте без та за допомогою математичних
  функцій Сі її цілу та дробову частину, найменше ціле число, що більше
  x та найбільше ціле, що менше x, а також його округлене значення.
  Перевірте результат роботи для від'ємного числа.
\item
  Ввести в двох різних рядках послідовно два дійсних числа та обчислити
  значення їх різниці та добутку. Результат вивести в десятковому
  представленні (з фіксованою крапкою).
\item
  Ввести два дійсних числа записаних через пробіли в одному рядку та
  обчислити значення їх середнього арифметичного та середнього
  гармонічного. Результат вивести в науковому та десятковому
  представленні.
\item
  Три дійсні числа вводяться як рядок вигляду
\end{enumerate}

\begin{quote}
А=ххх.ххх, B=xxExxx C=xxx.xxxx , де ``A='',''B='', ``C='' символи, що
повинні бути присутніми та ігноруються при введенні (Бажано не
використовувати рядковий тип при введенні).

Обчисліть їх середнє арифметичне та середнє гармонічне та виведіть у
науковому та форматі з фіксованою крапкою.
\end{quote}

\begin{enumerate}
\def\labelenumi{\arabic{enumi})}
\item
  Ввести дійсне число від 0 до 10000 та вивести його 8 ступінь з
  точністю до 20 знаків до десяткової коми та 4 значками після
  десяткової коми.
\item
  На терміналі вводяться 20 цифр. Перші 10 цифр -- це перше ціле число,
  останні 10 цифр -- друге. Введіть їх (не використовуючі рядковий тип)
  та обчисліть і виведіть їх суму.
\end{enumerate}

\begin{quote}
8) Вивести на екран текст:

а) б)

a a a a-\/-\/-\/-\/-\/-\/-\/-\/-\/-a

a a а \textbar{} a \textbar{}

a a a a-\/-\/-\/-\/-\/-\/-\/-\/-\/-a

де a -- введене з клавіатури дійсне число менше 100 (прослідкуйте, щоб
воно а) мало не більше 5 значущих цифр, б) мало рівно 5 значущих цифр).
\end{quote}

\subsubsection{0.2. Використання стандартних математичних
функцій}\label{ux432ux438ux43aux43eux440ux438ux441ux442ux430ux43dux43dux44f-ux441ux442ux430ux43dux434ux430ux440ux442ux43dux438ux445-ux43cux430ux442ux435ux43cux430ux442ux438ux447ux43dux438ux445-ux444ux443ux43dux43aux446ux456ux439}

\begin{enumerate}
\def\labelenumi{\arabic{enumi})}
\item
  Ввести дійсне число х та обчислити значення функції тригонометричного
  косинуса для нього.
\item
  Обчислити гіпотенузу \emph{c} прямокутного трикутника за катетами
  \emph{a} та \emph{b}.
\item
  Обчислити площу трикутника \emph{S} за трьома сторонами \emph{a},
  \emph{b}, \emph{c}.
\item
  Обчислити площу еліпса за координатами його радіусів.
\item
  В трикутнику відомо довжини всіх сторін. Обчислити довжини його:
\end{enumerate}

\begin{quote}
а) медіан;

б) бісектрис;

в) висот.

6) Трикутник заданий величинами своїх кутів та радіусом вписаного кола.
Обчисліть його площу.
\end{quote}

7) Трикутник заданий довжиною своїх сторін. Знайти та вивести величину
кутів трикутника у радіанах та градусах.

8) Обчислити відстань від точки \(\left( x_{0},y_{0} \right)\) до:

\begin{quote}
а) заданої точки \(\left( x,y \right);\)

б) заданої прямої \(\mathrm{\text{ax}} + \mathrm{\text{by}} + c = 0\);

в) точки перетину прямих \(x + \mathrm{\text{by}} + c = 0\) і
\(\mathrm{\text{ax}} + y + c = 0,\) де
\(\mathrm{\text{ab}} \neq 1\mathrm{.}\)

9) Знайти об'єм циліндра, якщо відомо його радіус основи та висоту.

10) Знайти об'єм конуса, якщо відомо його радіус основи та висоту.

11) Знайти об'єм тора з внутрішнім радіусом \emph{r} і зовнішнім
радіусом \emph{R.}

12) Знайти корені квадратного рівняння з коефіцієнтами a,b,c, якщо
відомо, що обидва корені в ньому існують. Перевірте ваш розв'язок на
коефіцієнтах рівняння a=3,b=100,c=2.
\end{quote}

\subsubsection{0.3. Декларація та використання
функцій}\label{ux434ux435ux43aux43bux430ux440ux430ux446ux456ux44f-ux442ux430-ux432ux438ux43aux43eux440ux438ux441ux442ux430ux43dux43dux44f-ux444ux443ux43dux43aux446ux456ux439}

\begin{quote}
1) Напишіть функцію, яка за найменшу кількість арифметичних операцій,
обчислює значення многочлена для введеного з клавіатури значення
\emph{x}:

а) \(y = x^{4} + 2x^{2} + 1;\) б)\(y = x^{4} + x^{3} + x^{2} + x + 1;\)

в)
\(y = x^{5} + 5x^{4} + \mathrm{10}x^{3} + \mathrm{10}x^{2} + 5x + 1;\)
г) \(y = x^{9} + x^{3} + 1;\)

ґ) \(y = \mathrm{16}x^{4} + 8x^{3} + 4x^{2} + 2x + 1;\) д)
\(y = x^{5} + x^{3} + x\mathrm{.}\)

2) Скласти функцію для обчислення значення многочлена від двох змінних
для введеної з клавіатури пари чисел \((x,y)\):

а)
\(f\left( x,y \right) = x^{3} + 3x^{2}y + 3\mathrm{\text{xy}}^{2} + y^{3};\)

б) \(f\left( x,y \right) = x^{2}y^{2} + x^{3}y^{3} + x^{4}y^{4};\)

в)
\(f\left( x,y \right) = x + y + x^{2} + y^{2} + x^{3} + y^{3} + x^{4} + y^{4}\mathrm{.}\)

3) Напишіть функцію Rosenbrock2d(x,y) =
\(100(x^{2} - y)^{2} + (x - 1)^{2}\) та перевірте її результат на
довільних трьох парах дійсних чисел.

4) Трикутник заданий довжинами своїх сторін. Знайти периметр та площу
цього трикутника. Перевірте для значень сторін
\(a = 3,b = c = 3.5 + 3*2^{- 111}\)

5) Трикутник вводиться координатами своїх вершин, які вводяться так: в
першому рядку через пробіл два дійсних числа --- координати точки А,
пропускається рядок, в третьому рядку через пробіл два дійсних числа ---
координати Б, пропускається рядок, через пробіл --- координати точки С.
Підрахувати площу трикутника. (Вказівка: напишіть функції підрахунку
довжини відрізка та функції обчислення площі трикутника за довжинами
сторін)

6) Напишіть власні функції, що обчислюють наступні вирази та відповідні
власні функції, що будуть рахувати похідні даних функцій(Приклад,
функція \(f\left( x \right) = identity\left( x \right) = x\)\emph{,} її
похідна
\(g\left( x \right) = identity\_ derivative\left( x \right) = 1\)) :

\emph{а)}
\(f\left( x \right) = th\left( x \right) = \frac{(e^{x} - e^{- x})}{(e^{x} + e^{- x})}\)

б)
\(f\left( x \right) = \text{Bent\ }\left( x \right) = \frac{\sqrt{x^{2} + 1} - 1}{2} + x\)

в)
\(f\left( x \right) = \text{Softsign}\left( x \right) = \frac{x}{1 + \left| x \right|}\)

г) \(f\left( x \right) = arctg\left( x \right) = tg^{- 1}(x)\)

д) \(f\left( x \right) = gauss\left( x \right) = e^{- x^{2}}\)

е)
\(f\left( x \right) = \text{SoftPlus}\left( x \right) = ln(1 + e^{x})\)

ж) \(f\left( x \right) = sigmoid\left( x \right) = (1 + e^{- x})^{- 1}\)

з)
\(f\left( x \right) = invsqrt(x,\alpha) = \ \frac{x}{\sqrt{1 + \alpha x^{2}}}\)

і)
\(f\left( x \right) = sigmweight\left( x \right) = x(1 + e^{- x})^{- 1}\)
\end{quote}

\subsubsection{Локалізація}\label{ux43bux43eux43aux430ux43bux456ux437ux430ux446ux456ux44f}

\begin{enumerate}
\def\labelenumi{\arabic{enumi})}
\item
  Обчислить результати наступних виразів та вивести на екран напис
  українською мовою «Результат дорівнює:»:
\end{enumerate}

\begin{quote}
2+3; 4.5*56; 2/3.0.
\end{quote}

\begin{enumerate}
\def\labelenumi{\arabic{enumi})}
\item
  Виведіть напис : «Введить ім``я:»
\end{enumerate}

\begin{quote}
Введіть з нового рядка ваше ім'я (наприклад, «Вася» ) та виведіть
привітання вигляду «Привіт, Вася!»

3) Введіть два цілих числа, що позначають грошовий тип --- гривні та
копійки та виведіть значення як грошовий тип в англійських,
американських та українських локалізаціях.

4) Введіть два дійсних числа, які записані за допомогою десяткової коми
та виведіть їх середнє геометричне в такому ж форматі

5) Введіть дату (число, місяць, рік) та виведіть її значення в
німецький, американський та українських локалізаціях.
\end{quote}

\subsection{1. Цілі числа та умовні
конструкції}\label{ux446ux456ux43bux456-ux447ux438ux441ux43bux430-ux442ux430-ux443ux43cux43eux432ux43dux456-ux43aux43eux43dux441ux442ux440ux443ux43aux446ux456ux457}

\subsection{ 1.0. Цілі числа та їх
типи}\label{ux446ux456ux43bux456-ux447ux438ux441ux43bux430-ux442ux430-ux457ux445-ux442ux438ux43fux438}

\begin{quote}
1)Дано натуральне тризначне число. Знайти:

а) кількість одиниць, десятків і сотень цього числа;

б) суму цифр цього числа;

в) число, утворене при читанні заданого числа справа наліво.

2)Ввести натуральне тризначне число. Якщо в ньому всі 3 цифри різні, то
вивести всі числа, які утворюються при перестановці цифр заданого числа.
\end{quote}

3) Введіть три цілих числа, записаних через кому в одному рядку та
підрахуйте їх добуток якщо всі ці числа гарантовано по модулю менші а)
\(2^{10}\) б) \(2^{21}\)

4) Напишіть програму, що з'ясовує скільки байтів на цілий та довгий
цілий тип виділяє компілятор, а також чи підтримує він довгий тип та
скільки на нього виділяється байтів.

5) Напишіть функцію, що гарантовано приймає у якості аргументів 8-бітні
натуральні числа та обчислює їх добуток як гарантовано 16-бітне
натуральне число.

6) Введіть два натуральних 32-бітних числа та виведіть їх суму як
32-бітне число, якщо немає переповнення типу. В противному випадку
виведіть про це повідомлення. Аналогічно підрахуйте добуток двох цілих
32-бітних чисел.

7) На терміналі вводяться 30 цифр. Перші 15 цифр -- це перше ціле число,
останні 15 цифр -- друге. Введіть їх та обчисліть і виведіть їх добуток
за допомогою двох чисел.

8) З'ясуйте максимальну кількість 8-бітних цілих на вашому комп'ютері,
яку можна перемножити між собою та користуючись лише стандартними типами
Сі-бібліотеки отримати коректний результат.

\subsubsection{1.1. Прості умовні
конструкції}\label{ux43fux440ux43eux441ux442ux456-ux443ux43cux43eux432ux43dux456-ux43aux43eux43dux441ux442ux440ux443ux43aux446ux456ux457}

\begin{quote}
\emph{Спробуйте розв'язати наступні 5 задач з допомогою тернарного
оператору.}
\end{quote}

\begin{enumerate}
\def\labelenumi{\arabic{enumi})}
\item
  Визначити більше та менше з двох чисел, введених з клавіатури.
\end{enumerate}

\begin{quote}
2) Дано три дійсних числа. Скласти програму для знаходження числа:

a) найбільшого за модулем;

б) найменшого за модулем.

3)Дано три дійсних числа \emph{x, y} і \emph{z}. Скласти програму для
обчислення:

а)
\(\mathrm{\max}\left( x + y + z,\mathrm{\text{xy}} - \mathrm{\text{xz}} + \mathrm{\text{yz}},\mathrm{\text{xyz}} \right);\)
б)
\(\mathrm{\max}\left( \mathrm{\text{xy}},\mathrm{\text{xz}},\mathrm{\text{yz}} \right)\mathrm{.}\)

4) Дано три дійсних числа \emph{x,~y}~і \emph{z}. Визначити кількість:

а) різних серед них; б) однакових серед них;

в) чисел, що є більшими за їхнє середнє арифметичне значення;

г) чисел, що є більшими за введене з клавіатури число \(a\).

5) Обчислити значення функцій:

а) \(f\left( x \right) = \left| x \right|;\) б)
\(f\left( x \right) = \left| \left| x \right| - 1 \right| - 1;\)

в) \(f\left( x \right) = sign(x)\) г)
\(f\left( x \right) = \mathrm{\sin}\left| x \right|;\)

6) Перевірити, чи існує трикутник із заданими сторонами \emph{a,b,c}.
Якщо так, то визначити, який він: (гострокутний, прямокутний,
тупокутний).

7) Визначити, скільки розв'язків має рівняння та розв'язати його:

а) \(\mathrm{\text{ax}}^{2} + \mathrm{\text{bx}} + c = 0;\) б)
\(\mathrm{\text{ax}}^{4} + \mathrm{\text{bx}}^{2} + c = 0\mathrm{.}\)

8) Визначити, скільки розв'язків має система рівнянь і розв'язати її:

а) \(\left\{ \begin{matrix}
a_{1}x + b_{1}y + c_{1} = 0 \\
a_{2}x + b_{2}y + c_{2} = 0; \\
\end{matrix} \right.\ \) б) \(\left\{ \begin{matrix}
\left| x + y \right| = 1 \\
a_{2}x + b_{2}y + c_{2} = 0 \\
\end{matrix} \right.\ \)
\end{quote}

9) Знайти число точок пеpетину кола \(x^{2} + y^{2} = r^{2}\) з
відpізком \(x = a,\ b \leq y \leq b + c^{2}\) .

\begin{quote}
10) Скласти програму, яка по колу
\({(x - v)}^{2} + ({y - u)}^{2} = r^{2}\) та пpямій \(ax + by + c = 0\)
встановлює, який випадок має місце:

а) дві точки пеpетину;

б) одна точка дотику;

в) жодної спільної точки.

11) З'ясувати, чи пеpетинаються два кола на площині.

12) Задано два квадрати, сторони яких паралельні координатним осям.
З'ясувати, чи перетинаються вони. Якщо так, то знайти координати лівого
нижнього та правого верхнього кутів прямокутника, що є їхнім перетином.

13) Дано два прямокутники, сторони яких паралельні координатним осям.
Відомо координати лівого нижнього та правого верхнього кутів кожного з
прямокутників. Знайти координати лівого нижнього та правого верхнього
кутів мінімального прямокутника, що містить задані прямокутники.

14) Записати функції, що істинні тоді й тільки тоді, коли:

а) натуральне число n -- парне;

б) остання цифра числа n -- 0;

в) ціле число n кратне натуральному числу m;

г) натуральні числа n і k одночасно кратні натуральному числу m

ґ) сума першої і другої цифри двозначного натурального числа - двозначне
число;

д) число x більше за число y не менше, ніж на 6;

е) принаймні одне з чисел x, y або z більше за 100;

є) тільки одне з чисел x, y або z менше за 1000.

15) Створити функцію, яка перевіряє, чи належить початок координат
трикутнику, що заданий координатами своїх вершин.

16) Точка площини задана декартовими координатами (x, y). Перевірити, чи
належить вона трикутнику з вершинами А(y1, x1), B (x2, y2), C (x3, y3).

17) Точка простору задана декартовими координатами (x, y, z).
Перевірити, чи належить вона кулі з радіусом R i центром у початку
координат.

18) Точка простору задана декартовими координатами (x, y, z).
Перевірити, чи належить вона циліндру, вісь якого збігається з віссю O.
Висота дорівнює h, а нижня основа лежить у площині Oxy та має радіус r

19) Реалізуйте функції та напишіть відповідну до кожної з них функцію,
що буде рахувати їх похідні (за нескінченість прийміть число MAXDBL):

а) onestep(x) = \(\left\{ \begin{matrix}
1,x \geq 0 \\
0,x < 0 \\
\end{matrix} \right.\ \)

б)ReLu(x) =\(max(0,x)\)

в)pleakyReLu(x,a)= \(\left\{ \begin{matrix}
ax,\ x < 0 \\
0,\ x \geq 0 \\
\end{matrix} \right.\ \)

г) \includegraphics{media/image1.png}eLu(a,x) =\(\left\{ \begin{matrix}
a(e^{x} - 1),x < 0 \\
0,\ x \geq 0 \\
\end{matrix} \right.\ \)

д) sReLu(tl,tr,al,ar,x)=\(\left\{ \begin{matrix}
tl + al\left( x - tl \right),x \leq tl \\
0,tl < x < tr \\
tr + ar\left( x - tr \right),x \geq tr \\
\end{matrix} \right.\ \)

е) isReLu(a,x)= \(\left\{ \begin{matrix}
\frac{x}{\sqrt{1 + ax^{2}}},x < 0 \\
x,\ x \geq 0 \\
\end{matrix} \right.\ \)

ж) SoftExponential(a,x) = \(\left\{ \begin{matrix}
 - \frac{ln(1 - a(x + a)}{a},a < 0 \\
x,a = 0 \\
\frac{e^{\text{ax}} - 1}{a} + a,a > 0 \\
\end{matrix} \right.\ \)

з) sinc(x)= \(\left\{ \begin{matrix}
1,\ x = 0 \\
\frac{\sin x}{x},x \neq 0 \\
\end{matrix} \right.\ \)

\emph{Розв'яжіть задачі за допомогою команди вибору (альтернативи)}

20) Вводиться натуральне число, що означає кількість днів, що пройшли з
початку поточного року. Виведіть день тижня, на який припадає цей день.

21) Ввести натуральне число менше 10 (цифру) та вивести назву цієї цифри
рядком

22) За даним числом k (k\textless{}100) вивести в лінгвістично коректній
формі фразу «Йому k рок(ів, и, рік)»
\end{quote}

\subsection{1.2 Цикли}\label{ux446ux438ux43aux43bux438}

\begin{enumerate}
\def\labelenumi{\arabic{enumi})}
\item
  Скласти функцію обчислення за даним дійсним x та натуральним n число
  \(y = \sin(\sin(\ldots\sin(x)\ldots))\) \((\)
  \(\mathrm{\ }n\mathrm{\ \ raziv}).\)
\end{enumerate}

\begin{quote}
2) Скласти функції для обчислення значень многочленів і виконати їх при
заданих значеннях аргументів:

а)
\(y = x^{n} + x^{n - 1} + \ldots + x^{2} + x + 1\mathrm{\text{\ \ \ \ \ \ \ \ \ \ \ \ \ \ \ \ \ \ \ \ }}n = 3,x = 2;\)

б)
\(y = x^{2^{n}} + x^{2^{n - 1}} + \ldots + x^{4} + x^{2} + 1\mathrm{\text{\ \ \ \ \ \ \ \ \ \ \ \ \ \ \ }}n = 4,x = 1;\)

в)
\(y = x^{3^{n}} + x^{3^{n - 1}} + \ldots + x^{9} + x^{3} + 1\mathrm{\text{\ \ \ \ \ \ \ \ \ \ \ \ \ \ \ \ }}n = 3,x = 1;\)

г)
\(y = x^{2^{n}}y^{n} + x^{2^{n - 1}}y^{n - 1} + \ldots + x^{2}y + 1\mathrm{\text{\ \ \ \ \ \ \ \ \ \ }}n = 4,x = 1,y = 2;\)

д)
\(y = x^{1^{2}} + x^{2^{2}} + \ldots + x^{n^{2}}\mathrm{,\ \ \ \ \ \ \ \ \ \ \ \ \ \ \ \ \ \ \ \ \ \ \ \ \ \ \ \ \ }n = 5,x = - 1.\)

3) Вивести на екран такий рядок:

n! = 1*2*3*4*5*...*n,

де n -- введене з клавіатури натуральне число.

4) \textbf{Дано натуральне число} \(\text{n.}\) Написати програми
обчислення значень виразів при заданому значенні \(x\):

а) \(1 + (x - 1) + (x - 1)^{2} + \ldots + (x - 1)^{n};\)

б)
\(1 + \frac{1}{x^{2} + 1} + \frac{1}{(x^{2} + 1)^{2}} + \ldots + \frac{1}{(x^{2} + 1)^{n}};\)

в) \(x + (2x)^{2} + \ldots + ((n - 1)x)^{n - 1} + (nx)^{n};\)

г) \(1 + \sin x + \operatorname{}x + \ldots + \operatorname{}x.\)

5) Дано натуральне число \emph{n}. Скласти програму обчислення
факторіала \emph{y=n!}, використовуючи

а) цикл по діапазону із зростанням;

б) цикл по діапазону зі спаданням.

6) Скласти функцію обчислення подвійного факторіала натурального числа
\(n\mathrm{\text{\ \ }}y = n!!.\) Скласти функції обчислення виразу
\(y = n!n!!(n + 1)!!.\)

\emph{\emph{Вказівка}}. За означенням
\end{quote}

\[n!! = \left\{ \begin{matrix}
1 \cdot 3 \cdot 5 \cdot \ldots \cdot n,\mathrm{\ \ iakshcho\ }n - \mathrm{neparne,} \\
2 \cdot 4 \cdot 6 \cdot \ldots \cdot n,\mathrm{\ \ iakshcho\ }n - \mathrm{\ \ parne.} \\
\end{matrix} \right.\ \]

\begin{quote}
8) Скласти програму обчислення

а) \(\sqrt{2 + \sqrt{2 + \ldots + \sqrt{2}}}\) (\emph{п} коренів),

б) \(\sqrt{3 + \sqrt{6 + \ldots + \sqrt{3(n - 1) + \sqrt{3n}}}}.\)

9) Скласти програми обчислення значень многочленів

а)
\(y = nx^{n - 1} + (n - 1)x^{n - 2} + \ldots + 2x + 1,\mathrm{\text{\ \ \ }}(x < 1,n \geq 0);\)

б)
\(y = \sum_{k = 0}^{n}{kx^{k}(1 - x)^{n - k}},\mathrm{\text{\ \ \ \ \ \ \ \ \ \ \ \ \ \ \ \ \ \ \ \ \ \ \ \ \ \ }}(0 < x < 1,n \geq 0);\)

в)
\(y = 1 + \frac{x}{1!} + \frac{x^{2}}{2!} + \frac{x^{3}}{3!} + \ldots + \frac{x^{n}}{n!},\mathrm{\text{\ \ \ \ \ \ \ \ \ \ \ }}(\mathrm{diisne\ }x < 1,n \geq 0).\)

10) Для довільного цілого числа \(m \geq 1\)знайти найбільше ціле \(k\),
при якому \(4^{k} \leq m.\)

11) Для заданого натурального числа \(n\) одержати найменше число
вигляду \(2^{r}\), яке перевищує \(n\)\emph{.}

12) Знайдіть машинний нуль для вашого компілятора, тобто таке дійсне
число \(a > 0,\) що \(1 + a = 1\) буде істиною.

\emph{Вказівка:} в циклі ділить значення \(a\) на 2 доки не виконується
вказана вище рівність.

13) Ввести послідовність наступним чином: користувачу виводиться напис
``a{[}**{]}= '', де замість ** стоїть номер числа, що вводиться. Тобто
там виводяться написи ``a{[}0{]}= '', і після знаку рівності користувач
вводить число, ``a{[}1{]}= '', і після знаку рівності користувач вводить
число і так далі доки користувач не введе число 0. Після цього потрібно
вивести суму введених чисел (масив чисел заводити необов'язково).

14) Введіть послідовність цілих ненульових чисел (тобто введення
закінчується коли ми вводимо 0) та виведіть середнє арифметичне введених
чисел та середнє геометричне.

15) Введіть послідовність цілих ненульових чисел (тобто введення
закінчується коли ми вводимо 0). Визначити кількість змін знаку в цій
послідовності. Наприклад, у послідовності 1, −34, 8,14, −5, 0 знак
змінюється три рази.

16) Введіть послідовність натуральних ненульових чисел (тобто введення
закінчується коли ми вводимо 0). Визначити порядковий номер найменшого з
них.

17) Введіть послідовність дійсних ненульових чисел (тобто введення
закінчується коли ми вводимо 0). Визначити величину найбільшого серед
від`ємних членів цієї послідовності. Якщо від'ємних чисел немає вивести
найменший серед додатних членів.

18) Банк пропонує річну ставку по депозиту A та 15\% по вкладу додаються
до основної суми депозиту кожен рік. Ви кладете в цей банк D гривень.
Скільки років потрібно чекати, щоб сума вкладу зросла до очікуваної суми
P?

19) Скласти програми для обчислення елементів послідовностей. Операцію
піднесення до степені та функцію обчислення факторіалу не
використовувати.

а) \(x_{k} = \frac{x^{k}}{k}\ (k \geq 1);\) д)
\(x_{k} = \frac{x^{2k}}{(2k)!}\ (k \geq 0);\);

б) \(x_{k} = \frac{( - 1)^{k}x^{k}}{k}\ (k \geq 1);\) е)
\(x_{k} = \frac{x^{2k + 1}}{(2k + 1)!}\ (k \geq 0);\);

в) \(x_{k} = \frac{x^{k}}{k!}\ (k \geq 0);\) ж)
\(x_{k} = \frac{( - 1)^{k}x^{2k}}{(2k)!}\ (k \geq 0);\);

г) \(x_{k} = \frac{( - 1)^{k}x^{k}}{k!}\ (k \geq 0);\) з)
\(x_{k} = \frac{( - 1)^{k}x^{2k + 1}}{(2k + 1)!}\ (k \geq 0);\)

\textbf{20)} Задане натуральне число \emph{n}. Скласти програми
обчислення добутків

а)
\(p = \left( 1 + \frac{1}{1^{2}} \right)\left( 1 + \frac{1}{2^{2}} \right)\ldots\left( 1 + \frac{1}{n^{2}} \right),\mathrm{\ \ \ \ n > 2};\)

б)
\(p = \left( 1 - \frac{1}{2^{2}} \right)\left( 1 - \frac{1}{3^{2}} \right)\ldots\left( 1 + \frac{1}{n^{2}} \right),\mathrm{\ \ \ \ n > 2.}\)

21) Скласти програму друку таблиці значень функції \(y = \sin x\) на
відрізку {[}0,1{]} з кроком \(h = 0.1.\)

22) Скласти програму визначення кількості тризначних натуральних чисел,
сума цифр яких дорівнює \(n\ (n > 1).\) Операцію ділення не
використовувати.

23) Дано \emph{n} цілих чисел. Скласти програму, що визначає, скільки з
них більші за своїх "сусідів", тобто попереднього та наступного чисел.

24) Задані натуральне число \emph{n}, дійсні числа
\(y_{1},\ldots y_{n}.\) Скласти програму визначення

а) \(\max(\left| z_{1} \right|,\ldots,\left| z_{n} \right|),\) де
\(z_{i} = \left\{ \begin{matrix}
\& y_{i},\mathrm{\ \ \ pri\ }\left| y_{i} \right| \leq 2, \\
\& 0.5,\mathrm{\ \ u\ inshikh\ vipadkakh\ \ \ } \\
\end{matrix} \right.\ \);

б) \(\min(\left| z_{1} \right|,\ldots,\left| z_{n} \right|),\) де
\(z_{i} = \left\{ \begin{matrix}
\& y_{i},\mathrm{\ \ \ pri\ }\left| y_{i} \right| \geq 1, \\
\& 2,\mathrm{\ \ u\ inshikh\ vipadkakh\ \ \ } \\
\end{matrix} \right.\ \);

в) \(z_{1} + z_{2} + \ldots + z_{n},\) де
\(z_{i} = \left\{ \begin{matrix}
\& y_{i},\mathrm{\ \ \ pri\ 0 <}\mathrm{y}_{i} < 10, \\
\& 1,\mathrm{\ \ u\ inshikh\ vipadkakh\ \ \ } \\
\end{matrix} \right.\ \)
\end{quote}

25) Дано натуральне число n. Викинути із запису числа n цифри 0 і 5,
залишивши порядок інших цифр. Наприклад, з числа 59015509 повинно вийти
919.

26) Знайти період десяткового дробу для відношення n/m для заданих
натуральних чисел n та m.

\begin{quote}
27*) Скоротити дріб n/m для заданих цілого числа n та натурального числа
m.

28*) Ввести натуральні числа a і b та натуральне число n. Чи можна
представити число n у вигляді n= k*a + m*b, де k та m --- натуральні
числа? Якщо можна --- то знайдіть такі числа k та m, що мають найменшу
суму модулів.

29) Представити дане натуральне число як суму двох квадратів натуральних
чисел. Якщо це неможливо представити як суму трьох квадратів. Якщо і це
неможливо, представити у вигляді суми чотирьох квадратів натуральних
чисел.

30) Знайти всі цілі корені кубічного рівняння . Вказівка: цілі корені
повинні бути дільниками (від'ємними або додатними дільниками вільного
члену d).

31) Напишіть функцію, яка розраховує для даного натурального числа n
значення функції Ойлера (кількість чисел від 1 до n, взаємно простих з
n).

32*) Ввести натуральне число \(d > 1\) та натуральне число m. Знайдіть
мінімальну кількість натуральних чисел вигляду \(\ x^{d}\ \) (d-ступенів
натуральних чисел) сума яких дорівнює m.
\end{quote}

\begin{enumerate}
\def\labelenumi{\arabic{enumi}.}
\item
  \textbf{Рекурентні співвідношення}
\end{enumerate}

\begin{quote}
1) Числами Фібоначчі називається числова послідовність
\(\left\{ F_{n} \right\}\), задана рекурентним співвідношенням другого
порядку
\(F_{0} = 0,F_{1} = 1,F_{k} = F_{k - 1} + F_{k},\ k = 2,3,\ldots\) .
Скласти функцію для обчислення \(F_{n}\ \) за номером члену.
\end{quote}

\begin{enumerate}
\def\labelenumi{\arabic{enumi})}
\item
  Маємо дійсне число \emph{a}. Скласти програми обчислення:
\end{enumerate}

\begin{quote}
а) серед чисел
\(1,1 + \frac{1}{2},1 + \frac{1}{2} + \frac{1}{3},\ldots\) першого,
більшого за \emph{;}

б) такого найменшого , що
\(1 + \frac{1}{2} + \ldots + \frac{1}{n} > a.\)
\end{quote}

\begin{enumerate}
\def\labelenumi{\arabic{enumi})}
\item
  Введіть натуральне число n. Далі утворить рекурентну послідовність
  \(a_{i}\) за наступним правилом: \(a_{0} = n\). Якщо \(a_{k}\) парне,
  то \(a_{k + 1} = a_{k}/2\) , якщо --- непарне,
  то\(\ a_{k + 1} = 4a_{k} + 1\) . Доведіть що для n\textless{}1000 ця
  послідовність буду збігатись до одиниці. Знайдіть серед цих n число,
  якому потрібно максимальна кількість кроків для досягнення одиниці.
\item
  Скласти програми для обчислення добутків:
\end{enumerate}

\begin{quote}
а) \(P_{n} = \prod_{i = 1}^{n}\left( 2 + \frac{1}{i!} \right);\) б)
\(P_{n} = \prod_{i = 1}^{n}\left( \frac{i + 1}{i + 2} \right);\)

в) \(P_{n} = \prod_{i = 1}^{n}\frac{1}{(i + 1)!};\); г)
\(P_{n} = \prod_{i = 1}^{n}\frac{1}{i^{i} + 1}.\)

\emph{\emph{Вказівка}}. Добуток \emph{P\textsubscript{n}} обчислити за
допомогою рекурентного співвідношення \(P_{0} = 1,\)
\(P_{k} = P_{k - 1}*a_{k},\) \(k = 1,2,\ldots,n,\)
\emph{k=}1,2\emph{,...,n,} де \(a_{k}\) - \emph{k}- тий множник.
\end{quote}

\begin{enumerate}
\def\labelenumi{\arabic{enumi})}
\item
  Скласти програми обчислення:
\end{enumerate}

\begin{quote}
а) номера найбільшого числа Фібоначчі, яке не перевищує задане число
\emph{a;}

б) номера найменшого числа Фібоначчі, яке більше заданого числа
\emph{a;}

в) суми всіх чисел Фібоначчі, які не перевищують 1000.
\end{quote}

\begin{enumerate}
\def\labelenumi{\arabic{enumi})}
\item
  \emph{Вводиться послідовність натуральних чисел (починаючи з першого
  члена) доки не введемо 0. Обчислити суму тих членів послідовності,
  порядкові номери яких - числа Фібоначчі.}
\item
  Скласти програми для обчислення найменшого додатного члена числових
  послідовностей, які задаються рекурентними співвідношеннями, та його
  номера
\item
  а)
  \(x_{n} = x_{n - 1} + x_{n - 2} + 100,\mathrm{\text{\ \ \ \ \ }}x_{1} = x_{2} = - 99,\mathrm{\text{\ \ \ }}n = 3,4,\ldots;\)
\end{enumerate}

\begin{quote}
б)
\(x_{n} = x_{n - 1} + x_{n - 2} + x_{n - 3} + 200,\mathrm{\text{\ \ \ \ \ }}x_{1} = x_{2} = x_{3} - 99,\mathrm{\text{\ \ \ }}n = 4,5,\ldots;\)

в)
\(x_{n} = x_{n - 1} + x_{n - 3} + 100,\mathrm{\text{\ \ \ \ \ }}x_{1} = x_{2} = x_{3} = - 99,\mathrm{\text{\ \ \ }}n = 4,5,\ldots\)
\end{quote}

\begin{enumerate}
\def\labelenumi{\arabic{enumi})}
\item
  Скласти програми для обчислення ланцюгових дробів
\end{enumerate}

\begin{quote}
а) \(b_{n} = b + \frac{1}{b + \frac{1}{b + \ddots + \frac{1}{b}};}\); б)
\(\lambda_{n} = 2 + \frac{1}{6 + \frac{1}{10 + \ddots + \frac{1}{4n + 2}};}\)

в)
\(x_{2n} = 1 + \frac{1}{2 + \frac{1}{1 + \frac{1}{2 + \frac{1}{1 + \ddots + \frac{1}{2}}}.};}\)

\emph{\emph{Вказівка}}. Використати рекурентні співвідношення

а)
\(b_{0} = b,b_{k} = b + \frac{1}{b_{k - 1}},\mathrm{\text{\ \ }}k = 1,2,\ldots,n;\)

в)
\(b_{0} = 4n + 2,b_{k} = 4(n - k) + 2 + \frac{1}{b_{k - 1}},\mathrm{\text{\ \ }}k = 1,2,\ldots,n.\)
\end{quote}

\begin{enumerate}
\def\labelenumi{\arabic{enumi})}
\item
  Скласти програми обчислення довільного елемента послідовностей,
  заданих рекурентними співвідношеннями
\item
  а)
  \(v_{0} = 1,v_{1} = 0.3,\mathrm{\text{\ \ \ \ \ \ \ \ }}v_{i} = (i + 2)v_{i - 2},\mathrm{\text{\ \ \ }}i = 2,3,\ldots\)
\end{enumerate}

\begin{quote}
б)
\(v_{0} = v_{1} = v_{2} = 1,\mathrm{\text{\ \ \ \ \ \ \ \ }}v_{i} = (i + 4)(v_{i - 1} - 1) + (i + 5)v_{i - 3},\mathrm{\text{\ \ }}i = 3,4,\ldots\)

в)
\(v_{0} = v_{1} = 0,\ v_{2} = \frac{3}{2}\mathrm{,\ \ \ }v_{i} = \frac{i - 2}{(i - 3)^{2} + 1}v_{i - 1} - v_{i - 2}v_{i - 3} + 1,\ i = 2,3,\ldots\)
\end{quote}

\begin{enumerate}
\def\labelenumi{\arabic{enumi})}
\item
  Скласти програму обчислення довільного елемента послідовності
  \(v_{n}\), визначеної системою співвідношень
\end{enumerate}

\[v_{0} = v_{1} = 1,\mathrm{\text{\ \ \ \ }}v_{i} = \frac{u_{i - 1} - v_{i - 1}}{\left| u_{i - 2} + v_{i - 1} \right| + 2},\mathrm{\text{\ \ \ }}i = 2,3,\ldots;\]

\begin{quote}
де
\(u_{0} = u_{1} = 0,\mathrm{\text{\ \ \ \ }}\mathrm{u}_{i} = \frac{u_{i - 1} - u_{i - 2}v_{i - 1} - v_{i - 2}}{1 + u_{i - 1}^{2} + v_{i - 1}^{2}},\mathrm{\text{\ \ \ }}i = 2,3,\ldots;\)
\end{quote}

\begin{enumerate}
\def\labelenumi{\arabic{enumi})}
\item
  Скласти програми для обчислення сум:
\end{enumerate}

\begin{quote}
а)
\(S_{n} = \sum_{k = 1}^{n}{2^{k}a_{k}},\mathrm{\ de\ \ }a_{1} = 0,a_{2} = 1,a_{k} = a_{k - 1} + k*a_{k - 2},\ k = 3,4,\ldots;\)
б)
\(S_{n} = \sum_{k = 1}^{n}\frac{3^{k}}{a_{k}},\mathrm{\ de\ \ }a_{1} = a_{2} = 1,\ a_{k} = \frac{a_{k - 1}}{k} + a_{k - 2},\mathrm{\text{\ \ }}k = 3,4,\ldots;\)

в)
\(S_{n} = \sum_{k = 1}^{n}\frac{k!}{a_{k}},\mathrm{\ de\ \ }a_{1} = a_{2} = 1,\ a_{k} = a_{k - 1} + \frac{a_{k - 1}}{2^{k}},\mathrm{\text{\ \ }}k = 3,4,\ldots;\)

г\emph{)}
\(S_{n} = \sum_{k = 1}^{n}{k!a_{k}},\mathrm{\ de\ \ }a_{1} = 0,a_{2} = 1,\ a_{k} = a_{k - 1} + \frac{a_{k - 2}}{(k - 1)!},\ k = 3,4,\ldots;\)

ґ)
\(S_{n} = \sum_{k = 1}^{n}\frac{a_{k}}{2^{k}},\mathrm{\ de\ \ }a_{1} = a_{2} = a_{3} = 1,\ a_{k} = a_{k - 1} + a_{k - 3},\mathrm{\text{\ \ }}k = 4,5,\ldots;\)

д)
\(S_{n} = \sum_{k = 1}^{n}{\frac{2^{k}}{k!}a_{k}},\mathrm{\ de\ \ }a_{0} = 1,a_{k} = ka_{k - 1} + \frac{1}{k},\mathrm{\text{\ \ }}k = 1,2,\ldots.\)
\end{quote}

\begin{enumerate}
\def\labelenumi{\arabic{enumi})}
\item
  Скласти програми для обчислення сум:
\end{enumerate}

\begin{quote}
а) \(S_{n} = \sum_{k = 1}^{n}\frac{2^{k}}{a_{k} + b_{k}},\) ,

де \(\left\{ \begin{matrix}
\& a_{1} = 0,a_{2} = 1, \\
\& a_{k} = \frac{a_{k - 1}}{k} + a_{k - 2}b_{k}, \\
\end{matrix} \right.\ \) \(\left\{ \begin{matrix}
\& b_{1} = 1,b_{2} = 0, \\
\& b_{k} = b_{k - 1} + a_{k - 1}, \\
\end{matrix} \right.\ \) \(k = 3,4,\ldots;\)

б) \(S_{n} = \sum_{k = 1}^{n}\frac{a_{k}b_{k}}{(k + 1)!},\)

де \(\left\{ \begin{matrix}
\& a_{1} = u, \\
\& a_{k} = 2b_{k - 1} + a_{k - 1}, \\
\end{matrix} \right.\ \) \(\left\{ \begin{matrix}
\& b_{1} = v, \\
\& b_{k} = 2a_{k = 1}^{2} + b_{k - 1}, \\
\end{matrix} \right.\ \) \(k = 2,3,\ldots;\)

\emph{u,v} -- задані дійсні числа;

в)
\(\ S_{n} = \sum_{k = 1}^{n}\frac{2^{k}}{{(1 + a}_{k} + b_{k}){k!}^{}}\)

де \(\left\{ \begin{matrix}
\& a_{1} = 1, \\
\& a_{k} = 3b_{k - 1} + 2a_{k - 1}, \\
\end{matrix} \right.\ \) \(\left\{ \begin{matrix}
\& b_{1} = 1, \\
\& b_{k} = 2a_{k - 1} + b_{k - 1}, \\
\end{matrix} \right.\ \) \(k = 2,3,\ldots;\)

г) \(S_{n} = \sum_{k = 1}^{n}\left( \frac{a_{k}}{b_{k}} \right)^{k},\)

де \(\left\{ \begin{matrix}
\& a_{0} = 1,a_{1} = 2, \\
\& a_{k} = b_{k - 2} + \frac{b_{k}}{2}, \\
\end{matrix} \right.\ \) \(\left\{ \begin{matrix}
\& a_{0} = 5,b_{1} = 5, \\
\& b_{k} = b_{k - 2}^{2} - a_{k - 1}, \\
\end{matrix} \right.\ \) \(k = 2,3,\ldots;\)

д) \(S_{n} = \sum_{k = 1}^{n}\frac{a_{k}}{1 + b_{k}},\)

де \(\left\{ \begin{matrix}
\& a_{0} = 1, \\
\& a_{k} = b_{k - 1}a_{k - 1}, \\
\end{matrix} \right.\ \) \(\left\{ \begin{matrix}
\& b_{0} = 1, \\
\& b_{k} = b_{k - 1} + a_{k - 1}, \\
\end{matrix} \right.\ \) \(k = 1,2,\ldots.\)\emph{.}
\end{quote}

\begin{enumerate}
\def\labelenumi{\arabic{enumi})}
\item
  Скласти програми для обчислення добутків
\end{enumerate}

\begin{quote}
а) \(P_{n} = \prod_{k = 0}^{n}{\frac{a_{k}}{3^{k}},}\) де
\(\left\{ \begin{matrix}
\& a_{0} = a_{1} = 1,\ a_{2} = 3, \\
\& a_{k} = a_{k - 3} + \frac{a_{k - 2}}{2^{k - 1}}, \\
\end{matrix} \right.\ \), \(k = 3,4,\ldots;\)

б) \(P_{n} = \prod_{k = 1}^{n}{a_{k}b_{k},}\)

де \(\left\{ \begin{matrix}
\& a_{1} = 1, \\
\& a_{k} = \left( \sqrt{b_{k - 1}} + a_{k - 1} \right)/5, \\
\end{matrix} \right.\ \) \(\left\{ \begin{matrix}
\& b_{1} = 1, \\
\& b_{k} = 2b_{k - 1} + 5a_{k - 1}^{2}, \\
\end{matrix} \right.\ \) \(k = 2,3,\ldots\)\emph{.}
\end{quote}

\begin{enumerate}
\def\labelenumi{\arabic{enumi})}
\item
  Реалізувати функцію яка з`ясовує, чи входить задана цифра до запису
  заданого натурального числа.
\item
  Реалізувати функцію "обернення" (запису в оберненому порядку цифр)
  заданого натурального числа.
\end{enumerate}

\begin{quote}
\emph{Вказівка.} Для побудови числа використати рекурентне
співвідношення \(y_{0} = 0,y_{i} = y_{i - 1}*10 + a_{i},\) де \(a_{i}\)
\emph{-} наступна цифра числа \(n\) при розгляді цифр справа наліво.
\end{quote}

\begin{enumerate}
\def\labelenumi{\arabic{enumi})}
\item
  Скласти програму, яка визначає потрібний спосіб розміну будь-якої суми
  грошей до 99 коп. за допомогою монет вартістю 1, 2, 5, 10, 25, 50 коп.
\end{enumerate}

\begin{quote}
б) \emph{Розв'яжить цю задачу для будь-якого натурального числа m
(1\textless{}m\textless{}100000) копійок так щоб кількість монет при
цьому була найменша.}
\end{quote}

\begin{enumerate}
\def\labelenumi{\arabic{enumi})}
\item
  Скласти програми наближеного обчислення суми всіх доданків, абсолютна
  величина яких не менше ε\emph{\textgreater{}}0:
\end{enumerate}

\begin{quote}
а) \(y = \sin x = x - \frac{x^{3}}{3!} + \frac{x^{5}}{5!} - \ldots;\)

б) \(y = \cos x = 1 - \frac{x^{2}}{2!} + \frac{x^{4}}{4!} - \ldots;\)

в)
\(y = \operatorname{s}hx = x + \frac{x^{3}}{3!} + \frac{x^{5}}{5!} + \ldots;\)

г) \(y = chx = 1 + \frac{x^{2}}{2!} + \frac{x^{4}}{4!} + \ldots;\)

д) \(y = e^{x} = 1 + \frac{x}{1!} + \frac{x^{2}}{2!} + \ldots;\)

е)
\(y = \ln(1 + x) = x - \frac{x^{2}}{2!} + \frac{x^{3}}{3!} - \ldots,\mathrm{\text{\ \ \ \ \ }}(\left| x \right| < 1);\)

ж)
\(y = \frac{1}{1 + x} = 1 - x + x^{2} - x^{3} + \ldots,\mathrm{\text{\ \ \ \ \ }}(\left| x \right| < 1);\)

з)
\(y = \ln\frac{1 + x}{1 - x} = 2*\left\lbrack \frac{x}{1} + \frac{x^{3}}{3} + \frac{x^{5}}{5} + \ldots \right\rbrack\mathrm{,\ \ \ \ \ }(\left| x \right| < 1);\)

і)
\(y = \frac{1}{(1 + x)^{2}} = 1 - 2*x + 3*x^{2} - \ldots,\mathrm{\text{\ \ \ \ \ }}(\left| x \right| < 1);\)

к)
\(y = \frac{1}{(1 + x)^{3}} = 1 - \frac{2*3}{2}x + \frac{3*4}{2}x^{2} - \frac{4*5}{2}x^{3} + \ldots,\mathrm{\text{\ \ \ \ \ }}(\left| x \right| < 1);\)

л)
\(y = \frac{1}{1 + x^{2}} = 1 - x^{2} + x^{4} - x^{6} + \ldots,\mathrm{\text{\ \ \ \ \ }}(\left| x \right| < 1);\)

м)
\(y = \sqrt{1 + x} = 1 + \frac{1}{2}x - \frac{1}{2*4}x^{2} + \frac{1*3}{2*4*6}x^{3} - \ldots,\mathrm{\text{\ \ \ \ \ }}(\left| x \right| < 1);\)

н)
\(y = \frac{1}{\sqrt{1 + x}} = 1 - \frac{1}{2}x + \frac{1*3}{2*4}x^{2} - \frac{1*3*5}{2*4*6}x^{3} - \ldots,\mathrm{\text{\ \ \ \ \ }}(\left| x \right| < 1);\)

о)
\(y = \mathrm{\text{arc}}\sin x = x + \frac{1}{2}\frac{x^{3}}{3!} + \frac{1*3}{2*4}\frac{x^{5}}{5!} + \ldots,\mathrm{\text{\ \ \ \ \ \ \ }}(\left| x \right|\mathrm{< 1}).\)

\emph{\emph{Вказівка}}. Суму \emph{y} обчислювати за допомогою
рекурентного співвідношення
\(S_{0} = 0,\ S_{k} = S_{k - 1} + a_{k},\ k = 1,2,\ldots,\) де
\(a_{k} - k\)\emph{-}тий доданок, для обчислення якого також складається
рекурентне співвідношення. В якості умови повторення циклу розглядається
умова \(\left| a_{k} \right| \geq \varepsilon.\)
\end{quote}

\begin{enumerate}
\def\labelenumi{\arabic{enumi})}
\item
  Маємо дійсні числа
  \(x,\varepsilon\ (x \neq 0,\varepsilon > 0)\)\emph{.} Обчислити з
  точністю \(\varepsilon\) нескінченну суму і вказати кількість
  врахованих доданків.
\end{enumerate}

\begin{quote}
а) \(\sum_{k = 0}^{\infty}\frac{x^{2k}}{2k!};\) б)
\(\sum_{k = 0}^{\infty}\frac{( - 1)^{k}x^{k}}{(k + 1)^{2}};\)

в) \(\sum_{k = 0}^{\infty}\frac{x^{2k}}{2^{k}k!};\) г)
\(\sum_{k = 0}^{\infty}\frac{( - 1)^{k}x^{2k + 1}}{k!(2k + 1)!}.\).

Рекурсія
\end{quote}

\begin{enumerate}
\def\labelenumi{\arabic{enumi})}
\item
  Маємо ціле \(n > 2\). Скласти програму для обчислення всіх простих
  чисел з діапазону \(\left\lbrack 1,n \right\rbrack.\)
\item
  Скласти програму друку всіх простих дільників заданого натурального
  числа.
\item
  Скласти програму, яка визначає чи є задане натуральне число n
  досконалим, тобто рівним сумі всіх своїх (додатних) дільників, крім
  самого цього числа (наприклад, число 6 - досконале: 6=1+2+3 ).
\end{enumerate}

\begin{quote}
\emph{Вказівка}. Шукаємо суму \emph{S} всіх дільників заданого числа
\emph{n}. Якщо \emph{S=n,} то число, яке перевіряємо, є досконалим.
Перша ідея полягає в знаходженні дільників числа \emph{n} в діапазоні
{[}1\emph{, n div} 2{]}. У відповідності з другою ідеєю пошук ведеться
тільки між 1 та \(\sqrt{n}\) і якщо дільник знайдений, то до суми
\emph{S} додаються як дільник, так і частка.
\end{quote}

\begin{enumerate}
\def\labelenumi{\arabic{enumi})}
\item
  Дано натуральне число \emph{k} . Скласти програму одержання
  \emph{к}-тої цифри послідовності
\end{enumerate}

\begin{quote}
а) 110100100010000 ... , в якій виписані підряд степені 10;

б) 123456789101112 ... , в якій виписані підряд всі натуральні числа;

в) 149162536 ... , в якій виписані підряд квадрати всіх натуральних
чисел;

г) 01123581321 ... , в якій виписані підряд всі числа Фібоначчі.
\end{quote}

\begin{enumerate}
\def\labelenumi{\arabic{enumi})}
\item
  Скласти програму знаходження кореня рівняння \(tgx = x\)на відрізку
  {[}0,001;1,5{]} із заданою точністю \(\varepsilon\), використовуючи
  метод ділення відрізку навпіл.
\item
  Знайти корінь рівняння \(x^{3} + 4x^{2} + x - 6 = 0,\) який міститься
  на відрізку {[}0,2{]}, з заданою точністю
\end{enumerate}

\begin{quote}
\emph{\emph{Вказівка.}} Одним з методів розв`язування рівняння є метод
хорд, який полягає в обчисленні елементів послідовності
\end{quote}

\[u_{0} = a,\backslash n\]

\begin{quote}
до виконання умови \(\left| u_{n} - u_{n - 1} \right| < \varepsilon\). В
умовах нашої задачі \(a = 0,b = 2,\ y(x) = x^{3} + 4x^{2} + x - 6.\)
\end{quote}

\subsection{1.3. Бітові
операції}\label{ux431ux456ux442ux43eux432ux456-ux43eux43fux435ux440ux430ux446ux456ux457}

\begin{enumerate}
\def\labelenumi{\arabic{enumi})}
\item
  Чому дорівнюють наступні вирази: 3\textless{}\textless{}2,
  5\textgreater{}\textgreater{}2, 5 \& 3, n \&1, n \textbar{} 1, n\^{}n,
  \textasciitilde{}0.
\end{enumerate}

\begin{quote}
\emph{В даних задачах k-м бітом числа вважається k-тий біт молодших
розрядів, причому перший біт вважається нульовим. }
\end{quote}

\begin{enumerate}
\def\labelenumi{\arabic{enumi})}
\item
  Ввести натуральне 8-бітове число n\textless{}64 і вивести в
  десятковому вигляді число 2\textsuperscript{n} використовуючи бітові
  операції.
\item
  Ввести ціле число n та натуральне k і вивести ціле число, яке у якого
  k-й біт встановлений в 1, а всі інші біти збігаються з бітами числа n
  на тих же позиціях. Наприклад, якщо введені 9 і 1, відповіддю буде 11.
\item
  Ввести натуральне довге число
  \protect\hypertarget{__DdeLink__4_8035966481}{}{}M та натуральне k.
  Встановіть її k-тий біт рівним нулеві та виведіть отримане число в
  десятковому та шістнадцятковому вигляді.
\item
  Ввести натуральне 64 бітне число M. Встановіть її біт рахуючи справа
  (старші розряди) j рівним одиниці та виведіть отримане число в
  десятковому та шістнадцятковому вигляді.
\item
  Визначить номер першого значущого (ненульового) зліва та справа біта
  натурального числа M.
\item
  Поміняйте місцями перші 8 біт та останні 8 біт натурального числа,
  виведіть отримане число в десятковому та шістнадцятковому вигляді.
\item
  Підрахуйте найбільшу кількість одиничок серед бітів даного числа, що
  йдуть підряд.
\item
  Написати функцію, результатом якого є дане значення x, у якого
  молодший нульовий біт встановлений в 1.
\item
  Написати функцію, результатом якого є дане значення x, у якого все
  біти встановлені в 1, крім молодших n бітів.
\item
  \emph{Описати словами результат наступного виразу: x \& (x-1).}
\item
  \emph{Описати словами результат наступного виразу: x \& (-x).}
\item
  Написати функцію, результатом якого є дане значення x, у якого
  молодший нульовий біт та найстарший біт встановлені в 1.
\item
  Написати функцію, результатом якого є дане значення x, у якого все
  біти встановлені в 1, крім молодших n бітів.
\item
  Підрахуйте кількість нулів серед бітів даного числа.
\item
  Знайдіть номер найстаршого значущого біта в даному 32-бітному числі.
\item
  Ввести натуральне число M. Встановіть її ліві n біт рівним нулеві та
  виведіть отримане число. Встановіть її праві n біт рівним нулеві та
  виведіть отримане число в десятковому та вісімковому вигляді.
  Розв'яжить задачу для типу M unsigned та long long unsigned.
\item
  Ввести натуральне число M. Поміняйте місцями біти її двійкового запису
  з номерами i та j (що теж вводяться) та виведіть отримане число в
  десятковому та шістнадцятковому вигляді.
\item
  Знайдіть кількість значущих (не рівних 0) бітів натурального числа.
\item
  За допомогою лише бітових операцій та операції декременту з'ясуйте чи
  є дане натуральне число ступенем двійки. Спробуйте з циклом та без
  циклу. (\emph{Вказівка}: подумайте, як виглядає бітове представлення
  декременту ступеню двійки, та використайте далі кон'юнкцію).
\item
  Ввести натуральні числа M та N та визначить скільки в них спільних
  одиничок бітового представлення. Визначить скільки в цих числах
  взагалі співпадає бітів.
\item
  Виведіть бітове (двійкове) представлення натурального числа.
\item
  Інвертуйте бітове представлення даного числа та виведіть двійкове
  представлення та десяткове для цієї інверсії.
\item
  Ввести ціле число n (байт) і вивести число, отримане в результаті
  циклічного зсуву числа n на один розряд вліво, тобто старший біт
  зсунутий в позицію молодшого, а всі інші біти зсуваються на один
  розряд вліво. Наприклад, якщо введено 130, відповіддю буде 5.
\item
  Визначити, скільки разів зустрічається 11 в двійковому поданні цілого
  додатного числа (в двійковому поданні 11110111 воно зустрічається 5
  разів).
\item
  Викреслити i-й біт з двійкового представлення натурального числа
  (молодші i-го біти залишаються на місці, старші зсуваються на один
  розряд вправо). Наприклад, якщо введені 11 і 2, відповіддю буде 7.
\item
  \emph{Ціле число m записується в двійковій системі та розряди в цьому
  записі переставляються в зворотному порядку. Отримане число ---
  результат функції BitReverse(m). (BitReverse(512)==1,
  BitReverse(513)==513... ). Вивести значення цієї функції для всіх
  чисел від N до M}
\item
  Напишіть функцію що визначає до якої архітектури (big, high, little
  endian) належить даний комп'ютер.
\item
  Напишіть функцію що визначає чи належить архітектура даної системи до
  little-endian, middle-endian чи big-endian. Напишіть функцію, що
  переводить дане ціле число з отриманої системи до middle-endian якщо
  ця система не middle-endian.
\end{enumerate}

Комплексні числа

\protect\hypertarget{_Hlk63616136}{}{}Розв'язати дані задачі
використовуючи заголовочний файл complex.h та типи \textbf{float
\_Complex}

\textbf{1) Визначити функції для введення та виведення комплексного
числа у одному рядку}

2) Визначити функції для обчислення за введеним комплексним числом:

а) аргументу;

б) модуля комплексного числа.

\begin{enumerate}
\def\labelenumi{\arabic{enumi})}
\item
  Скласти програму обчислення значень багаточлена з комплексними
  коефіцієнтами в заданій комплексній точці.
\item
  Скласти програму обчислення коренів квадратного рівняння з заданими
  комплексними коефіцієнтами.
\item
  Скласти програми обчислення суми всіх доданків, модуль яких не менше ε
  \textgreater{} 0, у комплексній точці \emph{z} та порівняйте обчислені
  значення з результатами відповідних комплексних функцій math.h або
  tgmath.h
\end{enumerate}

а)

б)

в)

г)

ґ)

д)
\(\ln(1 + z) = z - \frac{z^{2}}{2!} + \frac{z^{3}}{3!} - \ldots + ( - 1)^{n}\frac{z^{n}}{n!} + \ldots,\mathrm{\text{\ \ \ \ \ }}(\left| z \right| < 1);\)

е)
\(\mathrm{\text{arctg}}\mathrm{\ }z = z - \frac{z^{3}}{3!} + \frac{z^{5}}{5!} - \ldots + ( - 1)^{n}\frac{z^{2n + 1}}{\left( 2n + 1 \right)!} + \ldots;\mathrm{\text{\ \ \ }}\left( \left| z \right| < 1 \right).\)

\begin{enumerate}
\def\labelenumi{\arabic{enumi})}
\item
  За допомогою формули Кардано розв'яжить кубічне рівняння з
  комплексними коефіцієнтами
\item
  За допомогою формули Ферарі розв'яжить рівняння четвертого порядку з
  дійсними коефіцієнтами в комплексних числах.
\end{enumerate}

\subsection{ 2. Масиви та
вказівники}\label{ux43cux430ux441ux438ux432ux438-ux442ux430-ux432ux43aux430ux437ux456ux432ux43dux438ux43aux438}

\subsection{2.0. Лінійні
масиви}\label{ux43bux456ux43dux456ux439ux43dux456-ux43cux430ux441ux438ux432ux438}

\begin{enumerate}
\def\labelenumi{\arabic{enumi})}
\item
  \begin{quote}
  Ініціалізуйте масив 5 цілих чисел в програмі довільним чином. Введіть
  дійсне число та знайдіть кількість чисел у вашому масиві, що менше зі
  це число.
  \end{quote}
\item
  \begin{quote}
  Масив заповнений таким чином: 5, 112, 4, 3. Вивести його елементи
  навпаки (3,4,112,5). При цьому використання циклу є обов'язковим.
  \end{quote}
\item
  \begin{quote}
  Заповнити масив типу double з~10 елементів з клавіатури (по черзі в
  циклі вводяться всі елементи) і знайти суму всіх елементів більших за
  число Ейлера \(e\).
  \end{quote}
\item
  \begin{quote}
  Масив типу int з 5 елементів заповнюється з клавіатури. Знайти і
  вивести на екран максимальне значення у вашому масиві.
  \end{quote}
\item
  \begin{quote}
  Знайти суму всіх парних і непарних елементів масиву натуральних чисел.
  Масив з 7 елементів заповнюється з клавіатури.
  \end{quote}
\item
  \begin{quote}
  Написати функцію, що вводить послідовність дійсних чисел наступним
  чином: користувачу виводиться напис ``a{[}**{]}= '', де замість **
  стоїть номер числа, що вводиться. Тобто там виводяться написи
  ``a{[}0{]}= '', і після знаку рівності користувач вводить число,
  ``a{[}1{]}= '', і після знаку рівності користувач вводить число і так
  далі доки користувач не введе число 0. Після цього функція повертає
  кількість введених чисел та змінює аргумент, що відповідає масиву
  чисел (кількість чисел не перевищує 100).
  \end{quote}
\item
  Написати функції, що
\end{enumerate}

\begin{quote}
а) вводить n-вимірний вектор дійсних чисел;

б) виводить n-вимірний вектор дійсних чисел;

в) рахує суму двох векторів (результат : аргумент функції --- масив);

г) рахує скалярний добуток двох векторів.

Протестувати роботи цих функцій: ввести в головній програмі розмірність
векторів, 2 вектори цієї розмірності та підрахувати їх суму та скалярний
добуток і вивести результати.
\end{quote}

\begin{enumerate}
\def\labelenumi{\arabic{enumi})}
\item
  \begin{quote}
  Написати функцію, що вводить послідовність ненульових цілих чисел,
  введення завершується при вводі нуля. Кількість елементів масиву
  обмежена числом 20. Визначити кількість добуток та середнє гармонічне
  цієї послідовності.
  \end{quote}
\item
  \begin{quote}
  Вводиться масив натуральних чисел заданого розміру N:
  \end{quote}
\end{enumerate}

\begin{quote}
а) визначити скільки серед цих чисел повних квадратів непарних чисел;

б) визначити скільки серед цих чисел парних повних кубів;

в) визначити скільки серед цих чисел n-тих ступенів цілих чисел (для
всіх n\textgreater{}1);

г) визначити скільки серед них цілих ступенів двійки;

д) визначити скільки серед них повних квадратів, що кратні трьом;

е) визначити скільки серед них простих чисел;

ж) визначити скільки серед них чисел Фібоначчі;

з) визначити скільки серед них чисел, у яких 5-й, 6-й та 8-й біт
двійкового запису дорівнюють 1;

і) визначити скільки серед них чисел, які містять рівно 5 біт в
двійковому записі, що дорівнюють 1;

к) визначити скільки серед них чисел, у яких сума цифр в десятковому
запису ділиться на 7;

10) Задані натуральне число \(n\)\emph{,} дійсні числа
\(a_{1},a_{2},\ldots,a_{n}.\) Скласти функції для знаходження:

а) \(\max\left( a_{1},a_{2},\ldots,a_{n} \right);\) б)
\(\min\left( a_{1},a_{2},\ldots,a_{n} \right);\)

в) \(\max\left( a_{2},a_{4},\ldots \right);\) г)
\(\min\left( a_{1},a_{3},\ldots \right);\)

д)
\(\min\left( a_{2},a_{4},\ldots \right) + \max\left( a_{1},a_{3},\ldots \right);\)

е)
\(\max\left( \left| a_{1} \right|,\ldots,\left| a_{n} \right| \right);\)
ж) \(\max\left( - a_{1},a_{2}, - a_{3}\ldots,( - 1)^{n}a_{n} \right);\)

з)
\(\left( \min\left( a_{1},\ldots,a_{n} \right) \right)^{2} - \min\left( a_{1n}^{2},\ldots,a_{n}^{2} \right).\)

11) Аргументи функції - натуральне число n та цілі числа
\(a_{1},a_{2},\ldots,a_{n}.\) Скласти функції знаходження:

а) \(\min\left( a_{1},2a_{2},\ldots,na_{n} \right);\)

б) \(\min\left( a_{1} + a_{2},\ldots,a_{n - 1} + a_{n} \right);\)

в)
\(\max\left( a_{1},a_{1}a_{2},\ldots,a_{1}a_{2}\ldots a_{n} \right);\)

г) кількості парних серед \(a_{1!},a_{2!},\ldots,a_{k!}\)
(k!\textless{}n);

д) кількості повних квадратів
серед\(\ a_{1},a_{2},\ldots,a_{n}(k < n)\);

е) кількості квадратів непарних чисел серед
\(a_{1^{2}},a_{2^{2}},\ldots,a_{k^{2}}(k^{2} < n).\)

13) Скласти функції для обчислення

а) Значення многочлена Чебишова заданого степеню \(n\) в точці \(x\)

\(T_{0}(x) = 1,\mathrm{\text{\ \ }}T_{1}(x) = x,\)

\(T_{n}(x) = 2xT_{n - 1}(x) - T_{n - 2}(x),\mathrm{\text{\ \ }}n = 2,3,\ldots;\)

та функцію, що виводить коефіцієнти поліному ступеня n\textless{}256.

б) многочлена Ерміта заданого степеню \(n\) в точці \(x\)

\(H_{0}(x) = 1,\mathrm{\text{\ \ }}H_{1}(x) = 2x,\)

\(H_{n}(x) = 2xH_{n - 1}(x) - 2(n - 1)H_{n - 2}(x),\mathrm{\text{\ \ \ \ \ \ }}n = 2,3,\ldots\)

та функцію, що виводить коефіцієнти поліному ступеня n\textless{}256.

14) Точка площини задана декартовими координатами (x, y). Перевірити, чи
належить вона багатокутнику з вершинами P1(y1, x1), P2(x2, y2),
\ldots{}.,Pn (xn, yn).
\end{quote}

\begin{enumerate}
\def\labelenumi{\arabic{enumi})}
\setcounter{enumi}{16}
\item
  В цілочисельному масиві A{[}N{]} знайдіть моду, тобто елемент, що
  зустрічається найбільшу кількість разів. Якщо таких елементів декілька
  виведіть всі такі елементи.
\item
  В цілочисельному масиві A{[}N{]} знайдіть елемент, що є найближчим до
  середнього арифметичного найбільшого та найменшого елементу масиву.
\item
  \begin{quote}
  Напишіть функцію, яка в дійсному масиві A{[}N{]} знаходить середнє
  відхилення (варіацію) масиву.
  \end{quote}
\item
  \begin{quote}
  Знайдіть в даному цілому числі цифру десяткового запису, яка
  зустрічається найбільшу кількість разів. Якщо їх декілька, виведіть
  найбільшу цифру.
  \end{quote}
\item
  \begin{quote}
  Напишіть функцію, яка за заданим масивом значень
  \(\left\{ x_{i} \right\}_{i = 1}^{d}\) обчислює:
  \end{quote}
\end{enumerate}

\begin{quote}
\includegraphics[width=3.23958in,height=0.66667in]{media/image9.png}

22) Біля прилавка в магазині вишикувалася черга з n покупців, кожен з
яких став у чергу в час \(t_{i}\) (i = 1, ...,n). Час обслуговування
продавцем t-го покупця \({t'}_{i}\) (i = 1, ...,n). Нехай дано
натуральне n і дійсні
\(\left\{ t_{i} \right\}_{i = 1..n},\ \left\{ {t'}_{i} \right\}_{i = 1..n}\).
Отримати \(\left\{ c_{i} \right\}_{i = 1..n},\) де \(c_{i}\) - час
перебування i-го покупця в черзі (i = 1..n). Вказати номер покупця, для
обслуговування якого продавцеві потрібно найменше часу.

23) В деяких видах спортивних змагань виступ кожного спортсмена
незалежно оцінюється деякими суддями, потім з усієї сукупності оцінок
видаляються найбільш висока і найнижча, а для решти оцінок обчислюється
середнє арифметичне, яке і йде в залік спортсмену. Якщо найбільш високу
оцінку виставило декілька суддів, то з сукупності оцінок видаляється
лише одна така оцінка; аналогічно надходять з найбільш низькими
оцінками. Дано натуральне число n, дійсні числа
\(a_{1},a_{2},\cdots,a_{n}\). Вважаючи, що
\(a_{1},a_{2},\cdots,a_{n}\ \)оцінки, виставлені суддями одному з
учасників змагань, визначити оцінку, яка піде в залік цього спортсмену.

24) По заданим значенням коефіцієнтів поліномів P(x) та Q(x) знайдіть
значення коефіцієнтів поліному P(Q(x)).
\end{quote}

\begin{enumerate}
\def\labelenumi{\arabic{enumi})}
\setcounter{enumi}{16}
\item
  \begin{quote}
  * В цілочисельному масиві A{[}N{]} (не обов'язково впорядкованому)
  знайдіть медіану, тобто величину, що ділить ряд навпіл: по обидві
  сторони від неї знаходиться однакова кількість одиниць сукупності.
  Тобто, якщо кількість чисел непарна, обирається елемент, що є середнім
  за зростанням. Наприклад, для впорядкованого набору чисел 1, 3, 3, 6,
  7, 8, 9 медіаною є четверте із них, число 6. Якщо кількість елементів
  парна, тоді медіану зазвичай визначають як \emph{середнє} значення між
  двома числами по середині впорядкованого масиву. Наприклад, для
  наступного набору 1, 2, 3, 4, 5, 6, 8, 9 - медіана є середнім
  значенням для двох чисел по середині: вона дорівнюватиме (4 +
  5)/2=4.5.
  \end{quote}
\end{enumerate}

\begin{quote}
15) * Обчислити коефіцієнти багаточлена з заданими дійсними коренями
x{[}0{]},x{[}1{]}, \ldots{}, x{[}n{]}. Кількість коефіцієнтів обмежена
числом 100.

16) Побудувати N-розрядний код Грея (1\textless{}N\textless{}64). Кодом
Грея зветься така послідовність дворозрядних двійкових чисел, в яких
кожні два сусідніх а також перше й останнє числа відрізняються лише
одним розрядом. Так, для N=2 код Грея наступний: 00,01,11,10. Для N=3:
000,001,011,010,110,111,101. Переведіть всі числа з цього двійкового
коду до десяткової системи числення.
\end{quote}

\begin{enumerate}
\def\labelenumi{\arabic{enumi})}
\setcounter{enumi}{16}
\item
  \begin{quote}
  Заданий масив натуральних чисел a{[}N{]}. Знайти мінімальне натуральне
  число, яке не можна представити як суму елементів цього масиву. Сума
  може складатись і з одного елементу, але кожен елемент може туди
  входити лише один раз.
  \end{quote}
\item
  \begin{quote}
  Наступний спосіб призначений для шифрування послідовностей нулів і
  одиниць (або ж, наприклад, точок і тире). Нехай
  \(a_{1},\ \ \ldots,\ \ a_{n} -\) така послідовність. Те, що
  пропонується в якості її шифру, \(-\)це послідовність
  \(b_{1},\ \ \ldots,\text{\ b}_{n}\) , утворена по наступному закону:
  \end{quote}
\end{enumerate}

\[b_{1} = a_{1},\ \ b_{i} = \left\{ \begin{matrix}
1,iakshcho\ a_{i} = a_{i - 1}, \\
 \\
\ \ \ \ \ 0\ v\ inshomu\ vipadku \\
\end{matrix} \right.\ \left( i = 2,\ ...,n \right)\]

\begin{quote}
Користуючись викладеним способом:
\end{quote}

\begin{enumerate}
\def\labelenumi{\alph{enumi})}
\item
  \begin{quote}
  Зашифрувати дану послідовність;
  \end{quote}
\item
  \begin{quote}
  Розшифрувати дану послідовність.
  \end{quote}
\item
  \begin{quote}
  "Виправлення помилок". Нехай по деякому каналу зв'язку передається
  повідомлення, що має вигляд послідовності нулів і одиниць (або,
  аналогічно, крапок і тире). Через перешкод можливий помилковий прийом
  деяких сигналів: нуль може бути сприйнятий як одиниця і навпаки. Можна
  передавати кожен сигнал тричі, замінюючи, наприклад, послідовність 1,
  0, 1 послідовністю 1, 1, 1, 0, 0, 0, 1, 1. Три послідовні цифри при
  розшифровці замінюються тієї цифрою, яка зустрічається серед них
  принаймні двічі. Таке укроювання сигналів істотно підвищує ймовірність
  правильного прийому повідомлення. Написати програму розшифровки.
  \end{quote}
\end{enumerate}

\subsection{\texorpdfstring{\emph{\emph{2.1. Двовимірні та}}
багатовимірні
\emph{\emph{масиви}}}{2.1. Двовимірні та багатовимірні масиви}}\label{ux434ux432ux43eux432ux438ux43cux456ux440ux43dux456-ux442ux430-ux431ux430ux433ux430ux442ux43eux432ux438ux43cux456ux440ux43dux456-ux43cux430ux441ux438ux432ux438}

\begin{enumerate}
\def\labelenumi{\arabic{enumi})}
\item
  Двовимірна матриця 3х3 ініціалізована числами
  \{\{1,2,3,\},\{4,5,6\},\{7,8,9\}\}. Транспонуйте цю матрицю, введіть
  натуральні числа N і M та замініть елемент, що рівний числу M (якщо
  він є в матриці) на число N. Виведіть отриману матрицю рядок за
  рядком.
\item
  Двовимірна матриця 3х3 ініціалізована дійсними числами \{\{1.0,
  2,3,\},\{4,5,6\},\{7,8,9\}\}. Транспонуйте цю матрицю, введіть
  натуральні числа I і J та дійсне число A замініть елемент з індексами
  IJ на число A (відслідкуйте при цьому коректність індексів). Виведіть
  отриману матрицю рядок за рядком.
\item
  Напишіть процедуру вводу двовимірної дійсної матриці довільного
  розміру m x n , яка вводить з підказкою для користувача (які індекси
  елементів) кожен елемент в одному рядку.
\item
  Напишіть процедуру вводу двовимірної цілої (дійсної) матриці
  довільного розміру m x n , яка вводить з підказкою для користувача
  (які індекси елементів) матрицю рядок за рядком (числа в рядку
  розділяються одним пробілом).
\item
  Дана матриця n*m з нулів та одиниць. Знайти найбільший за площиною
  прямокутник з одних одиниць.
\item
  В двовимірному масиві A{[}N,M{]} знайдіть суму елементів A{[}i,j{]},
  що i-j=k . Ціле число k може бути від'ємним, якщо таких елементів
  немає, то вивести нуль.
\end{enumerate}

\begin{quote}
\emph{\emph{Вирішіть завдання даної групи, оформивши рішення у вигляді
функцій генерації, виведення і обробки масивів. Передбачте в функції
генерації масиву введення кордонів діапазону випадкових чисел.}}
\end{quote}

\begin{enumerate}
\def\labelenumi{\arabic{enumi})}
\item
  Дана квадратна матриця порядку 2n + 1. Дзеркально відобразити її
  елементи відносно горизонтальної осі симетрії матриці.
\item
  Дано дійсні числа a\_1, \ldots{}, a\_\{N*N\}. Отримати дійсну
  квадратну матрицю порядку 8, елементами якої є числа a\_1, \ldots{},
  a\_\{N*N\}, розташовані в ній за схемою, яка наведена на малюнку.
\end{enumerate}

\begin{quote}
А1 А2 А3 А4

А12 А13 А14 А5

А11 А16 А15 А6

А10 А9 А8 А7
\end{quote}

\begin{enumerate}
\def\labelenumi{\arabic{enumi})}
\item
  Дана матриця розміру n * m. Поміняти місцями її стовпці так, щоб їх
  максимальні елементи утворювали спадаючу послідовність.
\item
  Знайдіть квадратну матрицю, зворотну даної з розміром n x n.
\item
  Дана квадратна матриця порядку 2n. Повернути її на 180 градусів в
  позитивному напрямку.
\item
  Заповнити двовимірний квадратний масив цілими числами від 1 до 100 по
  спіралі, як показано на наступному малюнку.
\item
  Дана матриця розміру n x m. Поміняти місцями стовпці, що містять
  мінімальний і максимальний елементи матриці.
\item
  Дано дві матриці n x m і m x k. Отримайте їх добуток.
\item
  Дана матриця розміру n х m. Поміняти місцями її рядки так, щоб їх
  максимальні елементи утворювали зростаючу послідовність.
\item
  У даній дійсної квадратної матриці порядку n знайти найбільший по
  модулю елемент.
\item
  Отримати квадратну матрицю порядку n - 1 шляхом викидання з вихідної
  матриці будь-якого рядка і стовпця, на перетині яких розташований
  елемент зі знайденим значенням. Виконуйте до тих пір, поки не
  залишиться останній елемент.
\item
  Дана квадратна матриця порядку 2n + 1. Дзеркально відобразити її
  елементи відносно побічної діагоналі матриці.
\item
  Дана дійсна квадратна матриця порядку 2n + 1. Отримати нову матрицю,
  повернувши її блоки, обмежені діагоналями, на 180 градусів.
\item
  Дана матриця розміру n x m. Поміняти місцями її перший і останній
  рядки, що містять тільки негативні елементи.
\item
  Дана цілочисельна матриця розміру n x m. Знайти елемент, який є
  максимальним у своєму рядку і мінімальним в своєму стовпці. Якщо такий
  елемент відсутній, то вивести 0.
\item
  Складіть програму циклічної перестановки стовпців двовимірного масиву
  m x k, при якій зсуві зсувається вправо на n стовпців.
\item
  Дана матриця розміру n x m. Поміняти місцями її стовпці так, щоб їх
  мінімальні елементи утворювали зростаючу послідовність.
\item
  Дана квадратна матриця порядку 2n + 1. Дзеркально відобразити її
  елементи відносно вертикальної осі симетрії матриці.
\item
  Дана квадратна матриця порядку 2n. Повернути її на 270 градусів в
  позитивному напрямку щодо її центру.
\item
  Дана матриця розміру n x m. Поміняти місцями рядки, що містять
  мінімальний і максимальний елементи матриці.
\item
  У квадратній таблиці обміняйте місцями елементи рядка і стовпця, на
  перетині яких знаходиться мінімальний з позитивних елементів.
\item
  Дана квадратна матриця порядку 2n. Повернути її на 90 градусів в
  позитивному напрямку щодо її центру.
\item
  Дана квадратна матриця порядку 2n + 1. Дзеркально відобразити її
  елементи відносно головної діагоналі матриці.
\item
  Складіть програму циклічної перестановки рядків двовимірного масиву m
  x k, при якій зсув відбувається вниз на n рядків.
\item
  Дана матриця розміру n x m. Поміняти місцями її перший і останній
  стовпці, що містять тільки позитивні елементи.
\item
  Заповнити двовимірний квадратний масив цілими числами від 1 до 100 по
  спіралі, починаючи від центру і закручуючи за годинниковою стрілкою.
\item
  Заповніть квадратну матрицю n x n за принципом латинського квадрата: в
  кожному рядку і кожному стовпці використовуються лише числа від 1 до n
  що не повторюються між собою.
\item
  Дана матриця дійсних коефіцієнтів. Впорядкувати її рядки по неспаданню
  перших елементів, суми значень рядків, величині найменших елементів
  рядків.
\end{enumerate}

\subsection{3.Виділення пам'яті, вказівники та
рядки}\label{ux432ux438ux434ux456ux43bux435ux43dux43dux44f-ux43fux430ux43cux44fux442ux456-ux432ux43aux430ux437ux456ux432ux43dux438ux43aux438-ux442ux430-ux440ux44fux434ux43aux438}

\subsection{3.2 Вказівники та виділення
пам'яті}\label{ux432ux43aux430ux437ux456ux432ux43dux438ux43aux438-ux442ux430-ux432ux438ux434ux456ux43bux435ux43dux43dux44f-ux43fux430ux43cux44fux442ux456}

\begin{enumerate}
\def\labelenumi{\arabic{enumi})}
\item
  Ввести натуральне число n. Створити масив з n дійсних чисел та
  підрахувати суму квадратів елементів цього масиву.
\item
  Написати функцію, що вводить масив цілих чисел доки не введеться нуль
  через змінний аргумент та кількість елементів масиву повертається як
  результат роботи функції. Підрахувати кількість повних квадратів та
  кубів в цьому масиві.
\item
  Написати функцію, що вводить масив натуральних чисел доки не введеться
  нуль через кількість елементів масиву --- змінний аргумент, а роботи
  функції - вказівник. Підрахувати кількість ступенів двійки та трійки в
  цьому масиві.
\item
  Створити функцію, що вводить n-вимірний вектор, виділяючи відповідну
  пам'ять та функцію, що відповідно очищує пам'ять. Напишіть програму,
  що вводить два вектори, підраховує та створює як окремий масив їх
  векторний добуток, якщо це можливо, та в будь-якому варіанті коректно
  завершує програму без витоків пам'яті.
\item
  Створити функцію, що вводить дійсну квадратну n-вимірну матрицю (n
  задається як аргумент функції), виділяючи відповідну пам'ять та
  функцію, що відповідно очищує пам'ять. Напишіть програму, що вводить
  дві матриці, підраховує та створює як окремий масив їх добуток, якщо
  це можливо, та в будь-якому варіанті коректно завершує програму без
  витоків пам'яті.
\item
  Створити функцію, що вводить матрицю цілих чисел довільних
  розмірностей, виділяючи відповідну пам'ять (розміри масивів) та
  функцію, що відповідно очищує пам'ять. Напишіть функцію, що підраховує
  ранг матриці. Коректно протестуйте роботу цих функцій.
\item
  Створити функцію, що вводить матриці довільних розмірностей, виділяючи
  відповідну пам'ять та функцію, що відповідно очищує пам'ять. Напишіть
  програму, що вводить масив таких матриць, підраховує та створює як
  окремий масив добуток всього масиву матриць, якщо це можливо, та в
  будь-якому варіанті коректно завершує програму без витоків пам'яті.
\item
  Користувачу надається можливість декілька разів вводити розмірність
  вектору дійсних чисел та самі ці значення. Після кожного вводу
  потрібно підрахувати середнє арифметичне та дисперсію всіх введених
  значень.
\item
  Петя та Вася кожен день на протязі
  \protect\hypertarget{__DdeLink__55546_11145444801}{}{}N днів вимірюють
  декілька (від 0 до 1000) разів температуру повітря (хоча інколи хтось
  може забути це зробити). Створіть програму, що дозволить їм ввести ці
  результати за кожен день спостережень та підрахує середню температуру
  кожного з цих днів, де сумарна кількість вимірювань була більше 1.
  Програма повинна передбачити, що після вводу цих N днів вони можуть
  захотіти ввести наступні M днів таки спостережень. Передбачте
  можливість коректного завершення при нестачі ресурсів ПК для
  зберігання та обробки даних.
\item
  * В масиві натуральних чисел A{[}N{]} всі числа є меншими 16. Напишіть
  функцію, що зберігає дані цього масиву у масиві N/2 чисел типу
  uint8\_t (тобто в кожному числі uint8\_t зберігається два числа масиву
  A{[}i{]}).
\item
  *В масиві натуральних чисел A{[}N{]} всі числа є меншими 64. Напишіть
  функцію, що зберігає дані цього масиву у масиві {[}N*4/3{]} чисел типу
  uint8\_t (тобто в кожних трьох числах uint8\_t зберігається чотири
  числа масиву A{[}i{]}).
\item
  **В масиві натуральних чисел A{[}N{]} всі числа є меншими \(2^{k}\).
  Знайдіть це число k та напишіть функцію, що зберігає цей масив в N*k
  біт найбільш економічним чином (int A{[}3{]}, k=5 → uint8 B{[}2{]}
  ,тобто використовує 16 біт, або int A{[}8{]}, k=14 → uint16 B{[}7{]} ,
  тобто використовує 112 біт) та функцію що повертає числа з масиву B у
  масив A.
\item
  \textbf{Вирішіть завдання виконуючи наступні вимоги:}
\end{enumerate}

\textbf{Сформувати динамічний двовимірний масив, заповнити його
випадковими числами і вивести на екран. }

\begin{enumerate}
\def\labelenumi{\alph{enumi})}
\item
  Додати рядок із заданим номером.
\item
  Додати стовпець із заданим номером.
\item
  Додати рядок в кінець матриці.
\item
  Додати стовпець в кінець матриці.
\item
  Додати рядок в початок матриці.
\item
  Додати стовпець в початок матриці.
\item
  Додати К рядків в кінець матриці.
\item
  Додати К стовпців в кінець матриці.
\item
  Додати К рядків в початок матриці.
\item
  Додати К стовпців в початок матриці.
\item
  Видалити рядок з номером К.
\item
  Видалити стовпець з номером К.
\item
  Видалити рядки, починаючи з рядка К1 і до рядка К2.
\item
  Видалити стовпці, починаючи з стовпця К1 і до стовпчика К2.
\item
  Видалити всі парні рядки.
\item
  Видалити всі парні стовпці.
\item
  Видалити всі рядки, в яких є хоча б один нульовий елемент.
\item
  Видалити всі стовпці, в яких є хоча б один нульовий елемент.
\item
  Видалити рядок, в якій знаходиться найбільший елемент матриці.
\item
  Додати рядки після кожної парної рядки матриці.
\item
  Додати стовпці після кожного парного стовпця матриці.
\item
  Додати К рядків, починаючи з рядка з номером N.
\item
  Додати К стовпців, починаючи зі стовпчика з номером N.
\item
  Додати рядок після рядка, що містить найбільший елемент.
\item
  Додати стовпець після стовпця, що містить найбільший елемент.
\item
  Додати рядок після рядка, що містить найменший елемент.
\item
  Додати стовпець після стовпця, що містить найменший елемент.
\item
  Видалити рядок і стовпець, на перетині яких знаходиться найбільший
  елемент масиву.
\end{enumerate}

\begin{quote}
\protect\hypertarget{_Hlk48903540}{}{}3.1. Рядки Сі (Null-terminated
strings)

1) Надрукувати заданий рядок:

а) виключивши з нього всі цифри і подвоївши знаки '+' та '-';

б) виключивши з нього всі знаки '+', безпосередньо за якими знаходиться
цифра;

в) виключивши з нього всі літери '\emph{в}', безпосередньо перед якими
знаходиться літера '\emph{с}';

г) замінивши в ньому всі пари '\emph{ph}' на літеру '\emph{f}';

д) виключивши з нього всі зайві пропуски, тобто з кількох, що йдуть
підряд, залишити один.

2) Дано рядок, серед символів якого є принаймні одна кома, а може й
немає її. Знайти номер

а) першої по порядку коми;

б) останньої по порядку коми;

в) кількості ком.

3) Виключити з заданого рядка групи символів, які знаходяться між '(' та
')'. Самі дужки теж мають бути виключені. Перевірте перед цим, що дужки
розставлено правильно (парами) та всередині кожної пари дужок немає
інших дужок.

4) Заданий рядок, серед символів якого міститься двокрапка ':'. Якщо її
немає -- вивести весь рядок. Отримати як масив всі символи, що
розташовані:

а) до першої двокрапки включно;

б) після першої двокрапки;

в) між першою і другою двокрапкою. Якщо другої двокрапки немає, то
отримати всі символи, розміщені після єдиної двокрапки.

5) Заданий текст надрукувати по рядках, розуміючи під рядком або
наступні 6 символів, якщо серед них немає коми (знак оклику, питання),
або частину тексту до коми включно.

6) Задана послідовність символів, яка має вигляд:

\emph{d\textsubscript{1}} ± \emph{d\textsubscript{2}} ± \emph{...} ±
\emph{d\textsubscript{n}}

(\emph{d\textsubscript{i }}-- натуральні числа,
\emph{n}\textgreater{}1), за якою знаходиться знак рівності. Перевірити,
що рядок задовольняє вказаний вигляд та обчислити значення цієї
алгебраїчної суми.
\end{quote}

7) Задане натуральне число \emph{n}. Надрукувати в заданій системі
числення b цілі числа від 0 до \emph{n}.

\begin{quote}
8) В заданий рядок входять тільки цифри та літери. Визначити, чи
задовольняє він наступній властивості:

а) рядок є десятковим записом числа, кратного 9 (6, 4);

б) рядок починається з деякої ненульової цифри, за якою знаходяться
тільки літери і їх кількість дорівнює числовому значенню цієї цифри;

в) рядок містить (крім літер) тільки одну цифру, причому її числове
значення дорівнює довжині рядка;

г) сума числових значень цифр, які входять в рядок, дорівнює довжині
рядка;

д) рядок співпадає з початковим (кінцевим, будь-яким) відрізком ряду
0123456789;

е) рядок складається тільки з цифр, причому їх числові значення
складають арифметичну прогресію (наприклад, 3 5 7 9, 8 5 2, 2).

9) Пересвідчитись, що заданий рядок відповідає запису сімнадцяткового
числа (цифри `a'-`g' можуть бути як великого так і маленького регістру,
але обов'язково одного того самого регістру) та вивести його у
десятковому вигляді.

10) Знайти у даному рядку символ та довжину найдовшої послідовності
однакових символів, що йдуть підряд.

11) Скласти програму підрахунку загального числа входжень символів '+',
'-', '*' у рядок \emph{А}.

12) Скласти програму перетворення рядка \emph{А}, замінивши у ньому всі
знаки оклику '!' крапками '.', кожну крапку -- трьома крапками '...',
кожну зірочку '*'- знаком '+'.

13) Рядок називається симетричним, якщо його символи, рівновіддалені від
початку та кінця рядка, співпадають. Порожній рядок вважається
симетричним. Перевірити рядок \emph{A} на симетричність.

14) Скласти програму видалення із рядка \emph{А} всіх входжень заданої
групи символів.

15) Скласти програму перетворення слова \emph{А}, видаливши у ньому
кожний символ '*' та подвоївши кожний символ, відмінний від '*'.

16) Скласти функцію підрахунку найбільшої кількості цифр, що йдуть
підряд у рядку \emph{А}.

17) Скласти функцію підрахунку числа входжень у рядок \emph{А} заданої
послідовності літер.

18) Скласти функцію, яка за рядком \emph{А} та символом \emph{S} будує
новий рядок, отриманий заміною кожного символу, що слідує за \emph{S},
заданим символом \emph{С}.

19) Скласти функцію перетворення рядка \emph{А} видаленням із нього всіх
ком, які передують першій крапці, та заміною у ньому знаком '+' усіх
цифр '3', які зустрічаються після першої крапки.

20) Скласти функцію виведення на друк усіх цифр, які входять в заданий
рядок, та окремо - решту символів, зберігаючи при цьому взаємне
розташування символів у кожній з цих двох груп.

21) Рядок називається монотонним, якщо він складається з зростаючої або
спадної послідовності символів. Скласти функцію перевірки монотонності
рядка.

22) Перевірити, чи складається рядок з

а) 2 симетричних підрядків;

б) n симетричних підрядків.

23) Знайти символ, кількість входжень якого у рядок \emph{A}

а) максимальна;

б) мінімальна.

24) Дано рядок \emph{A}, що містить послідовність слів. Скласти
програми, що визначають:

а) кількість усіх слів;

б) кількість слів, що починаються із заданого символу \emph{c};

в) кількість слів, що закінчуються заданим символом \emph{c};

г) кількість слів, що починаються й закінчуються заданим символом
\emph{c};

ґ) кількість слів, що починаються й закінчуються однаковим символом.

25) Виділити з рядка \emph{A} найбільший підрядок, перший і останній
символи якого співпадають.

26) Виділити з рядка найбільший монотонний підрядок, коди послідовних
символів якого відрізняються на 1.

27) Замінити всі пари однакових символів рядка, які йдуть підряд, одним
символом. Наприклад, рядок \emph{`aabcbb'} перетворюється у
\emph{`abcb'}.

28) Побудувати рядок \emph{S} з рядків \emph{S1}, \emph{S2} так, щоб у
\emph{S} входили

а) ті символи \emph{S1}, які не входять у S2;

а) всі символи \emph{S1}, які не входять у \emph{S2}, та всі символи
\emph{S2}, які не входять у \emph{S1}.

29) Видалити з рядка симетричні початок та кінець. Наприклад, рядок
\emph{`abcdefba'} перетворюється у \emph{`cdef'}.

30) Написати програму, яка виконує зсув по ключу (ключ задається) тільки
для малих латинських літер. Наприклад: вхідні дані anz -- рядок, 2 --
ключ. Результат: cpb.

31) Встановити, чи задовольняє заданий рядок заданому шаблону. Шаблон
--- це рядок, що складається з символів а також наступних спецсимволів:
символ «?» позначає будь-який символ, «*» означає будь-яку послідовність
символів, у тому числі порожню, а «+» будь-яку непорожню послідовність
символів (приклад, «ab*ra??da+ra»).

32) Хеш даного рядку (довжина рядку більше одиниці) обчислюється так:

а) Кожні послідовні 4 байти конкатинуються щоб утворити натуральне
число. Якщо кількість символів не кратна 4, то до рядка дописуються
потрібна кількість символів, що взята з кінця рядку справа наліво
(зеркальний падінг). Всі ці числа додаються за допомогою ``виключного
або'' (xor).

б) Кожні послідовні 4 байти конкатинуються щоб утворити натуральне
число. Якщо кількість символів не кратна 4, то до рядка дописуються
потрібна кількість нулевих символів (нульовий падінг). До всіх цих чисел
додається за допомогою ``виключного або'' номер по порядку цього числа.
Потім всі ці числа додаються за допомогою ``виключного або''.

в) Береться просте число p. Кожен послідовні байт множиться на
p\textsuperscript{i}, де I -- номер по порядку цього числа та береться
остача від ділення на 2\textsuperscript{32}. Потім всі ці числа
додаються по модулю 2\textsuperscript{32}.

33) Реалізувати функцію виведення на друк тільки маленьких літер
українського алфавіту, які входять в заданий рядок.

34) Заданий рядок, який складається з великих літер українського
алфавіту. Скласти програму перевірки впорядкованості цих літер за
алфавітом.

35) Скласти програму виведення на друк в алфавітному порядку усіх різних
маленьких українських літер, які входять до даного рядка.

36) Як показують численні експерименти, розбиття українського слова на
частини для переносу з одного рядки на іншу з великою ймовірністю
виконується правильно, якщо користуватися наступними простими прийомами:
\end{quote}

\begin{itemize}
\item
  Дві підряд голосні можна розділити, якщо першій з них передує
  приголосна, а за другою йде хоча б одна буква (буква
  \(i\ \ pri\ ts'omu\ rozgliadaiet'sia\) разом з попередньою голосною як
  єдине ціле).
\item
  Дві йдуть підряд приголосні можна розділити, якщо першій з них передує
  голосна, а в тій частині слова, яка йде за другою приголосною, є хоча
  б одна голосна (літера `ь' разом з попередньою приголосною
  розглядаються як єдине ціле).
\item
  Якщо не вдається застосувати пункти 1), 2), то слід спробувати розбити
  слово так, щоб перша частина містила більш ніж одну букву і
  закінчувалася б на голосну, а друга містила хоча б одну голосну.
\item
  Імовірність правильного розбиття збільшується, якщо попередньо
  скористатись хоча б неповним списком приставок з голосними літерами, і
  спробувати перш за все виділити слова з такими приставками.
\end{itemize}

\begin{quote}
Дано текст, який є українським словом. Виконати поділ його на частини
для переносу.

37) Для більшості російських іменників, які закінчуються на -онок і
-енок, множина утворюється від іншої основи. Як правило, це відбувається
за зразком: цыпленок- цыплята, мышонок - мышата і т. д. (в новій основі
перед останньою буквою \(\text{m\ }\) пишеться \(a\ \ abo\ \ ia\ \ \) в
залежності від попередньої літери: якщо це шипляча, то \(\ a\), в іншому
випадку \(- ia)\). Є слова-винятки, з яких вкажемо наступні: ребенок
(дети), бесенок (бесенята), опенок (опята), звонок (звонки), позвонок
(позвонки), подонок (подонки), колонок (колонки), жаворонок (жаворонки),
бочонок(бочонки). Є ще ряд маловживаних слів-винятків, які ми не
розглядаємо. Дано текст, серед символів якого є пробіли. Група символів,
що передує першому пробілу -- є російським словом, закінчується на
\(- onok\ \ \)або \(- enok\). Отримати це слово у множині.

38) Дано натуральне число \(n\), символ \(s\)
(\(n \leq 1000\),\(\text{s\ }\)- одна з букв і, р, д, в, т, п, яка
вказує відмінок -називний, родовий, давальний, знахідний, орудний,
місцевий, окличний). Записати кількісний числівник, що означає запис
числа \(n\) у відповідному відмінку.

\protect\hypertarget{_Hlk48904419}{}{}4. Файли

4.0. Символьні файли (файли, що містять послідовності символів)
\end{quote}

\begin{enumerate}
\def\labelenumi{\arabic{enumi})}
\item
  Дано символьний файл F. Побудувати файл G, утворений із
\end{enumerate}

\begin{quote}
файлу F:

а) зміною всіх його великих літер однойменними малими;

б) записом його компонент у зворотному порядку.
\end{quote}

\begin{enumerate}
\def\labelenumi{\arabic{enumi})}
\item
  Дано символьний файл, що складається не менш ніж із 2 компонент.
  Визначити, чи є два перших символи файлу цифрами. Якщо так, то
  виявити, чи є число, утворене цими цифрами, парним.
\item
  Задано символьні файли F і G. Записати до файлу H спочатку
\end{enumerate}

\begin{quote}
компоненти файлу F, потім -- файлу G зі збереженням порядку.

4) Дано символьний файл. Скласти підпрограми для:

а) додавання в його кінець заданого символу;

б) додавання в його початок заданого символу;

в) підрахунку кількості входжень до файлу заданого символу;

г) визначення входження до файлу заданої комбінації символів;

д) вилучення заданого символу;

е) вилучення інших входжень кожного символу.
\end{quote}

\begin{enumerate}
\def\labelenumi{\arabic{enumi})}
\item
  Скласти функцію перевірки рівності файлів, виконаної за один перегляд
  їхнього змісту. Символьні файли рівні, коли вони складаються з тих
  самих слів в тому ж порядку. Слова відокремлюються одним чи більше
  пробілами.
\item
  Дано символьний файл. Групи символів, що відокремлені пропусками
  (одним або кількома) і не містять пропусків усередині, називатимемо
  словами. Скласти підпрограми для:
\end{enumerate}

\begin{quote}
а) знаходження найдовшого слова у файлі;

б) визначення кількості слів у файлі;

в) вилучення з файлу зайвих пропусків і всіх слів, що складаються з

однієї літери;

г) вилучення всіх пропусків на початку рядків, у кінці рядків і між
словами (крім одного);

д) вставки пропусків до рядків рівномірно між словами так, щоб довжина
всіх рядків (якщо в них більше 1 слова) була 80 символів і кількість

пропусків між словами в одному рядку відрізнялась не більш ніж на 1

(вважати, що рядки файлу мають не більш ніж 80 символів).

Результат записати до файлу H.
\end{quote}

\begin{enumerate}
\def\labelenumi{\arabic{enumi})}
\item
  Підрахувати кількість слів в даному символьному файлі, які починаються
  з даної послідовності літер. Врахуйте можливість перенесення складів
  одного слова в різні рядки
\end{enumerate}

\begin{quote}
4.1. Текстові файли

\protect\hypertarget{_Hlk65238588}{}{}Організуйте роботу з текстовим
файлом. Вихідні файли не передбачають зміни. Змінені дані збережіть в
іншому файлі.
\end{quote}

\begin{enumerate}
\def\labelenumi{\arabic{enumi})}
\item
  Дано два текстові файли з іменами Name1 і Name2. Додати в кінець
  кожного рядка файлу Name1 відповідний рядок файлу Name2. Якщо файл
  Name2 коротший файлу Name1, то виконайте перехід до початку файлу
  Name2.
\item
  Організувати текстовий файл, що складається з N рядків. Визначити
  максимальний і мінімальний розмір рядків в файлі і вивести їх в інший
  файл.
\item
  Дан текстовий файл з ім'ям NameT. Підрахувати число повторень в ньому
  малих латинських літер ('a' - 'z') і створити файл з ім'ям NameS,
  рядки якого мають вигляд: "\textless{}літера\textgreater{} -
  \textless{}число повторень даної літери\textgreater{}". Літери,
  відсутні в тексті, в файл не включати. Рядки впорядкувати за спаданням
  кількості повторень літер, а при однаковій кількості повторень - по
  зростанню кодів літер.
\item
  Дан символ с (прописна латинська літера) і текстовий файл. Створити
  текстовий файл, який містить всі слова з вихідного файлу, що
  починаються цією літерою (як великої, так і малої). Розділові знаки,
  розташовані на початках і в кінцях слів, не враховувати. Якщо вихідний
  файл не містить відповідних слів, залишити результуючий файл порожнім.
\item
  У відсортоване файл прізвищ додати нове прізвище, не порушивши його
  впорядкованість.
\item
  Дан текстовий файл. Створити файл, що містить всі символи, які
  зустрілися в тексті, включаючи пробіл і знаки пунктуації (без
  повторень). Символи розташовувати в порядку зростання їх кодів.
\item
  Організувати текстовий файл f що складається з N рядків. Після цього
  організувати файли h і g., де у файлі h записуються рядки файлу f які
  займають непарні позиції, а в файлі g парні.
\item
  Дан текстовий файл f. Створити файл g, що містить всі символи, які
  зустрілися в тексті, включаючи пробіл і знаки пунктуації (без
  повторень). Символи розташовувати в порядку проходження у вихідному
  файлі.
\item
  Дано ціле число N і текстовий файл з ім'ям Name1, що містить один
  абзац тексту, вирівняний по лівому краю. Відформатувати текст так, щоб
  його ширина не перевищувала N позицій, і вирівняти текст по лівому
  краю. Прогалини в кінці рядків видалити. Зберегти відформатований
  текст в новому текстовому файлі з іменемName2.
\item
  Організувати текстовий файл f, що складається з N рядків. Організувати
  заміну символів в файлі. "Старий" символ і "новий" символ запитуються
  і вводяться з клавіатури. Зміна вивести в другий файл.
\item
  Дан текстовий файл. Вивести в інший файл найдовші слова тексту (з
  урахуванням розділових знаків, розташованих на початку та в кінці
  слів).
\item
  Додати в вказане місце файлу задану кількість рядків, починаючи з
  зазначеного місця іншого файлу. Місце задається номером рядка.
  Результат вивести в третій файл.
\item
  У файлі зберігаються назви товарів і ціни в гривнях 1997 р Створити
  новий файл, перетворивши ціни товару в рублі і копійки 1998 року,
  додавши найменування "грн." і "коп.". У зазначений рік ціни зменшилися
  в 1000 разів.
\item
  Видалити задану кількість рядків із зазначеного місця файлу. Зміни
  вивести в другий файл. Якщо дію неможливо, вивести про це повідомлення
  на екран і в вихідний файл
\item
  Організувати текстовий файл f, що складається з N рядків. Після цього
  створити текстовий файл g, що містить рядки текстового файлу f в
  зворотному порядку.
\item
  Дан файл, який містить текст, вирівняний по лівому краю (довжина
  кожного рядка не перевищує 50 символів). Вирівняти його по правому
  краю, додавши в початок кожної непорожній рядки необхідну кількість
  прогалин. Вирівняний текст записати в інший файл.
\item
  Організувати текстовий файл, що складається з N рядків. Вивести на
  екран і в інший файл рядки, розмір яких більше середнього розміру
  рядка в файлі.
\item
  Дан текстовий файл. Створити файл, що містить всі знаки пунктуації,
  які зустрілися в текстовому файлі в тому ж порядку.
\item
  Організувати текстовий файл, що складається з N рядків. Замінити в
  файлі все маленькі латинські літери на великі і вивести це в інший
  файл.
\item
  Дан текстовий файл. Вивести в інший файл найкоротші слова тексту (з
  урахуванням розділових знаків, розташованих в кінці слів). Коротке
  слово не є порожнім.
\item
  Організувати текстовий файл, що складається з N рядків. Замінити в
  ньому все рядки даної довжини новим рядком. Довжину замінних рядків і
  вміст нового рядка запитується і вводиться з клавіатури. Якщо таких
  рядків немає, то дані не змінювати. Зміна вивести в новий файл.
\end{enumerate}

\begin{quote}
\protect\hypertarget{_Hlk65238644}{}{}\textbf{Організуйте роботу з
текстовим файлом. Вхідний файл потрібно змінити згідно вказаних умов,
тобто вхідний та вихідні файли співпадають.}
\end{quote}

\begin{enumerate}
\def\labelenumi{\arabic{enumi})}
\item
  Дано число N і текстовий файл. Видалити з файлу рядки з номерами,
  кратними N. Порожні рядки не враховувати і не видаляти. Якщо рядки з
  необхідними номерами відсутня, то залишити файл без змін. Зміна
  вивести в другий файл.
\item
  \begin{quote}
  Дан текстовий файл, що містить текст, вирівняний по лівому краю
  (довжина кожного рядка не перевищує 50 символів). Вирівняти його по
  центру, додавши в початок кожної непорожній рядки необхідну кількість
  прогалин. Рядки непарної довжини перед центруванням доповнювати зліва
  прогалиною. Вирівняний текст записати в інший файл.
  \end{quote}
\item
  \begin{quote}
  Організувати текстовий файл, що складається з N рядків. Перетворити
  файл, видаливши в кожній його рядку зайві пробіли. Зміни вивести в
  другий файл.
  \end{quote}
\item
  \begin{quote}
  Дан файл з текстом із символів латинського алфавіту. Зашифрувати файл,
  виконавши циклічний зсув кожної букви вперед на n позицій в алфавіті.
  Розділові знаки і пропуски не змінювати.
  \end{quote}
\item
  \begin{quote}
  Дано числа N1, N2 і текстовий файл. Видалити з файлу рядки з номерами
  між N1, N2, не включаючи меж. Зміни вивести в другий файл. Якщо
  виконати видалення неможливо, видайте про це повідомлення на екран і в
  вихідний файл.
  \end{quote}
\item
  \begin{quote}
  Дан файл з текстом із символів латинського алфавіту, цифр та знаків.
  Замініть всі цифри їх назвами на англійській мові.
  \end{quote}
\item
  \begin{quote}
  Організувати текстовий файл f складається з N рядків. Після цього
  організувати файли h і g. У файл h записати рядки файлу f непарної
  довжини, в файл g парної довжини.
  \end{quote}
\end{enumerate}

\begin{quote}
29) Визначити функцію, яка:

а) підраховує кількість порожніх рядків;

б) обчислює максимальну довжину рядків текстового файлу.

30) Визначити процедуру виведення:

а) усіх рядків текстового файлу;

б) рядків, які містять більше 60 символів.

31) Визначити функцію, що визначає кількість рядків текстового файлу,
які:

а) починаються із заданого символу;

б) закінчуються заданим символом;

в) починаються й закінчуються одним і тим самим символом;

г) що складаються з однакових символів.

33) В даному текстовому файлі знаходиться англомовний текст. Вирівняйте
його по лівий та правий границі так щоб розподіл слів у рядках був
найбільш рівномірним.

35) Визначити процедуру, яка переписує до текстового файлу G усі

рядки текстового файлу F:

а) із заміною в них символу '0' на '1', і навпаки;

б) в інвертованому вигляді.

36) Визначити процедуру пошуку найдовшого рядка в текстовому

файлі. Якщо таких рядків кілька, знайти перший із них.

37) Визначити процедуру, яка переписує компоненти текстового

файлу F до файлу G, вставляючи до початку кожного рядка один сим-

вол пропуску. Порядок компонент не має змінюватися.

38) У текстовому файлі записано непорожню послідовність дійсних чисел,
які розділяються пропусками. Визначити функцію обчислення найбільшого з
цих чисел.

39) У текстовому файлі F записано послідовність цілих чисел, як

розділяються пропусками. Визначити процедуру запису до текстового

файлу g усіх додатних чисел із F.

40) У текстовому файлі кожний рядок містить кілька натуральних

чисел, які розділяються пропусками. Числа визначають вигляд геометричної
фігури (номер) та її розміри. Прийнято такі домовленості:

відрізок прямої задається координатами своїх кінців і має номер 1;

прямокутник задається координатами верхнього лівого й нижнього

правого кутів і має номер 2;

коло задається координатами центра й радіусом і має номер 3.

Визначити процедури обчислення:

а) відрізка з найбільшою довжиною;

б) прямокутника з найбільшим периметром;

в) кола з найменшою площею.

41) У файлі записані координати точок на площині задані парою цілих
чисел. Точки записуються в форматі : ( х1 , х2 ) (х1 , х2) , \ldots{} -
саме так через коми та дужки. Створити файл, в якому будуть записані
координати всіх відрізків з точок цього файлу, при цьому ці відрізки
відсортовані за зростанням довжини.

42) У файлі записані координати Точок в просторі задані трійкою цілих
чисел. Точки записуються в форматі : х1 , х2 , х3 ; х1 , х2, х3;
\ldots{} Знайти відрізок з точок цього файлу, що має найбільшу довжину.

43) У файлі записані координати матеріальних точок на площині задані
парою цілих чисел та масою(дійсне число). Точки записуються в форматі :
{[}х1 , y1, m1 {]}, {[}х2 , y2, m2{]} , \ldots{} \textbf{- саме так
через коми та дужки. Знайдіть дві точки} з найбільшим важілем сили (m*(х
+y)).

45) У файлі записані дати , що задані трійкою цілих чисел у форматі
(чч1./мм1/рр1),(чч2./мм2/рр2), \ldots{} - саме в такому форматі.
Створити файл, в якому будуть записано найстарша та найсвіжіша дати
(врахуйте, що роки дат з 1951 по 2049).

46) У файлі записані дати , що задані двома цілими числами та рядком
(англійські назви місяця) у форматі: чч1 місяць1 рік1; чч1 місяць1
рік1;\ldots{} Знайти різницю в днях між найстаршою та найсвіжішою датою.

47) Відомості про учня складаються з його імені, прізвища та назви

класу (рік навчання та літери), в якому він вчиться. Дано файл, який

містить відомості про учнів школи. Скласти підпрограми, які дозволяють:

а) визначити, чи є в школі учні з однаковим прізвищем;

б) визначити, чи є учні з однаковим прізвищем у паралельних класах;

в) визначити, чи є учні з однаковим прізвищем у певному класі;

г) відповісти на питання а)-в) стосовно учнів, у яких збігаються ім'я та

прізвище;

ґ) визначити, в яких класах налічується більше 35 учнів;

д) визначити, на скільки учнів у восьмих класах більше, ніж у десятих;

е) зібрати у файл відомості про учнів 9-10-х класів, розташувавши

спочатку відомості про учнів класу 9 а, потім -- 9 б тощо;

є) отримати список учнів даного класу за зразками:

Прізвище Ім'я

Прізвище І.

І.Прізвище.

48) Дано файл, який містить ті самі відомості про учнів школи, що й

в попередній задачі, і додатково оцінки, отримані учнями на іспитах із

заданих предметів. Скласти процедури для:

а) визначення кількості учнів, які не мають оцінок, нижче 4;

б) побудови файлу, який містить відомості про кращих учнів ш

що мають оцінки, не нижче 4;

в) друкування відомостей про учнів, які мають принаймні одну

довільну оцінку, у вигляді прізвища та ініціалів, назви класу, предмету
та оцінки.

49) Відомості про автомобіль складаються з його марки, номеру та

прізвища власника. Дано файл, який містить відомості про кілька
автомобілів. Скласти процедури знаходження:

а) прізвищ власників номерів автомобілів певної марки;

б) кількості автомобілів кожної марки.

50) Дано файл, який містить відомості про книжки. Відомості про кожну
книгу -- це прізвище автора, назва та рік видання. Скласти процедури
пошуку:

а) назв книг певного автора, виданих із 1960 р.;

б) книг із заданою назвою. Якщо така книжка є, то надрукувати прізвища
авторів і рік видання.

51) Дано файл, який містить номери телефонів співробітників установи:
вказуються прізвище співробітника, його ініціали та номер телефону.
Визначити процедуру пошуку телефону співробітника за його прізвищем та
ініціалами.

52) Дано файл з відомостями про кубики: розмір кожного (довжини ребра у
см), його колір (червоний, жовтий, зелений, синій) і матеріалу
(дерев'яний, металевий, картонний). Скласти процедури пошуку:

а) кількості кубиків кожного з перелічених кольорів, їх сумарний об'єм

б) кількості дерев'яних кубиків із ребром 3 см і металевих кубиків

ребром, більшим за 5 см.

53) Відомості про учнів (ПІБ, клас, дата народження) записуються до
файлу певного формату. Створіть функції для запису та редагування даних
у файлі.

Напишіть функцію, що записує в окремий файл в тому ж форматі учнів, що
містять всі оцінки більше 10.

Формат файлу:
\end{quote}

\begin{enumerate}
\def\labelenumi{\alph{enumi})}
\item
  JSON
\item
  CVS
\item
  XML
\end{enumerate}

54) Відомості про предмет (Викладач, класи якім він викладається, час
читання) записуються до файлу певного формату. Створіть функції для
запису та редагування даних у файлі. Напишіть функцію, що записує в
окремий файл сумарну кількість годин для кожного викладача.

\begin{quote}
Формат файлу:
\end{quote}

\begin{enumerate}
\def\labelenumi{\alph{enumi})}
\item
  JSON
\item
  CVS
\item
  XML
\end{enumerate}

55) Використовуючи дані з попередніх двох задач, напишіть функцію, що
записує в окремий файл середню оцінку кожного учня.

\begin{quote}
Формат файлів:
\end{quote}

\begin{enumerate}
\def\labelenumi{\alph{enumi})}
\item
  JSON
\item
  CVS
\item
  \begin{quote}
  XML
  \end{quote}
\end{enumerate}

4.2. Робота з файлами

\begin{enumerate}
\def\labelenumi{\arabic{enumi}.}
\item
  В даній директорії з підкаталогами відшукати всі файли з розширенням
  *.с та поміняти їх на відповідні файли з розширенням *.срр
\item
  В даній директорії з підкаталогами відшукати всі файли з розширенням
  *.cpp та поміняти там коментарі вигляду // (до кінця рядку) на
  коментарі де в початку рядку /* \ldots{} та в кінці рядку */
\item
  В даній директорії з підкаталогами відшукати всі файли з розширенням
  *.txt які модифіковані раніше заданої дати та видалити їх
\item
  В даній директорії з підкаталогами відшукати всі файли з розширенням
  *.txt які створені раніше ніж рік тому та перенести їх в іншу (задану)
  директорію
\item
  В даній директорії з підкаталогами відшукати всі файли з розширеннями
  Word які менше 10 мб та замінити їх видаляє їх.
\item
  Створити форму яка дозволяє ввести шлях до директорії та виводить
  середній розмір текстових файлів у цій директорії.
\end{enumerate}

4.3. Статичні та глобальні змінні

Розв'язати ці задачі використовуючи глобальні змінні та розв'язати ці
задачі використовуючи статичні змінні. Чим відрізняються версії програм
з глобальним та локальними змінними?

\begin{enumerate}
\def\labelenumi{\arabic{enumi})}
\item
  Реалізуйте функцію, яка виводить повідомлення скільки разів вона була
  викликана з головної функції до кожного з цих викликів. В
  імплементації головний функції зробіть так щоб користувач мав
  можливість викликати цю функцію скільки завгодно разів (наприклад,
  вводив кількість викликів)
\item
  Реалізуйте дві функції зі змінним аргументом: перша додає до аргументу
  1, друга ділить його націло на 2. Після кожного виклику однієї з цих
  функцій в головній програмі повинно виводитись повідомлення, яка з цих
  функцій викликалась частіше.
\item
  Реалізуйте функцію, що може викликатись не більше фіксованої кількості
  разів. Ця кількість разів вводиться в головній програмі або через
  командний рядок.
\end{enumerate}

\begin{quote}
\protect\hypertarget{_Hlk48905535}{}{}5. Структури

5.0 Описи структури
\end{quote}

1) Визначити типи структури для зображення наступних понять та функції
їх вводу-виводу:

\begin{quote}
а) ціна (гривні, копійки);

б) час (година, хвилина, секунда);

в) дата (число, місяць, рік);

г) адреса (місто, вулиця, будинок, квартира);

ґ) семінар (предмет, викладач, № групи, день тижня, години занять,

аудиторія);

д) бланк вимоги на книгу (відомості про книгу: шифр, автор, назва;

відомості про читача: № читацького квитка, прізвище; дата замовлення);

е) поле шахової дошки (напр., а5, b8);

є) коло (радіус, координати центра);

ж) прямокутник зі сторонами, паралельними осям координат (Точка А, Точка
Б). Точка --- дві дійсні координати;

сфера в просторі;

прямокутний паралеліпіпед (сторони якого паралельні осям координат);

поліном довільного ступеня (дійсні коефіцієнти --- безрозмірний масив).

2) Використовуючи тип Поле шахової дошки описати булеву функцію, яка
перевіряє, чи може ферзь за один крок перейти з одного заданого поля
шахової дошки на інше задане поле.

3) Визначимо тип Rational (Раціональне число) як:

typedef struct \{

int numerator; // чисельник

unsigned int denominator; // знаменник

\} Rational;

Визначити функції для:

а) обчислення суми двох раціональних чисел;

б) обчислення добутку двох раціональних чисел;

в) порівняння двох раціональних чисел;

г) зведення раціонального числа до нескоротного виду.

58) Використовуючи опис типу Дата, визначити функції обчислення:

а) дати вчорашнього дня;

б) дня тижня за його датою в поточному році.

5) Задано масив розмірності N, компонентами якого є структури, що
містять відомості про вершини гір. У відомостях про кожну вершину
вказуються назва гори та її висота. Визначити функції введення/виведення
гір та функції пошуку назви найвищої вершини та виведення висоти вершини
з заданою назвою (якщо вершини з такою назвою немає в масиви --- вивести
відповідне повідомлення).

6) Відомо вартість і "вік" кожної з N моделей легкових автомобілів.

Визначити середню вартість автомобілів, вік яких більший за 5 років.

7) Відомо інформацію про ціну та наклад кожного з N журналів.

Знайти середню вартість журналів, наклад яких менший за 10000 при-

мірників.

8) Відомі дані про масу й об'єм N предметів, виготовлених із різ-

них матеріалів. Знайти предмет, густина матеріалу якого найбільша.

9) Відомі дані про чисельність населення (у мільйонах жителів) та

площі N держав. Знайти країну з мінімальною щільністю населення.

10) Задано масив С розмірності N, компонентами якого є відомості про
мешканців деяких міст. Інформація про кожного мешканця містить його
прізвище, назву міста, місцеву адресу у вигляді вулиці, будинку,
квартири. Визначити процедуру пошуку двох будь-яких жителів, що мешкають
у різних містах за однаковою адресою.

11) Відомо дані про вартість кожного з N найменувань товарів:

кількість гривень, кількість копійок. Скласти підпрограми пошуку:

а) найдешевшого товару в магазині;

б) найдорожчого товару в магазині;

в) товару, вартість якого відрізняється від середньої вартості товару

в магазині не більш ніж на 5 гривень:

12) Задано масив Р розмірності N, компонентами якого є записи,

що містять анкети службовців деякого закладу. У кожній анкеті вказується
прізвище та ім'я службовця, його стать, дата народження у вигляді числа,
місяця, року. Визначити підпрограми пошуку:

а) посади, яку обіймає найбільша кількість співробітників;

б) співробітників з однаковими іменами;

в) співробітників, прізвища яких починаються із заданої літери;

г) найстаршого з чоловіків цього закладу;

ґ) співробітників, вік яких менший за середній по організації;

д) пенсійного віку (урахувати, що пенсійний вік чоловіків і жінок --
різний).

13) Задано масив Р, компонентами якого Рi є записи, що містять дані про
людину на ім'я i з указаного списку. Кожне дане складається зі статі
людини та її зросту. Визначити підпрограми для:

а) обчислення середнього зросту жінок;

б) пошуку найвищого чоловіка;

в) перевірки, чи є дві людини, однакові на зріст.

14) Задано масив розмірності N, компоненти якого містять інформацію про
студентів деякого вишу. Відомості про кожного студента містять дані про
його прізвище, ім'я, по батькові, стать, вік, курс. Визначити процедуру
пошуку:

а) найпоширеніших чоловічих і жіночих імен;

б) прізвищ та ініціалів усіх студентів, вік яких є найпоширенішим.

15) Задано масив розмірності N, компонентами якого є відомості про
складання іспитів студентами деякого вишу. Інформація про кожного
студента задана в такому вигляді: прізвище, номер групи, оцінка\_1,
оцінка\_2, оцінка\_3. Визначити процедуру пошуку:

а) студентів, що мають заборгованості принаймні з одного з предметів;

б) предмета, складеного найуспішніше;

в) студентів, що склали всі іспити на 5 і 4.

Визначити універсальний тип, який допускає зображення точки

на площині у прямокутній або полярній системі координат (3-тє поле --
тип координат). Побудувати функцію обчислення площі трикутника з
вершинами A, B, C.

5.1. Файли бінарні

1) Нехай множина цілих чисел задана у файлі. Визначити:

а) процедуру введення множини;

б) процедуру виведення множини;

в) процедуру доповнення множини;

г) процедуру видалення елемента з множини;

ґ) функцію, що дає відповідь, чи входить елемент до множини;

д) функцію, що дає відповідь, чи порожня множина;

е) функцію, що знаходить максимальний елемент множини;

є) функцію, що знаходить мінімальний елемент множини;

ж) процедуру об'єднання множин;

з) процедуру різниці множин;

и) процедуру перетину множин;

і) функцію обчислення ваги множини;

ї) функцію обчислення діаметра множини;

й) функцію, що за множиною A знаходить підмножину всіх таких її
елементів, для яких справедлива умова Q(х), x∈A;

к) функцію, що з'ясовує, чи є множина A підмножиною множини В;

л) функцію, що з'ясовує, чи дорівнює множина A множині В.

2) Дано файл, компоненти якого є записи (koef, st) -- коефіцієнт і

степінь членів полінома (koef ≠ 0). Визначити підпрограми для виконання
таких дій над поліномом:

а) введення полінома;

б) друк полінома;

в) обчислення похідної від полінома;

г) обчислення невизначеного інтеграла від полінома;

ґ) упорядкування за степенями елементів полінома;

д) приведення подібних серед елементів полінома;

е) додавання, віднімання двох поліномів;

є) множення двох поліномів;

ж) знаходження частки та залишку від ділення двох поліномів;

з) знаходження полінома за лінійної заміни змінної x = dx + c, d ≠0;

и) знаходження полінома за заміни змінної x = d/x, d ≠ 0;

і) знаходження ступеня поліному;

ї) з'ясування, чи має поліном корені, рівні нулю, і визначення їхньої
кратності;

й) знаходження максимального за умовою Q(t) коефіцієнта серед
коефіцієнтів полінома, які задовольняють умову G(t);

к) знаходження мінімального за умовою Q(t) коефіцієнта серед
коефіцієнтів полінома, які задовольняють умову G(t);

л) знаходження значення полінома в заданій точці.

3) \protect\hypertarget{_Hlk65238097}{}{}Дано файл, компоненти якого є
дійсними числами. Скласти підпрограми для обчислення:

а) суми компонент файлу;

б) кількості від'ємних компонент файлу;

в) останньої компоненти файлу;

г) найбільшого зі значень компонент файлу;

ґ) найменшого зі значень компонент файлу з парними номерами;

д) суми найбільшого та найменшого зі компонент;

е) різниці першої й останньої компоненти файлу;

є) кількості компонент файлу, які менші за середнє арифметичне всіх

його компонент.
\end{quote}

\begin{enumerate}
\def\labelenumi{\arabic{enumi})}
\item
  \begin{quote}
  \protect\hypertarget{_Hlk65237989}{}{}Дано файл, компоненти якого є
  цілими числами. Скласти підпрограми для обчислення:
  \end{quote}
\end{enumerate}

\begin{quote}
а) кількості парних чисел серед компонент;

б) кількості квадратів непарних чисел серед компонент;

в) різниці між найбільшим парним і найменшим непарним числами

компонент;

г) кількості компонент у найдовшій зростаючій послідовності компонент
файлу.

4) Дано файл F, компоненти якого є цілими числами. Побудувати

файл G, який містив би всі компоненти файлу F:

а) що є парними числами;

б) що діляться на 3 і на 5;

в) що є точними квадратами;

г) записані у зворотному порядку;

ґ) за винятком повторних входжень одного й того самого числа.

5) Використовуючи файл F, компоненти якого є цілими числами,

побудувати файл G, що містить усі парні числа файлу F, і файл H -- усі

непарні. Послідовність чисел зберігається.

6) Задано натуральне число n та файл F, компоненти якого є цілими
числами. Побудувати файл G, записавши до нього найбільше значення перших
n компонент файлу F, потім -- наступних n компонент тощо. Розглянути два
випадки:

а) кількість компонент файлу ділиться на n;

б) кількість компонент файлу не ділиться на n. Остання компонента файлу
g має дорівнювати найбільшій із компонент файлу F, які утворюють останню
(неповну) групу.

7) Дано файл F, компоненти якого є цілими числами. Файл містить

рівне число додатних і від'ємних чисел. Використовуючи допоміжний файл

H, переписати компоненти файлу F до файлу G так, щоб у файлі G:

а) не було двох сусідніх чисел одного знаку;

б) спочатку йшли додатні, потім -- від'ємні числа;

в) числа йшли таким чином: два додатних, два від'ємних тощо (при-

пускається, що число компонент у файлі F ділиться на 4).

8) Дано файл F, компонентами якого є записи (структури) вигляду

struct T \{

unsigned Key; // ключ

char Data{[}10{]}; // дані

\};

Такий файл називатимемо впорядкованим за ключами, якщо записи в ньому
розташовуються в порядку зростання (спадання) ключів. Скласти процедуру
пошуку запису за ключем у впорядкованому файлі. Скласти процедуру
вилучення запису із заданим ключем:

а) з впорядкованого файлу;

б) з невпорядкованого файлу.

9) Багаж пасажира характеризується номером пасажира, кількістю

речей і їхньою загальною вагою. Дано файл пасажирів, який містить
прізвища пасажирів, і файл, що містить інформацію про багаж кілько

пасажирів (номер пасажира -- це номер запису у файлі пасажирів)

Скласти процедури для:

а) знаходження пасажира, у багажі якого середня вага однієї речі

відрізняється не більш ніж на 1 кг від загальної середньої ваги речей;

б) визначення пасажирів, які мають більше двох речей, і пасажирів

кількість речей у яких більша за середню кількість речей;

в) видачі відомостей про пасажира, кількість речей у багажі якого н

менша, ніж у будь-якому іншому багажі, а вага речей -- не більша, ніж

будь-якому іншому багажі із цією самою кількістю речей;

г) визначення, чи мають принаймні два пасажири багажі, які не
відрізняються за кількістю речей і відрізняються вагою не більш ніж на 1
кг (якщо такі пасажири є, то показати їхні прізвища);

ґ) визначення пасажира, багаж якого складається з однієї речі вагою не
менше 30 кг.

10) Дано файл, який містить відомості про іграшки: указано назву

іграшки (напр., м'яч, лялька, конструктор тощо), її вартість у гривнях і

вікові межі для дітей, яким іграшка призначається (напр., для дітей від

двох до п'яти років). Скласти процедури:

а) пошуку назв іграшок, вартість яких не перевищує 40 грн, призначених
дітям п'яти років;

б) пошуку назв іграшок, призначені дітям і чотирьох, і десяти років;

в) пошуку назв найдорожчих іграшок (ціна яких відрізняється від ціни

найдорожчої іграшки не більш ніж на 50 грн);

г) визначення ціни найдорожчого конструктора;

ґ) визначення ціни всіх кубиків;

д) пошуку двох іграшок, що призначені дітям трьох років, сумарна

вартість яких не перевищує 20 грн;

е) пошуку конструктора ціною 22 грн, призначеного дітям від п'яти до

десяти років. Якщо такої іграшки немає, то занести відомості про її від-

сутність до файлу.

11) Дано файл, який містить відомості про прямокутники: указано

номер прямокутника у файлі, координати верхнього лівого кута, нижнього
правого кута прямокутника. Скласти процедуру пошуку прямокутника

з найбільшою площею й визначення цієї площі.

12) \protect\hypertarget{_Hlk65238015}{}{}У двох файлах міститься
таблиця футбольного турніру, у першому -- записано назви команд; у
другому -- результати матчів, що зберігаються у записах типу T\_Match

typedef struct \{

unsigned int n1, n2;

unsigned int b1, b2;

\} T\_Match;

Тут у структурі типу T\_Match поля n1, n2 -- номери першої і другої

команд (тобто номери назв команд у файлі команд); b1, b2 -- кількість

м'ячів, забитих першою та другою командами, відповідно.

Кожній команді за перемогу нараховується 3 очки, за нічию -- 1, за

поразку -- 0.

Із двох команд, які мають однакову кількість очок, першою вважається та,
що має кращу різницю забитих і пропущених м'ячів;

за однакової різниці має більше забитих м'ячів;

за всіма однаковими попередніми показниками визначається жеребкуванням
(для жеребкування використати генератор випадкових чисел).

Знайти команду, яка є лідером.

Вказівка. Описати підпрограми створення файлів команд і матчів,

додавання результату матчу, визначення лідера.

13) Файл бази даних з малюнками містить на початку ціле 32-бітне число
2051, потім ціле 32-бітне число -- кількість малюнків, а наступні два
32-бітних числа -- кількість пікселів висоту та ширину кожного малюнку у
пікселах. При цьому ці числа задані в форматі high-indian (MSB first).
Наступний вміст файлу -- беззнакові натуральні байти (K*n*m байтів),
кожен з яких -- значення яскравостей пікселів (число від 0 до 255)
кожного з цих малюнків, що проходяться у порядку зліва-направо та
зверху-вниз.

Напишіть функцію, що перевіряє даний файл (заданий ім'ям) на
відповідність даному формату, та виводить масив яскравостей малюнка з
заданим номером, якщо такий номер та сам файл коректно задані. В
противному випадку вивести змістовне повідомлення про помилку.

14) Для представлення баз даних, що мітять тензори часто використовують
формат IDX (IDX file format), який має наступну форму:

magic\_number -- 32-бітове число у форматі high-indian (MSB first), в
якому перші 2 байти нулі, третій байт описує тип даних: якщо 0x08
-unsigned byte, 0x09 -- signed byte, 0x0B -- short(2 bytes), 0x0C -- int
(4 bytes), 0x0D -- float (4 bytes), 0x0E -- double (8 bytes), четвертий
байт -- кількість N розмірностей тензору;

size 1 - 32-бітове число у форматі high-indian (MSB first) величина
першої розмірності;

size 2 - 32-бітове число у форматі high-indian (MSB first) величина
другої розмірності;

**

size N - 32-бітове число у форматі high-indian (MSB first) величина N-ої
розмірності;

далі йдуть дані вказаного у першому числі формату:

data - Cі-масив даних у форматі high-indian (MSB first).

Напишіть функцію, що перевіряє даний файл (заданий ім'ям) на
відповідність даному формату, та виводить координату тензору, що задана
аргументом функції. В випадку, коли це не можливо, вивести змістовне
повідомлення про помилку.
\end{quote}

\subsubsection{5.2 Командний
рядок}\label{ux43aux43eux43cux430ux43dux434ux43dux438ux439-ux440ux44fux434ux43eux43a}

1) Напишіть програму, що приймає з командного рядку 1 цілий аргумент та
виведіть його квадрат. Якщо аргументів 2 або більше, або жодного --
виведіть повідомлення про помилку.

2) Напишіть програму, що приймає з командного рядку 3 дійсних аргументи
та виводить їх середнє гармонічне. Якщо аргументів більше трьох, або
менше -- виведіть повідомлення про помилку. Якщо серед них є нуль ---
інше повідомлення про помилку.

3) Введіть з командного рядочку ім'я текстового файлу та підрахуйте
кількість рядків в цьому файлі. Виведіть повідомлення про помилку якщо
щось негаразд.

4) Введіть з командного рядочку ім'я декількох текстових файлів (їх
повинно бути більше одного) та підрахуйте середню щільність символів на
рядок в цих файлах.

5) Введіть через командний рядочок наступного вигляду:

-filename name -rows rows ,

імя файлу (name) та кількість рядків (rows),

параметри -filename та -rows -- це обов'язкові літерали в рядку.

Якщо формат команди не такий як приведений вище -- виведіть повідомлення
про помилку та підказку. Якщо все вірно, створіть відповідний бінарний
файл, що містить вказану кількість цілих чисел від 0 до rows.

6) Введіть через командний рядочок рядок наступного вигляду:

-filename name -rows rows -cols cols

імя файлу (name) та кількість рядків (rows) та стовпчиків(cols),

параметри -filename та -rows, cols -- це обов'язкові літерали в рядку.

Якщо формат команди не такий як приведений вище -- виведіть повідомлення
про помилку та підказку. Якщо все вірно, створіть відповідний текстовий
файл, що містить вказану кількість рядків заповнену cols нулями через
табуляцію.

7) Введіть через командний рядочок рядок наступного вигляду:

-filename1 name1 -filename2 name2 -rows,

rows імя файлу (name) та кількість рядків (rows) параметри -filename1 та
-filename2, це обовязкові літерали в рядку.

А параметр -rows rows може бути необов'язковий.

Якщо формат команди не такий як приведений вище -- виведіть повідомлення
про помилку та підказку. Якщо все вірно, порівняйте чи співпадає в даних
двох файлах перші rows рядків з точністю до пробілів, якщо параметр rows
не вказаний -- файли порівнюються повністю за всіма рядками.

Напишіть програму, яка приймає ціле число як аргумент командного рядка і
знаходить усі його дільники.

Напишіть програму, яка приймає в якості аргументу командного рядка ім'я
текстового файлу. Відкрийте цей файл і прочитайте його по одному слову
(підказка: використовувати \textgreater{}\textgreater{}). Збережіть
кожне слово у вектор \textless{}string\textgreater{}. Примусити всі
слова в нижній регістр, відсортувати їх, видалити всі дублікати та
надрукувати результати.

\subsubsection{5.3. Змінні
оточення}\label{ux437ux43cux456ux43dux43dux456-ux43eux442ux43eux447ux435ux43dux43dux44f}

\begin{enumerate}
\def\labelenumi{\arabic{enumi})}
\item
  Напишіть функцію, яка визначає тип операційної системи даного
  комп'ютера.
\item
  Напишіть функцію, яка записує вміст даного файлу в новий файл, що
  знаходиться в системній директорії.
\item
  Напишіть функцію, що визначає чи існує в системі змінна JAVA\_PATH, та
  якщо немає, то встановлює цю змінну коректним шляхом.
\item
  Напишіть функцію, що визначає чи існує в системі змінна JAVA\_PATH, а
  в поточній директорії файл file1.java та якщо є, то запускає з консолі
  команду `JAVA\_PATH file1.java'
\end{enumerate}

\subsubsection{5.4. Тип
перерахування}\label{ux442ux438ux43f-ux43fux435ux440ux435ux440ux430ux445ux443ux432ux430ux43dux43dux44f}

\begin{enumerate}
\def\labelenumi{\arabic{enumi})}
\item
  Створіть та реалізуйте за допомогою перерахування базові функції
  вводу-виводу для наступних сутностей:
\end{enumerate}

\begin{enumerate}
\def\labelenumi{\alph{enumi})}
\item
  день тижня;
\item
  місяць у році;
\item
  колір спектру;
\item
  шахова фігура.
\end{enumerate}

\begin{enumerate}
\def\labelenumi{\arabic{enumi})}
\item
  Опишіть тип -- структуру Card для карти з колоди для преферансу. Для
  цього створіть перерахування Масть= \{Піка, Трефи, Бубна, Чирва\} та
  Ранг =\{7,8,9,10, 'Jack', `Queen','King','Ace'\}. Реалізуйте логічну
  функцію beat(Card x, Card y, Масть z), що вказує чи бє перша карта
  другу, а третій параметр вказує яка масть є козирною.
\item
  \protect\hypertarget{_Hlk48906187}{}{}Створіть перелік величин довжини
  (мм, см, дм, м, км) та реалізуйте функцію яка за введеною довжиною та
  величною виміру виводить довжину в метрах.
\item
  Створіть перерахування Відмінок= \{ім, бат, дат, \ldots{} \} та за
  вказаним відмінком провідмінюйте задані слова -- програмування, мова,
  комп'ютер.
\item
  Створіть перерахування Голосні, яке містить всі англійські(українськи)
  голосні та за допомогою цього типу визначить яка кількість складів в
  даному реченні (вважаючи, що склад містить лише одну голосну).
\item
  Створіть перерахування Course=\{N,S,W,E\} та Order=\{Forward, Back,
  Left,Right\}. В нас задано початковий курс корабля та масив команд як
  він рухався. Виведіть кінцевий напрямок корабля. Введіть також
  швидкість судна та масив дійсних чисел, що відповідає часу -- скільки
  воно рухалося за даним курсом та за допомогою цих даних визначте на
  яку абсолютну відстань від початкової змістився корабель.
\end{enumerate}

\subsubsection{5.5.
Об'єднання}\label{ux43eux431ux454ux434ux43dux430ux43dux43dux44f}

\begin{enumerate}
\def\labelenumi{\arabic{enumi})}
\item
  Визначити універсальний тип, що дозволяє представляти точку на площині
  в декартовій та полярних координатах. Введіть дві точки та обчисліть
  довжину відрізку на даних точках.
\item
  Визначити універсальний грошовий тип, що може представляти вартість
  або в гривнях та копійках, або лише в копійках з методом, що дозволяє
  при цьому правильно відображати ті самі вартості.
\item
  Визначити універсальний тип, що дозволяє представляти вектор в як дві
  точки та як точку та вектор до другої точки. Введіть три вектори та
  з'ясуйте чи колінеарні вони.
\item
  Визначити універсальний тип, що дозволяє представляти точку в просторі
  в декартовій, полярній та сферичних координатах. Введіть дві точки та
  обчисліть довжину відрізку на даних точках.
\item
  Визначити тип Пласка Фігура, що включає Круг, Квадрат, Трикутник,
  Прямокутник, Трапеція. Реалізуйте функції обчислення периметру та
  площі фігури.
\item
  Визначте тип, що дозволяє зберігати число або будь-якого числового
  типу (double, int, unsigned) або рядки «Нескінченість» та
  «Невизначеність». Реалізуйте арифметичні операції для цього типу які
  коректно працюють з діленням та іншими операціями для всіх можливих
  комбінаціях значень та типів.
\end{enumerate}

\subsection{6.0 Введення-виведення
Сі++}\label{ux432ux432ux435ux434ux435ux43dux43dux44f-ux432ux438ux432ux435ux434ux435ux43dux43dux44f-ux441ux456}

1) \protect\hypertarget{_Hlk65238415}{}{}Ввести в двох різних рядках
послідовно два дійсних числа x та y та обчислити значення x в ступені y.
Результат вивести в десятковому та науковому представленні.

2) \protect\hypertarget{_Hlk65238515}{}{}Ввести декілька (невідомо
зазделегідь скільки) дійсних числа записаних через коми та обчислити
значення функції log() для кожного з них. Якщо значення виходить за межі
області вивести слово ``None'', для інших значень результат вивести в
науковому та десятковому представленні шириною 5 символів.

3) Три додатніх дійсні числа вводяться як рядок вигляду

А=ххх.ххх, B=xxExxx C=xxx.xxxx

Обчисліть їх середнє гармонійне та виведіть у науковому та звичайному
форматі.

4) Ввести дійсне число від -10000 до 10000 та вивести його k-ту ступінь
з точністю до 20 знаків до десяткової коми та 4 значками після
десяткової коми.

5) \protect\hypertarget{_Hlk65238442}{}{}На терміналі вводяться 10*n
цифр. Перші 10 цифр -- це перше натуральне число, наступні 10 -- друге і
так далі. Введіть всі ці числа в масив розміру n та обчисліть і виведіть
їх суму (вважайте що сума влазить в точність unsigned long long ).

6) Вивести на екран таблицю, слідкуючи, щоб виведення було рівним та
кількість цифр після коми була або 0 або 2:

+++++++++++++++ +++++++++++

+число + 1 + 2 + 3 + 4 + 5

++++++++++++++++++++++++++++

+експонента+ 1 +1.44 + 1.69 + 2

++++++++++++++++ ++++++++++

\begin{enumerate}
\def\labelenumi{\arabic{enumi})}
\item
  Ввести з текстового файлу та з консолі натуральне число n та масиви з
  n цілих чисел \(\left\{ m_{i} \right\}_{i = 1}^{n}\) та дійсних чисел
  \(\left\{ x_{i} \right\}_{i = 1}^{n}\). Обчислить та виведіть у файл
  числа \(\left\{ x_{i}^{m_{i}} \right\}_{i = 1}^{n}\).
\end{enumerate}

7) \protect\hypertarget{_Hlk65238464}{}{}Вхідний потік містить набір
цілих чисел Ai (0 ≤ Ai ≤ 1018), відділений один від іншого довільною
кількістю пробілів і переводів рядків. Розмір вхідного потоку не
перевищує 256 КБ. Для кожного числа Ai, починаючи з останнього та
завершуючи першим, в окремому рядку вивести його квадратний корінь не
менш ніж з чотирма знаками після десяткової крапки.

Приклад:

\textbf{Вхід:}

1427 0

876652098643267843

5276538

\textbf{Вихід: }

2297.0716

936297014.1164

0.0000

37.7757

8)* \protect\hypertarget{_Hlk65238487}{}{}Розглянемо послідовність чисел
\(a_{i}\) , i = 0, 1, 2, \ldots{}, що задовольняють умовам:

\(a_{0} = 0\), \(a_{1} = 1\), \(a_{2i} = a_{i}\) a,
\(a_{2i + 1} = {2a}_{i} + 1\) для кожного i = 1, 2, 3, \ldots{} .

Напишіть програму, яка для заданого значення n знаходить максимальне
серед чисел \(a_{0},a_{1},\cdots,a_{n}\). Вхідні дані складаються з
декількох тестів (не більше 10). Кожен тест - рядок, в якому записано
ціле число n (1 ≤ n ≤ 99 999). В останньому рядку вхідних даних записано
число 0. Для кожного n у виводі запишіть максимальне значення.

\begin{quote}
\emph{9)* Створити текстовий (.txt) файл з 100,000,000 рядків з числами
в діапазоні від 0 до 99,999,999:}
\end{quote}

\emph{формат чисел - 8 нулів (1 = 00000001, 65535 = 00065535) , діапазон
від 0 до 99999999, всі числа розташовані в випадковому порядку без
повторів (кожен рядок -- унікальне число)}

\emph{Приклад.}

\begin{quote}
\emph{00306453 }

\emph{99645283 }

\emph{70000021 }

\emph{06847127 }
\end{quote}

\subsection{7.0 Рядки С++}\label{ux440ux44fux434ux43aux438-ux441}

В даній групі задач потрібно реалізувати функції та в тих функціях де
потрібно виводити рядок зробіть 2 варіанти: 1) Результат записати в
новий рядок. 2) Результат замінює рядок, що є аргументом функції.

\begin{enumerate}
\def\labelenumi{\arabic{enumi})}
\item
  Даний рядок, що складається з символів латинського алфавіту,
  розділених пробілами (одним або декількома). Перетворити кожне слово в
  рядку, видаливши з нього всі входження першої літери цього слова
  (кількість пропусків між словами не змінювати).
\item
  Даний рядок, що складається з символів латинського алфавіту,
  розділених пробілами (одним або декількома). Визначити кількість слів,
  які починаються і закінчуються однією і тією ж буквою.
\item
  У мові використовується латинський алфавіт. Дієслово минулого часу
  виходить з дієслова теперішнього часу зміною порядку проходження
  голосних (а, о, u, i, е) на зворотний. Приголосні літери залишаються
  на своїх місцях. Наприклад, дієслово padbote перетворюється в pedbota.
  Здається дієслово теперішнього часу. Перетворити його в дієслово
  минулого часу і надрукувати.
\item
  Даний рядок -- речення з символів латинського алфавіту. Вивести
  найкоротший слово в реченні (якщо таких слів кілька, то вивести перше
  з них).
\item
  Даний рядок, що складається з символів латинського алфавіту,
  розділених пробілами (одним або декількома). Визначити кількість слів,
  які містять рівно три букви «А».
\item
  Даний рядок із символів латинського алфавіту. Перевірте правильність
  розстановки тега \textless{}td\textgreater{}: кожному відкритого тегу
  повинен відповідати закритий \textless{}/ td\textgreater{}.
\item
  Даний рядок, що складається з символів латинського алфавіту,
  розділених пробілами (одним або декількома). Визначити довжину
  найдовшого слова.
\item
  Даний рядок, що складається з символів латинського алфавіту,
  розділених пробілами (одним або декількома). Вивести рядок, що містить
  ці ж слова, але розділені одним символом '.' (точка, крапка). В кінці
  крапку не ставити.
\item
  Даний рядок, що складається з символів латинського алфавіту,
  розділених пробілами (одним або декількома). Перетворити кожне слово в
  рядку, видаливши з нього всі входження останньої літери цього слова
  (кількість пропусків між словами не змінювати).
\item
  Речення складається з слів, розділених одним або декількома
  пропусками. Написати програму, що друкує все слова, що закінчуються на
  заданий символ.
\item
  У реченні, що складається зі слів, відокремлених одним пропуском,
  замінити першу букву у слів, що настають за словами die, der, das, на
  прописну.
\item
  Даний рядок, що складається з символів латинського алфавіту,
  розділених пробілами (одним або декількома). Перетворити кожне слово в
  рядку видаливши з нього всі входження заданого символу (кількість
  пропусків між словами не змінювати).
\item
  Даний рядок-речення з символів латинського алфавіту. Перетворити рядок
  так, щоб кожне слово починалося з великої літери.
\item
  Даний рядок-речення з символів латинського алфавіту. Вивести найдовше
  слово в реченні (якщо таких слів кілька, то вивести останнє з них).
\item
  Визначити, скільки разів в рядку зустрічається задане слово.
\item
  У записці слова зашифровані - кожне з них записано навпаки.
  Розшифрувати повідомлення.
\item
  Даний рядок з восьми цифрових символів. Переведіть її в формат дати
  "dd-mm-yyyy" і перевірте коректність такої дати.
\item
  Даний рядок, що складається з символів латинського алфавіту,
  розділених пробілами (одним або декількома). Визначити кількість слів,
  які містять введений символ.
\item
  З'ясуйте, чи є серед введених символів всі букви, що входять в задане
  слово.
\item
  Речення складається з слів, розділених одним або декількома
  пропусками. Написати програму, що друкує все слова, що починаються на
  введений символ.
\item
  У англійському реченні слова розділені одним пропуском. У всіх словах,
  наступних за артиклями a, an та the, першу букву замінити на прописну.
  Написати програму, що виконує цю роботу.
\item
  Написати програму, що визначає, який відсоток слів в англійському
  тексті містить подвоєну приголосну.
\item
  У мові використовується латинський алфавіт, причастя завжди
  закінчується суфіксом "ings". Задана рядок слів, в якій слова
  відокремлюються одним або декількома пропусками. Надрукувати причастя
  з цього рядку.
\item
  Даний рядок з малих символів латинського алфавіту. Замініть кожен
  символ на наступний за ним за алфавітом, символ 'z' замініть на 'a'.
\item
  Даний рядок із символів латинського алфавіту. Замініть всі входження
  рядків ``one'', "two","three",\ldots{},''nine'' на символи `1',
  '2','3',\ldots{},'9'.
\item
  Відредагувати задане речення, видаляючи з нього ті слова, які
  зустрічаються в реченні задану кількість разів.
\item
  Визначте, який відсоток символи кожного слова складають з символів
  даного речення.
\item
  Дан текст, що складається з символів латинського алфавіту, пробілів і
  знаків пунктуації. Знайдіть найпоширенішу голосну букву (без
  урахування регістру).
\item
  Даний рядок. Групи символів, що відокремлені пропусками (одним або
  кількома) і не містять пропусків усередині, називатимемо словами.
  Скласти підпрограми для:
\end{enumerate}

а) знаходження найдовшого слова;

б) визначення кількості слів

в) вилучення з рядку зайвих пропусків і всіх слів, що складаються з
однієї літери;

г) вилучення всіх пропусків на початку рядків, у кінці рядків і між
словами (крім одного);

д) вставки пропусків до рядків рівномірно між словами так, щоб довжина
всіх рядків (якщо в них більше 1 слова) була 80 символів і кількість

пропусків між словами в одному рядку відрізнялась не більше ніж на 1

(вважати, що рядки файлу мають не більш ніж 80 символів).

30) В заданий рядок входять тільки цифри та літери. Перевірте це.
Визначити, чи задовольняє він наступній властивості:

а) рядок є десятковим записом числа, кратного 9 (6, 4);

б) рядок починається з деякої ненульової цифри, за якою знаходяться
тільки літери і їх кількість дорівнює числовому значенню цієї цифри;

в) рядок містить (крім літер) тільки одну цифру, причому її числове
значення дорівнює довжині рядка;

г) сума числових значень цифр, які входять в рядок, дорівнює довжині
рядка;

д) рядок співпадає з початковим (кінцевим, будь-яким) відрізком ряду
0123456789;

е) рядок складається тільки з цифр, причому їх числові значення
складають арифметичну прогресію (наприклад, 3 5 7 9, 8 5 2, 2).

\subsection{}\label{section-1}

\subsection{8. ООП (об'єктно-орієнтоване
програмування)}\label{ux43eux43eux43f-ux43eux431ux454ux43aux442ux43dux43e-ux43eux440ux456ux454ux43dux442ux43eux432ux430ux43dux435-ux43fux440ux43eux433ux440ux430ux43cux443ux432ux430ux43dux43dux44f}

\begin{quote}
\protect\hypertarget{_Hlk57988688}{}{}
\end{quote}

Питання по Лекції:

1) Що таке класи і які шляхи визначення класів в Сі++?

2) Яким чином можна визначити методи класу?

3) Приватний та публічний доступ до членів та методів. Яка різниця?

4) Які методи в класі визначені за замовченням? Як і коли потрібно ці
методи визначати самостійно?

5) Шляхи визначення конструктору класу. Як викликати конструктор в
головній функції?

6) Статичні члени та методи класу. Як визначити і коли вони потрібні?

7) Дружні класи та методи. Як вони використовуються?

Вправи:

1) а) Визначити клас раціональне число з членами: nominator --- ціле
число, denominator --- натуральне число. Визначити методи введення та
виведення з терміналу, методи додавання та множення раціонального числа

б) Зробіть члени класу приватними та визначить методи ініціалізації
окремо чисельника і знаменника (при цьому не дайте користувачу
можливість ініціалізувати знаменник нулем)

в) Створіть приватний метод класу для скорочення раціонального числа
через НСД

г) Визначить конструктори класу який ініціалізує за замовченням
раціональне число одиницями та конструктор, що ініціалізує його двома
довільними числами

2) Визначить клас Вектор, що ініціалізується кількістю елементів масиву
N та виділяє при цьому пам'ять під N дійсних чисел. Створіть методи для
заповнення членів цього масиву (через конструктор та окремим методом) та
конкретного елементу вектору за номером. Визначить деструктор та
копіконструктор

3) В класі Monomial з лекції за допомогою статичного члену заборонить
визначати більш ніж декілька екземплярів класу.

Визначить свою дружню функцію для цього класу для виведення його в
текстовий файл.

\subsection{8.1. Опис
класів}\label{ux43eux43fux438ux441-ux43aux43bux430ux441ux456ux432}

\begin{enumerate}
\def\labelenumi{\arabic{enumi})}
\item
  Описати клас \textbf{Точка} (на площині). Реалізуйте методи введення,
  виведення. Описати клас \textbf{Відрізок} (на площині), що складається
  з 2-х точок та містить крім введення/виведення методи підрахунку
  середини відрізку, довжини відрізку. \emph{За допомогою визначення
  порожньої Точки реалізуйте метод перетину двох відрізків, що повертає
  Точку (у випадку, якщо цих точок декілька виведіть будь-яку з них, а
  якщо жодної -- порожній відрізок).}
\item
  Описати клас \textbf{Коло} (на площині), що задається координатами
  центру та радіусом. Описати методи отримання довжини діаметру, площі
  та периметру кола, перетину двох кіл (повертає відповідно 0,1 або 2
  точки як масив через змінний аргумент).
\item
  Описати клас \textbf{Прямокутник}. Сторони прямокутника паралельні
  осям координат. Для прямокутника задані координати лівого верхнього
  кута та довжини сторін. Описати методи отримання довжини кожної зі
  сторін, площі та периметру, перетину двох прямокутників (якщо перетин
  порожній -- поверніть Прямокутник вигляду(-1,-1,-1,-1)).
\item
  Описати клас \textbf{Трикутник}. Основа трикутника паралельна осі
  \emph{x} координат. Для трикутника задані лівий нижній кут та довжини
  2 сторін. Описати методи отримання довжини кожної зі сторін, кутів,
  площі та периметру.
\item
  Описати класи розділивши інтерфейс та реалізацію та заборонивши
  введення некоректних даних, з методами введення/виведення та де
  можливо додавання:
\end{enumerate}

А) \textbf{Час} (години, хвилини, секунди)

Б) \textbf{Дата}(рік, місяць, день)

В) \textbf{Валюта}( назва валюти, значення, центи(копійки))

\begin{enumerate}
\def\labelenumi{\arabic{enumi})}
\item
  Описати клас ігрова \textbf{Дошка}(визначається розміром та назвою
  гри: шашки (міжнародні, російські та турецькі), шахи, нарди) та
  \textbf{Фігура} (назва, гра, масив можливих ходів -- ходи описуються в
  термінах зрозумілих класу Дошка)
\item
  Написати клас Book (Книжка) та реалізувати програму пошуку книжки за
  авторами та назвою в каталозі (каталог -- масив книжок, що
  зберігається у файлі).
\end{enumerate}

\subsection{8.2. Конструктори та перевантаження
операторів}\label{ux43aux43eux43dux441ux442ux440ux443ux43aux442ux43eux440ux438-ux442ux430-ux43fux435ux440ux435ux432ux430ux43dux442ux430ux436ux435ux43dux43dux44f-ux43eux43fux435ux440ux430ux442ux43eux440ux456ux432}

\begin{enumerate}
\def\labelenumi{\arabic{enumi})}
\item
  Опишість клас Раціональне\_число як пару (чисельник, знаменник).
  Реалізуйте метод введення (з перевіркої коректості вводу), виведення
  та зведення дробу до незворотного вигляду. Також у класі перевантажте
  основні арифметичні оператори, оператори порівняння та інші оператори,
  що необхідні для роботи з раціональними числами.
\end{enumerate}

\begin{quote}
Використовуючи цей клас, розв'яжіть такі задачі:

а) знайдіть найбільше за модулем серед послідовності раціональних чисел

б) підрахуйте суму 20-ти членів ряду за формулою Грегорі
\end{quote}

\[\frac{\pi}{4} = 1 - \frac{1}{3} + \frac{1}{5} - \frac{1}{7} + \ldots\]

\begin{enumerate}
\def\labelenumi{\arabic{enumi})}
\item
  Опишіть класи Matrix3 та Vector3, що є відповідно матрицею розмірності
  3 на 3 та тривімірним вектором. Перевантажте математичні оператори для
  цих класів та спеціальні методи (множення матриці на вектор у тому
  числі). Оператор abs() перевантажте для матриці методом, що визначає
  її норму. Для матриці опишіть метод det(), що повертає визначник цієї
  матриці.
\item
  Описати клас Dynamic\_Array (Динамічний\_Масив), реалізувати методи
  створення та видалення масиву, читання та зміни елемента. Із
  використанням динамічних масивів розв'язати задачу: у двох масивах
  містяться коефіцієнти поліномів степеню m і n, відповідно. Отримати
  скалярний добуток цих поліномів.
\end{enumerate}

\protect\hypertarget{__DdeLink__7879_2123939799}{}{}4)Описати клас
Поліном та реалізувати методи: введення поліному, виведення поліному,
обчислення значення поліному у точці x, взяття похідної поліному, суми,
різниці та добутку поліномів. Використати цей клас для розв'язання
задачі: ввести 2 поліноми P1, P2 та рядок, який містить вираз, що
залежить від 2 поліномів. Наприклад,

P1 + P2*P1 -- P2

Обчислити поліном, який буде значенням цього виразу.

\emph{Вказівка}: поліном представити у вигляді масиву змінної довжини.

\begin{enumerate}
\def\labelenumi{\arabic{enumi}.}
\setcounter{enumi}{1}
\item
  \textbf{Статичні методи та класи. }
\end{enumerate}

5)На базі класу Точка напишіть програму, що дозволяє вводити
багатокутник з будь якої кількості вершин вводячи точки доки користувач
не відповість на запитання «Ввести точку?» - «Ні». Після цього виведіть
інформацію про кількість точок у багатокутнику та виведе його периметр.

\subsection{Наслідування}\label{ux43dux430ux441ux43bux456ux434ux443ux432ux430ux43dux43dux44f}

Для наступних задач будемо вважати, що клас Person описано таким чином:

\textbf{class} \textbf{Person\{} //Клас Особа

string name; //прізвище

unsigned byear\textbf{;//}рік народження

public:

\textbf{int} input()\textbf{\{} //ввести особу

\textbf{cin\textgreater{}\textgreater{}}name;

\textbf{cin\textgreater{}\textgreater{}byer;}

\textbf{\}}

\textbf{void} \textbf{print()\{ //}вивести особу

\textbf{cout\textless{}\textless{}}name\textless{}\textless{}'',''\textless{}\textless{}byear\textbf{\textless{}\textless{}endl;}

\}

\begin{enumerate}
\def\labelenumi{\arabic{enumi})}
\item
  Описати клас Знайомий на базі класу Person.
\end{enumerate}

У цьому класі повинно бути як мінімум одне додаткове поле «номер
телефону» а також методи введення та виведення інформації про знайомого.

Використати цей клас для побудови класу телефонного довідника (кількість
знайомих обмежена числом 100).

Передбачити дії: створення довідника, додавання запису про знайомого,
пошуку номера телефону за прізвищем та заміни номера телефону.

Телефонний довідник зберігає дані про знайомих у файлі.

\emph{\emph{Вказівка}}: телефонний довідник представити у вигляді класу
що зчитує дані з (текстового) файлу.

\begin{enumerate}
\def\labelenumi{\arabic{enumi})}
\item
  Описати клас Пасажир на базі класу Person. Клас містить дані про місце
  відправлення та місце слідування, а також місце пасажира. Створіть
  клас Каса, який дозволяє додавати та виводити інформацію про
  Пасижирів, містить методи пошуку по прізвищу, місцям відправлення,
  прибуття та місцю. Також серед заданого масиву місць у потягу знайдіть
  місце яке не зайняте (у випадку якщо таких місць декілька -- виведіть
  найменше за значенням, якщо їх немає відповідне повідомлення).
\end{enumerate}

\emph{\emph{Вказівка}}: інформацію про пасажирів представити у вигляді
бінарного файлу.

\begin{enumerate}
\def\labelenumi{\arabic{enumi})}
\item
  Описати клас Студент на базі класу Person.
\end{enumerate}

У класі Студент повинна бути інформація про оцінки отримані ним протягом
сесії (за 5-ти бальною та 100 бальною шкалами).

Скласти програму для обчислення нарахованої студентам стипендії в
залежності від результатів сесії:

\begin{itemize}
\item
  За старим підходом нарахування стипендії (середній бал за всі іспити
  має бути не меншим ніж 4 за 5-ти бальною шкалою).
\item
  З новим підходом нарахування стипендії (стипендію отримують 40\% від
  загального числа студентів, які є найкращими по рейтингу)
\end{itemize}

\emph{\emph{Вказівка}}: інформацію про студентів представити у вигляді
масиву. Дані зчитувати з клавіатури.

\begin{enumerate}
\def\labelenumi{\arabic{enumi})}
\item
  На базі класу \textbf{Точка} (на площині) створіть клас Точка3Д (точка
  в просторі). Реалізуйте методи введення, виведення. Аналогічно на базі
  Відрізка2Д реалізуйте клас Відрізок3Д. Методи
  введення\textbackslash{}виведення, визначення довжини відрізка та
  визначення чи перетинаються 2 відрізка.
\item
  Реалізувати клас СЛОВО, який має члени типу Рядок: ПРИСТАВКА,
  ПРИСТАВКА2, КОРІНЬ, СУФІКС, ЗАКІНЧЕННЯ (клас повинен мати геттери та
  сеттери).
\end{enumerate}

Створіть наслідники цього класу: ГЛАГОЛ, ІМЕННИК, ПРИКМЕТНИК.

Реалізуйте для них методи: Род, Число, Лице, Відмінок -- які будуть
відповідним чино змінювати (якщо це можливо) дане слово.

Створіть декілька слів, що є екземплярами ГЛАГОЛу, ІМЕННИКу, ПРИКМЕТНИКу
та виконайте відповідні методи для них щоб можна було побачити
результат.

\protect\hypertarget{_Hlk54461599}{}{}

\textbf{Лекція 9.} Перевантаження методів. Перевантаження бінарних та
унарних операторів.

Стандартний клас рядок. Конструктори та методи класу рядок. Приклади
використання рядків.

Наслідування. Типи наслідування в Сі++.

Наслідування та абстрактні класи. Віртуальні методи. Множинне
наслідування та проблеми з ним пов'язані. Віртуальне наслідування.

\textbf{Питання.}

\begin{enumerate}
\def\labelenumi{\arabic{enumi})}
\item
  \protect\hypertarget{_Hlk57988736}{}{}Що таке перевантаження методів?
  Чому воно зручно в мовах зі строгою типізацією?
\end{enumerate}

\begin{enumerate}
\def\labelenumi{\arabic{enumi})}
\item
  Чим перевантаження операторів відрізняється від перевантаження інших
  методів?
\item
  Які оператори не можна перевантажувати? Коли перевантаження операторів
  може бути набезпечним?
\item
  Чому при перевантаженні операторів вводу-виводу нам потрібно ключове
  слово friend?
\item
  В файлі string.hpp приведений код, що реалізує інтерфейс класу рядок
  Сі++. Скільки конструкторів в цьому коді? Скільки копіконструкторів?
  Скільки та які оператори є перевантаженими?
\item
  Як видалити підрядок, використовуючи методи класу String?
\item
  Які типи наслідування є на Сі++ та яка між ними різниця?
\item
  Поясніть на прикладі, що таке раннє та пізнє зв'язування
\item
  Що таке чисто віртуальний клас та чисто віртуальний метод? Коли вони
  потрібні?
\item
  Як реалізувати множинне наслідування на Сі++?
\item
  Що робити та які шляхи правильного множинного наслідування якщо й
  класи батьки й клас-син мають метод з однаковою назвою? Що зміниться,
  якщо це не метод, а перевантажений оператор?
\end{enumerate}

\textbf{Вправи:}

\begin{enumerate}
\def\labelenumi{\arabic{enumi})}
\item
  В класі Раціональній дріб з попередньої лекції напишіть методи
  введення, виведення (cin\textgreater{}\textgreater{},
  cout\textless{}\textless{}) та оператори віднімання, ділення як
  перевантажені оператори. Тобто з типом Раціональній дріб можна тепер
  працювати як зі стандартним типом. Чому краще перевантажити два
  оператори віднімання?
\item
  Напишіть функцію часткового спліттінгу рядку. Тобто функція, що
  приймає рядок та повертає перше слово з рядку (роздільник -- задається
  як аргумент функції)
\end{enumerate}

\begin{enumerate}
\def\labelenumi{\arabic{enumi})}
\item
  Напишіть функцію, що приймає рядок та повертає масив (як
  аргумент-змінний) всі дійсні числа, що містяться в рядку (роздільник
  -- задається як аргумент функції)
\item
  Створіть клас Людина (члени: ПІБ, стать, вік) та його наслідники
  Студент (додано: курс, група, ВУЗ), Викладач (додано: ВУЗ, посада,
  з.п.). Методи введення, виведення, конструктори для різної кількості
  вхідних даних.
\end{enumerate}

Створіть клас Аспірант, що є наслідником і студента і викладача.
Коректно визначте член ВУЗ для нього.

Наслідування та віртуальні методи

\begin{enumerate}
\def\labelenumi{\arabic{enumi})}
\item
  Реалізувати наступні класи:
\end{enumerate}

Описати клас \textbf{Прямокутник}. Сторони прямокутника паралельні осям
координат. Для прямокутника задані лівий верхній кут та довжини сторін.
Описати методи отримання довжини кожної з сторін, площі прямокутника,
периметру, метод знаходження перетину двох прямокутників. Методи
переміщення прямокутника. Скласти програму створення заданої кількості
прямокутників та знаходження їх спільного перетину.

Описати клас \textbf{Трикутник}. Основа трикутника паралельна осі
\emph{x} координат. Для трикутника задані лівий нижній кут (координати)
та довжини сторін. Описати методи отримання довжини кожної зі сторін.
Описати методи отримання довжини кожної з сторін, площі прямокутника,
периметру, метод знаходження перетину двох прямокутників. Методи
переміщення прямокутника. Скласти програму створення заданої кількості
прямокутників та знаходження їх спільного перетину.

Описати клас \textbf{Трикутник}. Основа трикутника паралельна осі
\emph{x} координат. Для трикутника задані лівий нижній кут (координати)
та довжини сторін. Описати методи отримання довжини кожної зі сторін.
Описати методи отримання довжини кожної з сторін, площі, периметру,
метод знаходження перетину двох трикутників. Методи переміщення. Скласти
програму створення заданої кількості трикутників та знаходження їх
спільного перетину.

Описати клас \textbf{Еліпс}. Для нього є заданими фокуси та радіуси.
Описати методи отримання геометричних характеристик. Описати методи
отримання довжини радіусів, площі, периметру, метод знаходження площі
перетину двох еліпсів. Методи переміщення та повороту. Скласти програму
створення заданої кількості еліпсів та знаходження їх спільного
перетину.

Створити клас Фігура, який є базою.

Опишіть класи для таких геометричних фігур та реалізуйте зазначені
методи:

\begin{enumerate}
\def\labelenumi{\alph{enumi})}
\item
  Клас Трапеція. У цьому класі реалізуйте операції знаходження периметра
  і площі;
\item
  Клас Паралелограм. У цьому класі реалізуйте операції знаходження
  периметра і площі.
\item
  Клас Круг. Реалізуйте методи відшукання площі круга, довжини кола,
  цього круга.
\item
  Клас Піраміда. Реалізуйте методи пошуку площі бічної поверхні і
  об'єму;
\item
  Клас П'ятикутник, що містить масив вершин. Реалізуйте метод перевірки
  чи є цей п'ятикутник опуклим.
\item
  Клас Багатокутник. Реалізуйте метод перевірки чи є цей багатокутник
  опуклим.
\end{enumerate}

Дано список фігур вищенаведених класів. Серед фігур, що належать до
перших трьох класів знайдіть фігуру, що має найбільшу площу та периметр
(довжину кола). Також знайдіть всі опуклі багатокутники

6) Опишіть класи

\begin{quote}
1. \textbf{Гість}, що містить всю необхідну інформацію про жильця
деякого готелю: ім'я, період проживання тощо.

2. \textbf{Кімната}, що містить інформацію про кімнату готелю у тому
числі вартість проживання за добу.

3. \textbf{Готель}, що містить список кімнат цього готелю, інформацію
про те ким і коли вони зайняті, а також методи на кшталт тощо.
\end{quote}

Використовуючи вищенаведені класи розв'язати задачі:

\begin{quote}
а) Вивести відомість про кількість вільних кімнат у готелі;

б) Пошуку вільної кімнати у зазначений період;

в) Поселити жильця на вказаний термін;

г) Вартості проживання жильця у зазначений період;

д) Прибутку, який отримає готель за вказаний період;

е) Пошуку гостя у готелі (у заданий період);
\end{quote}

7) Опишіть клас Фігура, що інкапсулює основні геометричні характеристики
та методи. Для фігури визначено методи:

\begin{enumerate}
\def\labelenumi{\arabic{enumi}.}
\item
  calculateVolume() -- віртуальний метод, що обчислює міру фігури (для
  плоскої фігури -- площу, для об'ємної -- відповідно об'єм).
\item
  getVolume() -- що повертає міру фігури.
\end{enumerate}

Від класу Фігура наслідуються такі класи

\begin{itemize}
\item
  Трикутник
\item
  Прямокутник
\item
  Трапеція
\item
  Паралелограм
\item
  Круг
\item
  Куля
\item
  Трикутна Піраміда (який успадковується від класу Трикутник)
\item
  Чотирикутна піраміда (який успадковується від класу Прямокутник)
\item
  Паралелепіпед (який успадковується від класу Прямокутник)
\end{itemize}

\begin{quote}
Нехай дано список фігур. Серед заданих фігур, знайдіть фігуру, що має
найбільшу міра якої є найбільшою
\end{quote}

8)Опишіть клас Pet -- домашня тварина, що має метод to\_feed(feed,
count) -- годувати (feed -- тип корму, count -- кількість).

Клас Pet має віртуальні методи

\begin{enumerate}
\def\labelenumi{\arabic{enumi}.}
\item
  to\_sniff () («нюхати» -- визначає, чи може їсти тварина заданий тип
  корму),
\item
  to\_ask() («просити» -- метод повертає True, якщо тип корму не
  підходить або тварина ще хоче їсти і виводить на екран прохання
  «тваринною мовою», наприклад, «Мяв\ldots{}» для кота),
\item
  to\_eat() (їсти, якщо тип корму підходить).
\end{enumerate}

Клас Pet має нащадки -- Cat, Dog, Parrot (папуга), у яких перевизначено
вищезгадані віртуальні методи.

Задано список тварин та список кормів (тип та загальна вага). Пропонуючи
по черзі кожній тварині порцію їжі, потрібно нагодувати всіх тварин.
Якщо корму не вистачить -- вивести відповідне повідомлення.

9) Опишіть клас Car, що має метод go(distance), який змінює пройдений
кілометраж автомобілем та залишок пального. Метод go(\ldots{}) залежить
від віртуального методу fuelPerKm(), який визначає скільки потрібно
пального автомобілю для проїзду одного кілометру. Нехай Personal
(легковий автомобіль) і Truck (вантажівка) -- класи, що наслідують клас
Car і перевизначають метод fuelPerKm(). При цьому потрібно врахувати, що
цей метод залежить від кількості пасажирів (+10\% на кожного пасажира)
для авто класу Personal або ваги вантажу для Truck (+25\% на кожну тонну
вантажу). Визначити чи зможе задане авто проїхати задану відстань.

10) Задано клас Flower, що має нащадками конкретні класи квітів (напр.,
тюльпан, троянд, тощо). Ви зайшли у квітковий магазин у якому продаються
різні типи квітів. Необхідно зібрати букет з квітів (букет може містити
квітки одного класу) та визначити:

\begin{enumerate}
\def\labelenumi{\arabic{enumi}.}
\item
  Його вартість.
\item
  Скільки часу зможе тішити букет очі (до моменту поки не зів'яне перша
  квітка).
\item
  Колір, що домінує у цьому букеті.
\item
  Чи припустимий цей букет за інтенсивністю запаху.
\end{enumerate}

\protect\hypertarget{_Hlk54461890}{}{}\textbf{Лекція 10. Перетворення
типів Сі++. Виключення Сі++.}

\protect\hypertarget{_Hlk57988995}{}{}\textbf{Питання.}

\begin{enumerate}
\def\labelenumi{\arabic{enumi})}
\item
  Які варіанти перетворень стандартних типів один між іншим можливі в
  Сі++?
\item
  Яким перетворенням краще скористатись для перетворень між цілими
  типами? Яким при перетворення цілих до дійсного та навпаки?
\item
  Чим відрізняються перетворення вгору та вниз? Яке перетворення типу
  краще для перетворення вгору, а яке вниз?
\item
  Чому не можна відловити виключення при діленні на нуль в Сі++ зі
  стандартними типами?
\item
  Як створити власне виключення в Сі++? Як його коректно обробити?
\item
  Яке виключення дозволяє коректно обробити static\_cast?
\item
  Як складнощі виникають якщо виключення виникає в деструкторі класу?
\item
  Як коректно працювати з виключенням, що виникає в конструкторі класу?
\end{enumerate}

\textbf{Вправи:}

\begin{enumerate}
\def\labelenumi{\arabic{enumi})}
\item
  В класі Раціональній дріб з попередньої лекції перепишіть методи
  введення (cin\textgreater{}\textgreater{}) та конструктор і сеттери,
  щоб вони кидали виключення при ініціалізації знаменнику нулем.
  Коректно обробить в коді це виключення.
\item
  Напишіть дружню функцію запису Раціонального дробу в файл, яка буде
  викидати виключення при некоректному відкритті файлу та обробить його
  в тілі програми.
\item
  Ви вже створили клас Людина (члени: ПІБ, стать, вік) та його
  наслідники Студент (додано: курс, група, ВУЗ), Викладач (додано: ВУЗ,
  посада, з.п.). Методи введення, виведення, конструктори для різної
  кількості вхідних даних.
\end{enumerate}

Створіть клас Аспірант, що є наслідником і студента і викладача.
Коректно визначте член ВУЗ для нього.

Створить програму що буде вводити масив Людей, серед яких є Студенти,
Викладачі, Аспіранти. Без створення нових членів класу виведіть коректно
ВУЗ для кожного екземпляру масиву.

\subsection{Виключення}\label{ux432ux438ux43aux43bux44eux447ux435ux43dux43dux44f}

1)Скласти підпрограму та програму для обчислення значення натурального
числа за заданим рядком символів, який є записом цього числа у системі
числення за основою b (\(2 \leq b \leq 16\)). Використати функцію, яка
за заданим символом повертає відповідну цифру у системі числення за
основою b. Використати у цій функції твердження про стан програми assert
для перевірки того, що відповідний символ є цифрою у системі числення за
основою b. Обробити у підпрограмі помилку неправильного символу рядка та
показати змістовне повідомлення про помилку.

2)Скласти функцію та програму для обчислення суми всіх доданків, модуль
яких не менше ε \textgreater{} 0, у комплексній точці \emph{z}

\(\text{arctg}\left( z \right) = z - \frac{z^{3}}{3} + \frac{z^{5}}{5} - \cdots + {( - 1)}^{n}\frac{z^{2n + 1}}{2n + 1} + \cdots,\ \ \ \ (\left| z \right| < 1)\).

Використати у цій функції твердження про стан програми для перевірки
того, що параметр \emph{z} відповідає заданій умові та зробить обробку
всіх можливих виключень -- включаючи некоректне введення та виділення
пам'яті під масиви. Обробити у програмі помилку неправильного значення
\emph{z} та показати змістовне повідомлення про помилку.

3)Задані натуральне число \emph{і} файл \emph{f}, компоненти якого є
цілими числами. Побудувати файл \emph{g}, записавши в нього найбільше
значення перших \emph{n} компонент файлу \emph{f}, потім-наступних
\emph{n} компонент і т.д. Розглянути два випадки:

а) число компонент файлу ділиться на \emph{n};

б) число компонент файлу не ділиться на \emph{n}.

В цьому випадку остання компонента файлу \emph{g} повинна дорівнювати
найбільшій із компонент файлу \emph{f}, які утворюють останню (неповну)
групу.

Забезпечити обробку помилок при роботі з файлами.

4)У текстовому файлі записана непорожня послідовність дійсних чисел, які
розділяються пропусками в одному рядку та можуть бути розташовані у
різних рядках. Визначити функцію обчислення найбільшого з цих чисел.

Забезпечити обробку помилок, якщо у файлі зустрічаються не дійсні числа.

5)Описати клас Трьохбайтне ціле число для роботи з цілими числами,
представленими трьома байтами. Інтервал представлення при цьому -- від
-2\textsuperscript{23} до 2\textsuperscript{23}-1. Операції не можуть
вивести за межі інтервалу представлення. Наприклад,
2\textsuperscript{23}-1 + 1 == -2\textsuperscript{23} й т.д. Якщо
результат операції виводить за межі інтервалу представлення, повинна
ініціюватися помилка переповнення.

Перевизначити у цьому класі операції +, -, *, //, \%.

Описати також 3 класи обробки помилок для трьохбайтних цілих чисел:
загальний клас обробки помилок та два його підкласи для обробки помилки
переповнення та помилки ділення на 0.

Використати цей клас для розв'язання задач:

а) обчислення \emph{n}!

б) обчислення \emph{x\textsuperscript{n}}, де \emph{x} -- ціле, \emph{n}
-- невід'ємне ціле.

Забезпечити обробку помилок при виконанні обчислень.

6)Описати клас Поліном та реалізувати методи: введення поліному,
виведення поліному, обчислення значення поліному у точці x, взяття
похідної поліному, суми, різниці та добутку поліномів.

Описати також клас обробки помилок при неправильному введенні поліному
(степінь -- не невід'ємне ціле число, коефіцієнт -- не дійсне число) та
забезпечити ініціювання помилки при неправильному введенні.

Використати цей клас для розв'язання задачі: ввести 2 поліноми P1, P2 та
рядок, який містить вираз, що залежить від 2 поліномів. Наприклад,

P1 + P2*P1 -- P2

Обчислити поліном, який буде значенням цього виразу.

Забезпечити обробку помилок неправильного введення поліному.

\emph{\emph{Вказівка:}} поліном представити у вигляді словника.

7)Описати клас для реалізації мультимножини на базі масиву чисел розміру
N=100. Мультимножина - це множина в якій для кожного елемента
запам'ятовується не лише його входження, але й кількість входжень.

Кількість входжень елемента \emph{k} (\(0 \leq k \leq n\)) у
мультимножину - це значення елемента словника з ключем \emph{k}.

Реалізувати дії над мультимножинами:

1) зробити мультимножину порожньою;

2) чи є мультимножина порожньою;

3) додати елемент до мультимножини;

4) забрати елемент з мультимножини (кількість входжень елемента
зменшується на 1, якщо елемент не входить - відмова);

5) кількість входжень елемента у мультимножину;

6) об'єднання двох мультимножин (в результаті об'єднання кількість
входжень елемента визначається як максимальна з двох мультимножин);

7) перетин двох мультимножин (в результаті кількість входжень елемента
визначається як мінімальна з двох мультимножин);

Описати клас обробки помилки взяття елементу, який не входить до
мультимножини.

З використанням класу розв'язати задачі:

а) перевірити, чи складаються рядки \emph{S1}, \emph{S2} з одних і тих
же символів, які входять у ці рядки однакову кількість разів;

б) перевірити, чи вірно, що всі символи рядка \emph{S1}, входять також у
рядок \emph{S2}, причому не меншу кількість разів, ніж у \emph{S1}.

Забезпечити обробку помилок.

\textbf{Лекція 11-12. Шаблони. Стандартна бібліотека шаблонів STL}

\protect\hypertarget{_Hlk57989145}{}{}\textbf{Питання.}

\begin{enumerate}
\def\labelenumi{\arabic{enumi})}
\item
  Як створити функцію-шаблон? В яких ситуаціях вона корисна?
\item
  Як створити клас-шаблон? Що потрібно зробити якщо шаблоном є лише
  єдиний метод класу?
\item
  З яких частин складається бібліотека шаблонів Сі++?
\item
  Для чого потрібні контейнери-адаптори? Які контейнери-адаптори
  визначені в Сі++?
\item
  Які контейнери прямого доступу визначені в Сі++?
\item
  Яка різниця між контейнерами list, forward\_list, vector, array?
\item
  Які асоціативні контейнери існують в Сі++? Що додає приставка multi до
  назви контейнера?
\item
  Які переваги array або vector перед стандартним масивом чи
  вказівником?
\item
  Які коректні шляхи ініціалізації заданими числами вектору? Стеку?
  Відображення?
\item
  Для яких стандартних класів-шаблонів не визначений метод push\_back()?
  Чому? Як в ці класи додаються елементи?
\item
  Як визначити кількість елементів будь-якого контейнеру?
\item
  Які коректні шляхи ітерації по вектору? Мультивідображенню? Будь-якому
  контейнеру?
\item
  Які типи ітераторів існують?
\item
  Що таке придикат та функтор? Як їми скористатись?
\item
  Як скористатись алгоритмами сортування? Акумульованої суми? Бінарного
  пошуку?
\end{enumerate}

\textbf{Вправи:}

\begin{enumerate}
\def\labelenumi{\arabic{enumi})}
\item
  Перепишіть функцію шаблон для пошуку максимуму, так щоб вона працювала
  для всіх стандартних числових типів. Що потрібно зробити, щоб вона
  запрацювала і для типу Раціонального дробу з попередніх лекцій?
  (Вказівка: щось потрібно визначити для класу Раціональний дріб)
\item
  Створіть власну реалізацію класу шаблону Стек. Перевірте її роботу за
  допомогою стандартного класу Стек з STL.
\item
  В текстовому файлі міститься текст (слова відокремлені лише одним
  пробілом). За допомогою відображення виведіть частотну характеристику
  слів та літер у тексті.
\item
  Створіть клас-шаблон Поліном, який приймає вектор чисел (будь-якого
  типу) --- вектор (на базі стандартного класу Вектор) коефіцієнтів
  поліному. Методи: введення-виведення, додавання, множення та
  обчислення значення. Перевірте, що клас працює коректно для дійсних,
  цілих чисел та для типу Раціональний дріб з попередніх завдань.
\end{enumerate}

\subsection{10.0
Класи-шаблони}\label{ux43aux43bux430ux441ux438-ux448ux430ux431ux43bux43eux43dux438}

\begin{enumerate}
\def\labelenumi{\arabic{enumi})}
\item
  Створити клас-шаблон BlackBox БлекБокс, який містить конструктор
  (порожній та від масиву (вказівника) будь-якого типу), метод push(),
  що дозволяє додати елемент певного типу, та метод pop(), що видає та
  видаляє випадковий елемент, що вже міститься в класі та виключення,
  якщо БлекБокс порожній, метод xpop(), що просто повертає випадковий
  елемент цього класу.
\item
  Створити клас-шаблон Mediana, який містить конструктор (порожній та
  від масиву (вказівника) будь-якого типу, що містить операції
  порівняння), метод push(), що дозволяє додати елемент будь-якого типу,
  що містить операції порівняння, та метод pop(int n), що видає та
  видаляє елемент, з номером n по порядку, або виключення, якщо n більше
  розміру всіх елементів, метод mediana(), що повертає медіану елементів
  цього класу.
\end{enumerate}

\subsection{11.0 Стандартна бібліотека
}\label{ux441ux442ux430ux43dux434ux430ux440ux442ux43dux430-ux431ux456ux431ux43bux456ux43eux442ux435ux43aux430}

\begin{enumerate}
\def\labelenumi{\arabic{enumi})}
\item
  Введіть відображення в якому ключ --- це слово, а значення декілька
  слів, які визначають це слово. При цьому порочного кола немає.
  Пронумеруйте слова таким чином, щоб слова з більшим номером
  визначались лише словами з меншими номерами.
\item
  Біля прилавка в магазині вишикувалася черга з п покупців. Час
  обслуговування продавцем i-го покупця число
  \(t_{i},\ i = 1,\cdots,n\). Нехай дано натуральне n і дійсні числа
  \(t_{1},t_{2},\cdots,t_{n}\). Отримати \(c_{1},c_{2},\cdots,c_{n}\) де
  з \(c_{i}\ \)-- час перебування i-го покупця в черзі
  \(i = 1,\cdots,n\). Вказати номер покупця, для обслуговування якого
  продавцеві потрібно найменше часу.
\item
  Створити структуру або клас Пасажир, який містить ім'я пасажиру та як
  мінімум два додаткових поля: «місто відправлення» та «місто прибуття»,
  а також методи введення та виведення інформації про пасажира та
  розрахунку плати за білет. Використати цей клас для розрахунку плати
  за білети усіх пасажирів. Вважати що маршрути зберігаються у масиві
  структур (місто1, місто2, відстань), а також те, що плата за білет
  пропорційна відстані та відома плата за 1 км відстані.
\item
  В деяких видах спортивних змагань виступ кожного спортсмена незалежно
  оцінюється деякими суддями, потім з усієї сукупності оцінок
  видаляються найбільш висока і найнижча, а для решти оцінок
  обчислюється середнє арифметичне, яке і йде в залік спортсмену. Якщо
  найбільш високу оцінку виставило кілька суддів, то з сукупності оцінок
  видаляється лише одна така оцінка; аналогічно надходять з найбільш
  низькими оцінками. Дано натуральне число n, дійсні числа
  \(a_{1},a_{2},\cdots,a_{n}\) (\(n \geq 3\))(масив реалізується як
  вектор). Вважаючи, що числа \(a_{1},a_{2},\cdots,a_{n}\) - це оцінки,
  виставлені суддями одному з учасників змагань, визначити оцінку, яка
  піде в залік цього спортсмену. Нехай в нас є декілька спортсменів
  (вектор векторів) з оцінками по заданому правилу (кількість суддів в
  кожного спортсмена може бути різна). Знайдіть переможця.
\item
  Ввести n d-вимірних векторів x (n, d вводяться з клавіатури) та
  обчислити значення функції f(x) (реалізувати її) для кожного з цих x.
\end{enumerate}

\includegraphics[width=3.23889in,height=0.66667in]{media/image9.png}

\begin{enumerate}
\def\labelenumi{\arabic{enumi})}
\item
  Створить список цілих чисел List і число X. Не використовуючи
  допоміжних об'єктів і не змінюючи розміру списку, переставити елементи
  списку так, щоб спочатку йшли числа, що не перевищують X, а потім
  числа що є більшими за X.
\item
  Заданий файл з текстом англійською мовою. Виділити все різні слова.
  Для кожного слова підрахувати частоту його входження. Слова, що
  відрізняються регістром літер, вважати різними. Використовувати Map.
\item
  З використанням Set виконати попарне підсумовування довільного
  кінцевого ряду чисел за такими правилами: на першому етапі
  підсумовуються попарно сусідні числа, на другому етапі підсумовуються
  результати першого етапу і т. д. до тих пір, поки не залишиться одне
  число.
\item
  На базі шаблону List реалізувати структуру зберігання чисел з
  підтримкою наступних операцій:

  \begin{itemize}
  \item
    додавання / видалення числа;
  \item
    пошук числа, найбільш близького до заданого (тобто модуль різниці
    мінімальний).
  \end{itemize}
\item
  У вхідному файлі розташовані два набору позитивних чисел; між наборами
  -- від'ємне число. Побудувати два списки C1 і С2, елементи яких
  містять відповідно числа 1-го і 2-го набору таким чином, щоб усередині
  одного списку числа були впорядковані по зростанню. Потім об'єднати
  списки C1 і С2 в один відсортований список.
\item
  На площині задано N точок. Вивести в файл описи всіх прямих, які
  проходять більш ніж через одну точку із заданих. Для кожної прямий
  вказати, через скільки точок вона проходить. Використовувати клас
  MultiMap.
\item
  На клітковому аркуші намальований круг. Вивести в файл опису всіх
  клітин, цілком лежать всередині кола в порядку зростання відстані від
  клітини до центру кола. Використовувати клас PriorityQueue.
\item
  На площині задано N відрізків. Знайти точку перетину двох відрізків,
  що має мінімальну абсцису. Використовувати клас Map.
\item
  На клітковому аркуші паперу зафарбована частина клітин. Виділити все
  різні фігури, які утворилися при цьому. Фігурою вважається набір
  зафарбованих клітин, які сусідні один з одного при руху в чотирьох
  напрямах. Дві фігури є різними, якщо їх не можна сумістити поворотом
  на кут, кратний 90 градусам, і паралельним переносом. Використовуйте
  клас MultiSet.
\item
  Дана матриця з цілих чисел. Знайти в ній прямокутну підматрицю, що
  складається з максимальної кількості однакових елементів.
  Використовувати клас Stack.
\item
  Реалізувати структуру «чорний ящик» на базі Queue, що зберігає множину
  чисел і має внутрішній лічильник K, спочатку рівний нулю. структура
  повинна підтримувати операції додавання числа в множину і повернення
  K-го по мінімальності числа з множини.
\item
  У файлі записані координати точок на площині задані парою цілих чисел.
  Точки записуються в форматі : ( х1 , х2 ) (х1 , х2) , \ldots{} - саме
  так через коми та дужки. Створити файл, в якому будуть записані
  координати всіх відрізків з точок цього файлу, при цьому ці відрізки
  відсортовані за зростанням довжини.
\item
  У файлі записані координати Точок в просторі задані трійкою цілих
  чисел. Точки записуються в форматі : х1 , х2 , х3 ; х1 , х2, х3 ;
  \ldots{}
\end{enumerate}

Створити файл, в якому будуть записані відрізки з точок цього файлу, при
цьому ці відрізки відсортовані за зростанням довжини.

\begin{enumerate}
\def\labelenumi{\arabic{enumi})}
\item
  У файлі записані координати Точок на площині задані парою цілих чисел
  та масою(дійсне число). Точки записуються в форматі : (х1 , х2): m1 ,
  (х1 , х2): m2 , \ldots{} Створити файл, в якому будуть записані
  відрізки з точок цього файлу, при цьому ці точки відсортовані за
  важилем сили (m1*(х1 +х2)).
\item
  У файлі записані дати , що трійкою цілих чисел у форматі: чч/мм/рр,
  \ldots{} Створити файл, в якому будуть записані дати з цього файлу без
  повторень, при цьому ці дати відсортовані за спадання дати (врахуйте,
  що роки дат з 1951 по 2049).
\item
  У файлі записані дати , що двома цілими числами та рядком (англійські
  або числові назви місяця) у форматі: чч1 місяць1 рік1, чч2 місяць2
  рік2\ldots{}Вивести дати без повторень з цього файлу у форматі:
  рік1/місяць1/число1, рік2/місяць1/число2,... (місяць заданий назвою)
  при цьому ці дати відсортовані за зростанням дати
\item
  Нехай значення функції f(n)- кількість літер у письмовому
  представленні числа n (f(1)=4 („один``), f(3)=3(«три»), f(42)=8
  («сорок два», а(2001) =13 («дві тисячи один»))). Знайдіть всі числа до
  10000, для яких f(n) = n. (Вказівка: Використовуйте
  відображення(словник) для зберігання кількості літер у представленні
  цифри)
\item
  Дана послідовність (вектор) з n чисел. Знайдіть кількість інверсій в
  цій послідовності, тобто таких пар чисел в яких більше число
  знаходиться лівіше за менше число (використайте тут стандартні
  алгоритми STL).
\item
  \begin{quote}
  Напишіть функцію, як повертає а) суму найбільших k чисел даного
  вектору, б) масив з k найменших чисел даного вектору, якщо k не
  перевищує розмір масиву та а) нуль б) порожній масив в протилежному
  випадку.
  \end{quote}
\end{enumerate}

\begin{quote}
1. Створіть генератор, який повертає поточне значення clock () (у
\textless{}ctime\textgreater{}). Створіть список
\textless{}clock\_t\textgreater{} і заповніть його своїм генератором за
допомогою create\_n (). Видаліть усі дублікати зі списку та роздрукуйте
його на cout за допомогою copy ().

2. За допомогою transform () і toupper () (у
\textless{}cctype\textgreater{}) напишіть один виклик функції, який
перетворить рядок на всі великі літери.

3. Створіть шаблон об'єкта функції Sum, який буде накопичувати всі
значення в діапазоні при використанні з for\_each ().

4. Напишіть генератор анаграм, який приймає слово як аргумент командного
рядка і створює всі можливі перестановки літер.

5. Напишіть генератор анаграм речень, який приймає речення як аргумент
командного рядка і створює всі можливі перестановки слів у реченні. (Це
залишає слова в спокої і просто рухає їх навколо).

6. Створіть ієрархію класів з базовим класом B та похідним класом D.
Помістіть функцію віртуального члена void f () у B таким чином, щоб вона
надрукувала повідомлення, що вказує, що було викликано B sf (), і
перевизначити цю функцію для D щоб надрукувати інше повідомлення.
Створіть вектор \textless{}B *\textgreater{} і заповніть його об'єктами
B і D. Використовуйте for\_each () для виклику f () для кожного з
об'єктів у вашому векторі.

7. Напишіть програму, яка знаходить усі спільні слова між двома вхідними
файлами, використовуючи set\_intersection (). Змініть його, щоб показати
слова, які не є спільними, за допомогою set\_symmetric\_difference ().

12. Створіть програму, яка, отримуючи ціле число в командному рядку,
створює таблицю факторіалів з усіх факторіалів, включаючи число в
командному рядку. Для цього напишіть генератор для заповнення вектора
\textless{}int\textgreater{}, а потім використовуйте парциальну\_суму ()
зі стандартним об'єктом функції.
\end{quote}

Створіть шаблон класу Matrix, який створений з вектору \textless{}vector
\textless{}T\textgreater{}\textgreater{}. Надайте його дружньому методу
ostream \& operator \textless{}\textless{} (ostream \&, const Matrix \&)
для відображення матриці. Створіть наступні двійкові операції,
використовуючи об'єкти функції STL, де це можливо: оператор + (const
Matrix \&, const Matrix \&) для додавання матриці, оператор * (const
Matrix \&, const vector \textless{}int\textgreater{} \&) для множення
матриці на вектор та оператор * ( const Matrix \&, const Matrix \&) для
множення матриць. Перевірте шаблон класу Matrix, використовуючи int і
float.

Використовуючи символи "\textasciitilde{}`! @ \# \$\% \^{} \& * () \_- +
=\} \{{[}{]} \textbar{} \textbackslash{} :;
"'\textless{}.\textgreater{},? /", згенеруйте кодову книгу,
використовуючи вхідний файл, вказаний у командному рядку як словник
слів. Не турбуйтеся про вилучення не алфавітних символів і не турбуйтеся
про регістр слів у файлі словника. Співставте кожну перестановку рядка
символів із таким словом, наприклад:

"= ') /\% {[}\}{]} \textbar{} \{* @ ?!" `,;\textgreater{} \& \^{} -
\textasciitilde{} \_: \$ +. \# (\textless{}\textbackslash{}" apple ",

\textbar{}{]} \textbackslash{} \textasciitilde{}\textgreater{} \#. +\%
(/ -\_ {[}` ':; =\} \{* "\$ \^{}! \&?), @ \textless{}"carrot ",

@ = \textasciitilde{} {[}'{]}. \textbackslash{} /
\textless{}-`\textgreater{} \# *) \^{}\% +, "; \&?! \_ \{: \textbar{}
\$\} " Carrot'' тощо .

Переконайтеся, що у вашій книзі кодів немає повторюваних кодів або слів.
Використовуйте lexicographic\_compare (), щоб виконати сортування кодів.
Використовуйте книгу кодів для кодування файлу словника. Розшифруйте
своє кодування файлу словника та переконайтеся, що ви повернули той
самий вміст.

Створіть алгоритм стилю STL transform\_if (), слідуючи першій формі
transform (), яка виконує перетворення лише на об'єктах, які
задовольняють одинарний предикат. Об'єкти, які не задовольняють
предикату, опускаються з результату. Потрібно повернути новий кінцевий
ітератор.

Створіть алгоритм стилю STL, який є перевантаженою версією for\_each (),
яка слідує за другою формою перетворення () і займає два діапазони
введення, щоб він міг передавати об'єкти другого діапазону введення a
двійковій функції, яку він застосовує до кожного об'єкта першого
діапазону.

Армія хоче набрати людей зі свого виборчого списку служб. Вони вирішили
набрати тих, хто записався на службу в 1997 році, починаючи від
найстаршого і закінчуючи молодшим. Згенеруйте довільну кількість людей
(надайте їм такі дані, як вік та рік, зареєстровані) у вектор. Розділіть
вектор так, щоб ті, хто вступив у 1997 році, були упорядковані на
початку списку, починаючи від наймолодшого до найстаршого, а решту
частину списку залишали сортувати за віком.

Створіть клас «Місто» з даними про населення, висоту та погоду. Зробіть
погоду переліченою за допомогою \{ДОЩИТЬ, СНІЖНО, ХМАРНО, ЯСНО\}.
Створіть клас, який генерує об'єкти Town. Створіть назви міст (незалежно
від того, мають вони сенс чи ні, це не має значення) або витягніть їх з
Інтернету. Переконайтеся, що назва всього міста має маленькі регістри, а
дублікатів назв немає. Для простоти радимо зводити назви міст одним
словом. Для населення, висот та погодних полів створіть генератор, який
випадковим чином генеруватиме погодні умови, популяції в межах
{[}100-1000000) та висоти між {[}0, 8000) футами. Заповніть вектор
об'єктами міста. Перепишіть вектор у новий файл під назвою Towns.txt.

Відбувся бебі-бум, що призвело до збільшення населення на 10\% у кожному
місті. Оновіть дані про місто за допомогою transform (), перепишіть дані
назад у файл.

28. Знайдіть міста з найбільшим і найменшим населенням. Для цієї вправи
застосуйте оператор \textless{}для вашого класу Town. Також спробуйте
реалізувати функцію, яка повертає true, якщо її перший параметр менше,
ніж другий. Використовуйте його як предикат для виклику
використовуваного вами алгоритму.

Знайдіть усі міста на висоті 2500--3500 футів включно. За необхідності
реалізуйте оператори рівності для класу Town.

Нам потрібно розмістити аеропорт на певній висоті, але розташування не є
проблемою. Впорядкуйте свій список міст так, щоб не було дублікатів
(дублікат означає, що жодні дві висоти не знаходяться в одному діапазоні
100 футів. До таких класів належать {[}100, 199), {[}200, 199) і т.д.
Відсортуйте цей список за зростанням принаймні двома різними способами,
використовуючи об'єкти функції в \textless{}functional\textgreater{}.
Зробіть те ж саме для порядку зменшення. За необхідності впроваджуйте
реляційні оператори для міста.

Створіть довільну кількість випадкових чисел у масиві на основі стеку.
Використовуйте max\_element (), щоб знайти найбільше число в масиві.
Поміняйте його номером у кінці масиву. Знайдіть наступне найбільше число
та росташуйте його в масиві в позиції перед попереднім числом.
Продовжуйте це робити, доки всі елементи не будуть переміщені. Коли
алгоритм буде завершено, ви отримаєте відсортований масив. (Це
сортування виділенням)

Напишіть програму, яка знімає телефонні номери з файлу (що також містить
імена та іншу відповідну інформацію) та змінює номери, що починаються з
222 на 863. Обов'язково збережіть старі номери. Формат файлу такий:

222 8945

756 3920

222 8432

тощо

Напишіть програму, яка за прізвищем знайде кожного з цим прізвищем із
відповідним номером телефону. Використовуйте алгоритми, які мають справу
з діапазонами (upper\_bound, lower\_bound, equal\_range тощо). Сортуйте
за прізвищем, що діє як первинний ключ, а за іменем, що діє як вторинний
ключ. Припустимо, що ви прочитаєте імена та номери з файлу, формат якого
буде таким. (Обов'язково впорядкуйте їх так, щоб прізвища були
впорядковані, а імена впорядковані в межах прізвищ.):

Джон Доу 345 9483

Нік Бонем 349 2930

Джейн Доу 283 2819

Отримавши файл із даними, подібними до наведених нижче, витягніть із
нього всі державні абревіатури та помістіть їх в окремий файл. (Зверніть
увагу, що ви не можете залежати від номера рядка для типу даних. Дані
містяться на випадкових рядках.)

Складіть клас Employee із двома членами даних: hours та hourlyPay.
Працівник також повинен мати функцію calcSalary(), яка повертає
заробітну плату за цього працівника. Генеруйте довільну погодинну оплату
праці та години для довільної кількості працівників. Зберігайте вектор
\textless{}Співробітник *\textgreater{}. Дізнайтеся, скільки грошей
компанія витратить за цей період оплати праці.

Порівняйте роботу функцій sort(), partial\_sort() та nth\_element() один
проти одного і з'ясуйте, чи дійсно варто використовувати одне із слабких
сортувань, коли вони можуть спрацювати коректно.

Міські об'єкти. Створіть назви міст (незалежно від того, мають вони сенс
чи ні, це не має значення) або витягніть їх з Інтернету. Переконайтеся,
що назва всього міста має маленькі регістри, а дублікатів назв немає.
Для простоти радимо зводити назви міст одним словом. Для населення,
висот та погодних полів створіть генератор, який випадковим чином
генеруватиме погодні умови, популяції в межах {[}100-1000000) та висоти
між {[}0, 8000) футами. Заповніть вектор об'єктами міста. Перепишіть
вектор у новий файл під назвою Towns.txt.

\subsection{12. Випадкові
числа}\label{ux432ux438ux43fux430ux434ux43aux43eux432ux456-ux447ux438ux441ux43bux430}

Випадкові числа

Маємо натуральні а, с, m, s0 такі, що m --- найбільше з них. Визначимо
послідовність натуральних чисел s0, s1,\ldots{},sn наступним чином: sn
дорівнює \(s_{n} = a*s_{n - 1} + c(mod\ m)\), та розглянемо
послідовність

\(r_{n} = \frac{s_{n} + 1}{m + 1}\).

Послідовність r буде імітувати рівномірно розподілені в інтервалі (0, 1)
випадкові числа, якщо:

а) \({m = 2}^{n}\), де k --- натуральне число;

б) при діленні числа \(a\) на 8 --- остаток дорівнює 5. Крім того,
\(\sqrt{m} < a < m - \sqrt{m}\);

в) Число с --- непарне, при цьому бажано щоб
\(\frac{s}{m} \approx 0.5 - \frac{\sqrt{3}}{6}\);

г) Число s0 можна обрати довільно в діапазоні від 0 до m-1.

Завдання:

а) Створити функцію, що буде генерувати числа а, с, m, s0, що
задовольняють вказаним умовам

б) створити на базі цієї послідовності генератор випадкових цілих чисел
та генератор випадкових дійсних чисел

в) створити цей генератор таким чином, щоб він генерував майже завжди
різні числа при першому виклику (визначайте нове s0) та враховував
попередні виклики при нових викликах (використовуйте статичні глобальні
змінні та хеш часу)

г) Отримайте цим датчиком 1000 чисел та оцінить рівномірність розподілу:
розбийте інтервал (0, 1) на N інтервалів рівної довжини та знайдіть
варіацію серед чисел, що туди потрапили.

Описані методи повинні бути описаними в заголовочному файлі на Сі та як
методи класу на Сі++(клас відповідно містить приватні члени для а, с, m,
s0). В тестовий програмі перевірте зокрема коректність ГВЧ за критерієм
Хі-квадрат.

Random-2

Маємо натуральні а, с, m, s0 такі, що m --- найбільше з них. Визначимо
послідовність натуральних чисел s0, s1, . . .sn наступним чином: sn
дорівнює \(s_{n} = a*s_{n - 1} + c(mod\ m)\), та розглянемо
послідовність

\(r_{n} = \frac{s_{n} + 1}{m + 1}\) .

Послідовність r\_i буде імітувати рівномірно розподілені в інтервалі (0,
1) випадкові числа, якщо:

a) \({m = 2}^{n}\), де k --- натуральне число;

b) c та m --- взаємно прості числа;

c) (a-1) ділиться на всі прості числа p, що є дільниками m;

d) b*(a-1) ділиться на 4, якщо m ділиться на 4.

Завдання:

а) Створити функцію, що буде генерувати числа а, с, m, s0, що
задовольняють вказаним умовам

б) створити на базі цієї послідовності генератори випадкових цілих чисел
та генератор випадкових дійсних чисел

в) створити цей генератор таким чином, щоб він генерував майже завжди
різні числа при першому виклику (визначайте нове s0) та враховував
попередні виклики при нових викликах (використайте статичний член
структури або класу та \emph{хеш часу})

г) на базі даного ГВЧ створить метод, що генерує n-вимірні випадкові
вектори дійсних чисел, кожні дві координати яких є некорельовані.

Напишіть код, який перевіряє на достатньо великій вибірці, що вони
дійсно некорельовані.

д) На базі методу Монте-Карло підрахуйте об``єм n-вимірної сфери.

\begin{quote}
За допомогою ГВЧ отримати:

ж) 10 натуральних чисел, що більше 20;

з) n цілих чисел в діапазоні -150, 150;

и) n не­відємних дійсних чисел, менших 3.14;

й) 15 чисел, серед яких 7 двійок та 8 трійок;

к) перестановку чисел 1, ..., 12, тобто послідовність чисел р1, ...,
р12, в яку входить кожне з чисел 1, ..., 12;

л) 28 малих латинських літер;

м) 15 великих латинських літер без повторів

3) Використовуючи розподіли зі стандартної бібліотеки Сі++:

а) Побудувати 100 перших членів випадкової послідовності з нулів і
одиниць, в яких нуль і одиниця рівноймовірні, тобто послідовності з
розподілом ( 0.5, 0.5)

б) Побудувати 100 перших членів випадкової послідовності з цифр 1, 2, 3,
4, 5, 6, в який всі ці цифри рівноймовірні.

в) Побудувати 100 перших членів випадкової послідовності з нулів и
одиниць, в який нуль зустрічається з ймовірністю 1/4, а одиниця з
імовірністю 3/4,

г) Побудувати 100 перших членів випадкової послідовності слів «камінь»,
«ножиці», «бумага», в який ці три слова равноімовірні.

д) Побудувати 100 перших членів випадкової послідовності слів «камінь»,
«ножиці», «бумага», в який слово «камінь» зустрічається з ймовірністю
1/3, слово «ножиці»--- з ймовірністю 1/2, слово «бумага»---с ймовірністю
1/6.

4)Побудувати послідовність випадкових величин, що задовольняються
розподілу:

а) Пуасона

б) Гауса

в) експоненційного

г) Стьюдента

д) Фішера

Використовуйте srand для створення 100 чисел. (Розмір чисел не має
значення.) Знайдіть, які числа у вашому діапазоні є конгруентними модулю
23 (тобто вони мають однаковий залишок, коли їх ділити на 23). Виберіть
самостійно випадкове число вручну та визначте, чи перебуває воно у
вашому діапазоні, поділивши кожне число у списку на ваше число та
перевіривши, чи результат дорівнює 1, а не просто використовуючи find ()
зі своїм значенням. 15. Заповніть вектор
\textless{}подвійний\textgreater{} цифрами, що представляють кути в
радіанах. Використовуючи функціональний склад об'єкта, візьміть синус
усіх елементів у вашому векторі (див. \textless{}cmath\textgreater{}).
16. Перевірте швидкість свого комп'ютера. Викличте srand (час (0)), а
потім створіть масив випадкових чисел. Знову викличить srand (time (0))
і згенеруйте однакову кількість випадкових чисел у другому масиві.
Скористайтесь рівним (), щоб перевірити, чи однакові масиви. (Якщо ваш
комп'ютер досить швидкий, час (0) поверне одне і те ж значення в обох
випадках, коли його викликають.) Якщо масиви неоднакові, відсортуйте їх
та скористайтеся невідповідністю (), щоб побачити, де вони
відрізняються. Якщо вони однакові, збільште довжину масиву та повторіть
спробу.

Олімпіадні задачи
\end{quote}

\textbf{Задача по математиці.}

Задачник містить N задач, пронумерованих від 1 до N. У вчительки є
магнітики з цифрами. На початку уроку вона прикріплює їх на дошку таким
чином, щоби утворилися номери K задач, які розбираються на уроці. Яка
кількість та яких саме магнітиків з цифрами потрібна для того, щоб
вчителька могла записати номери всіх K задач?

\begin{quote}
\textbf{Лотерейні квітки}

Дано: Масив - таблиця з 5000 лотерейних квитків:\\
id білета, виграш квітка (від 0 до 100), а також масив - таблиця 20
учасників лотереї:\\
id учасника, кількість квитків, бажана сума виграшу з квитків.

Сума виграшу квитків дорівнює сумі бажаної суми виграшу з білетів всіх
учасників\\
Сума кількості всіх квитків дорівнює сумі кількості всіх квитків всіх
користувачів Потрібно кожному квітку співставити учасника так, щоби
виконувались умови:\\
- кожен учасник отримав вказану кількість квитків\\
- сума виграшу з усіх квитків кожного учасника була максимально близька
до

бажаної суми виграшів (задане число).

\textbf{Розшифровка чисел}

Є база даних цілих чисел:

42498910\\
40522543\\
38356813\\
39343454\\
40724853\\
41975176\\
43487650\\
46448082\\
47105757\\
48291314\\
...

В них зашифровано RGB представлення кольору

Дано частину співставлення:

33591293 ff9515\\
33591785 ec9615\\
37699777 c9453b\\
37707949 b2633e\\
49345525 f5f4f1\\
49081842 f3eeed

Знайти та реалізувати алгоритм шифрування/розшифрування та отримати HEX
або RGB чисел.

\textbf{Число Карпекара}

Розглянемо натуральне число, що більше 1 та менше 9999, в десятковому
запису якого повинно бути принаймні дві різні цифри ( Наприклад, 3993 -
ок, а 3333 ні). Якщо ціле число менше 1000 заповніть їх нулями так, щоб
вони мали 4 цифри, наприклад, ціле число 10 буде 0010.

Для цього числа (наприклад, 9837) проведіть наступні операції:

1) відсортуйте цифри за зростанням, тобто отримайте 3789;

2) відсортуйте цифри за спаданням, тобто отримайте 9873;

3) відніміть ці два числа, тобто 9873 - 3789 = 6084.

4) якщо це число не дорівнює попередньому числу, повторіть процедуру.

Перевірьте, що для кожного числа, яке задовольняє потрібним
властивостям, ця процедура буде збігатися до єдиного числа (сталої
Карпекара) та виведіть його, а також число яке збігається до нього за
найбільшу кількість ітерацій (якщо їх декілька - виведіть найменше з
них).

\textbf{Похідна багаточлена}

Багаточлен з цілими коефецієнтами задається в текстовому рядку, де вони
без пробілу записані за допомогою цифр та знаків +/-(перед числами),
*(перед змінною) та \^{}(перед ступенем), а також ідентифікатором
змінної (х). Багаточлен може бути поданий на вхід як багаточлен з
неприведеними та невідсортованими доданками.

Знайти та вивести похідну многочлена. Багаточлен може бути великий, але
тільки з невід'ємними цілими ступенями і цілими коефіцієнтами. Виведення
повинно бути без пробілів і в порядку спадання ступенів.

Приклади:1) x\^{}2+x - результат: 2*x+1

2) 2*x\^{}100+100*x\^{}2 - результат: 200*x\^{}99+200*x

3) -x\^{}2-x\^{}3 - результат: -3*x\^{}2-2*x

4) x+x+x+x+x+x+x+x+x+x - результат: 10

5) x\^{}10000+x+1 - результат: 10000*x\^{}9999+1

Розв'яжить цю задачу якщо замість х може бути будь-який ідентифікатор.
\end{quote}

\end{document}
