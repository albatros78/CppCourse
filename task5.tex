\documentclass[]{article}
\usepackage{lmodern}
\usepackage{amssymb,amsmath}
\usepackage{ifxetex,ifluatex}


\usepackage[utf8]{inputenc}
\usepackage[english,russian,ukrainian]{babel}

\usepackage{fixltx2e} % provides \textsubscript
\ifnum 0\ifxetex 1\fi\ifluatex 1\fi=0 % if pdftex
  \usepackage[T1]{fontenc}
  \usepackage[utf8]{inputenc}
\else % if luatex or xelatex
  \ifxetex
    \usepackage{mathspec}
  \else
    \usepackage{fontspec}
  \fi
  \defaultfontfeatures{Ligatures=TeX,Scale=MatchLowercase}
\fi
% use upquote if available, for straight quotes in verbatim environments
\IfFileExists{upquote.sty}{\usepackage{upquote}}{}
% use microtype if available
\IfFileExists{microtype.sty}{%
\usepackage{microtype}
\UseMicrotypeSet[protrusion]{basicmath} % disable protrusion for tt fonts
}{}
\usepackage[unicode=true]{hyperref}
\hypersetup{
            pdfborder={0 0 0},
            breaklinks=true}
\urlstyle{same}  % don't use monospace font for urls
\usepackage{graphicx,grffile}
\makeatletter
\def\maxwidth{\ifdim\Gin@nat@width>\linewidth\linewidth\else\Gin@nat@width\fi}
\def\maxheight{\ifdim\Gin@nat@height>\textheight\textheight\else\Gin@nat@height\fi}
\makeatother
% Scale images if necessary, so that they will not overflow the page
% margins by default, and it is still possible to overwrite the defaults
% using explicit options in \includegraphics[width, height, ...]{}
\setkeys{Gin}{width=\maxwidth,height=\maxheight,keepaspectratio}
\IfFileExists{parskip.sty}{%
\usepackage{parskip}
}{% else
\setlength{\parindent}{0pt}
\setlength{\parskip}{6pt plus 2pt minus 1pt}
}
\setlength{\emergencystretch}{3em}  % prevent overfull lines
\providecommand{\tightlist}{%
  \setlength{\itemsep}{0pt}\setlength{\parskip}{0pt}}
\setcounter{secnumdepth}{0}
% Redefines (sub)paragraphs to behave more like sections
\ifx\paragraph\undefined\else
\let\oldparagraph\paragraph
\renewcommand{\paragraph}[1]{\oldparagraph{#1}\mbox{}}
\fi
\ifx\subparagraph\undefined\else
\let\oldsubparagraph\subparagraph
\renewcommand{\subparagraph}[1]{\oldsubparagraph{#1}\mbox{}}
\fi

\date{}


\usepackage{enumitem}
\makeatletter
\newcommand{\xslalph}[1]{\expandafter\@xslalph\csname c@#1\endcsname}
\newcommand{\@xslalph}[1]{%
    \ifcase#1\or а\or б\or в\or г\or д\or e\or є\or ж\or з\or i%
    \or й\or к\or л\or м\or н\or о\or п\or р\or с\or т%
    \or у\or ф\or х\or ц\or ч\or ш\or ю\or я\or аа\or бб\or вв %
    \else\@ctrerr\fi%
}
\AddEnumerateCounter{\xslalph}{\@xslalph}{m}
\makeatother


\begin{document}

\section*{ Методичні рекомендації з курсу «Мова програмування С++» }

Вступ

1. Лінійні програми на Сі. Введення/виведення. Дійсний тип даних.

2. Використання математичної бібліотеки С. Створення власних функцій

3. Цілі типи Сі. Умовні конструкції.

4. Цикли.

5. Цикли. Рекурентні співвідношення. Рекурсія

6. Бітові операції

7. Статичні масиви. Лінійні масиви та багатовимірні масиви

8. Динамічні масиви. Робота з вказівниками

9. Робота з рядком, що закінчується нулем на С.

10. Структури. Створення власного типу

11. Робота з бінарним файлами на Сі

12. Введення/виведення на С++. Робота з текстовими файлами

13. Робота з класом рядок на С++.

14. Створення власних класів. Інкапсуляція.

15. Робота з класами. Наслідування та поліморфізм.

16. Перетворення типів та робота з виключеннями.

17. Створення шаблонів функцій та шаблонів класів

18. Стандартна бібліотека С++. Послідовні контейнери.

19. Стандартна бібліотека С++. Асоціативні контейнери.

20. Стандартна бібліотека С++. Алгоритми та функтори.

\subsection{ ВСТУП }

Мета цього посібника, надати студенту завдання для того, щоб практично
оволодіти потрібними навичками програмування на мовах С та С++ в рамках
дисципліни «Мова програмування С++». Теми обиралися автором таким чином,
щоб найбільш швидким темпом здобути навичкі для практичного
програмування за 20 занять, тому деякі теми та розділи програмування на
С та С++, які автор вважає занадто складним або не обовязковими з точки
зору практики програмування, не входять до цього задачника, а винесені
на самостійну роботу або в якості завдань на курсові проекти.

Завдання посібника розділені на 20 лабораторних робіт, кожна з яких
присвячена окремій темі, що вивчається в дисципліні. Завдання та теми
підбиралися таким чином, щоб вивчення синтаксису мови виходило
поступовим тому послідовне виконання лабораторних робіт є найкращим для
засвоєння та набуття відповідних навичок. Тому наполегливо рекомендуємо
дотримуватися послідовного виконання лабораторних робіт.

Матеріал кожної лабораторної роботи посібника складається з п'яти
блоків: контрольних запитань, завдань для аудиторної роботи та трьох
блоків завдань для самостійної роботи. Під час підготовки до практичного
заняття, студент повинен опрацювати блок контрольних запитань та знати
вичерпні відповіді на них. Блок завдань для аудиторної містять перелік
типових задач відповідної теми. Ці завдання студент має виконати
протягом практичного заняття самостійно або під керівництвом викладача.
Завдання для самостійної роботи студент виконує самостійно та звітує про
їхнє виконання викладачу. Як було зазначено вище, завдання для
самостійної роботи складається з трьох блоків, перший з яких є
обов'язковим для виконання.

Другий блок завдань є ідентичним по складності основному блоку завдань
для самостійної роботи та призначений для кращого засвоєння матеріалу.

Третій блок завдань складається з задач підвищеної складності та вимагає
від студента не лише досконалого опанування методів поточної теми, а й
матеріалу, що виходить за межі нормативного курсу.


\newpage
\subsection{ 5. Цикли. Рекурентні співвідношення. Рекурсія }
\setcounter{subsection}{1}


\begin{itemize}
\item
  Яким чином обчислити числа Фібоначчі на Сі за допомогою циклів?
\item
  Який загальний метод обчислення рекурентних послідовномтей для Сі?
\item
  Що таке рекурсія та як її застосувати для обчислення, наприклад,
  факторіалу? Чисел Фібоначчі?
\item
  Що таке бінарний пошук та як його застосувати?
\end{itemize}

\begin{enumerate}
\def\labelenumi{\arabic{enumi})}
\item
  Маємо дійсне число \emph{a}. Скласти програми обчислення:
\begin{itemize}
\item серед чисел
\(1,1 + \frac{1}{2},1 + \frac{1}{2} + \frac{1}{3},\ldots\)першого,
більшого за $a$;

\item такого найменшого $n>0$, що
\(1 + \frac{1}{2} + \ldots + \frac{1}{n} > a.\)

\end{itemize}

\item
  Числами Фібоначчі називається числова послідовність
  \(\left\{ F_{n} \right\}\), задана рекурентним співвідношенням другого
  порядку
  \(F_{0} = 0,F_{1} = 1,F_{k} = F_{k - 1} + F_{k},\ k = 2,3,\ldots\).

Скласти функції:
\begin{enumerate}[label=\xslalph*)]
\item
для обчислення \(F_{n}\ \)за номером члену;
\item номера найбільшого числа Фібоначчі, яке не перевищує задане число
$a$;
\item номера найменшого числа Фібоначчі, яке більше заданого числа
$a$;
\item суми всіх чисел Фібоначчі, які не перевищують 1000.

\end{enumerate}

\item
  Введіть натуральне число n. Далі утворить рекурентну послідовність
  \(a_{i}\)за наступним правилом: \(a_{0} = n\). Якщо \(a_{k}\) парне,
  то \(a_{k}\), якщо --- непарне, то\(a_{k + 1} = 4a_{k} + 1\). Доведіть
  що для n\textless{}1000 ця послідовність буду містити член рівний
  одиниці. Знайдіть серед цих n число, якому потрібно максимальна
  кількість кроків для досягнення одиниці.
\item
  Скласти програми для обчислення добутків:
\begin{enumerate}[label=\xslalph*)]
\item \(P_{n} = \prod_{i = 1}^{n}\left( 2 + \frac{1}{i!} \right);\) 
\item
\(P_{n} = \prod_{i = 1}^{n}\left( \frac{i + 1}{i + 2} \right);\)
\item
\(P_{n} = \prod_{i = 1}^{n}\frac{1}{(i + 1)!};\); г)
\(P_{n} = \prod_{i = 1}^{n}\frac{1}{i^{i} + 1}.\)
\end{enumerate}

\emph{\emph{Вказівка}}. Добуток \emph{P\textsubscript{n}} обчислити за
допомогою рекурентного співвідношення
\(P_{0} = 1,P_{k} = P_{k - 1}*a_{k},k = 1,2,\ldots,n,\)($k=1,2,\ldots,n$)
де \(a_{k}\)- $k$-тий множник.

\item
  Скласти програми для обчислення найменшого додатного члена числових
  послідовності, які задаються рекурентними співвідношеннями, та його
  номера:
\(x_{n} = x_{n - 1} + x_{n - 3} + 100, x_{1} = x_{2} = x_{3} = - 99, n = 3,4,\ldots;\)

\item
  Скласти програми для обчислення ланцюгових дробів
\begin{enumerate}[label=\xslalph*)]
\item \(b_{n} = b + \frac{1}{b + \frac{1}{b + \ddots + \frac{1}{b}};}\); 
\item
\(\lambda_{n} = 2 + \frac{1}{6 + \frac{1}{10 + \ddots + \frac{1}{4n + 2}};}\)
\item
\(x_{2n} = 1 + \frac{1}{2 + \frac{1}{1 + \frac{1}{2 + \frac{1}{1 + \ddots + \frac{1}{2}}}.};}\)
\end{enumerate}
\emph{\emph{Вказівка}}. Використати рекурентні співвідношення

а)
\(b_{0} = b,b_{k} = b + \frac{1}{b_{k - 1}}, \; k = 1,2,\ldots,n\);

б)
\(b_{0} = 4n + 2,b_{k} = 4(n - k) + 2 + \frac{1}{b_{k - 1}},\; k = 1,2,\ldots,n\).

\item
  Скласти програми для обчислення суми:
\end{enumerate}

\(S_{n} = \sum_{k = 1}^{n}\frac{2^{k}}{a_{k} + b_{k}},\) ,

де \(\left\{ \begin{matrix}
 a_{1} = 0,a_{2} = 1, \\
 a_{k} = \frac{a_{k - 1}}{k} + a_{k - 2}b_{k}, \\
\end{matrix} \right.\ \) \(\left\{ \begin{matrix}
 b_{1} = 1,b_{2} = 0, \\
 b_{k} = b_{k - 1} + a_{k - 1}, \\
\end{matrix} \right.\ \) \(k = 3,4,\ldots;\)



\emph{Самостійна}

\begin{enumerate}
\def\labelenumi{\arabic{enumi})}
\setcounter{enumi}{7}
\item
  Скласти програми обчислення довільного елемента послідовностей,
  заданих рекурентними співвідношеннями
\begin{enumerate}[label=\xslalph*)]
\item
\(v_{0} = 1,v_{1} = 0.3, v_{i} = (i + 2)v_{i - 2}, i = 2,3,\ldots\)

\item
\(v_{0} = v_{1} = v_{2} = 1, \; v_{i} = (i + 4)(v_{i - 1} - 1) + (i + 5)v_{i - 3},\; i = 3,4,\ldots\)

\item
\(v_{0} = v_{1} = 0,\ v_{2} = \frac{3}{2}\;v_{i} = \frac{i - 2}{(i - 3)^{2} + 1}v_{i - 1} - v_{i - 2}v_{i - 3} + 1,\; i = 2,3,\ldots\)

\end{enumerate}

\item
  Скласти програму обчислення довільного елемента послідовності
  \(v_{n}\), визначеної системою співвідношень

\[v_{0} = v_{1} = 1,v_{i} = \frac{u_{i - 1} - v_{i - 1}}{\left| u_{i - 2} + v_{i - 1} \right| + 2},i = 2,3,\ldots;\]

де
\(u_{0} = u_{1} = 0,u_{i} = \frac{u_{i - 1} - u_{i - 2}v_{i - 1} - v_{i - 2}}{1 + u_{i - 1}^{2} + v_{i - 1}^{2}},i = 2,3,\ldots;\)


\item
  Скласти програми для обчислення сум:
\begin{enumerate}[label=\xslalph*)]
\item
\(S_{n} = \sum_{k = 1}^{n}{2^{k}a_{k}},\textup{дe \ }a_{1} = 0,a_{2} = 1,a_{k} = a_{k - 1} + k*a_{k - 2},\ k = 3,4,\ldots;\)
\item
\(S_{n} = \sum_{k = 1}^{n}\frac{3^{k}}{a_{k}},\textup{дe \ } a_{1} = a_{2} = 1,\ a_{k} = \frac{a_{k - 1}}{k} + a_{k - 2},\; k = 3,4,\ldots;\)

\item
\(S_{n} = \sum_{k = 1}^{n}\frac{k!}{a_{k}},\textup{дe \ } a_{1} = a_{2} = 1,\ a_{k} = a_{k - 1} + \frac{a_{k - 1}}{2^{k}},\; k = 3,4,\ldots;\)

\item
\(S_{n} = \sum_{k = 1}^{n}{k!a_{k}},\textup{дe \ } a_{1} = 0,a_{2} = 1,\ a_{k} = a_{k - 1} + \frac{a_{k - 2}}{(k - 1)!},\; k = 3,4,\ldots;\)
\item
\(S_{n} = \sum_{k = 1}^{n}\frac{a_{k}}{2^{k}},\textup{дe \ } a_{1} = a_{2} = a_{3} = 1,\ a_{k} = a_{k - 1} + a_{k - 3},\; k = 4,5,\ldots;\)
\item
\(S_{n} = \sum_{k = 1}^{n}{\frac{2^{k}}{k!}a_{k}},\textup{дe \ } a_{0} = 1,a_{k} = ka_{k - 1} + \frac{1}{k},\; k = 1,2,\ldots.\)

\end{enumerate}

\item
  Скласти програми для обчислення сум:
\begin{enumerate}[label=\xslalph*)]
\item \(S_{n} = \sum_{k = 1}^{n}\frac{3^{2k + 1}}{a_{k}*b_{k} + 1},\) ,

де \(\left\{ \begin{matrix}
 a_{1} = 2,a_{2} = 1, \\
 a_{k} = \frac{a_{k}}{k + 1} + a_{k - 2} + b_{k}, \\
\end{matrix} \right.\ \) \(\left\{ \begin{matrix}
 b_{1} = 1,b_{2} = 0, \\
 b_{k} = 2b_{k - 1} + a_{k - 1}, \\
\end{matrix} \right.\ \) \(k = 3,4,\ldots;\)

\item \(S_{n} = \sum_{k = 1}^{n}\frac{a_{k}b_{k}}{(k + 1)!},\)

де \(\left\{ \begin{matrix}
 a_{1} = u, \\
 a_{k} = 2b_{k - 1} + a_{k - 1}, \\
\end{matrix} \right.\ \) \(\left\{ \begin{matrix}
 b_{1} = v, \\
 b_{k} = 2a_{k = 1}^{2} + b_{k - 1}, \\
\end{matrix} \right.\ \) \(k = 2,3,\ldots;\)

\emph{u,v} -- задані дійсні числа;
\item
\(\ S_{n} = \sum_{k = 1}^{n}\frac{2^{k}}{{(1 + a}_{k} + b_{k}){k!}^{}}\)

де \(\left\{ \begin{matrix}
 a_{1} = 1, \\
 a_{k} = 3b_{k - 1} + 2a_{k - 1}, \\
\end{matrix} \right.\ \) \(\left\{ \begin{matrix}
 b_{1} = 1, \\
& b_{k} = 2a_{k - 1} + b_{k - 1}, \\
\end{matrix} \right.\ \) \(k = 2,3,\ldots;\)
\item \(S_{n} = \sum_{k = 1}^{n}\left( \frac{a_{k}}{b_{k}} \right)^{k},\)

де \(\left\{ \begin{matrix}
 a_{0} = 1,a_{1} = 2, \\
 a_{k} = b_{k - 2} + \frac{b_{k}}{2}, \\
\end{matrix} \right.\ \) \(\left\{ \begin{matrix}
 a_{0} = 5,b_{1} = 5, \\
 b_{k} = b_{k - 2}^{2} - a_{k - 1}, \\
\end{matrix} \right.\ \) \(k = 2,3,\ldots;\)
\item \(S_{n} = \sum_{k = 1}^{n}\frac{a_{k}}{1 + b_{k}},\)

де \(\left\{ \begin{matrix}
 a_{0} = 1, \\
 a_{k} = b_{k - 1}a_{k - 1}, \\
\end{matrix} \right.\ \) \(\left\{ \begin{matrix}
 b_{0} = 1, \\
 b_{k} = b_{k - 1} + a_{k - 1}, \\
\end{matrix} \right.\ \) \(k = 1,2,\ldots.\)\emph{.}

\end{enumerate}

\item
  Скласти програми для обчислення добутків
\begin{enumerate}[label=\xslalph*)]
\item \(P_{n} = \prod_{k = 0}^{n}{\frac{a_{k}}{3^{k}},}\) де
\(\left\{ \begin{matrix}
 a_{0} = a_{1} = 1,\ a_{2} = 3, \\
 a_{k} = a_{k - 3} + \frac{a_{k - 2}}{2^{k - 1}}, \\
\end{matrix} \right.\ \), \(k = 3,4,\ldots;\)

\item \(P_{n} = \prod_{k = 1}^{n}{a_{k}b_{k},}\)

де \(\left\{ \begin{matrix}
 a_{1} = 1, \\
 a_{k} = \left( \sqrt{b_{k - 1}} + a_{k - 1} \right)/5, \\
\end{matrix} \right.\ \) \(\left\{ \begin{matrix}
 b_{1} = 1, \\
 b_{k} = 2b_{k - 1} + 5a_{k - 1}^{2}, \\
\end{matrix} \right.\ \) \(k = 2,3,\ldots\)\emph{.}

\end{enumerate}

\item
  Реалізувати функцію яка з`ясовує, чи входить задана цифра до запису
  заданого натурального числа.
\item
  Реалізувати функцію "обернення" (запису в оберненому порядку цифр)
  заданого натурального числа.

\emph{\emph{Вказівка. Для побудови числа використати рекурентне
співвідношення}} \(y_{0} = 0,y_{i} = y_{i - 1}*10 + a_{i},\)\emph{де}
\(a_{i}\) \emph{- наступна цифра числа} \(n\)\emph{при розгляді цифр
справа наліво.}

\item
  Скласти програми наближеного обчислення суми всіх доданків, абсолютна
  величина яких не менше $\varepsilon > 0 $:
\begin{enumerate}[label=\xslalph*)]
\item \(y = \sin x = x - \frac{x^{3}}{3!} + \frac{x^{5}}{5!} - \ldots\);
\item \(y = \cos x = 1 - \frac{x^{2}}{2!} + \frac{x^{4}}{4!} - \ldots\);
\item
\(y = \sinh (x) = x + \frac{x^{3}}{3!} + \frac{x^{5}}{5!} + \ldots\);
\item 
\(y = \cosh (x) = 1 + \frac{x^{2}}{2!} + \frac{x^{4}}{4!} + \ldots\);
\item \(y = e^{x} = 1 + \frac{x}{1!} + \frac{x^{2}}{2!} + \ldots\);
\item
\(y = \ln(1 + x) = x - \frac{x^{2}}{2!} + \frac{x^{3}}{3!} - \ldots,(\left| x \right| < 1)\);
\item
\(y = \frac{1}{1 + x} = 1 - x + x^{2} - x^{3} + \ldots,(\left| x \right| < 1)\);
\item
\(y = \ln\frac{1 + x}{1 - x} = 2*\frac{x}{1} + \frac{x^{3}}{3} + \frac{x^{5}}{5} + \ldots, (\left| x \right| < 1)\);
\item
\(y = \frac{1}{(1 + x)^{2}} = 1 - 2*x + 3*x^{2} - \ldots,(\left| x \right| < 1)\);
\item
\(y = \frac{1}{(1 + x)^{3}} = 1 - \frac{2*3}{2}x + \frac{3*4}{2}x^{2} - \frac{4*5}{2}x^{3} + \ldots,(\left| x \right| < 1)\);
\item
\(y = \frac{1}{1 + x^{2}} = 1 - x^{2} + x^{4} - x^{6} + \ldots,(\left| x \right| < 1)\);
\item
\(y = \sqrt{1 + x} = 1 + \frac{1}{2}x - \frac{1}{2*4}x^{2} + \frac{1*3}{2*4*6}x^{3} - \ldots,(\left| x \right| < 1)\);
\item
\(y = \frac{1}{\sqrt{1 + x}} = 1 - \frac{1}{2}x + \frac{1*3}{2*4}x^{2} - \frac{1*3*5}{2*4*6}x^{3} - \ldots,(\left| x \right| < 1)\);
\item
\(y = \arcsin (x) = x + \frac{1}{2}\frac{x^{3}}{3!} + \frac{1*3}{2*4}\frac{x^{5}}{5!} + \ldots,(\left| x \right| < 1)\).

\end{enumerate}

\emph{\emph{Вказівка}}. Суму $y$ обчислювати за допомогою
рекурентного співвідношення
\(S_{0} = 0,\ S_{k} = S_{k - 1} + a_{k},\ k = 1,2,\ldots,\) де
\(a_{k} - k\)-тий доданок, для обчислення якого також складається
рекурентне співвідношення. В якості умови повторення циклу розглядається
умова \(\left| a_{k} \right| \geq \varepsilon.\)

\item
  Ввести дійсні числа
  \(x,\varepsilon\ (x \neq 0,\varepsilon > 0)\)\emph{.} Обчислити з
  точністю \(\varepsilon\) нескінченну суму і вказати кількість
  врахованих доданків.
\begin{enumerate}[label=\xslalph*)]
\item \(\sum_{k = 0}^{\infty}\frac{x^{2k}}{2k!};\) 
\item \(\sum_{k = 0}^{\infty}\frac{( - 1)^{k}x^{k}}{(k + 1)^{2}};\)
\item \(\sum_{k = 0}^{\infty}\frac{x^{2k}}{2^{k}k!};\) г)
\(\sum_{k = 0}^{\infty}\frac{( - 1)^{k}x^{2k + 1}}{k!(2k + 1)!}.\)
\end{enumerate}

\end{enumerate}

Додаткові задачі:

\begin{enumerate}
\def\labelenumi{\arabic{enumi})}
\setcounter{enumi}{16}
\item
  Дано натуральне число $k$ . Скласти програму одержання $k$-тої цифри послідовності
\begin{enumerate}[label=\xslalph*)]
\item 110100100010000 ... , в якій виписані підряд степені 10;
\item 123456789101112 ... , в якій виписані підряд всі натуральні числа;
\item 149162536 ... , в якій виписані підряд квадрати всіх натуральних
чисел;
\item 01123581321 ... , в якій виписані підряд всі числа Фібоначчі.

\end{enumerate}

\item
  Скласти програму знаходження кореня рівняння \(tgx = x\) на відрізку
  {[}0,001;1,5{]} із заданою точністю \(\varepsilon\), використовуючи
  метод ділення відрізку навпіл.
\item
  Знайти корінь рівняння \(x^{3} + 4x^{2} + x - 6 = 0,\) який міститься
  на відрізку {[}0,2{]}, з заданою точністю \(\varepsilon\).

\emph{\emph{Вказівка.}} Одним з методів розв`язування рівняння є метод
хорд, який полягає в обчисленні елементів послідовності 

\(u_{0} = a / n, \;  u_{n} = u_{n-1} - \frac{y(u_{n-1}) (u_{n-1} -u_{0})}{y(u_{n-1}) -y(u_{0})} \) 

до виконання умови \(\left| u_{n} - u_{n - 1} \right| < \varepsilon\). В
умовах нашої задачі \(a = 0,b = 2,\ y(x) = x^{3} + 4x^{2} + x - 6.\)

\item
  а)Скласти програму, яка визначає потрібний спосіб розміну будь-якої
  суми грошей до 99 коп. за допомогою монет вартістю 1, 2, 5, 10, 25, 50
  коп.

  б) Розв'яжить цю задачу для будь-якого натурального числа $m$
($1 <m<100000$) копійок так щоб кількість монет при
цьому була найменша.
\end{enumerate}

\end{document}

