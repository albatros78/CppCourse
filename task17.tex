\documentclass[]{article}
\usepackage{lmodern}
\usepackage{amssymb,amsmath}
\usepackage{ifxetex,ifluatex}


\usepackage[utf8]{inputenc}
\usepackage[english,russian,ukrainian]{babel}

\usepackage{fixltx2e} % provides \textsubscript
\ifnum 0\ifxetex 1\fi\ifluatex 1\fi=0 % if pdftex
  \usepackage[T1]{fontenc}
  \usepackage[utf8]{inputenc}
\else % if luatex or xelatex
  \ifxetex
    \usepackage{mathspec}
  \else
    \usepackage{fontspec}
  \fi
  \defaultfontfeatures{Ligatures=TeX,Scale=MatchLowercase}
\fi
% use upquote if available, for straight quotes in verbatim environments
\IfFileExists{upquote.sty}{\usepackage{upquote}}{}
% use microtype if available
\IfFileExists{microtype.sty}{%
\usepackage{microtype}
\UseMicrotypeSet[protrusion]{basicmath} % disable protrusion for tt fonts
}{}
\usepackage[unicode=true]{hyperref}
\hypersetup{
            pdfborder={0 0 0},
            breaklinks=true}
\urlstyle{same}  % don't use monospace font for urls
\usepackage{graphicx,grffile}
\makeatletter
\def\maxwidth{\ifdim\Gin@nat@width>\linewidth\linewidth\else\Gin@nat@width\fi}
\def\maxheight{\ifdim\Gin@nat@height>\textheight\textheight\else\Gin@nat@height\fi}
\makeatother
% Scale images if necessary, so that they will not overflow the page
% margins by default, and it is still possible to overwrite the defaults
% using explicit options in \includegraphics[width, height, ...]{}
\setkeys{Gin}{width=\maxwidth,height=\maxheight,keepaspectratio}
\IfFileExists{parskip.sty}{%
\usepackage{parskip}
}{% else
\setlength{\parindent}{0pt}
\setlength{\parskip}{6pt plus 2pt minus 1pt}
}
\setlength{\emergencystretch}{3em}  % prevent overfull lines
\providecommand{\tightlist}{%
  \setlength{\itemsep}{0pt}\setlength{\parskip}{0pt}}
\setcounter{secnumdepth}{0}
% Redefines (sub)paragraphs to behave more like sections
\ifx\paragraph\undefined\else
\let\oldparagraph\paragraph
\renewcommand{\paragraph}[1]{\oldparagraph{#1}\mbox{}}
\fi
\ifx\subparagraph\undefined\else
\let\oldsubparagraph\subparagraph
\renewcommand{\subparagraph}[1]{\oldsubparagraph{#1}\mbox{}}
\fi

\date{}


\usepackage{enumitem}
\makeatletter
\newcommand{\xslalph}[1]{\expandafter\@xslalph\csname c@#1\endcsname}
\newcommand{\@xslalph}[1]{%
    \ifcase#1\or а\or б\or в\or г\or д\or e\or є\or ж\or з\or i%
    \or й\or к\or л\or м\or н\or о\or п\or р\or с\or т%
    \or у\or ф\or х\or ц\or ч\or ш\or ю\or я\or аа\or бб\or вв%
    \else\@ctrerr\fi%
}
\AddEnumerateCounter{\xslalph}{\@xslalph}{m}
\makeatother


\begin{document}


\newpage
\subsection{16. Перетворення типів та робота з виключеннями}
\setcounter{subsection}{1}



\begin{itemize}
\item
  Які варіанти перетворень стандартних типів один між іншим можливі в
  Сі++?
\item
  Яким перетворенням краще скористатись для перетворень між цілими
  типами? Яким при перетворення цілих до дійсного та навпаки?
\item
  Чим відрізняються перетворення вгору та вниз? Яке перетворення типу
  краще для перетворення вгору, а яке вниз?
\item
  Чому не можна відловити виключення при діленні на нуль в Сі++ зі
  стандартними типами?
\item
  Як створити власне виключення в Сі++? Як його коректно обробити?
\item
  Яке стандартне виключення дозволяє коректно обробити static\_cast?
\item
  Як складнощі виникають якщо виключення виникає в деструкторі класу?
\item
  Як коректно працювати з виключенням, що виникає в конструкторі класу?
\end{itemize}

Задачі для аудиторної роботи

\begin{enumerate}
\def\labelenumi{\arabic{enumi})}

\item
  В класі Раціональній дріб з попередньої лекції напишіть методи
  введення, виведення (cin\textgreater{}\textgreater{},
  cout\textless{}\textless{}) та оператори віднімання, ділення як
  перевантажені оператори. Тобто з типом Раціональній дріб можна тепер
  працювати як зі стандартним типом. Чому краще перевантажити два
  оператори віднімання? Перепишіть методи введення
  (cin\textgreater{}\textgreater{}) та конструктор і сеттери, щоб вони
  кидали виключення при ініціалізації знаменнику нулем. Коректно
  обробить в коді це виключення. Напишіть дружню функцію запису
  Раціонального дробу в файл, яка буде викидати виключення при
  некоректному відкритті файлу та обробить його в тілі програми.

\item
  Створіть клас Людина (члени: ПІБ, стать, вік) та його наслідники
  Студент (додано: курс, група, ВУЗ, Викладач (додано: ВУЗ, посада,
  з.п.). Методи введення, виведення, конструктори для різної кількості
  вхідних даних. Створіть клас Аспірант, що є наслідником і студента і викладача.
 Коректно визначте член ВУЗ для нього. 

 Створить програму що буде вводити масив Людей, серед яких є Студенти,
Викладачі, Аспіранти. Без створення нових членів класу виведіть коректно
ВУЗ для кожного екземпляру масиву. Забезпечте обробку помилок для коректного вводу людей.
\end{enumerate}

Задачі для самостійної роботи
\begin{enumerate}
\def\labelenumi{\arabic{enumi})}
\setcounter{enumi}{2}
\item
Скласти функцію для обчислення значення натурального
числа за заданим рядком символів, який є записом цього числа у системі
числення за основою $b$ (\(2 \leq b \leq 16\)). Використати функцію, яка
за заданим символом повертає відповідну цифру у системі числення за
основою $b$. Використати у цій функції твердження про стан програми assert
для перевірки того, що відповідний символ є цифрою у системі числення за
основою $b$. Обробити помилку неправильного символу рядка та
показати змістовне повідомлення про помилку створивши власне виключення.

\item
Скласти власний клас для комплексного типу з методами введення/виведення 
та арифметичним операціями. Напишіть функцію для обчислення суми всіх доданків, модуль
яких не менше $\varepsilon \ge 0$, у комплексній точці $z$:

\(\text{arctg}\left( z \right) = z - \frac{z^{3}}{3} + \frac{z^{5}}{5} - \cdots + {( - 1)}^{n}\frac{z^{2n + 1}}{2n + 1} + \cdots,\ \ \ \ (\left| z \right| < 1)\).

Використати у цій функції твердження про стан програми для перевірки
того, що параметр $z$ відповідає заданій умові та зробить обробку
всіх можливих виключень -- включаючи некоректне введення та виділення
пам'яті під масиви. Обробити у програмі помилку неправильного значення
$z$ та показати змістовне повідомлення про помилку.

\item
Описати клас Трьохбайтне ціле число для роботи з цілими числами,
представленими трьома байтами. Інтервал представлення при цьому --- від
$-2^{23}$ до $2^{23}-1$. 
Зробіть методи та конструктор вводу, що оброблюють введено ціле число
та кидають виключення при некоректному вводі та перезаватажте арифметичні дії. 
Арифметичні дії не повинні дозволяти переповнення інтервалу представлення, 
тобто $2^{23}-1 + 1$ --- помилка. Якщо
результат операції виводить за межі інтервалу представлення, повинна
ініціюватися помилка переповнення. 
Перевизначити у цьому класі операції +, -, *.
Описати також 3 класи обробки помилок для трьохбайтних цілих чисел:
загальний клас обробки помилок та два його підкласи для обробки помилки
переповнення та помилки ділення на 0.

Використати цей клас для розв'язання задач:
\begin{itemize}
\item
обчислення $n!$;
\item
обчислення $x^{n}$, де $x$ --- ціле, $n$ --- натуральне.
\end{itemize}
Забезпечити обробку помилок при виконанні обчислень.


\item
Створіть клас для роботи з бінарними файлами, в яких записані цілі числа.
В класи визначені члени: ім'я файлу, кількість чисел у файлі.
Реалізуйте методи, введення чисел з консолі в файл, створення файлу з масиву чисел,
виведення змісту файлу на консоль, повернути число за даним номером, 
додавання до файлу масиву чисел в кінець, видалення числа за даним номером.
Забезпечити обробку помилок при роботі з файлами. 
Створіть відповідні виключення --- проблеми при створенні файлу,
проблеми при читанні з файлу, некоректні номери чи кількість чисел. 

\item
Створіть клас для роботи з текстовими файлами, в яких записані дійсні числа
які розділяються пропусками в одному рядку та можуть бути розташовані у
різних рядках.
В класи визначені члени: ім'я файлу, кількість чисел у файлі, кількість рядків файлу.
Реалізуйте методи:
\begin{itemize}
\item
введення чисел з консолі в файл рядок за рядком, 
\item
створення файлу з двовимірного масиву чисел,
\item
виведення змісту файлу на консоль, повернути число за даним номером, 
\item
додавання до файлу масиву чисел в кінець новим рядком, 
\item
видалення числа за даним номером рядку та місцем в ньому.
\end{itemize}

Створіть відповідні виключення --- проблеми при створенні файлу,
проблеми при читанні з файлу, некоректні номери чи кількість чисел.
Забезпечити обробку помилок, якщо у файлі, що читаються, зустрічаються не дійсні числа.

\item

Описати клас Поліном, що заданий ступенем та масивом дійсних коефіцієнтів
та реалізувати методи: введення поліному з консолі та рядку,
виведення поліному, обчислення значення поліному у точці x, взяття
похідної поліному, суми, різниці та добутку поліномів.

Описати також клас обробки помилок при неправильному введенні поліному
(ступінь -- не невід'ємне ціле число, коефіцієнт -- не дійсне число) та
забезпечити ініціювання помилки при неправильному введенні.
Забезпечити обробку помилок неправильного введення поліному в основній програмі.

\item
Створіть клас роботи з рядком, який має настпну властивість: 
користувач задає власноруч допустиму множину символів, з яких може складатись цей рядок.
Члени класу: масив допустимих символів та його довжина,
масив введених символів та його довжина.
Методи класу:
\begin{itemize}
\item
перезавантажте методи введення/виведення в/з консолі та в/з текствого файлу;
\item
методи зміни(додавання/видалення) допустимих символів;
\item
довжина рядку;
\item
конкатинація рядків (при цьому допустимі символи --- це перетин 
множин допустимих символів, 
тобто після конкатинації в нас може зменшитися ітоговий рядок);
\item
хеш рядку (ваш будь-який розумний варіант хешу).
\end{itemize}
Забезпечити ініціювання помилки при неправильному введенні та роботі з рядками 
та роботі з файлами.

\item

Реалізуйте клас Вектор, що ініціалізується кількістю елементів масиву N
  та виділяє при цьому пам'ять під N дійсних чисел. Створіть методи для
  заповнення членів цього масиву (через конструктор та окремим методом)
  та конкретного елементу вектору за номером. 
  Написати методи для введення/виведення таких векторів з файлу,
  скалярного та векторного добутку (за можливості) для цих векторів та обробіть
  за допомогою виключень проблеми з введенням та арифметичними операціями та методами 
 доступу над векторами. Також спробуйте врахувати можливі проблеми з пам'яттю.

 
\end{enumerate}

\end{document}
