\documentclass[]{article}
\usepackage{lmodern}
\usepackage{amssymb,amsmath}
\usepackage{ifxetex,ifluatex}


\usepackage[utf8]{inputenc}
\usepackage[english,russian,ukrainian]{babel}

\usepackage{fixltx2e} % provides \textsubscript
\ifnum 0\ifxetex 1\fi\ifluatex 1\fi=0 % if pdftex
  \usepackage[T1]{fontenc}
  \usepackage[utf8]{inputenc}
\else % if luatex or xelatex
  \ifxetex
    \usepackage{mathspec}
  \else
    \usepackage{fontspec}
  \fi
  \defaultfontfeatures{Ligatures=TeX,Scale=MatchLowercase}
\fi
% use upquote if available, for straight quotes in verbatim environments
\IfFileExists{upquote.sty}{\usepackage{upquote}}{}
% use microtype if available
\IfFileExists{microtype.sty}{%
\usepackage{microtype}
\UseMicrotypeSet[protrusion]{basicmath} % disable protrusion for tt fonts
}{}
\usepackage[unicode=true]{hyperref}
\hypersetup{
            pdfborder={0 0 0},
            breaklinks=true}
\urlstyle{same}  % don't use monospace font for urls
\usepackage{graphicx,grffile}
\makeatletter
\def\maxwidth{\ifdim\Gin@nat@width>\linewidth\linewidth\else\Gin@nat@width\fi}
\def\maxheight{\ifdim\Gin@nat@height>\textheight\textheight\else\Gin@nat@height\fi}
\makeatother
% Scale images if necessary, so that they will not overflow the page
% margins by default, and it is still possible to overwrite the defaults
% using explicit options in \includegraphics[width, height, ...]{}
\setkeys{Gin}{width=\maxwidth,height=\maxheight,keepaspectratio}
\IfFileExists{parskip.sty}{%
\usepackage{parskip}
}{% else
\setlength{\parindent}{0pt}
\setlength{\parskip}{6pt plus 2pt minus 1pt}
}
\setlength{\emergencystretch}{3em}  % prevent overfull lines
\providecommand{\tightlist}{%
  \setlength{\itemsep}{0pt}\setlength{\parskip}{0pt}}
\setcounter{secnumdepth}{0}
% Redefines (sub)paragraphs to behave more like sections
\ifx\paragraph\undefined\else
\let\oldparagraph\paragraph
\renewcommand{\paragraph}[1]{\oldparagraph{#1}\mbox{}}
\fi
\ifx\subparagraph\undefined\else
\let\oldsubparagraph\subparagraph
\renewcommand{\subparagraph}[1]{\oldsubparagraph{#1}\mbox{}}
\fi

\date{}


\usepackage{enumitem}
\makeatletter
\newcommand{\xslalph}[1]{\expandafter\@xslalph\csname c@#1\endcsname}
\newcommand{\@xslalph}[1]{%
    \ifcase#1\or а\or б\or в\or г\or д\or e\or є\or ж\or з\or i%
    \or й\or к\or л\or м\or н\or о\or п\or р\or с\or т%
    \or у\or ф\or х\or ц\or ч\or ш\or ю\or я\or аа\or бб\or вв%
    \else\@ctrerr\fi%
}
\AddEnumerateCounter{\xslalph}{\@xslalph}{m}
\makeatother


\begin{document}


\newpage
\subsection{14. Створення власних класів. Інкапсуляція.}
\setcounter{subsection}{1}


\begin{itemize}
\item
Що таке класи і які шляхи визначення класів в Сі++?
\item
Яким чином можна визначити методи класу?
\item
Приватний та публічний доступ до членів та методів. Яка різниця?
\item
Які методи в класі визначені за замовченням? Як і коли потрібно ці
методи визначати самостійно?
\item 
Шляхи визначення конструктору класу. Як викликати конструктор в
головній функції?
\item
Статичні члени та методи класу. Як визначити і коли вони потрібні?
\item 
Дружні класи та методи. Як вони використовуються?
\end{itemize}

Задачі для аудиторної роботи
\begin{enumerate}
\def\labelenumi{\arabic{enumi})}

\item 

Визначити клас раціональне число з членами: nominator --- ціле
число, denominator --- натуральне число. Визначить наступне:
\begin{itemize}
\item
методи введення та виведення з терміналу;
\item 
методи додавання та множення раціонального числа;
\item
зробіть члени класу приватними та визначить методи ініціалізації
окремо чисельника і знаменника (при цьому не дайте користувачу
можливість ініціалізувати знаменник нулем);
\item
cтворіть приватний метод класу для скорочення раціонального числа
через НСД;
\item визначить конструктори класу який ініціалізує за замовченням
раціональне число одиницями та конструктор, що ініціалізує його двома
довільними числами;
\item також у класі перевантажте основні арифметичні оператори, оператори
порівняння та інші оператори, що необхідні для роботи з раціональними
числами.
\end{itemize}

Використовуючи цей клас, розв'яжіть такі задачі:
\begin{itemize}
\item
знайдіть найменше раціональне число в масиві раціональних чисел;
\item
підрахуйте суму ряду за формулою Грегорі з точністю 0.01:

\[\frac{\pi}{4} = 1 - \frac{1}{3} + \frac{1}{5} - \frac{1}{7} + \ldots\]

\end{itemize}

\item
  На базі класу Точка напишіть програму, що дозволяє вводити
  багатокутник з будь якої кількості вершин вводячи точки доки
  користувач не відповість на запитання «Ввести точку?» - «Ні». Після
  цього виведіть інформацію про кількість точок у багатокутнику та
  виведіть його периметр.
\item
  Визначить клас Поліном, що ініціалізується кількістю елементів масиву
  N та виділяє при цьому пам'ять під N дійсних чисел. Створіть методи
  для заповнення членів цього масиву (через конструктор та окремим
  методом) та конкретного коефіцієнту за номером. Визначить деструктор
  та копіконструктор. 

  Визначить свою дружню функцію для цього класу для введення/виведення
його з консолі у бінарний файл.


\end{enumerate}

Задачі для самостійної роботи

Описати класи розділивши інтерфейс та реалізацію та заборонивши введення
некоректних даних, з методами введення/виведення та іншими:

\begin{enumerate}
\def\labelenumi{\arabic{enumi})}
\setcounter{enumi}{3}
\item
  Описати клас \textbf{Точка} на площині. Реалізуйте методи введення,
  виведення. Описати клас \textbf{Відрізок} на площині, що складається
  з 2-х точок та містить крім введення/виведення методи підрахунку
  середини відрізку, довжини відрізку. За допомогою визначення
  порожньої Точки реалізуйте метод перетину двох відрізків, що повертає
  Точку (у випадку, якщо цих точок декілька виведіть будь-яку з них, а
  якщо жодної -- порожній відрізок). Описати клас \textbf{Трикутник} з 
  методами введення/виведення, періметру та площі.
 

\item
  Описати клас \textbf{Коло} на площині, що задається координатами
  центру та радіусом. Описати методи отримання довжини діаметру, площі
  та периметру кола, перетину двох кіл (повертає відповідно 0,1 або 2
  точки як масив через змінний аргумент).

\item
  Описати клас \textbf{Прямокутник}. Сторони прямокутника паралельні
  осям координат. Для прямокутника задані координати лівого верхнього
  кута та довжини сторін. Описати методи отримання довжини кожної зі
  сторін, площі та периметру, перетину двох прямокутників (якщо перетин
  порожній -- поверніть Прямокутник вигляду(-1,-1,-1,-1)).
\item
  Описати клас \textbf{Трикутник}. Основа трикутника паралельна осі
  $x$ координат. Для трикутника задані лівий нижній кут та довжини
  2 сторін. Описати методи отримання довжини кожної зі сторін, кутів,
  площі та периметру.

\item
  Описати класи з методами визначення різниці:
\begin{enumerate}[label=\xslalph*)]
\item \textbf{Час} (години, хвилини, секунди);
\item \textbf{Дата}(рік, місяць, день).
\end{enumerate}

\item
  Описати клас ігрова \textbf{Дошка}(визначається розміром та назвою
  гри: шашки (міжнародні, російські та турецькі, шахи, нарди) та
  \textbf{Фігура} (назва, гра, масив можливих ходів -- ходи описуються в
  термінах зрозумілих класу Дошка).

\item
  Описати класи з методами додавання та різниці:
\begin{enumerate}[label=\xslalph*)]
\item \textbf{Валюта}( назва валюти, значення, центи(копійки));
\item \textbf{Товар} (назва товару, вартість, валюта в який вимірюється
вартість, одиниця в який вимірюються товар).
\end{enumerate}

\item
  Створити клас \textbf{Book} (Книжка -- назва, автор, кількість сторінок, рік
  видання) та реалізувати програму пошуку книжки за авторами та назвою в
  каталозі (каталог -- масив книжок, що зберігається у файлі).
  
\item
  Визначить клас \textbf{Вектор}, що ініціалізується кількістю елементів масиву N
  та виділяє при цьому пам'ять під N дійсних чисел. Створіть методи для
  заповнення членів цього масиву (через конструктор та окремим методом)
  та конкретного елементу вектору за номером. Визначить деструктор та
  копіконструктор. Із використанням динамічних масивів розв'язати
  задачу: у двох масивах містяться коефіцієнти векторів степеню $m$ і $n$
  відповідно. Написати методи для введення/виведення таких векторів з файлу,
  скалярного та векторного добутку (за можливості) для цих векторів, або
 змістовного повідомлення, чому така операція неможлива.
 
\item
  Опишіть класи \textbf{Matrix3} та \textbf{Vector3}, що є відповідно матрицею розмірності
  3х3 та тривімірним вектором. Перевантажте математичні оператори для
  цих класів та спеціальні методи (множення матриці на вектор у тому
  числі). Функцію abs() визначте для матриці та вектору як визначення
  норми. Для матриці опишіть метод det(), що повертає визначник цієї
  матриці.

\item
Створіть клас для реалізації гри «Хрестики-нолики», який має наступні методи: 
\begin{itemize}
\item
малювання початкового стану за допомогою символів '|' та '\_'; 
\item
малювання символу в даному полі за допомогою символів пробілу, 'O' та 'X'; 
\item
приймання ходу гравця з клавіатури (з превіркою коректності вводу, 
унеможивленням введення гравцем некоректного ходу та можливість виходу з гри);
\item
перевірка на те що гра закінчилось та визначення результату гри. 
\end{itemize}
В головній програмі розіграйте партію для перевірки даних методів.

\item
Опишіть класи:
\begin{itemize}
\item
\textbf{Гість}, що містить всю необхідну інформацію про жильця
деякого готелю: ім'я, період проживання, номер в отелі тощо.
\item
\textbf{Готель}, що містить масив номерів отелю, вартість кожного з них і т.п.
\end{itemize}
Використовуючи вищенаведені класи розв'язати задачі:
\begin{itemize}
\item відомість про кількість вільних кімнат у готелі;
\item пошуку вільної кімнати у зазначений період;
\item вартості проживання даного жильця у зазначений період;
\item кімната гостя у готелі (у заданий період).
\end{itemize}

\item
Визначити клас Квадратне рівняння. Реалізувати методи для пошуку коренів,
екстремумів, а також інтервалів убування / зростання. Створити масив об'єктів і 
визначити найбільші і найменші значення коренів.
\item
Визначити клас Інтервал с урахуванням включення/невключення країв. 
Створити методи по знаходженню перетину і об'єднанню інтервалів, 
причому інтервали, що немають спільних точок, перетинатися /об'єднуватися неможуть. 
Створить масив з $n$ інтервалів і визначить відстань між найбільш віддаленими кінцями.

\item
Визначити клас Точка на площині (в просторі) та в часі. 
Задати рух точки у певному напрямку. Створити методи по знаходженню швидкості та прискорення точки. 
Перевірити для двох точок можливість перетину траєкторій. 
Визначити відстань між двома точками в заданний момент часу.
Ввести масив точок та підрахувати кількість всіх перетинів траєкторій за даний період часу

\end{enumerate}

Додаткові задачі:

\begin{enumerate}
\def\labelenumi{\arabic{enumi})}
\setcounter{enumi}{18}
\item
Доповніть задачу 3) методами ініціалізації через рядок та текстові й бінарні файли.
Реалізувати методи: введення поліному, виведення поліному, обчислення
значення поліному у точці $x$, взяття похідної поліному, суми, різниці та
добутку поліномів. Використати цей клас для розв'язання задачі: ввести два
поліноми $P1$, $P2$ та рядок, який містить вираз, що залежить від двох
поліномів (наприклад, $P1 + P2*(P1- P2) $). Обчислити поліном, який буде значенням цього виразу.

\end{enumerate}

\end{document}
