\documentclass[]{article}
\usepackage{lmodern}
\usepackage{amssymb,amsmath}
\usepackage{ifxetex,ifluatex}


\usepackage[utf8]{inputenc}
\usepackage[english,russian,ukrainian]{babel}

\usepackage{fixltx2e} % provides \textsubscript
\ifnum 0\ifxetex 1\fi\ifluatex 1\fi=0 % if pdftex
  \usepackage[T1]{fontenc}
  \usepackage[utf8]{inputenc}
\else % if luatex or xelatex
  \ifxetex
    \usepackage{mathspec}
  \else
    \usepackage{fontspec}
  \fi
  \defaultfontfeatures{Ligatures=TeX,Scale=MatchLowercase}
\fi
% use upquote if available, for straight quotes in verbatim environments
\IfFileExists{upquote.sty}{\usepackage{upquote}}{}
% use microtype if available
\IfFileExists{microtype.sty}{%
\usepackage{microtype}
\UseMicrotypeSet[protrusion]{basicmath} % disable protrusion for tt fonts
}{}
\usepackage[unicode=true]{hyperref}
\hypersetup{
            pdfborder={0 0 0},
            breaklinks=true}
\urlstyle{same}  % don't use monospace font for urls
\usepackage{graphicx,grffile}
\makeatletter
\def\maxwidth{\ifdim\Gin@nat@width>\linewidth\linewidth\else\Gin@nat@width\fi}
\def\maxheight{\ifdim\Gin@nat@height>\textheight\textheight\else\Gin@nat@height\fi}
\makeatother
% Scale images if necessary, so that they will not overflow the page
% margins by default, and it is still possible to overwrite the defaults
% using explicit options in \includegraphics[width, height, ...]{}
\setkeys{Gin}{width=\maxwidth,height=\maxheight,keepaspectratio}
\IfFileExists{parskip.sty}{%
\usepackage{parskip}
}{% else
\setlength{\parindent}{0pt}
\setlength{\parskip}{6pt plus 2pt minus 1pt}
}
\setlength{\emergencystretch}{3em}  % prevent overfull lines
\providecommand{\tightlist}{%
  \setlength{\itemsep}{0pt}\setlength{\parskip}{0pt}}
\setcounter{secnumdepth}{0}
% Redefines (sub)paragraphs to behave more like sections
\ifx\paragraph\undefined\else
\let\oldparagraph\paragraph
\renewcommand{\paragraph}[1]{\oldparagraph{#1}\mbox{}}
\fi
\ifx\subparagraph\undefined\else
\let\oldsubparagraph\subparagraph
\renewcommand{\subparagraph}[1]{\oldsubparagraph{#1}\mbox{}}
\fi

\date{}


\usepackage{enumitem}
\makeatletter
\newcommand{\xslalph}[1]{\expandafter\@xslalph\csname c@#1\endcsname}
\newcommand{\@xslalph}[1]{%
    \ifcase#1\or а\or б\or в\or г\or д\or e\or є\or ж\or з\or i%
    \or й\or к\or л\or м\or н\or о\or п\or р\or с\or т%
    \or у\or ф\or х\or ц\or ч\or ш\or ю\or я\or аа\or бб\or вв %
    \else\@ctrerr\fi%
}
\AddEnumerateCounter{\xslalph}{\@xslalph}{m}
\makeatother


\begin{document}

\section*{ Методичні рекомендації з курсу «Мова програмування С++» }

Вступ

1. Лінійні програми на Сі. Введення/виведення. Дійсний тип даних.

2. Використання математичної бібліотеки С. Створення власних функцій

3. Цілі типи Сі. Умовні конструкції.

4. Цикли.

5. Цикли. Рекурентні співвідношення. Рекурсія

6. Бітові операції

7. Статичні масиви. Лінійні масиви та багатовимірні масиви

8. Динамічні масиви. Робота з вказівниками

9. Робота з рядком, що закінчується нулем на С.

10. Структури. Створення власного типу

11. Робота з бінарним файлами на Сі

12. Введення/виведення на С++. Робота з текстовими файлами

13. Робота з класом рядок на С++.

14. Створення власних класів. Інкапсуляція.

15. Робота з класами. Наслідування та поліморфізм.

16. Перетворення типів та робота з виключеннями.

17. Створення шаблонів функцій та шаблонів класів

18. Стандартна бібліотека С++. Послідовні контейнери.

19. Стандартна бібліотека С++. Асоціативні контейнери.

20. Стандартна бібліотека С++. Алгоритми та функтори.

\subsection{ ВСТУП }

Мета цього посібника, надати студенту завдання для того, щоб практично
оволодіти потрібними навичками програмування на мовах С та С++ в рамках
дисципліни «Мова програмування С++». Теми обиралися автором таким чином,
щоб найбільш швидким темпом здобути навичкі для практичного
програмування за 20 занять, тому деякі теми та розділи програмування на
С та С++, які автор вважає занадто складним або не обовязковими з точки
зору практики програмування, не входять до цього задачника, а винесені
на самостійну роботу або в якості завдань на курсові проекти.

Завдання посібника розділені на 20 лабораторних робіт, кожна з яких
присвячена окремій темі, що вивчається в дисципліні. Завдання та теми
підбиралися таким чином, щоб вивчення синтаксису мови виходило
поступовим тому послідовне виконання лабораторних робіт є найкращим для
засвоєння та набуття відповідних навичок. Тому наполегливо рекомендуємо
дотримуватися послідовного виконання лабораторних робіт.

Матеріал кожної лабораторної роботи посібника складається з п'яти
блоків: контрольних запитань, завдань для аудиторної роботи та трьох
блоків завдань для самостійної роботи. Під час підготовки до практичного
заняття, студент повинен опрацювати блок контрольних запитань та знати
вичерпні відповіді на них. Блок завдань для аудиторної містять перелік
типових задач відповідної теми. Ці завдання студент має виконати
протягом практичного заняття самостійно або під керівництвом викладача.
Завдання для самостійної роботи студент виконує самостійно та звітує про
їхнє виконання викладачу. Як було зазначено вище, завдання для
самостійної роботи складається з трьох блоків, перший з яких є
обов'язковим для виконання.

Другий блок завдань є ідентичним по складності основному блоку завдань
для самостійної роботи та призначений для кращого засвоєння матеріалу.

Третій блок завдань складається з задач підвищеної складності та вимагає
від студента не лише досконалого опанування методів поточної теми, а й
матеріалу, що виходить за межі нормативного курсу.


\newpage
\subsection{ 2. Використання математичної бібліотеки С. Створення власних функцій }
\setcounter{subsection}{1}

Питання по темі 2:

\begin{itemize}
\item
  Як підключити математичні функції та скомпілювати програму, що
  використовує sin та arctan?

\item
  Як узнати скільки максимальна кількість значущих цифр в даному
  дійсному типі? Максимальну експоненту та мантису?

\item
  Як записати власну функцію на Сі? Як запустити її зі сталими
  аргументами та як з аргументами, що є змінними, в програмі?

\item
  Що таке головна функція (драйвер функція)?

\item
  Як перевірити роботу функції в головній функції якщо ми знаємо 
її значення в деяких точках? 

\end{itemize}

Аудиторні завдання:

\begin{enumerate}
\def\labelenumi{\arabic{enumi})}
\item
  Ввести дійсне число х та обчислити значення функції тригонометричного
  косинуса для нього.
\item
  Обчислити гіпотенузу $c$ прямокутного трикутника за катетами
  $a$ та $b$.
\item
  Обчислити площу трикутника $S$ за трьома сторонами $a$,
  $b$, $c$.
\item
  Обчислити відстань від точки \((x_{0},y_{0})\) до:
\begin{enumerate}[label=\xslalph*)]
\item заданої точки \((x,y)\);
\item заданої прямої \(ax + by + c = 0\);
\item точки перетину прямих \(x + by + c = 0\) і
\(ax + y + c = 0,\ \) де 
\(ab \neq 1\).
\end{enumerate}

\item
  Напишіть функцію, яка за найменшу кількість арифметичних операцій,
  обчислює значення многочлена для введеного з клавіатури значення
  $x$:
  \begin{enumerate}[label=\xslalph*)]
  \item \(y = x^{4} + 2x^{2} + 1\); 
  \item \(y = x^{4} + x^{3} + x^{2} + x + 1\);
  \item \(y = x^{5} + 5x^{4} + 10x^{3} + 10x^{2} + 5x + 1\);
  \item \(y = x^{9} + x^{3} + 1\);
  \item \(y = 16x^{4} + 8x^{3} + 4x^{2} + 2x + 1\); 
  \item \(y = x^{5} + x^{3} + x\).
  \end{enumerate}

\item
  Напишіть функцію $ Rosenbrock2d(x,y) = 100(x^{2} - y)^{2} + (x - 1)^{2}$ 
 та перевірте її результат на довільних трьох парах дійсних чисел.

\item
  Трикутник вводиться координатами своїх вершин, які вводяться так: в
  першому рядку через пробіл два дійсних числа --- координати точки А,
  пропускається рядок, в третьому рядку через пробіл два дійсних числа
  --- координати Б, пропускається рядок, через пробіл --- координати
  точки С. Підрахувати площу трикутника. (Вказівка: напишіть функції
  підрахунку довжини відрізка та функції обчислення площі трикутника за
  довжинами сторін)
\end{enumerate}

Завдання для самостійної роботи

\begin{enumerate}
\def\labelenumi{\arabic{enumi})}
\setcounter{enumi}{7}
\item
  Обчислити площу еліпса за координатами його радіусів.
\item
  В трикутнику відомо довжини всіх сторін. Обчислити довжини його:
  \begin{enumerate}[label=\xslalph*)]
   \item
    медіан,
   \item
    бісектрис,
    \item
    висот.
  \end{enumerate}
\item
  Трикутник заданий величинами своїх кутів та радіусом вписаного кола.
  Обчисліть його площу.
\item
  Трикутник заданий довжиною своїх сторін. Знайти та вивести величину
  кутів трикутника у радіанах та градусах.
\item
  Знайти об'єм циліндра, якщо відомо його радіус основи та висоту.
\item
  Знайти об'єм конуса, якщо відомо його радіус основи та висоту.
\item
  Знайти об'єм тора з внутрішнім радіусом \(r\) і зовнішнім радіусом
  \(R\).
\item
  Знайти корені квадратного рівняння з коефіцієнтами \(a,b,c\), якщо відомо,
  що обидва корені в ньому існують. Перевірте ваш розв'язок на
  коефіцієнтах рівняння \(a=3,b=100,c=2\).
\item
  Скласти функцію для обчислення значення многочлена від двох змінних
  для введеної з клавіатури пари чисел \((x,y)\):
  \begin{enumerate}[label=\xslalph*)]
    \item
    \(f(x,y) = x^{3} + 3x^{2}y + 3xy^{2} + y^{3};\)
    \item
    \(f(x,y) = x^{2}y^{2} + x^{3}y^{3} + x^{4}y^{4};\)
    \item
    \(f(x,y) = x + y + x^{2} + y^{2} + x^{3} + y^{3} + x^{4} + y^{4}\).
  \end{enumerate}

\item
  Напишіть власні функції, що обчислюють наступні вирази та відповідні
  власні функції, що будуть рахувати похідні даних функцій(Приклад,
  функція \(f(x) = identity(x) = x\), її похідна
  \(g(x) = \textrm{identity\_derivative}(x) = 1\)) :


  \begin{enumerate}[label=\xslalph*)]
  \item   \(f(x) = th(x) = \frac{(e^{x} - e^{-x})}{(e^{x} + e^{-x})}\);
\item \(f(x) = bent(x) = \frac{\sqrt{x^{2} + 1} - 1}{2} + x\);
\item \(f(x) = softSign(x) = \frac{x}{1 + |x|}\);
\item \(f(x) = arctg(x) = tg^{-1}(x)\);
\item\(f(x) = gauss(x) = e^{-x^{2}}\);
\item \(f(x) = softPlus(x) = \ln(1 + e^{x})\);
\item \(f(x) = sigmoid(x) = {(1 + e^{-x})}^{-1}\);
\item \(f(x) = invsqrt(x,\alpha) = \frac{x}{\sqrt{1 + \alpha x^{2}}}\);
\item\(f(x) = sigmweight(x) = x*{(1 + e^{-x})}^{-1}\).

 \end{enumerate}
\end{enumerate}

\end{document}

