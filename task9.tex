\documentclass[]{article}
\usepackage{lmodern}
\usepackage{amssymb,amsmath}
\usepackage{ifxetex,ifluatex}


\usepackage[utf8]{inputenc}
\usepackage[english,russian,ukrainian]{babel}

\usepackage{fixltx2e} % provides \textsubscript
\ifnum 0\ifxetex 1\fi\ifluatex 1\fi=0 % if pdftex
  \usepackage[T1]{fontenc}
  \usepackage[utf8]{inputenc}
\else % if luatex or xelatex
  \ifxetex
    \usepackage{mathspec}
  \else
    \usepackage{fontspec}
  \fi
  \defaultfontfeatures{Ligatures=TeX,Scale=MatchLowercase}
\fi
% use upquote if available, for straight quotes in verbatim environments
\IfFileExists{upquote.sty}{\usepackage{upquote}}{}
% use microtype if available
\IfFileExists{microtype.sty}{%
\usepackage{microtype}
\UseMicrotypeSet[protrusion]{basicmath} % disable protrusion for tt fonts
}{}
\usepackage[unicode=true]{hyperref}
\hypersetup{
            pdfborder={0 0 0},
            breaklinks=true}
\urlstyle{same}  % don't use monospace font for urls
\usepackage{graphicx,grffile}
\makeatletter
\def\maxwidth{\ifdim\Gin@nat@width>\linewidth\linewidth\else\Gin@nat@width\fi}
\def\maxheight{\ifdim\Gin@nat@height>\textheight\textheight\else\Gin@nat@height\fi}
\makeatother
% Scale images if necessary, so that they will not overflow the page
% margins by default, and it is still possible to overwrite the defaults
% using explicit options in \includegraphics[width, height, ...]{}
\setkeys{Gin}{width=\maxwidth,height=\maxheight,keepaspectratio}
\IfFileExists{parskip.sty}{%
\usepackage{parskip}
}{% else
\setlength{\parindent}{0pt}
\setlength{\parskip}{6pt plus 2pt minus 1pt}
}
\setlength{\emergencystretch}{3em}  % prevent overfull lines
\providecommand{\tightlist}{%
  \setlength{\itemsep}{0pt}\setlength{\parskip}{0pt}}
\setcounter{secnumdepth}{0}
% Redefines (sub)paragraphs to behave more like sections
\ifx\paragraph\undefined\else
\let\oldparagraph\paragraph
\renewcommand{\paragraph}[1]{\oldparagraph{#1}\mbox{}}
\fi
\ifx\subparagraph\undefined\else
\let\oldsubparagraph\subparagraph
\renewcommand{\subparagraph}[1]{\oldsubparagraph{#1}\mbox{}}
\fi

\date{}


\usepackage{enumitem}
\makeatletter
\newcommand{\xslalph}[1]{\expandafter\@xslalph\csname c@#1\endcsname}
\newcommand{\@xslalph}[1]{%
    \ifcase#1\or а\or б\or в\or г\or д\or e\or є\or ж\or з\or i%
    \or й\or к\or л\or м\or н\or о\or п\or р\or с\or т%
    \or у\or ф\or х\or ц\or ч\or ш\or ю\or я\or аа\or бб\or вв%
    \else\@ctrerr\fi%
}
\AddEnumerateCounter{\xslalph}{\@xslalph}{m}
\makeatother


\begin{document}


\newpage
\subsection{ 9. Робота з рядком, що закінчується нулем на С }
\setcounter{subsection}{1}

\begin{itemize}
\item
Які є символьні типи в Сі/Сі++? Як їх коректно ввести/вивести на Сі? Які
є функції для роботи з символьним типом?
\item
Як ініціалізувати рядок на Сі? Як ввести/вивести рядок?
\item
Як порівняти два рядки? Як конкатенувати два рядки?
\item
Як з'ясувати, що даний рядок є словом? Натуральним числом?
\item
Як ввести речення та підрахувати кількість слів у ньому?
\item
Які варіанти є для переведення рядку в ціле число? Дійсне число? Як
обробити помилку цих приведень?
\item
Як перевести ціле число в рядок на Сі? Дійсне число?
\end{itemize}

Задачі для аудиторної роботи
\begin{enumerate}
\def\labelenumi{\arabic{enumi})}
\item
Дано рядок, серед символів якого є принаймні одна кома, а може й
немає її. Знайти номер:
\begin{enumerate}[label=\xslalph*)]
\item першої по порядку коми;
\item останньої по порядку коми;
\item кількості ком.
\end{enumerate}

\item
Надрукувати заданий рядок:
 \begin{enumerate}[label=\xslalph*)]
 \item виключивши з нього всі цифри і подвоївши знаки '+' та '-';
 \item виключивши з нього всі знаки '+', безпосередньо за якими знаходиться
цифра;
 \item виключивши з нього всі літери '\emph{b}', безпосередньо перед якими
знаходиться літера '\emph{c}';
 \item замінивши в ньому всі пари '\emph{ph}' на літеру '\emph{f}';
 \item виключивши з нього всі зайві пропуски, тобто з кількох, що йдуть
підряд, залишити один.
\end{enumerate}

\item
Виключити з заданого рядка групи символів, які знаходяться між '(' та
')'. Самі дужки теж мають бути виключені. Перевірте перед цим, що дужки
розставлено правильно (парами) та всередині кожної пари дужок немає
інших дужок.

\item
Задана послідовність символів, яка має вигляд:\\
\emph{d\textsubscript{1}} ± \emph{d\textsubscript{2}} ± \emph{...} ±
\emph{d\textsubscript{n }} ($n \ge 1 $, а текст \emph{d\textsubscript{i }} -- це натуральні
числа), за якою знаходиться знак рівності.
Напишить функцію, яка перевряє що рядок задовольняє вказаний вигляд та обчислити значення
цієї алгебраїчної суми. В противному випадку повернути найменше можливе ціле число.

\item
Задане натуральне число $n$ ($n<100$). Надрукувати в заданій системі числення $b$ ($1<b<17$)
цілі числа від 0 до $n$.
\end{enumerate}

Задачі для самостійної роботи

\begin{enumerate}
\def\labelenumi{\arabic{enumi})}
\setcounter{enumi}{5}
\item Заданий рядок, серед символів якого міститься двокрапка ':'.
 Отримати всі символи, розташовані:
\begin{enumerate}[label=\xslalph*)]
\item до першої двокрапки включно;
\item після першої двокрапки;
\item між першою і другою двокрапкою. Якщо другої двокрапки немає, 
то отримати всі символи, розміщені після єдиної двокрапки.
\end{enumerate}

\item
  Заданий текст надрукувати по рядках, розуміючи під рядком або наступні
  6 символів, якщо серед них немає коми (знак оклику, питання), або
  частину тексту до коми включно.
\item
  Знайти у даному рядку символ та довжину найдовшої послідовності
  однакових символів, що йдуть підряд.
\item
  Скласти програму підрахунку загального числа входжень символів '+',
  '-', '*' у рядок \emph{А}.
\item
  Скласти програму перетворення рядка \emph{А}, замінивши у ньому всі
  знаки оклику '!' крапками '.', кожну крапку -- трьома крапками '...',
  кожну зірочку '*'-- знаком '+'.
\item
  Рядок називається симетричним, якщо його символи, рівновіддалені від
  початку та кінця рядка, співпадають. Порожній рядок вважається
  симетричним. Перевірити рядок \emph{A} на симетричність.
\item
  Скласти програму видалення із рядка \emph{А} всіх входжень заданої
  групи символів.
\item
  Скласти програму перетворення слова \emph{А}, видаливши у ньому кожний
  символ '*' та подвоївши кожний символ, відмінний від '*'.
\item
  Скласти функцію підрахунку найбільшої кількості цифр, що йдуть підряд
  у рядку \emph{А}.
\item
  Скласти функція підрахунку числа входжень у рядок \emph{А} заданої
  послідовності літер.
\item
  Скласти функцію, яка за рядком \emph{А} та символом \emph{s} будує
  новий рядок, отриманий заміною кожного символу, наступного за
  \emph{s}, заданим символом \emph{c}.
\item
  Cкласти функцію перетворення рядка \emph{А} видаленням із нього всіх
  ком, які передують першій крапці, та заміною у ньому знаком '+' усіх
  цифр '3', які зустрічаються після першої крапки.
\item
  Cкласти функцію виведення на друк усіх цифр, які входять в заданий
  рядок, та окремо решти символів, зберігаючи при цьому взаємне
  розташування символів у кожній з цих двох груп.
\item
  Рядок називається монотонним, якщо він складається з зростаючої або
  спадної послідовності символів. Cкласти функцію перевірки
  монотонності рядка.
\item
За допомогою масиву рядків виведіть на екран зображення вісей координат на ділянці
[-5,5]x[-5,5] з підписами кожної вісі та точки О. Введіть дві координати (x,y) 
в цьому інтервалі та за допомогою символу „X“ виведіть знаходження цієї точки. 
Для цього створіть функції додавання точки до існуючих та перемалювання точки.

\item
  В заданий рядок входять тільки цифри та літери. Визначити, чи
  задовольняє він наступній властивості:
  \begin{enumerate}[label=\xslalph*)]
\item рядок є десятковим записом числа, кратного 9 (6, 4);
\item рядок починається з деякої ненульової цифри, за якою знаходяться
тільки літери і їх кількість дорівнює числовому значенню цієї цифри;
\item рядок містить (крім літер) тільки одну цифру, причому її числове
значення дорівнює довжині рядка;
\item сума числових значень цифр, які входять в рядок, дорівнює довжині
рядка;
\item рядок співпадає з початковим (кінцевим, будь-яким) відрізком ряду
0123456789;
\item рядок складається тільки з цифр, причому їх числові значення
складають арифметичну прогресію (наприклад, 3 5 7 9, 8 5 2, 2).
  \end{enumerate}

\item
  Перевірити, чи складається рядок з:
 \begin{enumerate}[label=\xslalph*)]
 \item 2 симетричних підрядків;
 \item $n$ симетричних підрядків (для всіх можливих $n>1$).
 \end{enumerate}

\item
  Знайти символ, кількість входжень якого у рядок \emph{A}
\begin{enumerate}[label=\xslalph*)]
  \item максимальна;
  \item мінімальна.
\end{enumerate}
\item
Дано рядок \emph{A}, що містить послідовність слів. Скласти функції, що визначають:
\begin{enumerate}[label=\xslalph*)]
\item кількість усіх слів;
\item кількість слів, що починаються із заданого символу \emph{c};
\item кількість слів, що закінчуються заданим символом \emph{c};
\item кількість слів, що починаються й закінчуються заданим символом \emph{c};
\item кількість слів, що починаються й закінчуються однаковим символом.
\end{enumerate}

\item 
  Виділити з рядка \emph{A} найбільший підрядок, перший і останній
  символи якого співпадають.
\item
  Виділити з рядка найбільший монотонний підрядок, коди послідовних
  символів якого відрізняються на 1.
\item
  Замінити всі пари однакових символів рядка, які йдуть підряд, одним
  символом. Наприклад, рядок \emph{`aabcbb'} перетворюється у
  \emph{`abcb'}.
\item
  Побудувати рядок \emph{S} з рядків \emph{S1}, \emph{S2} так, щоб у
  \emph{S} входили:
\begin{enumerate}[label=\xslalph*)]
\item ті символи \emph{S1}, які не входять у S2;
\item всі символи \emph{S1}, які не входять у \emph{S2}, та всі символи
\emph{S2}, які не входять у \emph{S1}.
\end{enumerate}

\item
Видалити з рядка симетричні початок та кінець. Наприклад, рядок
\emph{`abcdefba'} перетворюється у \emph{`cdef'}. 

\item
Написати функцію, яка виконує зсув по ключу (ключ задається) для малих 
та великих латинських літер. Наприклад: вхідні дані \emph{`Any`} -- рядок, 3 -- ключ.
Результат: \emph{`Dpq`}.

\end{enumerate}

Додаткові задачі:

\begin{enumerate}
\def\labelenumi{\arabic{enumi})}
\setcounter{enumi}{30}
\item
  Встановити, чи задовольняє заданий рядок заданому шаблону. Шаблон ---
  це рядок, що складається з символів а також наступних спецсимволів:
  символ «?» позначає будь-який символ, «*» означає будь-яку
  послідовність символів, у тому числі порожню, а «+» будь-яку непорожню
  послідовність символів (приклад, «ab*ra??da+ra»).

\item
 Напишить функцію обчислення хешу рідку. Хеш даного рядку
 (довжина рядку більше одиниці) це ціле число, 
що відповідає рядку та обчислюється за наступними варіантами:
\end{enumerate}

\begin{enumerate}[label=\xslalph*)]
\item Кожні послідовні 4 байти конкатинуються щоб утворити натуральне
число. Якщо кількість символів не кратна 4, то до рядка дописуються
потрібна кількість символів, що взята з кінця рядку справа наліво
(зеркальний падінг). Всі ці числа додаються за допомогою ``виключного
або'' (xor).
\item Кожні послідовні 4 байти конкатинуються щоб утворити натуральне
число. Якщо кількість символів не кратна 4, то до рядка дописуються
потрібна кількість нулевих символів (нульовий падінг. До всіх цих чисел
додається за допомогою ``виключного або'' номер по порядку цього числа.
Потім всі ці числа додаються за допомогою ``виключного або''.
\item Береться просте число p. Кожен послідовні байт множиться на $p^{i}$,
 де $i$ --- номер по порядку цього числа та береться
остача від ділення на $2^{32}$. Потім всі ці числа додаються по модулю $2^{32}$.
\end{enumerate}

\end{document}
